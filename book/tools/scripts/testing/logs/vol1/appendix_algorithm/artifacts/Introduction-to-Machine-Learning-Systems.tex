% Options for packages loaded elsewhere
% Options for packages loaded elsewhere
\PassOptionsToPackage{unicode,linktoc=all,pdfwindowui,pdfpagemode=FullScreen,pdfpagelayout=TwoPageRight}{hyperref}
\PassOptionsToPackage{hyphens}{url}
\PassOptionsToPackage{dvipsnames,svgnames,x11names}{xcolor}
%
\documentclass[
  9pt,
  letterpaper,
  abstract,
  titlepage]{scrbook}
\usepackage{xcolor}
\usepackage{amsmath,amssymb}
\setcounter{secnumdepth}{3}
\usepackage{iftex}
\ifPDFTeX
  \usepackage[T1]{fontenc}
  \usepackage[utf8]{inputenc}
  \usepackage{textcomp} % provide euro and other symbols
\else % if luatex or xetex
  \usepackage{unicode-math} % this also loads fontspec
  \defaultfontfeatures{Scale=MatchLowercase}
  \defaultfontfeatures[\rmfamily]{Ligatures=TeX,Scale=1}
\fi
\usepackage{lmodern}
\ifPDFTeX\else
  % xetex/luatex font selection
\fi
% Use upquote if available, for straight quotes in verbatim environments
\IfFileExists{upquote.sty}{\usepackage{upquote}}{}
\IfFileExists{microtype.sty}{% use microtype if available
  \usepackage[]{microtype}
  \UseMicrotypeSet[protrusion]{basicmath} % disable protrusion for tt fonts
}{}
% Make \paragraph and \subparagraph free-standing
\makeatletter
\ifx\paragraph\undefined\else
  \let\oldparagraph\paragraph
  \renewcommand{\paragraph}{
    \@ifstar
      \xxxParagraphStar
      \xxxParagraphNoStar
  }
  \newcommand{\xxxParagraphStar}[1]{\oldparagraph*{#1}\mbox{}}
  \newcommand{\xxxParagraphNoStar}[1]{\oldparagraph{#1}\mbox{}}
\fi
\ifx\subparagraph\undefined\else
  \let\oldsubparagraph\subparagraph
  \renewcommand{\subparagraph}{
    \@ifstar
      \xxxSubParagraphStar
      \xxxSubParagraphNoStar
  }
  \newcommand{\xxxSubParagraphStar}[1]{\oldsubparagraph*{#1}\mbox{}}
  \newcommand{\xxxSubParagraphNoStar}[1]{\oldsubparagraph{#1}\mbox{}}
\fi
\makeatother


\providecommand{\tightlist}{%
  \setlength{\itemsep}{0pt}\setlength{\parskip}{0pt}}\usepackage{longtable,booktabs,array}
\usepackage{calc} % for calculating minipage widths
% Correct order of tables after \paragraph or \subparagraph
\usepackage{etoolbox}
\makeatletter
\patchcmd\longtable{\par}{\if@noskipsec\mbox{}\fi\par}{}{}
\makeatother
% Allow footnotes in longtable head/foot
\IfFileExists{footnotehyper.sty}{\usepackage{footnotehyper}}{\usepackage{footnote}}
\makesavenoteenv{longtable}
\usepackage{graphicx}
\makeatletter
\newsavebox\pandoc@box
\newcommand*\pandocbounded[1]{% scales image to fit in text height/width
  \sbox\pandoc@box{#1}%
  \Gscale@div\@tempa{\textheight}{\dimexpr\ht\pandoc@box+\dp\pandoc@box\relax}%
  \Gscale@div\@tempb{\linewidth}{\wd\pandoc@box}%
  \ifdim\@tempb\p@<\@tempa\p@\let\@tempa\@tempb\fi% select the smaller of both
  \ifdim\@tempa\p@<\p@\scalebox{\@tempa}{\usebox\pandoc@box}%
  \else\usebox{\pandoc@box}%
  \fi%
}
% Set default figure placement to htbp
\def\fps@figure{htbp}
\makeatother
% definitions for citeproc citations
\NewDocumentCommand\citeproctext{}{}
\NewDocumentCommand\citeproc{mm}{%
  \begingroup\def\citeproctext{#2}\cite{#1}\endgroup}
\makeatletter
 % allow citations to break across lines
 \let\@cite@ofmt\@firstofone
 % avoid brackets around text for \cite:
 \def\@biblabel#1{}
 \def\@cite#1#2{{#1\if@tempswa , #2\fi}}
\makeatother
\newlength{\cslhangindent}
\setlength{\cslhangindent}{1.5em}
\newlength{\csllabelwidth}
\setlength{\csllabelwidth}{3em}
\newenvironment{CSLReferences}[2] % #1 hanging-indent, #2 entry-spacing
 {\begin{list}{}{%
  \setlength{\itemindent}{0pt}
  \setlength{\leftmargin}{0pt}
  \setlength{\parsep}{0pt}
  % turn on hanging indent if param 1 is 1
  \ifodd #1
   \setlength{\leftmargin}{\cslhangindent}
   \setlength{\itemindent}{-1\cslhangindent}
  \fi
  % set entry spacing
  \setlength{\itemsep}{#2\baselineskip}}}
 {\end{list}}
\usepackage{calc}
\newcommand{\CSLBlock}[1]{\hfill\break\parbox[t]{\linewidth}{\strut\ignorespaces#1\strut}}
\newcommand{\CSLLeftMargin}[1]{\parbox[t]{\csllabelwidth}{\strut#1\strut}}
\newcommand{\CSLRightInline}[1]{\parbox[t]{\linewidth - \csllabelwidth}{\strut#1\strut}}
\newcommand{\CSLIndent}[1]{\hspace{\cslhangindent}#1}

% =============================================================================
% LATEX HEADER CONFIGURATION FOR MLSYSBOOK PDF
% =============================================================================
% This file contains all LaTeX package imports, custom commands, and styling
% definitions for the PDF output of the Machine Learning Systems textbook.
%
% Key Features:
% - Harvard crimson branding throughout
% - Custom part/chapter/section styling
% - Professional table formatting with colored headers
% - Margin notes with custom styling
% - TikZ-based part dividers
% - Page numbering (Roman for frontmatter, Arabic for mainmatter)
%
% Note: This file is included via _quarto-pdf.yml and affects PDF output only.
% HTML/EPUB styling is handled separately via CSS files.
% =============================================================================

% =============================================================================
% PACKAGE IMPORTS
% =============================================================================

% Layout and positioning
% \usepackage[outercaption, ragged]{sidecap}  % Commented out to make figure captions inline instead of in margin
\usepackage{adjustbox}      % Adjusting box dimensions
\usepackage{afterpage}      % Execute commands after page break
\usepackage{morefloats}     % Increase number of floats
\usepackage{array}          % Enhanced table column formatting
\usepackage{atbegshi}       % Insert content at page beginning
%\usepackage{changepage}     % Change page dimensions mid-document
\usepackage{emptypage}      % Clear headers/footers on empty pages

% Language and text
\usepackage[english]{babel} % English language support
\usepackage{microtype}      % Improved typography and hyphenation

% Captions and floats
\usepackage{caption}
% Caption styling configuration
%\captionsetup[table]{belowskip=5pt}
\captionsetup{format=plain}
\DeclareCaptionLabelFormat{mylabel}{#1
#2:\hspace{1.0ex}}
\DeclareCaptionFont{ninept}{\fontsize{7pt}{8}\selectfont #1}

% Figure captions: Small font, bold label, ragged right
\captionsetup[figure]{labelfont={bf,ninept},labelsep=space,
belowskip=2pt,aboveskip=6pt,labelformat=mylabel,
justification=raggedright,singlelinecheck=false,font={ninept}}

% Table captions: Small font, bold label, ragged right
\captionsetup[table]{belowskip=6pt,labelfont={bf,ninept},labelsep=none,
labelformat=mylabel,justification=raggedright,singlelinecheck=false,font={ninept}}

% Typography fine-tuning
\emergencystretch=5pt       % Allow extra stretch to avoid overfull boxes

% Utility packages
\usepackage{etoolbox}       % For patching commands and environments

% Page layout and headers
\usepackage{fancyhdr}       % Custom headers and footers
\usepackage{geometry}       % Page dimensions and margins

% Graphics and figures
\usepackage{graphicx}       % Include graphics
\usepackage{float}          % Improved float placement
\usepackage[skins,breakable]{tcolorbox} % Coloured and framed text boxes
\tcbset{before upper=\setlength{\parskip}{3pt}}

% Tables
\usepackage{longtable}      % Multi-page tables

% Fonts and typography
\usepackage{fontspec}       % Font selection for LuaLaTeX
\usepackage{mathptmx}       % Times-like math fonts
\usepackage{newpxtext}      % Palatino-like font for body text

% Colors and visual elements
\usepackage[dvipsnames]{xcolor}  % Extended color support
\usepackage{tikz}           % Programmatic graphics
\usetikzlibrary{positioning}
\usetikzlibrary{calc}
\usepackage{tikzpagenodes}  % TikZ positioning relative to page

% Code listings
\usepackage{listings}       % Code highlighting

% Hyperlinks
\usepackage{hyperref}       % Clickable links in PDF

% Conditional logic
\usepackage{ifthen}         % If-then-else commands

% Math symbols
\usepackage{amsmath}        % AMS math extensions
\usepackage{amssymb}        % AMS math symbols
\usepackage{latexsym}       % Additional LaTeX symbols
\usepackage{pifont}         % Zapf Dingbats symbols
\providecommand{\blacklozenge}{\ding{117}}  % Black diamond symbol

% Lists
\usepackage{enumitem}       % Customizable lists

% Margin notes and sidenotes
\usepackage{marginfix}      % Fixes margin note overflow
\usepackage{marginnote}     % Margin notes
\usepackage{sidenotes}      % Academic-style sidenotes
\renewcommand\raggedrightmarginnote{\sloppy}
\renewcommand\raggedleftmarginnote{\sloppy}

% Typography improvements
\usepackage{ragged2e}       % Better ragged text
\usepackage[all]{nowidow}   % Prevent widows and orphans
\usepackage{needspace}      % Ensure minimum space on page

% Section formatting
\usepackage[explicit]{titlesec}  % Custom section titles
\usepackage{tocloft}        % Table of contents formatting

% QR codes and icons
\usepackage{fontawesome5}   % Font Awesome icons
\usepackage{qrcode}         % QR code generation
\qrset{link, height=15mm}

% =============================================================================
% FLOAT CONFIGURATION
% =============================================================================
% Allow more floats per page to handle figure-heavy chapters
\extrafloats{200}
\setcounter{topnumber}{12}       % Max floats at top of page
\setcounter{bottomnumber}{12}    % Max floats at bottom of page
\setcounter{totalnumber}{24}     % Max floats per page
\setcounter{dbltopnumber}{8}     % Max floats at top of two-column page
\renewcommand{\topfraction}{.95}  % Max fraction of page for top floats
\renewcommand{\bottomfraction}{.95}
\renewcommand{\textfraction}{.05}  % Min fraction of page for text
\renewcommand{\floatpagefraction}{.7}  % Min fraction of float page
\renewcommand{\dbltopfraction}{.95}

% Prevent "Float(s) lost" errors by flushing floats more aggressively
\usepackage{placeins}  % Provides \FloatBarrier

% =============================================================================
% COLOR DEFINITIONS
% =============================================================================
% Harvard crimson - primary brand color used throughout
\definecolor{crimson}{HTML}{A51C30}

% Quiz element colors
\definecolor{quiz-question-color1}{RGB}{225,243,248}  % Light blue background
\definecolor{quiz-question-color2}{RGB}{17,158,199}   % Blue border
\definecolor{quiz-answer-color1}{RGB}{250,234,241}    % Light pink background
\definecolor{quiz-answer-color2}{RGB}{152,14,90}      % Magenta border

% =============================================================================
% LIST FORMATTING
% =============================================================================
% Tighter list spacing for academic style
\def\tightlist{}
\setlist{itemsep=1pt, parsep=1pt, topsep=0pt,after={\vspace{0.3\baselineskip}}}
\let\tightlist\relax

\makeatletter
\@ifpackageloaded{framed}{}{\usepackage{framed}}
\@ifpackageloaded{fancyvrb}{}{\usepackage{fancyvrb}}
\makeatother

\makeatletter
%New float "codelisting" has been updated
\AtBeginDocument{%
\floatstyle{ruled}
\newfloat{codelisting}{!htb}{lop}
\floatname{codelisting}{Listing}
\floatplacement{codelisting}{!htb}
\captionsetup[codelisting]{labelfont={bf,ninept},labelformat=mylabel,
  singlelinecheck=false,width=\linewidth,labelsep=none,font={ninept}}%
\renewenvironment{snugshade}{%
   \def\OuterFrameSep{3pt}%
   \def\FrameCommand{\fboxsep=5pt\colorbox{shadecolor}}%
   \MakeFramed{\advance\hsize-\width\FrameRestore}%
   \leftskip 0.5em \rightskip 0.5em%
   \small% decrease font size
   }{\endMakeFramed}%
}
\makeatother

%The space before and after the verbatim environment "Highlighting" has been reduced
\fvset{listparameters=\setlength{\topsep}{0pt}\setlength{\partopsep}{0pt}}
\DefineVerbatimEnvironment{Highlighting}{Verbatim}{framesep=0mm,commandchars=\\\{\}}

\makeatletter
\renewcommand\fs@ruled{\def\@fs@cfont{\bfseries}\let\@fs@capt\floatc@ruled
\def\@fs@pre{\hrule height.8pt depth0pt \kern2pt}%
\def\@fs@post{\kern2pt\hrule\relax}%
\def\@fs@mid{\kern2pt\hrule\kern1pt}%space between float and caption
\let\@fs@iftopcapt\iftrue}
\makeatother


% =============================================================================
% HYPHENATION RULES
% =============================================================================
% Explicit hyphenation points for technical terms to avoid bad breaks
\hyphenation{
  light-weight
  light-weight-ed
  de-vel-op-ment
  un-der-stand-ing
  mod-els
  prin-ci-ples
  ex-per-tise
  com-pli-cat-ed
  blue-print
  per‧for‧mance
  com-mu-ni-ca-tion
  par-a-digms
  hy-per-ten-sion
  a-chieved
}

% =============================================================================
% CODE LISTING CONFIGURATION
% =============================================================================
% Settings for code blocks using listings package
\lstset{
breaklines=true,              % Automatic line wrapping
breakatwhitespace=true,       % Break at whitespace only
basicstyle=\ttfamily,         % Monospace font
frame=none,                   % No frame around code
keepspaces=true,              % Preserve spaces
showspaces=false,             % Don't show space characters
showtabs=false,               % Don't show tab characters
columns=flexible,             % Flexible column width
belowskip=0pt,               % Minimal spacing
aboveskip=0pt
}

% =============================================================================
% PAGE GEOMETRY
% =============================================================================
% MIT Press trim size: 7" x 10" (per publisher specifications)
% This is a standard academic textbook format providing good readability
% for technical content with figures and code blocks.
% Wide outer margin accommodates sidenotes/margin notes.
\geometry{
  paperwidth=7in,
  paperheight=10in,
  top=0.875in,
  bottom=0.875in,
  inner=0.875in,              % Inner margin (binding side)
  outer=1.75in,               % Outer margin (includes space for sidenotes)
  footskip=30pt,
  marginparwidth=1.25in,      % Width for margin notes
  twoside                     % Different left/right pages
}

% =============================================================================
% SIDENOTE STYLING
% =============================================================================
% Custom sidenote design with crimson vertical bar
\renewcommand{\thefootnote}{\textcolor{crimson}{\arabic{footnote}}}

% Save original sidenote command
\makeatletter
\@ifundefined{oldsidenote}{
  \let\oldsidenote\sidenote%
}{}
\makeatother

% Redefine sidenote with vertical crimson bar
\renewcommand{\sidenote}[1]{%
  \oldsidenote{%
    \noindent
    \color{crimson!100}                        % Crimson vertical line
    \raisebox{0em}{%
      \rule{0.5pt}{1.5em}                      % Thin vertical line
    }
    \hspace{0.3em}                             % Space after line
    \color{black}                              % Reset text color
    \footnotesize #1                           % Sidenote content
  }%
}

% =============================================================================
% FLOAT HANDLING
% =============================================================================
% Patch LaTeX's output routine to handle float overflow gracefully
% The "Float(s) lost" error occurs in \@doclearpage when \@currlist is not empty
% This patch silently clears pending floats that can't be placed
\makeatletter
\let\orig@doclearpage\@doclearpage
\def\@doclearpage{%
  \ifx\@currlist\@empty\else
    \global\let\@currlist\@empty
    \typeout{Warning: Floats cleared to prevent overflow}%
  \fi
  \orig@doclearpage
}
\makeatother

% Additional safety for structural commands
\let\originalbackmatter\backmatter
\renewcommand{\backmatter}{%
  \clearpage%
  \originalbackmatter%
}

\let\originalfrontmatter\frontmatter
\renewcommand{\frontmatter}{%
  \clearpage%
  \originalfrontmatter%
}

\let\originalmainmatter\mainmatter
\renewcommand{\mainmatter}{%
  \clearpage%
  \originalmainmatter%
}

% =============================================================================
% PAGE HEADERS AND FOOTERS
% =============================================================================
% Ensure chapters use fancy page style (not plain)
\patchcmd{\chapter}{\thispagestyle{plain}}{\thispagestyle{fancy}}{}{}

% Main page style with crimson headers
\pagestyle{fancy}
\fancyhf{}                                              % Clear all
\fancyhead[LE]{\small\color{crimson}\nouppercase{\rightmark}}  % Left even: section
\fancyhead[RO]{\color{crimson}\thepage}                 % Right odd: page number
\fancyhead[LO]{\small\color{crimson}\nouppercase{\leftmark}}   % Left odd: chapter
\fancyhead[RE]{\color{crimson}\thepage}                 % Right even: page number
\renewcommand{\headrulewidth}{0.4pt}                    % Thin header line
\renewcommand{\footrulewidth}{0pt}                      % No footer line

% Plain page style (for chapter openings)
\fancypagestyle{plain}{
  \fancyhf{}
  \fancyfoot[C]{\color{crimson}\thepage}                % Centered page number
  \renewcommand{\headrulewidth}{0pt}
  \renewcommand{\footrulewidth}{0pt}
}

% =============================================================================
% KOMA-SCRIPT FONT ADJUSTMENTS
% =============================================================================
% Apply crimson color to all heading levels
\addtokomafont{disposition}{\rmfamily\color{crimson}}
\addtokomafont{chapter}{\color{crimson}}
\addtokomafont{section}{\color{crimson}}
\addtokomafont{subsection}{\color{crimson}}

% =============================================================================
% ABSTRACT ENVIRONMENT
% =============================================================================
\newenvironment{abstract}{
  \chapter*{\abstractname}
  \addcontentsline{toc}{chapter}{\abstractname}
  \small
}{
  \clearpage
}

% =============================================================================
% HYPERLINK CONFIGURATION
% =============================================================================
% Crimson-colored links throughout, two-page PDF layout
\hypersetup{
  linkcolor=crimson,
  citecolor=crimson,
  urlcolor=crimson,
  pdfpagelayout=TwoPageRight,   % Two-page spread view
  pdfstartview=Fit               % Initial zoom fits page
}

% =============================================================================
% PART SUMMARY SYSTEM
% =============================================================================
% Allows adding descriptive text below part titles
\newcommand{\partsummary}{}     % Empty by default
\newif\ifhaspartsummary%
\haspartsummaryfalse%

\newcommand{\setpartsummary}[1]{%
  \renewcommand{\partsummary}{#1}%
  \haspartsummarytrue%
}

% Additional colors for part page backgrounds
\definecolor{BrownLL}{RGB}{233,222,220}
\definecolor{BlueDD}{RGB}{62,100,125}
\colorlet{BlueDD}{magenta}

% ===============================================================================
% PART STYLING SYSTEM
% ===============================================================================
%
% This system provides three distinct visual styles for book organization:
%
% 1. NUMBERED PARTS (\part{title}) - For main book sections
%    - Roman numerals (I, II, III, etc.) in top right corner
%    - Crimson title with horizontal lines above/below
%    - "Part I" label in sidebar
%    - Used for: foundations, principles, optimization, deployment, etc.
%
% 2. UNNUMBERED PARTS (\part*{title}) - For special sections like "Labs"
%    - Division-style geometric background (left side)
%    - No Roman numerals
%    - Used for: labs section
%
% 3. DIVISIONS (\division{title}) - For major book divisions
%    - Clean geometric background with centered title
%    - Used for: frontmatter, main_content, backmatter
%
% The Lua filter (inject-parts.lua) automatically routes parts by {key:xxx} commands
% to the appropriate LaTeX command based on the key name.
% ===============================================================================

% NUMBERED PARTS: Roman numeral styling for main book sections
\titleformat{\part}[display]
{\thispagestyle{empty}}{}{20pt}{
\begin{tikzpicture}[remember picture,overlay]
%%%
%%
\node[crimson,align=flush right,
inner sep=0,outer sep=0mm,draw=none,%
anchor=east,minimum height=31mm, text width=1.2\textwidth,
yshift=-30mm,font={%
\fontsize{98pt}{104}\selectfont\bfseries}]  (BG) at (current page text area.north east){\thepart};
%
\node[black,inner sep=0mm,draw=none,
anchor=mid,text width=1.2\textwidth,
 minimum height=35mm, align=right,
node distance=7mm,below=of BG,
font={\fontsize{30pt}{34}\selectfont}]
(BGG)  {\hyphenchar\font=-1 \color{black}\MakeUppercase {#1}};
\draw [crimson,line width=3pt] ([yshift=0mm]BGG.north west) -- ([yshift=0mm]BGG.north east);
\draw [crimson,line width=2pt] ([yshift=0mm]BGG.south west) -- ([yshift=0mm]BGG.south east);
%
\node[fill=crimson,text=white,rotate=90,%
anchor=south west,minimum height=15mm,
minimum width=40mm,font={%
\fontsize{20pt}{20}\selectfont\bfseries}](BP)  at
(current page text area.south east)
{{\sffamily Part}~\thepart};
%
\path[red](BP.north west)-|coordinate(PS)(BGG.south west);
%
% Part summary box commented out for cleaner design
% \ifhaspartsummary
% \node[inner sep=4pt,text width=0.7\textwidth,draw=none,fill=BrownLL!40,
% align=justify,font={\fontsize{9pt}{12}\selectfont},anchor=south west]
% at (PS) {\partsummary};
% \fi
\end{tikzpicture}
}[]

\renewcommand{\thepart}{\Roman{part}}

% UNNUMBERED PARTS: Division-style background for special sections
\titleformat{name=\part,numberless}[display]
{\thispagestyle{empty}}{}{20pt}{
\begin{tikzpicture}[remember picture,overlay]
%%%
\coordinate(S1)at([yshift=-200mm]current page.north west);
\draw[draw=none,fill=BlueDD!7](S1)--++(45:16)coordinate(S2)-
|(S2|-current page.north west)--(current page.north west)coordinate(S3)--(S1);
%
\coordinate(E1)at([yshift=-98mm]current page.north west);
\draw[draw=none,fill=BlueDD!15](E1)--(current page.north west)coordinate(E2)
--++(0:98mm)coordinate(E3)--(E1);
%
\coordinate(D1)at([yshift=15mm]current page.south west);
\draw[draw=none,fill=BlueDD!40,opacity=0.5](D1)--++(45:5.5)coordinate(D2)
-|(D2|-current page.north west)--(current page.north west)coordinate(D3)--(D1);
%%%%
\path[red](S2)-|(S2-|current page.east)coordinate(SS2);
%PART
\node[crimson,align=flush right,inner sep=0,outer sep=0mm,draw=none,anchor=south,
font={\fontsize{48pt}{48}\selectfont\bfseries}]  (BG) at ($(S2)!0.5!(SS2)$){\hphantom{Part}};
%%%
\path[green]([yshift=15mm]D2)-|coordinate(TPD)(BG.south east);
\node[inner sep=0mm,draw=none,anchor=south east,%text width=0.9\textwidth,
align=right,font={\fontsize{40pt}{40}\selectfont}]
(BGG) at (TPD)  {\color{crimson}\MakeUppercase {#1}};%\MakeUppercase {}
\end{tikzpicture}
}

% Define \numberedpart command for numbered parts
\newcommand{\numberedpart}[1]{%
\FloatBarrier%  % Flush all pending floats before part break
\clearpage
\thispagestyle{empty}
\stepcounter{part}%
\begin{tikzpicture}[remember picture,overlay]
%%%
%%
\node[crimson,align=flush right,
inner sep=0,outer sep=0mm,draw=none,%
anchor=east,minimum height=31mm, text width=1.2\textwidth,
yshift=-30mm,font={%
\fontsize{98pt}{104}\selectfont\bfseries}]  (BG) at (current page text area.north east){\thepart};
%
\node[black,inner sep=0mm,draw=none,
anchor=mid,text width=1.2\textwidth,
 minimum height=35mm, align=right,
node distance=7mm,below=of BG,
font={\fontsize{30pt}{34}\selectfont}]
(BGG)  {\hyphenchar\font=-1 \color{black}\MakeUppercase {#1}};
\draw [crimson,line width=3pt] ([yshift=0mm]BGG.north west) -- ([yshift=0mm]BGG.north east);
\draw [crimson,line width=2pt] ([yshift=0mm]BGG.south west) -- ([yshift=0mm]BGG.south east);
%
\node[fill=crimson,text=white,rotate=90,%
anchor=south west,minimum height=15mm,
minimum width=40mm,font={%
\fontsize{20pt}{20}\selectfont\bfseries}](BP)  at
(current page text area.south east)
{{\sffamily Part}~\thepart};
%
\path[red](BP.north west)-|coordinate(PS)(BGG.south west);
%
% Part summary box commented out for cleaner design
% \ifhaspartsummary
% \node[inner sep=4pt,text width=0.7\textwidth,draw=none,fill=BrownLL!40,
% align=justify,font={\fontsize{9pt}{12}\selectfont},anchor=south west]
% at (PS) {\partsummary};
% \fi
\end{tikzpicture}
\clearpage
}



% DIVISIONS: Clean geometric styling with subtle tech elements
% Used for frontmatter, main_content, and backmatter divisions
\newcommand{\division}[1]{%
\FloatBarrier%  % Flush all pending floats before division break
\clearpage
\thispagestyle{empty}
\begin{tikzpicture}[remember picture,overlay]

% Clean geometric background (original design)
\coordinate(S1)at([yshift=-200mm]current page.north west);
\draw[draw=none,fill=BlueDD!7](S1)--++(45:16)coordinate(S2)-
|(S2|-current page.north west)--(current page.north west)coordinate(S3)--(S1);

\coordinate(E1)at([yshift=-98mm]current page.north west);
\draw[draw=none,fill=BlueDD!15](E1)--(current page.north west)coordinate(E2)
--++(0:98mm)coordinate(E3)--(E1);

\coordinate(D1)at([yshift=15mm]current page.south west);
\draw[draw=none,fill=BlueDD!40,opacity=0.5](D1)--++(45:5.5)coordinate(D2)
-|(D2|-current page.north west)--(current page.north west)coordinate(D3)--(D1);

% Subtle tech elements - positioned in white areas for better visibility
% Upper right white area - more visible
\draw[crimson!40, line width=0.8pt] ([xshift=140mm,yshift=-60mm]current page.north west) -- ++(40mm,0);
\draw[crimson!40, line width=0.8pt] ([xshift=150mm,yshift=-70mm]current page.north west) -- ++(30mm,0);
\draw[crimson!35, line width=0.7pt] ([xshift=160mm,yshift=-60mm]current page.north west) -- ++(0,-15mm);
\draw[crimson!35, line width=0.7pt] ([xshift=170mm,yshift=-70mm]current page.north west) -- ++(0,10mm);

% Circuit nodes - upper right
\fill[crimson!50] ([xshift=160mm,yshift=-60mm]current page.north west) circle (1.5mm);
\fill[white] ([xshift=160mm,yshift=-60mm]current page.north west) circle (0.8mm);
\fill[crimson!50] ([xshift=170mm,yshift=-70mm]current page.north west) circle (1.3mm);
\fill[white] ([xshift=170mm,yshift=-70mm]current page.north west) circle (0.6mm);

% Lower right white area - enhanced visibility
\draw[crimson!45, line width=0.9pt] ([xshift=140mm,yshift=-190mm]current page.north west) -- ++(45mm,0);
\draw[crimson!45, line width=0.9pt] ([xshift=150mm,yshift=-200mm]current page.north west) -- ++(35mm,0);
\draw[crimson!40, line width=0.8pt] ([xshift=160mm,yshift=-190mm]current page.north west) -- ++(0,-20mm);
\draw[crimson!40, line width=0.8pt] ([xshift=170mm,yshift=-200mm]current page.north west) -- ++(0,15mm);

% Additional connecting lines in lower right
\draw[crimson!35, line width=0.7pt] ([xshift=130mm,yshift=-180mm]current page.north west) -- ++(25mm,0);
\draw[crimson!35, line width=0.7pt] ([xshift=145mm,yshift=-180mm]current page.north west) -- ++(0,-25mm);

% Circuit nodes - lower right (more prominent)
\fill[crimson!55] ([xshift=160mm,yshift=-190mm]current page.north west) circle (1.6mm);
\fill[white] ([xshift=160mm,yshift=-190mm]current page.north west) circle (0.9mm);
\fill[crimson!55] ([xshift=170mm,yshift=-200mm]current page.north west) circle (1.4mm);
\fill[white] ([xshift=170mm,yshift=-200mm]current page.north west) circle (0.7mm);
\fill[crimson!50] ([xshift=145mm,yshift=-180mm]current page.north west) circle (1.2mm);
\fill[white] ([xshift=145mm,yshift=-180mm]current page.north west) circle (0.6mm);

% Title positioned in center - clean and readable
\node[inner sep=0mm,draw=none,anchor=center,text width=0.8\textwidth,
align=center,font={\fontsize{40pt}{40}\selectfont}]
(BGG) at (current page.center)  {\color{crimson}\MakeUppercase {#1}};

\end{tikzpicture}
\clearpage
}

% LAB DIVISIONS: Circuit-style neural network design for lab sections
% Used specifically for lab platform sections (arduino, xiao, grove, etc.)
\newcommand{\labdivision}[1]{%
\FloatBarrier%  % Flush all pending floats before lab division break
\clearpage
\thispagestyle{empty}
\begin{tikzpicture}[remember picture,overlay]
% Circuit background with subtle gradient
\coordinate(S1)at([yshift=-200mm]current page.north west);
\draw[draw=none,fill=BlueDD!5](S1)--++(45:16)coordinate(S2)-
|(S2|-current page.north west)--(current page.north west)coordinate(S3)--(S1);

% TOP AREA: Circuit lines in upper white space
\draw[crimson!50, line width=1.5pt] ([xshift=30mm,yshift=-40mm]current page.north west) -- ++(60mm,0);
\draw[crimson!40, line width=1pt] ([xshift=120mm,yshift=-50mm]current page.north west) -- ++(50mm,0);
\draw[crimson!50, line width=1.5pt] ([xshift=40mm,yshift=-70mm]current page.north west) -- ++(40mm,0);

% Connecting lines in top area
\draw[crimson!30, line width=1pt] ([xshift=60mm,yshift=-40mm]current page.north west) -- ++(0,-20mm);
\draw[crimson!30, line width=1pt] ([xshift=145mm,yshift=-50mm]current page.north west) -- ++(0,10mm);

% Neural nodes in top area
\fill[crimson!70] ([xshift=60mm,yshift=-40mm]current page.north west) circle (2.5mm);
\fill[white] ([xshift=60mm,yshift=-40mm]current page.north west) circle (1.5mm);
\fill[crimson!60] ([xshift=145mm,yshift=-50mm]current page.north west) circle (2mm);
\fill[white] ([xshift=145mm,yshift=-50mm]current page.north west) circle (1mm);
\fill[crimson!80] ([xshift=80mm,yshift=-70mm]current page.north west) circle (2mm);
\fill[white] ([xshift=80mm,yshift=-70mm]current page.north west) circle (1mm);

% BOTTOM AREA: Circuit lines in lower white space
\draw[crimson!50, line width=1.5pt] ([xshift=20mm,yshift=-200mm]current page.north west) -- ++(70mm,0);
\draw[crimson!40, line width=1pt] ([xshift=110mm,yshift=-210mm]current page.north west) -- ++(60mm,0);
\draw[crimson!50, line width=1.5pt] ([xshift=35mm,yshift=-230mm]current page.north west) -- ++(45mm,0);

% Connecting lines in bottom area
\draw[crimson!30, line width=1pt] ([xshift=55mm,yshift=-200mm]current page.north west) -- ++(0,-20mm);
\draw[crimson!30, line width=1pt] ([xshift=140mm,yshift=-210mm]current page.north west) -- ++(0,15mm);

% Neural nodes in bottom area
\fill[crimson!70] ([xshift=55mm,yshift=-200mm]current page.north west) circle (2.5mm);
\fill[white] ([xshift=55mm,yshift=-200mm]current page.north west) circle (1.5mm);
\fill[crimson!60] ([xshift=140mm,yshift=-210mm]current page.north west) circle (2mm);
\fill[white] ([xshift=140mm,yshift=-210mm]current page.north west) circle (1mm);
\fill[crimson!80] ([xshift=80mm,yshift=-230mm]current page.north west) circle (2mm);
\fill[white] ([xshift=80mm,yshift=-230mm]current page.north west) circle (1mm);

% SIDE AREAS: Subtle circuit elements on left and right edges
\draw[crimson!30, line width=1pt] ([xshift=15mm,yshift=-120mm]current page.north west) -- ++(20mm,0);
\draw[crimson!30, line width=1pt] ([xshift=175mm,yshift=-130mm]current page.north west) -- ++(15mm,0);
\fill[crimson!50] ([xshift=25mm,yshift=-120mm]current page.north west) circle (1.5mm);
\fill[white] ([xshift=25mm,yshift=-120mm]current page.north west) circle (0.8mm);
\fill[crimson!50] ([xshift=185mm,yshift=-130mm]current page.north west) circle (1.5mm);
\fill[white] ([xshift=185mm,yshift=-130mm]current page.north west) circle (0.8mm);

% Title positioned in center - CLEAN AREA
\node[inner sep=0mm,draw=none,anchor=center,text width=0.8\textwidth,
align=center,font={\fontsize{44pt}{44}\selectfont\bfseries}]
(BGG) at (current page.center)  {\color{crimson}\MakeUppercase {#1}};

\end{tikzpicture}
\clearpage
}

% Define \lab command for lab styling (different visual treatment)
\newcommand{\lab}[1]{%
\begin{tikzpicture}[remember picture,overlay]
%%%
% Different background pattern for labs
\coordinate(S1)at([yshift=-200mm]current page.north west);
\draw[draw=none,fill=BlueDD!15](S1)--++(45:16)coordinate(S2)-
|(S2|-current page.north west)--(current page.north west)coordinate(S3)--(S1);
%
\coordinate(E1)at([yshift=-98mm]current page.north west);
\draw[draw=none,fill=BlueDD!25](E1)--(current page.north west)coordinate(E2)
--++(0:98mm)coordinate(E3)--(E1);
%
\coordinate(D1)at([yshift=15mm]current page.south west);
\draw[draw=none,fill=BlueDD!60,opacity=0.7](D1)--++(45:5.5)coordinate(D2)
-|(D2|-current page.north west)--(current page.north west)coordinate(D3)--(D1);
%%%%
\path[red](S2)-|(S2-|current page.east)coordinate(SS2);
%LAB - Different styling
\node[crimson,align=flush right,inner sep=0,outer sep=0mm,draw=none,anchor=south,
font={\fontsize{48pt}{48}\selectfont\bfseries}]  (BG) at ($(S2)!0.5!(SS2)$){\hphantom{Workshop}};
%%%
\path[green]([yshift=15mm]D2)-|coordinate(TPD)(BG.south east);
\node[inner sep=0mm,draw=none,anchor=south east,%text width=0.9\textwidth,
align=right,font={\fontsize{40pt}{40}\selectfont}]
(BGG) at (TPD)  {\color{crimson}\MakeUppercase {#1}};%\MakeUppercase {}
\end{tikzpicture}
\thispagestyle{empty}
\clearpage
}

% =============================================================================
% SECTION FORMATTING
% =============================================================================
% All section levels use crimson color and are ragged right

% Section (Large, bold, crimson)
\titleformat{\section}
  {\normalfont\Large\bfseries\color{crimson}\raggedright}
  {\thesection}
  {0.5em}
  {#1}
\titlespacing*{\section}{0pc}{14pt plus 4pt minus 4pt}{6pt plus 2pt minus 2pt}[0pc]

% Subsection (large, bold, crimson)
\titleformat{\subsection}
  {\normalfont\large\bfseries\color{crimson}\raggedright}
  {\thesubsection}
  {0.5em}
  {#1}
\titlespacing*{\subsection}{0pc}{12pt plus 4pt minus 4pt}{5pt plus 1pt minus 2pt}[0pc]

% Subsubsection (normal size, bold, crimson)
\titleformat{\subsubsection}
  {\normalfont\normalsize\bfseries\color{crimson}\raggedright}
  {\thesubsubsection}
  {0.5em}
  {#1}
\titlespacing*{\subsubsection}{0pc}{12pt plus 4pt minus 4pt}{5pt plus 1pt minus 2pt}[0pc]

% Paragraph (run-in, bold, crimson, ends with period)
\titleformat{\paragraph}[runin]
  {\normalfont\normalsize\bfseries\color{crimson}}
  {\theparagraph}
  {0.5em}
  {#1}
  [\textbf{.}]
  \titlespacing*{\paragraph}{0pc}{6pt plus 2pt minus 2pt}{0.5em}[0pc]

% Subparagraph (run-in, italic, crimson, ends with period)
\titleformat{\subparagraph}[runin]
  {\normalfont\normalsize\itshape\color{crimson}}
  {\thesubparagraph}
  {0.5em}
  {#1}
  [\textbf{.}]
  \titlespacing*{\subparagraph}{0pc}{6pt plus 2pt minus 2pt}{0.5em}[0pc]

% =============================================================================
% CHAPTER FORMATTING
% =============================================================================
% Numbered chapters: "Chapter X" prefix, huge crimson title
\titleformat{\chapter}[display]
  {\normalfont\huge\bfseries\color{crimson}}
  {\chaptername\ \thechapter}
  {20pt}
  {\Huge #1}
  []

% Unnumbered chapters: no prefix, huge crimson title
\titleformat{name=\chapter,numberless}
  {\normalfont\huge\bfseries\color{crimson}}
  {}
  {0pt}
  {\Huge #1}
  []

\renewcommand{\chaptername}{Chapter}
% =============================================================================
% TABLE OF CONTENTS FORMATTING
% =============================================================================
\setcounter{tocdepth}{2}                      % Show chapters, sections, subsections

% TOC spacing adjustments for number widths and indentation
\setlength{\cftchapnumwidth}{2em}             % Chapter number width
\setlength{\cftsecnumwidth}{2.75em}           % Section number width
\setlength{\cftsubsecnumwidth}{3.25em}        % Subsection number width
\setlength{\cftsubsubsecnumwidth}{4em}        % Subsubsection number width
\setlength{\cftsubsecindent}{4.25em}          % Subsection indent
\setlength{\cftsubsubsecindent}{7.5em}        % Subsubsection indent

% Chapter entries in TOC: bold crimson with "Chapter" prefix
\renewcommand{\cftchapfont}{\bfseries\color{crimson}}
\renewcommand{\cftchappresnum}{\color{crimson}Chapter~}

% Custom formatting for division entries (styled like parts)
\newcommand{\divisionchapter}[1]{%
  \addvspace{12pt}%
  \noindent\hfil\bfseries\color{crimson}#1\hfil\par%
  \addvspace{6pt}%
}

% Adjust TOC spacing for "Chapter" prefix
\newlength{\xtraspace}
\settowidth{\xtraspace}{\cftchappresnum\cftchapaftersnum}
\addtolength{\cftchapnumwidth}{\xtraspace}

% Unnumbered chapters with TOC entry
\newcommand{\likechapter}[1]{%
    \chapter*{#1}
    \addcontentsline{toc}{chapter}{\textcolor{crimson}{#1}}
}

% =============================================================================
% PAGE NUMBERING SYSTEM
% =============================================================================
% Implements traditional book numbering:
% - Roman numerals (i, ii, iii...) for frontmatter
% - Arabic numerals (1, 2, 3...) for mainmatter
% Automatically switches at first numbered chapter
\makeatletter
\newif\if@firstnumbered%
\@firstnumberedtrue%
\newif\if@firstunnumbered%
\@firstunnumberedtrue%

\newcounter{lastRomanPage}
\setcounter{lastRomanPage}{1}

% Start document with Roman numerals (frontmatter)
\AtBeginDocument{
  \pagenumbering{roman}
  \renewcommand{\thepage}{\roman{page}}
}

% Intercept chapter command
\let\old@chapter\chapter%
\renewcommand{\chapter}{%
  \@ifstar{\unnumbered@chapter}{\numbered@chapter}%
}

% Numbered chapters: switch to Arabic on first occurrence
\newcommand{\numbered@chapter}[1]{%
  \if@firstnumbered%
    \cleardoublepage%
    \setcounter{lastRomanPage}{\value{page}}%
    \pagenumbering{arabic}%
    \@firstnumberedfalse%
  \else
    \setcounter{page}{\value{page}}%
  \fi
  \setcounter{sidenote}{1}                    % Reset footnote counter per chapter
  \old@chapter{#1}%
}

% Unnumbered chapters: stay in Roman numerals
\newcommand{\unnumbered@chapter}[1]{%
  \if@firstunnumbered%
    \clearpage
    \setcounter{lastRomanPage}{\value{page}}%
    \pagenumbering{roman}%
    \@firstunnumberedfalse%
  \fi
  \setcounter{sidenote}{1}
  \old@chapter*{#1}%
}
\makeatother

% =============================================================================
% TABLE SIZING AND SPACING
% =============================================================================
% Make tables slightly smaller to fit more content
\AtBeginEnvironment{longtable}{\scriptsize}

% Increase vertical spacing in table cells (default is 1.0)
\renewcommand{\arraystretch}{1.3}

% Prefer placing figures and tables at the top of pages
\makeatletter
\renewcommand{\fps@figure}{t}  % Default placement: top of page
\renewcommand{\fps@table}{t}   % Default placement: top of page
\makeatother

% =============================================================================
% LONGTABLE PAGE BREAKING FIXES (Windows compatibility)
% =============================================================================
% Prevent "Infinite glue shrinkage" errors on Windows LaTeX builds
% by giving longtable more flexibility in page breaking

% Allow more flexible page breaking (vs strict \flushbottom)
\raggedbottom

% Process more rows before attempting page break (default is 20)
\setcounter{LTchunksize}{50}

% Add extra stretch for longtable environments specifically
\AtBeginEnvironment{longtable}{%
  \setlength{\emergencystretch}{3em}%
  \setlength{\parskip}{0pt plus 1pt}%
}

% =============================================================================
% TABLE STYLING - Clean tables with crimson borders
% =============================================================================
% Professional table appearance with:
% - Clean white background (no colored rows)
% - Crimson-colored borders
% - Good spacing for readability
%
% Note: Headers are automatically bolded by Quarto when using **text** in source
\usepackage{booktabs}      % Professional table rules (\toprule, \midrule, \bottomrule)
\usepackage{colortbl}      % For colored borders (\arrayrulecolor)

% Global table styling - crimson borders
\setlength{\arrayrulewidth}{0.5pt}          % Thinner borders than default
%\arrayrulecolor{crimson}                    % Crimson borders matching brand

\setcounter{chapter}{0}

% =============================================================================
% DROP CAPS (Lettrine)
% =============================================================================
% Decorative large first letter at chapter openings, following the tradition
% of Hennessy & Patterson and other MIT Press textbooks.
% Usage in QMD: \lettrine{T}{he first sentence...}
\usepackage{lettrine}
\renewcommand{\LettrineFontHook}{\color{crimson}\bfseries}
\setcounter{DefaultLines}{3}          % Drop cap spans 3 lines
\renewcommand{\DefaultLoversize}{0.1} % Slight oversize for visual weight
\renewcommand{\DefaultLraise}{0}      % No vertical shift
\setlength{\DefaultNindent}{0.5em}    % Indent of continuation text
\setlength{\DefaultSlope}{0pt}        % No slope on continuation

% =============================================================================
% RUNNING HEADERS — Truncation Safety
% =============================================================================
% Long chapter/section titles can overflow the header. These marks truncate
% gracefully so headers stay within the text block.
\renewcommand{\chaptermark}[1]{%
  \markboth{\thechapter.\ #1}{}}
\renewcommand{\sectionmark}[1]{%
  \markright{\thesection\ #1}}

% =============================================================================
% EPIGRAPH ENVIRONMENT
% =============================================================================
% For chapter-opening quotations. Renders as right-aligned italic block
% with attribution in small caps below.
% Usage: \epigraph{Quote text}{Author Name, \textit{Source}}
\newcommand{\bookepigraph}[2]{%
  \vspace{1em}%
  \begin{flushright}%
    \begin{minipage}{0.75\textwidth}%
      \raggedleft\itshape\small #1\\[0.5em]%
      \normalfont\small --- #2%
    \end{minipage}%
  \end{flushright}%
  \vspace{1.5em}%
}

% =============================================================================
% THUMB INDEX TABS
% =============================================================================
% Colored tabs on the outer page edge for quick chapter navigation.
% Each Part gets a different vertical position; chapters within a Part
% share the same tab position. Visible when flipping through the book.
\newcounter{thumbindex}
\setcounter{thumbindex}{0}
\newlength{\thumbtabheight}
\setlength{\thumbtabheight}{16mm}     % Height of each tab
\newlength{\thumbtabwidth}
\setlength{\thumbtabwidth}{8mm}       % Width protruding from edge
\newlength{\thumbtabgap}
\setlength{\thumbtabgap}{1mm}         % Gap between tabs

% Advance to next thumb tab position (call at each \part)
\newcommand{\nextthumb}{%
  \stepcounter{thumbindex}%
}

% Draw the thumb tab on every page (placed in header via fancyhdr)
\newcommand{\drawthumb}{%
  \ifnum\value{thumbindex}>0%
    \begin{tikzpicture}[remember picture,overlay]
      \pgfmathsetmacro{\thumboffset}{%
        20 + (\value{thumbindex}-1) * (16 + 1)}  % mm from top
      \ifodd\value{page}%
        % Odd pages: tab on right edge
        \fill[crimson!80]
          ([yshift=-\thumboffset mm]current page.north east)
          rectangle +(-\thumbtabwidth, -\thumbtabheight);
        \node[white,font=\tiny\bfseries,rotate=90]
          at ([yshift=-\thumboffset mm - 0.5\thumbtabheight,
               xshift=-0.5\thumbtabwidth]current page.north east)
          {\Roman{thumbindex}};
      \else
        % Even pages: tab on left edge
        \fill[crimson!80]
          ([yshift=-\thumboffset mm]current page.north west)
          rectangle +(\thumbtabwidth, -\thumbtabheight);
        \node[white,font=\tiny\bfseries,rotate=-90]
          at ([yshift=-\thumboffset mm - 0.5\thumbtabheight,
               xshift=0.5\thumbtabwidth]current page.north west)
          {\Roman{thumbindex}};
      \fi
    \end{tikzpicture}%
  \fi
}

% Hook into fancyhdr to draw thumb on every content page
\AddToHook{shipout/foreground}{%
  \drawthumb%
}

% =============================================================================
% CROP / BLEED MARKS
% =============================================================================
% For final print submission, uncomment the line below to add crop marks.
% MIT Press production will advise on exact requirements.
% \usepackage[cam,center,width=7.5in,height=10.5in]{crop}

% =============================================================================
% PDF/A ARCHIVAL COMPLIANCE
% =============================================================================
% MIT Press increasingly requires PDF/A for long-term preservation.
% This embeds all fonts and removes transparency.
% Note: pdfx must be loaded early; if it conflicts with hyperref,
% MIT Press production can handle the conversion post-build.
% Uncomment when ready for final submission:
% \usepackage[a-3u]{pdfx}

% =============================================================================
% ENHANCED WIDOW / ORPHAN CONTROL
% =============================================================================
% Prevent single lines at top/bottom of pages and breaks before equations
\clubpenalty=10000          % No orphans (single first line at bottom)
\widowpenalty=10000         % No widows (single last line at top)
\displaywidowpenalty=10000  % No widow before display math
\predisplaypenalty=10000    % No page break just before display math
\postdisplaypenalty=0       % Allow break after display math (natural)
\usepackage{needspace}
\let\Needspace\needspace
\makeatletter
\@ifpackageloaded{float}{}{\usepackage{float}}
\floatstyle{plain}
\@ifundefined{c@chapter}{\newfloat{vid}{h}{lovid}}{\newfloat{vid}{h}{lovid}[chapter]}
\floatname{vid}{Video}
\newcommand*\listofvids{\listof{vid}{List of Videos}}
\makeatother
\makeatletter
\@ifpackageloaded{tcolorbox}{}{\usepackage[skins,breakable]{tcolorbox}}
\@ifpackageloaded{fontawesome5}{}{\usepackage{fontawesome5}}
\definecolor{quarto-callout-color}{HTML}{909090}
\definecolor{quarto-callout-note-color}{HTML}{0758E5}
\definecolor{quarto-callout-important-color}{HTML}{CC1914}
\definecolor{quarto-callout-warning-color}{HTML}{EB9113}
\definecolor{quarto-callout-tip-color}{HTML}{00A047}
\definecolor{quarto-callout-caution-color}{HTML}{FC5300}
\definecolor{quarto-callout-color-frame}{HTML}{acacac}
\definecolor{quarto-callout-note-color-frame}{HTML}{4582ec}
\definecolor{quarto-callout-important-color-frame}{HTML}{d9534f}
\definecolor{quarto-callout-warning-color-frame}{HTML}{f0ad4e}
\definecolor{quarto-callout-tip-color-frame}{HTML}{02b875}
\definecolor{quarto-callout-caution-color-frame}{HTML}{fd7e14}
\makeatother
\makeatletter
\@ifpackageloaded{bookmark}{}{\usepackage{bookmark}}
\makeatother
\makeatletter
\@ifpackageloaded{caption}{}{\usepackage{caption}}
\AtBeginDocument{%
\ifdefined\contentsname
  \renewcommand*\contentsname{Table of contents}
\else
  \newcommand\contentsname{Table of contents}
\fi
\ifdefined\listfigurename
  \renewcommand*\listfigurename{List of Figures}
\else
  \newcommand\listfigurename{List of Figures}
\fi
\ifdefined\listtablename
  \renewcommand*\listtablename{List of Tables}
\else
  \newcommand\listtablename{List of Tables}
\fi
\ifdefined\figurename
  \renewcommand*\figurename{Figure}
\else
  \newcommand\figurename{Figure}
\fi
\ifdefined\tablename
  \renewcommand*\tablename{Table}
\else
  \newcommand\tablename{Table}
\fi
}
\@ifpackageloaded{float}{}{\usepackage{float}}
\floatstyle{ruled}
\@ifundefined{c@chapter}{\newfloat{codelisting}{h}{lop}}{\newfloat{codelisting}{h}{lop}[chapter]}
\floatname{codelisting}{Listing}
\newcommand*\listoflistings{\listof{codelisting}{List of Listings}}
\makeatother
\makeatletter
\makeatother
\makeatletter
\@ifpackageloaded{caption}{}{\usepackage{caption}}
\@ifpackageloaded{subcaption}{}{\usepackage{subcaption}}
\makeatother
\newcommand{\fbxIconPath}{assets/images/icons/callouts}
\newcommand{\fbxIconFormat}{pdf}
\makeatletter
\@ifpackageloaded{tcolorbox}{}{\usepackage[many]{tcolorbox}}
\makeatother
%%%% ---foldboxy preamble ----- %%%%%

% Load xstring for string manipulation
\RequirePackage{xstring}

% Icon path and format configuration - can be overridden in filter-metadata
\providecommand{\fbxIconPath}{assets/images/icons/callouts}
\providecommand{\fbxIconFormat}{pdf}

% Helper command to include icon with hyphen-to-underscore conversion
% This ensures consistency: callout-quiz-question -> callout_quiz_question
\newcommand{\fbxIncludeIcon}[2]{%
  \StrSubstitute{#1}{-}{_}[\fbxIconName]%
  \includegraphics[width=#2]{\fbxIconPath/icon_\fbxIconName.\fbxIconFormat}%
}

% Legacy fallback colors (keep for compatibility)
\definecolor{fbx-default-color1}{HTML}{c7c7d0}
\definecolor{fbx-default-color2}{HTML}{a3a3aa}
\definecolor{fbox-color1}{HTML}{c7c7d0}
\definecolor{fbox-color2}{HTML}{a3a3aa}

% arguments: #1 typelabelnummer: #2 titel: #3
\newenvironment{fbx}[3]{%
\begin{tcolorbox}[
  enhanced,
  breakable,
  %fontupper=\fontsize{8pt}{10pt}\selectfont,  % 95% of body text (10pt -> 9.5pt)
  before skip=8pt,  % space above box (increased)
  after skip=8pt,   % space below box (increased)
  attach boxed title to top*={xshift=0pt},
  boxed title style={
  %fuzzy shadow={1pt}{-1pt}{0mm}{0.1mm}{gray},
  arc=1.5pt,
  rounded corners=north,
  sharp corners=south,
  top=6pt,          % Adjusted for ~40px equivalent height
  bottom=5pt,       % Adjusted for ~40px equivalent height
  overlay={
      \node [left,outer sep=0em, black,draw=none,anchor=west,
        rectangle,fill=none,inner sep=0pt]
        at ([xshift=4mm]frame.west) {\fbxIncludeIcon{#1}{4.2mm}};
    },
  },
  colframe=#1-color2,             % Border color (auto-generated from YAML)
  colbacktitle=#1-color1,         % Background color (auto-generated from YAML)
  colback=white,
  coltitle=black,
  titlerule=0mm,
  toprule=0.5pt,
  bottomrule=0.5pt,
  leftrule=2.2pt,
  rightrule=0.5pt,
  outer arc=1.5pt,
  arc=1.5pt,
  left=0.5em,       % increased left padding
  bottomtitle=1.5mm, % increased title bottom margin
  toptitle=1.5mm,    % increased title top margin
  title=\hspace{2.5em}\protect#2\hspace{0.5em}\protect#3, % Protect parameters
  extras middle and last={top=4pt} % increased continuation spacing
]}
{\end{tcolorbox}}


% boxed environment with right border
\newenvironment{fbxSimple}[3]{\begin{tcolorbox}[
  enhanced,
  breakable,
  %fontupper=\fontsize{8pt}{10pt}\selectfont,  % 95% of body text (10pt -> 9.5pt)
  before skip=8pt,  % space above box (increased)
  after skip=8pt,   % space below box (increased)
  attach boxed title to top*={xshift=0pt},
  boxed title style={
  %fuzzy shadow={1pt}{-1pt}{0mm}{0.1mm}{gray},
  arc=1.5pt,
  rounded corners=north,
  sharp corners=south,
  top=6pt,          % Adjusted for ~40px equivalent height
  bottom=5pt,       % Adjusted for ~40px equivalent height
  overlay={
      \node [left,outer sep=0em, black,draw=none,anchor=west,
        rectangle,fill=none,inner sep=0pt]
        at ([xshift=3mm]frame.west) {\fbxIncludeIcon{#1}{4.2mm}};
    },
  },
  colframe=#1-color2,             % Border color (auto-generated from YAML)
  colbacktitle=#1-color1,         % Background color (auto-generated from YAML)
  colback=white,
  coltitle=black,
  titlerule=0mm,
  toprule=0.5pt,
  bottomrule=0.5pt,
  leftrule=2.2pt,
  rightrule=0.5pt,
  outer arc=1.5pt,
  arc=1.5pt,
  left=0.5em,       % increased left padding
  bottomtitle=1.5mm, % increased title bottom margin
  toptitle=1.5mm,    % increased title top margin
  title=\hspace{2.5em}\protect#2\hspace{0.5em}\protect#3, % Protect parameters
  boxsep=1pt,
  extras first={bottom=0pt},
  extras last={top=0pt,bottom=-4pt},
  overlay first={
    \draw[line width=1pt,white] ([xshift=2.2pt]frame.south west)-- ([xshift=-0.5pt]frame.south east);
  },
  overlay last={
    \draw[line width=1pt,white] ([xshift=2.2pt]frame.north west)-- ([xshift=-0.5pt]frame.north east);
   }
]}
{\end{tcolorbox}}

%%%% --- end foldboxy preamble ----- %%%%%
%%==== colors from yaml ===%
\definecolor{callout-example-color1}{HTML}{F0F8F6}
\definecolor{callout-example-color2}{HTML}{148F77}
\definecolor{callout-principle-color1}{HTML}{F3F2FA}
\definecolor{callout-principle-color2}{HTML}{3D3B8E}
\definecolor{callout-code-color1}{HTML}{F2F4F8}
\definecolor{callout-code-color2}{HTML}{D1D7E0}
\definecolor{callout-lighthouse-color1}{HTML}{FDF8E6}
\definecolor{callout-lighthouse-color2}{HTML}{B8860B}
\definecolor{callout-theorem-color1}{HTML}{F5F0FF}
\definecolor{callout-theorem-color2}{HTML}{6B46C1}
\definecolor{callout-resource-videos-color1}{HTML}{E0F2F1}
\definecolor{callout-resource-videos-color2}{HTML}{20B2AA}
\definecolor{callout-notebook-color1}{HTML}{F2F7FF}
\definecolor{callout-notebook-color2}{HTML}{2C5282}
\definecolor{callout-definition-color1}{HTML}{F0F4F8}
\definecolor{callout-definition-color2}{HTML}{1B4F72}
\definecolor{callout-perspective-color1}{HTML}{F7F8FA}
\definecolor{callout-perspective-color2}{HTML}{4A5568}
\definecolor{callout-chapter-connection-color1}{HTML}{FDF2F7}
\definecolor{callout-chapter-connection-color2}{HTML}{A51C30}
\definecolor{callout-colab-color1}{HTML}{FFF5E6}
\definecolor{callout-colab-color2}{HTML}{FF6B35}
\definecolor{callout-takeaways-color1}{HTML}{FDF2F7}
\definecolor{callout-takeaways-color2}{HTML}{BE185D}
\definecolor{callout-quiz-question-color1}{HTML}{F0F0F8}
\definecolor{callout-quiz-question-color2}{HTML}{5B4B8A}
\definecolor{callout-resource-slides-color1}{HTML}{E0F2F1}
\definecolor{callout-resource-slides-color2}{HTML}{20B2AA}
\definecolor{callout-quiz-answer-color1}{HTML}{E8F2EA}
\definecolor{callout-quiz-answer-color2}{HTML}{4a7c59}
\definecolor{callout-resource-exercises-color1}{HTML}{E0F2F1}
\definecolor{callout-resource-exercises-color2}{HTML}{20B2AA}
\definecolor{callout-checkpoint-color1}{HTML}{E8F5E9}
\definecolor{callout-checkpoint-color2}{HTML}{2E7D32}
%=============%

\usepackage{hyphenat}
\usepackage{ifthen}
\usepackage{calc}
\usepackage{calculator}



\usepackage{graphicx}
\usepackage{geometry}
\usepackage{afterpage}
\usepackage{tikz}
\usetikzlibrary{calc}
\usetikzlibrary{fadings}
\usepackage[pagecolor=none]{pagecolor}


% Set the titlepage font families







% Set the coverpage font families

\usepackage{bookmark}
\IfFileExists{xurl.sty}{\usepackage{xurl}}{} % add URL line breaks if available
\urlstyle{same}
\hypersetup{
  pdftitle={Introduction to Machine Learning Systems},
  pdfauthor={Vijay Janapa Reddi},
  colorlinks=true,
  linkcolor={Maroon},
  filecolor={Maroon},
  citecolor={Blue},
  urlcolor={Blue},
  pdfcreator={LaTeX via pandoc}}


\title{Introduction to Machine Learning Systems}
\author{Vijay Janapa Reddi}
\date{February 1, 2026}
\begin{document}
%%%%% begin titlepage extension code

  \begin{frontmatter}

\begin{titlepage}
% This is a combination of Pandoc templating and LaTeX
% Pandoc templating https://pandoc.org/MANUAL.html#templates
% See the README for help

\thispagestyle{empty}

\newgeometry{top=-100in}

% Page color

\newcommand{\coverauthorstyle}[1]{{\fontsize{20}{24.0}\selectfont
{#1}}}

\begin{tikzpicture}[remember picture, overlay, inner sep=0pt, outer sep=0pt]

\tikzfading[name=fadeout, inner color=transparent!0,outer color=transparent!100]
\tikzfading[name=fadein, inner color=transparent!100,outer color=transparent!0]
\node[anchor=south west, rotate=0, opacity=1] at ($(current page.south west)+(0.225\paperwidth, 9)$) {
\includegraphics[width=\paperwidth, keepaspectratio]{assets/images/covers/cover-image-transparent-vol1.png}};

% Title
\newcommand{\titlelocationleft}{0.075\paperwidth}
\newcommand{\titlelocationbottom}{0.4\paperwidth}
\newcommand{\titlealign}{left}

\begin{scope}{%
\fontsize{52}{62.4}\selectfont
\node[anchor=north
west, align=left, rotate=0] (Title1) at ($(current page.south west)+(\titlelocationleft,\titlelocationbottom)$)  [text width = 0.9\paperwidth]  {{\nohyphens{Machine
Learning Systems}}};
}
\end{scope}

% Author
\newcommand{\authorlocationleft}{.925\paperwidth}
\newcommand{\authorlocationbottom}{0.175\paperwidth}
\newcommand{\authoralign}{right}

\begin{scope}
{%
\fontsize{20}{24.0}\selectfont
\node[anchor=north
east, align=right, rotate=0] (Author1) at ($(current page.south west)+(\authorlocationleft,\authorlocationbottom)$)  [text width = 6in]  {\coverauthorstyle{Vijay
Janapa Reddi\\}};
}
\end{scope}

% Footer
\newcommand{\footerlocationleft}{0.075\paperwidth}
\newcommand{\footerlocationbottom}{0.475\paperwidth}
\newcommand{\footerlocationalign}{left}

\begin{scope}
{%
\fontsize{25}{30.0}\selectfont
 \node[anchor=north west, align=left, rotate=0] (Footer1) at %
($(current page.south west)+(\footerlocationleft,\footerlocationbottom)$)  [text width = 0.9\paperwidth]  {{\nohyphens{Introduction
to}}};
}
\end{scope}

\end{tikzpicture}
\clearpage
\restoregeometry
%%% TITLE PAGE START

% Set up alignment commands
%Page
\newcommand{\titlepagepagealign}{
\ifthenelse{\equal{left}{right}}{\raggedleft}{}
\ifthenelse{\equal{left}{center}}{\centering}{}
\ifthenelse{\equal{left}{left}}{\raggedright}{}
}


\newcommand{\titleandsubtitle}{
% Title and subtitle
{{\huge{\bfseries{\nohyphens{Introduction to Machine Learning
Systems}}}}\par
}%
}
\newcommand{\titlepagetitleblock}{
\titleandsubtitle
}

\newcommand{\authorstyle}[1]{{\large{#1}}}

\newcommand{\affiliationstyle}[1]{{\large{#1}}}

\newcommand{\titlepageauthorblock}{
{\authorstyle{\nohyphens{Vijay Janapa
Reddi}{\textsuperscript{1}}\textsuperscript{,}{\textsuperscript{,*}}}}}

\newcommand{\titlepageaffiliationblock}{
\hangindent=1em
\hangafter=1
{\affiliationstyle{
{1}.~Harvard University


\vspace{1\baselineskip}
* \textit{Correspondence:}~Vijay Janapa Reddi~vj@eecs.harvard.edu
}}
}
\newcommand{\headerstyled}{%
{}
}
\newcommand{\footerstyled}{%
{\large{}}
}
\newcommand{\datestyled}{%
{February 1, 2026}
}


\newcommand{\titlepageheaderblock}{\headerstyled}

\newcommand{\titlepagefooterblock}{
\footerstyled
}

\newcommand{\titlepagedateblock}{
\datestyled
}

%set up blocks so user can specify order
\newcommand{\titleblock}{{

{\titlepagetitleblock}
}

\vspace{4\baselineskip}
}

\newcommand{\authorblock}{{\titlepageauthorblock}

\vspace{2\baselineskip}
}

\newcommand{\affiliationblock}{{\titlepageaffiliationblock}

\vspace{0pt}
}

\newcommand{\logoblock}{}

\newcommand{\footerblock}{}

\newcommand{\dateblock}{{\titlepagedateblock}

\vspace{0pt}
}

\newcommand{\headerblock}{}

\thispagestyle{empty} % no page numbers on titlepages


\newcommand{\vrulecode}{\textcolor{black}{\rule{\vrulewidth}{\textheight}}}
\newlength{\vrulewidth}
\setlength{\vrulewidth}{2pt}
\newlength{\B}
\setlength{\B}{\ifdim\vrulewidth > 0pt 0.05\textwidth\else 0pt\fi}
\newlength{\minipagewidth}
\ifthenelse{\equal{left}{left} \OR \equal{left}{right} }
{% True case
\setlength{\minipagewidth}{\textwidth - \vrulewidth - \B - 0.1\textwidth}
}{
\setlength{\minipagewidth}{\textwidth - 2\vrulewidth - 2\B - 0.1\textwidth}
}
\ifthenelse{\equal{left}{left} \OR \equal{left}{leftright}}
{% True case
\raggedleft % needed for the minipage to work
\vrulecode
\hspace{\B}
}{%
\raggedright % else it is right only and width is not 0
}
% [position of box][box height][inner position]{width}
% [s] means stretch out vertically; assuming there is a vfill
\begin{minipage}[b][\textheight][s]{\minipagewidth}
\titlepagepagealign
\titleblock

Prof.~Vijay Janapa Reddi

School of Engineering and Applied Sciences

Harvard University

\vspace{80mm}

With heartfelt gratitude to the community for their invaluable
contributions and steadfast support.

\vfill

February 1, 2026

\vfill
\par

\end{minipage}\ifthenelse{\equal{left}{right} \OR \equal{left}{leftright} }{
\hspace{\B}
\vrulecode}{}
\clearpage
%%% TITLE PAGE END
\end{titlepage}
\setcounter{page}{1}
\end{frontmatter}

%%%%% end titlepage extension code

% =============================================================================
% HALF-TITLE PAGE (Volume I)
% =============================================================================
% Standard academic book sequence: half-title -> blank -> title page -> copyright
% The half-title shows only the book title -- no author, no publisher, no date.
\thispagestyle{empty}
\begin{center}
\vspace*{0.3\textheight}
{\fontsize{24pt}{28pt}\selectfont\bfseries\color{crimson} Introduction to\\[0.4em] Machine Learning Systems}\\[2em]
{\large\itshape Volume~I}
\vfill
\end{center}
\clearpage
\thispagestyle{empty}\null\clearpage  % Blank verso (back of half-title)

\renewcommand*\contentsname{Table of contents}
{
\hypersetup{linkcolor=}
\setcounter{tocdepth}{2}
\tableofcontents
}
\listoffigures
\listoftables

\mainmatter
\bookmarksetup{startatroot}

\chapter*{Welcome to Volume I}\label{welcome-to-volume-i}
\addcontentsline{toc}{chapter}{Welcome to Volume I}

\markboth{Welcome to Volume I}{Welcome to Volume I}

\bookmarksetup{startatroot}

\chapter{Algorithm Foundations}\label{algorithm-foundations}

This appendix covers the mathematical and computational machinery that
powers neural networks. From the linear algebra at the heart of every
layer to the backpropagation algorithm that enables learning, these
foundations explain \emph{how} models compute and \emph{why} certain
implementation choices affect performance. The concepts here support the
deep learning foundations in
\textbf{?@sec-deep-learning-systems-foundations}, the framework
internals in \textbf{?@sec-ai-frameworks}, and the training strategies
in \textbf{?@sec-ai-training}.

\section{Linear Algebra for Neural
Networks}\label{sec-system-foundations-linear-algebra-neural-networks-f606}

Deep learning systems are, at their core, engines for transforming
massive matrices. While frameworks like PyTorch abstract away the raw
math, understanding the underlying linear algebra is essential for
performance engineering. Now that we know how numbers are stored (from
\textbf{?@sec-machine-foundations}), we must understand how they are
manipulated.

\textbf{Linear Algebra}, and specifically the General Matrix Multiply
(GEMM), is the computational engine of Deep Learning.

\phantomsection\label{callout-perspectiveux2a-1.1}
\begin{fbx}{callout-perspective}{Systems Perspective:}{Why This Matters}
\phantomsection\label{callout-perspective*-1.1}
Every neural network, regardless of architecture, spends most of its
time doing matrix multiplication. Understanding GEMM performance
characteristics explains why batch size affects throughput, why certain
layer dimensions are ``better'' than others, and how to interpret
profiler output.

\end{fbx}

\subsection{Tensor Operations and
Notation}\label{sec-system-foundations-tensor-operations-notation-de17}

We use \textbf{Einstein summation} notation throughout this book because
it makes complex operations explicit.\sidenote{Implemented as
\texttt{torch.einsum} in PyTorch and \texttt{np.einsum} in NumPy. It
provides a unified DSL for dot products, outer products, transposes, and
matrix multiplications. } Matrix multiplication \(C = AB\) becomes:

\[ C_{ij} = \sum_k A_{ik} B_{kj} \]

Or in einsum notation: \texttt{ik,kj-\textgreater{}ij}. This notation
extends naturally to the multi-dimensional operations in attention
mechanisms. For example, batched multi-head attention is
\texttt{bhid,bhjd-\textgreater{}bhij} (batch, head, sequence indices).

\subsection{Memory Layouts and
Performance}\label{sec-system-foundations-memory-layouts-performance-ae64}

Data layout in memory (row-major vs.~column-major) directly affects
cache efficiency. When iterating over a matrix, accessing contiguous
memory locations is dramatically faster than strided access. The
difference can be 10x to 100x in effective bandwidth.

A common optimization pattern: transpose tensors once before repeated
operations to ensure contiguous access in the hot loop. The one-time
transpose cost is amortized across many subsequent operations.

\subsection{The Dot Product as
Similarity}\label{sec-system-foundations-dot-product-similarity-879f}

The dot product \(\mathbf{a} \cdot \mathbf{b} = \sum a_i b_i\) is
geometrically equivalent to \(|\mathbf{a}| |\mathbf{b}| \cos \theta\).

In deep learning, this is our primary tool for measuring
\textbf{similarity}. In Attention mechanisms: * Query (\(Q\)) and Key
(\(K\)) vectors are dot-produced. * A large positive result means they
align (high similarity). * A result of zero means they are orthogonal
(unrelated).

\subsection{General Matrix Multiply
(GEMM)}\label{sec-system-foundations-general-matrix-multiply-gemm-5f92}

GEMM is the computational workhorse of deep learning. For matrices of
size \(M \times K\) and \(K \times N\), GEMM performs \(2MNK\)
floating-point operations (multiply-accumulate counts as two
operations).

The arithmetic intensity of GEMM scales linearly with matrix dimension.
For square \(n \times n\) matrices in \textbf{FP16} (2 bytes/element):

\[ \text{Intensity} = \frac{\text{Ops}}{\text{Bytes}} = \frac{2n^3}{3n^2 \times 2} = \frac{n}{3} \text{ FLOP/byte} \]

This explains several important phenomena:

\begin{itemize}
\tightlist
\item
  \textbf{Larger batches improve efficiency}: Batching increases the
  effective matrix dimensions, pushing workloads toward the
  compute-bound region of the roofline.
\item
  \textbf{Power-of-two dimensions help}: Hardware tensor cores are
  optimized for specific tile sizes (typically 16x16 or 32x32).
  Dimensions that align with these sizes avoid padding overhead.
\item
  \textbf{Small matrices are inefficient}: A 64x64 GEMM may achieve only
  10\% of peak throughput because it cannot fully utilize the hardware.
\end{itemize}

\subsection{Sparse Matrix
Formats}\label{sec-system-foundations-sparse-matrix-formats-7920}

When most elements in a matrix are zero, specialized storage formats
avoid wasting memory on zeros and enable computations that skip them
entirely.

The \textbf{Compressed Sparse Row (CSR)} format uses three arrays:

\begin{itemize}
\tightlist
\item
  \texttt{Values}: The non-zero elements, stored in row order
\item
  \texttt{Col\_Idx}: The column index of each non-zero element
\item
  \texttt{Row\_Ptr}: The starting position in \texttt{Values} for each
  row (length = num\_rows + 1)
\end{itemize}

CSR is essential for recommendation systems (sparse embedding tables)
and pruned models. For a matrix with \(N\) elements and \(K\) non-zeros,
CSR uses \(O(K)\) storage instead of \(O(N)\).

\textbf{Example}: A vocabulary embedding matrix of size
\(100,000 \times 10,000\) (1 billion parameters). * \textbf{Dense
(FP32)}: \(10^9 \times 4\) bytes = \textbf{4 GB}. * \textbf{Sparse (1\%
density)}: Storing only non-zeros requires roughly
\(10^7 \times (4 \text{ bytes value} + 4 \text{ bytes index}) \approx\)
\textbf{80 MB}. * \emph{Result}: A 50x reduction in memory footprint,
fitting a model that would otherwise OOM (Out of Memory).

\section{Tensor Programming
Primitives}\label{sec-system-foundations-tensor-programming-primitives-7baf}

The mathematical operations of linear algebra are implemented in
software through \textbf{Tensor Programming}. This layer translates
abstract math into concrete array manipulations, where bugs often
manifest as shape mismatches rather than logical errors.

\phantomsection\label{callout-perspectiveux2a-1.2}
\begin{fbx}{callout-perspective}{Systems Perspective:}{Why This Matters}
\phantomsection\label{callout-perspective*-1.2}
Your logic is correct, but your code crashes with a shape mismatch
error. Or worse, it runs but produces garbage because you broadcasted
dimensions incorrectly. Mastering tensor shapes, strides, and
broadcasting is the literacy of ML engineering.

\end{fbx}

\subsection{Computational Complexity Cheat
Sheet}\label{sec-appendix-complexity-cheat-sheet}

This table provides a quantitative reference for the most common
building blocks. Use these formulas for \textbf{Napkin Math} estimation
of model size and compute requirements \emph{before} you provision
hardware. If you know the layer type and input shape, you can predict
whether a model will fit in memory.

\begin{longtable}[]{@{}
  >{\raggedright\arraybackslash}p{(\linewidth - 6\tabcolsep) * \real{0.1562}}
  >{\raggedright\arraybackslash}p{(\linewidth - 6\tabcolsep) * \real{0.1953}}
  >{\raggedleft\arraybackslash}p{(\linewidth - 6\tabcolsep) * \real{0.2891}}
  >{\raggedleft\arraybackslash}p{(\linewidth - 6\tabcolsep) * \real{0.3438}}@{}}
\caption{\textbf{Deep Learning Tensor Primitives}: Summary of shapes,
parameters, and FLOP counts. Note: \(B\) is batch size, \(S\) is
sequence length, \(K\) is kernel size. The Attention FLOPs include QKV
projections and the \(S^2\) attention matrix
interactions.}\label{tbl-tensor-op-ref}\tabularnewline
\toprule\noalign{}
\begin{minipage}[b]{\linewidth}\raggedright
\textbf{Layer Type}
\end{minipage} & \begin{minipage}[b]{\linewidth}\raggedright
\textbf{Output Shape}
\end{minipage} & \begin{minipage}[b]{\linewidth}\raggedleft
\textbf{Parameters (\(P\))}
\end{minipage} & \begin{minipage}[b]{\linewidth}\raggedleft
\textbf{FLOPs (per Forward Pass)}
\end{minipage} \\
\midrule\noalign{}
\endfirsthead
\toprule\noalign{}
\begin{minipage}[b]{\linewidth}\raggedright
\textbf{Layer Type}
\end{minipage} & \begin{minipage}[b]{\linewidth}\raggedright
\textbf{Output Shape}
\end{minipage} & \begin{minipage}[b]{\linewidth}\raggedleft
\textbf{Parameters (\(P\))}
\end{minipage} & \begin{minipage}[b]{\linewidth}\raggedleft
\textbf{FLOPs (per Forward Pass)}
\end{minipage} \\
\midrule\noalign{}
\endhead
\bottomrule\noalign{}
\endlastfoot
\textbf{Linear} & \((B, N_{out})\) & \((N_{in} + 1) \times N_{out}\) &
\(2 \times B \times N_{in} \times N_{out}\) \\
\textbf{Conv2D} & \((B, C_{out}, H', W')\) &
\(K^2 \times C_{in} \times C_{out}\) &
\(2 \times H' \times W' \times P\) \\
\textbf{Attention} \textbf{(Single Head)} & \((B, S, d_{model})\) &
\(4 \times d_{model}^2\) & \(2 \times S^2 \times d_{model} +\)
\(8 \times S \times d_{model}^2\) \\
\textbf{LayerNorm} & \((B, S, d_{model})\) & \(2 \times d_{model}\) &
\(O(B \times S \times d_{model})\) \\
\end{longtable}

\subsection{Shapes and
Strides}\label{sec-system-foundations-shapes-strides-bfd9}

A tensor is a view over a contiguous block of memory.

\begin{itemize}
\tightlist
\item
  \textbf{Shape:} The dimensions of the tensor (e.g., \texttt{(3,\ 4)}).
\item
  \textbf{Stride:} The number of elements to skip in memory to move to
  the next element in a dimension.
\end{itemize}

\textbf{Key Insight:} Operations like \texttt{transpose()} or
\texttt{view()} often just change the \emph{strides}, not the data in
memory. This is fast (\(O(1)\)) but can lead to non-contiguous tensors
that fail in optimized kernels. \texttt{contiguous()} forces a memory
copy to realign data.

\subsection{Broadcasting}\label{sec-system-foundations-broadcasting-3387}

Broadcasting allows arithmetic operations on tensors of different
shapes. The rule is: compare dimensions from the last to the first. Two
dimensions are compatible if: 1. They are equal. 2. One of them is 1.

The dimension with size 1 is ``stretched'' to match the other. Note that
this stretching is \textbf{virtual}: the data is not copied in memory.
Instead, the stride for that dimension is set to 0, allowing the
hardware to read the same value repeatedly with \(O(1)\) memory
overhead.

\begin{figure}[htb]

\centering{

\pandocbounded{\includegraphics[keepaspectratio]{index_files/mediabag/17f74eb8c0c74d91498b42044d68c96dd0062c40.pdf}}

}

\caption{\label{fig-broadcasting-rules}\textbf{Tensor Broadcasting
Rules}: Two tensors are compatible if, starting from the trailing
(rightmost) dimension, the dimensions are equal or one of them is 1.
Dimensions of size 1 are `stretched' to match the other tensor. This
stretching is a virtual operation that modifies strides without
allocating new memory.}

\end{figure}%

\textbf{Example:} * A: \texttt{(32,\ 1,\ 64)} (Batch, Channels, Height)
* B: \texttt{(1,\ 128,\ 64)} (1, Width, Height) * Result:
\texttt{(32,\ 128,\ 64)}

Visualizing this expansion prevents silent logic bugs where you
accidentally create a massive tensor (e.g., \texttt{(Batch,\ Batch)}
matrix instead of element-wise \texttt{(Batch,)} vector).

\section{The Mechanics of
Learning}\label{sec-system-foundations-mechanics-learning-78b6}

With valid tensor programs, we can construct the training loops that
power learning. \textbf{Backpropagation} is the algorithm that
orchestrates these tensors to compute gradients, transforming a forward
prediction into a backward learning signal.

\phantomsection\label{callout-perspectiveux2a-1.3}
\begin{fbx}{callout-perspective}{Systems Perspective:}{Why This Matters}
\phantomsection\label{callout-perspective*-1.3}
When training fails (loss goes to NaN, gradients explode, memory runs
out), understanding what backpropagation actually does helps you
diagnose the problem. This section gives you the mental model to reason
about gradient flow and memory usage during training.

\end{fbx}

\subsection{The Chain Rule and Automatic
Differentiation}\label{sec-system-foundations-chain-rule-automatic-differentiation-e742}

For a composed function \(y = f(g(x))\), the derivative is
\(\frac{dy}{dx} = \frac{dy}{dg} \cdot \frac{dg}{dx}\).

Modern frameworks use \textbf{reverse-mode automatic differentiation},
which computes gradients for all \(N\) parameters in a single backward
pass. This is why training (one forward + one backward pass) has similar
compute cost to two inference passes, rather than \(N\) passes.

\subsection{The Backpropagation
Algorithm}\label{sec-system-foundations-backpropagation-algorithm-3175}

Backpropagation implements the chain rule efficiently through two
passes: forward to compute outputs, backward to compute gradients.

\begin{figure}[htb]

\centering{

\pandocbounded{\includegraphics[keepaspectratio]{index_files/mediabag/3bf4c2b8bfe41cd530e4def74001073f204f36fd.pdf}}

}

\caption{\label{fig-backprop-graph}\textbf{Backpropagation Computational
Graph}: A two-layer network showing the forward pass (black arrows) and
backward pass (red dashed arrows). Each node caches values during the
forward pass that are reused during the backward pass.}

\end{figure}%

\textbf{How to trace the computation.} Figure~\ref{fig-backprop-graph}
shows a simple two-layer network. Practice tracing both passes to
understand what happens during training:

\textbf{Forward pass (black arrows, left to right)}: Start at \(x\),
your input. Multiply by \(W_1\) to get hidden activation \(h\). Cache
\(h\) because you will need it later. Multiply \(h\) by \(W_2\) to get
output \(y\). Cache \(y\). Compare \(y\) to the target label to compute
loss \(L\).

At this point, you have computed the loss and your memory contains: the
input \(x\), the cached activation \(h\), the cached output \(y\), and
the loss \(L\). For a large model, these cached activations dominate
memory usage.

\textbf{Backward pass (red arrows, right to left)}: Now trace backward
from \(L\). The loss function tells you
\(\frac{\partial L}{\partial y}\), the gradient of loss with respect to
your prediction. This is where the error signal enters the network.

To compute \(\frac{\partial L}{\partial W_2}\), you need to know how
\(W_2\) affected \(y\). That requires the cached value of \(h\). The
chain rule gives you:
\(\frac{\partial L}{\partial W_2} = \frac{\partial L}{\partial y} \cdot h^T\).

To continue backward to \(W_1\), you need
\(\frac{\partial L}{\partial h}\), then multiply by the cached input
\(x\). Each step backward requires the activations cached during the
forward pass.

\subsection{The True Cost of Training
Memory}\label{the-true-cost-of-training-memory}

A common student mistake is to assume \(Memory = \text{Model Size}\).
This leads to immediate OOM errors. The actual memory footprint of
training is:

\[ M_{total} = M_{weights} + M_{gradients} + M_{optimizer} + M_{activations} \]

For a standard Adam optimizer in Mixed Precision: * \textbf{Weights}: 2
bytes (FP16/BF16) or 4 bytes (FP32). * \textbf{Gradients}: Same size as
weights. * \textbf{Optimizer State}: 8-12 bytes per parameter (Momentum
+ Variance + FP32 Master Weights). * \textbf{Activations}: The hidden
giant. \(O(Batch \times Sequence \times Layers \times Width)\).

\textbf{The Activation Explosion}: While weights are fixed (\(O(P)\)),
activations grow linearly with \textbf{Batch Size} and \textbf{Sequence
Length}. For Large Language Models, activations can be 10-50x larger
than the weights. This is why techniques like Gradient Checkpointing
(recomputing activations) and FlashAttention (tiling attention) are
mandatory, not optional.

\textbf{Memory Implication}: Cached activations dominate memory usage.
Storing activations for a 100-layer network consumes huge memory,
motivating \textbf{Gradient Checkpointing}
(\citeproc{ref-chen2016training}{Chen et al. 2016}) (recomputing
activations to save memory).\sidenote{Gradient checkpointing trades
compute for memory. Instead of storing all activations, you store only a
subset (checkpoints) and recompute the missing ones during the backward
pass. This reduces memory usage from \(O(N)\) to \(O(\sqrt{N})\) at the
cost of \(33\%\) more compute. }

\subsection{Computational Graphs and
Optimization}\label{sec-system-foundations-computational-graphs-optimization-b49d}

ML compilers represent models as directed acyclic graphs (DAGs). This
representation enables hardware-independent optimizations.

\textbf{Static Single Assignment (SSA)}: Compilers transform graphs into
SSA form where each variable is assigned exactly once. This makes data
dependencies explicit, enabling safe optimizations like \textbf{Operator
Fusion} (combining \texttt{Conv\ -\textgreater{}\ ReLU} into one kernel
to avoid memory round-trips).

\section{Fallacies and
Pitfalls}\label{sec-algorithm-foundations-fallacies-pitfalls}

\begin{tcolorbox}[enhanced jigsaw, rightrule=.15mm, leftrule=.75mm, arc=.35mm, breakable, coltitle=black, opacityback=0, toprule=.15mm, bottomtitle=1mm, opacitybacktitle=0.6, bottomrule=.15mm, colback=white, colframe=quarto-callout-warning-color-frame, left=2mm, titlerule=0mm, title=\textcolor{quarto-callout-warning-color}{\faExclamationTriangle}\hspace{0.5em}{Pitfall: Assuming sparse matrices always save memory.}, colbacktitle=quarto-callout-warning-color!10!white, toptitle=1mm]

\textbf{The Reality}: Sparse formats (CSR, COO) add metadata overhead
(indices). If a matrix is only 50\% sparse, the overhead (storing
indices) often exceeds the savings (skipping zeros). A rule of thumb:
sparsity usually needs to exceed 90\%--95\% to be worthwhile for
performance, though specialized hardware (like NVIDIA's 2:4 sparsity)
changes this calculus.

\end{tcolorbox}

\phantomsection\label{refs}
\begin{CSLReferences}{1}{0}
\bibitem[\citeproctext]{ref-chen2016training}
Chen, Tianqi, Bing Xu, Chiyuan Zhang, and Carlos Guestrin. 2016.
{``Training Deep Nets with Sublinear Memory Cost.''} \emph{arXiv
Preprint arXiv:1604.06174}, April.
\url{http://arxiv.org/abs/1604.06174v2}.

\end{CSLReferences}


\backmatter

\clearpage


\end{document}
