% Options for packages loaded elsewhere
% Options for packages loaded elsewhere
\PassOptionsToPackage{unicode,linktoc=all,pdfwindowui,pdfpagemode=FullScreen,pdfpagelayout=TwoPageRight}{hyperref}
\PassOptionsToPackage{hyphens}{url}
\PassOptionsToPackage{dvipsnames,svgnames,x11names}{xcolor}
%
\documentclass[
  9pt,
  letterpaper,
  abstract,
  titlepage]{scrbook}
\usepackage{xcolor}
\usepackage{amsmath,amssymb}
\setcounter{secnumdepth}{3}
\usepackage{iftex}
\ifPDFTeX
  \usepackage[T1]{fontenc}
  \usepackage[utf8]{inputenc}
  \usepackage{textcomp} % provide euro and other symbols
\else % if luatex or xetex
  \usepackage{unicode-math} % this also loads fontspec
  \defaultfontfeatures{Scale=MatchLowercase}
  \defaultfontfeatures[\rmfamily]{Ligatures=TeX,Scale=1}
\fi
\usepackage{lmodern}
\ifPDFTeX\else
  % xetex/luatex font selection
\fi
% Use upquote if available, for straight quotes in verbatim environments
\IfFileExists{upquote.sty}{\usepackage{upquote}}{}
\IfFileExists{microtype.sty}{% use microtype if available
  \usepackage[]{microtype}
  \UseMicrotypeSet[protrusion]{basicmath} % disable protrusion for tt fonts
}{}
% Make \paragraph and \subparagraph free-standing
\makeatletter
\ifx\paragraph\undefined\else
  \let\oldparagraph\paragraph
  \renewcommand{\paragraph}{
    \@ifstar
      \xxxParagraphStar
      \xxxParagraphNoStar
  }
  \newcommand{\xxxParagraphStar}[1]{\oldparagraph*{#1}\mbox{}}
  \newcommand{\xxxParagraphNoStar}[1]{\oldparagraph{#1}\mbox{}}
\fi
\ifx\subparagraph\undefined\else
  \let\oldsubparagraph\subparagraph
  \renewcommand{\subparagraph}{
    \@ifstar
      \xxxSubParagraphStar
      \xxxSubParagraphNoStar
  }
  \newcommand{\xxxSubParagraphStar}[1]{\oldsubparagraph*{#1}\mbox{}}
  \newcommand{\xxxSubParagraphNoStar}[1]{\oldsubparagraph{#1}\mbox{}}
\fi
\makeatother


\providecommand{\tightlist}{%
  \setlength{\itemsep}{0pt}\setlength{\parskip}{0pt}}\usepackage{longtable,booktabs,array}
\usepackage{calc} % for calculating minipage widths
% Correct order of tables after \paragraph or \subparagraph
\usepackage{etoolbox}
\makeatletter
\patchcmd\longtable{\par}{\if@noskipsec\mbox{}\fi\par}{}{}
\makeatother
% Allow footnotes in longtable head/foot
\IfFileExists{footnotehyper.sty}{\usepackage{footnotehyper}}{\usepackage{footnote}}
\makesavenoteenv{longtable}
\usepackage{graphicx}
\makeatletter
\newsavebox\pandoc@box
\newcommand*\pandocbounded[1]{% scales image to fit in text height/width
  \sbox\pandoc@box{#1}%
  \Gscale@div\@tempa{\textheight}{\dimexpr\ht\pandoc@box+\dp\pandoc@box\relax}%
  \Gscale@div\@tempb{\linewidth}{\wd\pandoc@box}%
  \ifdim\@tempb\p@<\@tempa\p@\let\@tempa\@tempb\fi% select the smaller of both
  \ifdim\@tempa\p@<\p@\scalebox{\@tempa}{\usebox\pandoc@box}%
  \else\usebox{\pandoc@box}%
  \fi%
}
% Set default figure placement to htbp
\def\fps@figure{htbp}
\makeatother
% definitions for citeproc citations
\NewDocumentCommand\citeproctext{}{}
\NewDocumentCommand\citeproc{mm}{%
  \begingroup\def\citeproctext{#2}\cite{#1}\endgroup}
\makeatletter
 % allow citations to break across lines
 \let\@cite@ofmt\@firstofone
 % avoid brackets around text for \cite:
 \def\@biblabel#1{}
 \def\@cite#1#2{{#1\if@tempswa , #2\fi}}
\makeatother
\newlength{\cslhangindent}
\setlength{\cslhangindent}{1.5em}
\newlength{\csllabelwidth}
\setlength{\csllabelwidth}{3em}
\newenvironment{CSLReferences}[2] % #1 hanging-indent, #2 entry-spacing
 {\begin{list}{}{%
  \setlength{\itemindent}{0pt}
  \setlength{\leftmargin}{0pt}
  \setlength{\parsep}{0pt}
  % turn on hanging indent if param 1 is 1
  \ifodd #1
   \setlength{\leftmargin}{\cslhangindent}
   \setlength{\itemindent}{-1\cslhangindent}
  \fi
  % set entry spacing
  \setlength{\itemsep}{#2\baselineskip}}}
 {\end{list}}
\usepackage{calc}
\newcommand{\CSLBlock}[1]{\hfill\break\parbox[t]{\linewidth}{\strut\ignorespaces#1\strut}}
\newcommand{\CSLLeftMargin}[1]{\parbox[t]{\csllabelwidth}{\strut#1\strut}}
\newcommand{\CSLRightInline}[1]{\parbox[t]{\linewidth - \csllabelwidth}{\strut#1\strut}}
\newcommand{\CSLIndent}[1]{\hspace{\cslhangindent}#1}

% =============================================================================
% LATEX HEADER CONFIGURATION FOR MLSYSBOOK PDF
% =============================================================================
% This file contains all LaTeX package imports, custom commands, and styling
% definitions for the PDF output of the Machine Learning Systems textbook.
%
% Key Features:
% - Harvard crimson branding throughout
% - Custom part/chapter/section styling
% - Professional table formatting with colored headers
% - Margin notes with custom styling
% - TikZ-based part dividers
% - Page numbering (Roman for frontmatter, Arabic for mainmatter)
%
% Note: This file is included via _quarto-pdf.yml and affects PDF output only.
% HTML/EPUB styling is handled separately via CSS files.
% =============================================================================

% =============================================================================
% PACKAGE IMPORTS
% =============================================================================

% Layout and positioning
% \usepackage[outercaption, ragged]{sidecap}  % Commented out to make figure captions inline instead of in margin
\usepackage{adjustbox}      % Adjusting box dimensions
\usepackage{afterpage}      % Execute commands after page break
\usepackage{morefloats}     % Increase number of floats
\usepackage{array}          % Enhanced table column formatting
\usepackage{atbegshi}       % Insert content at page beginning
%\usepackage{changepage}     % Change page dimensions mid-document
\usepackage{emptypage}      % Clear headers/footers on empty pages

% Language and text
\usepackage[english]{babel} % English language support
\usepackage{microtype}      % Improved typography and hyphenation

% Captions and floats
\usepackage{caption}
% Caption styling configuration
%\captionsetup[table]{belowskip=5pt}
\captionsetup{format=plain}
\DeclareCaptionLabelFormat{mylabel}{#1
#2:\hspace{1.0ex}}
\DeclareCaptionFont{ninept}{\fontsize{7pt}{8}\selectfont #1}

% Figure captions: Small font, bold label, ragged right
\captionsetup[figure]{labelfont={bf,ninept},labelsep=space,
belowskip=2pt,aboveskip=6pt,labelformat=mylabel,
justification=raggedright,singlelinecheck=false,font={ninept}}

% Table captions: Small font, bold label, ragged right
\captionsetup[table]{belowskip=6pt,labelfont={bf,ninept},labelsep=none,
labelformat=mylabel,justification=raggedright,singlelinecheck=false,font={ninept}}

% Typography fine-tuning
\emergencystretch=5pt       % Allow extra stretch to avoid overfull boxes

% Utility packages
\usepackage{etoolbox}       % For patching commands and environments

% Page layout and headers
\usepackage{fancyhdr}       % Custom headers and footers
\usepackage{geometry}       % Page dimensions and margins

% Graphics and figures
\usepackage{graphicx}       % Include graphics
\usepackage{float}          % Improved float placement
\usepackage[skins,breakable]{tcolorbox} % Coloured and framed text boxes
\tcbset{before upper=\setlength{\parskip}{3pt}}

% Tables
\usepackage{longtable}      % Multi-page tables

% Fonts and typography
\usepackage{fontspec}       % Font selection for LuaLaTeX
\usepackage{mathptmx}       % Times-like math fonts
\usepackage{newpxtext}      % Palatino-like font for body text

% Colors and visual elements
\usepackage[dvipsnames]{xcolor}  % Extended color support
\usepackage{tikz}           % Programmatic graphics
\usetikzlibrary{positioning}
\usetikzlibrary{calc}
\usepackage{tikzpagenodes}  % TikZ positioning relative to page

% Code listings
\usepackage{listings}       % Code highlighting

% Hyperlinks
\usepackage{hyperref}       % Clickable links in PDF

% Conditional logic
\usepackage{ifthen}         % If-then-else commands

% Math symbols
\usepackage{amsmath}        % AMS math extensions
\usepackage{amssymb}        % AMS math symbols
\usepackage{latexsym}       % Additional LaTeX symbols
\usepackage{pifont}         % Zapf Dingbats symbols
\providecommand{\blacklozenge}{\ding{117}}  % Black diamond symbol

% Lists
\usepackage{enumitem}       % Customizable lists

% Margin notes and sidenotes
\usepackage{marginfix}      % Fixes margin note overflow
\usepackage{marginnote}     % Margin notes
\usepackage{sidenotes}      % Academic-style sidenotes
\renewcommand\raggedrightmarginnote{\sloppy}
\renewcommand\raggedleftmarginnote{\sloppy}

% Typography improvements
\usepackage{ragged2e}       % Better ragged text
\usepackage[all]{nowidow}   % Prevent widows and orphans
\usepackage{needspace}      % Ensure minimum space on page

% Section formatting
\usepackage[explicit]{titlesec}  % Custom section titles
\usepackage{tocloft}        % Table of contents formatting

% QR codes and icons
\usepackage{fontawesome5}   % Font Awesome icons
\usepackage{qrcode}         % QR code generation
\qrset{link, height=15mm}

% =============================================================================
% FLOAT CONFIGURATION
% =============================================================================
% Allow more floats per page to handle figure-heavy chapters
\extrafloats{200}
\setcounter{topnumber}{12}       % Max floats at top of page
\setcounter{bottomnumber}{12}    % Max floats at bottom of page
\setcounter{totalnumber}{24}     % Max floats per page
\setcounter{dbltopnumber}{8}     % Max floats at top of two-column page
\renewcommand{\topfraction}{.95}  % Max fraction of page for top floats
\renewcommand{\bottomfraction}{.95}
\renewcommand{\textfraction}{.05}  % Min fraction of page for text
\renewcommand{\floatpagefraction}{.7}  % Min fraction of float page
\renewcommand{\dbltopfraction}{.95}

% Prevent "Float(s) lost" errors by flushing floats more aggressively
\usepackage{placeins}  % Provides \FloatBarrier

% =============================================================================
% COLOR DEFINITIONS
% =============================================================================
% Harvard crimson - primary brand color used throughout
\definecolor{crimson}{HTML}{A51C30}

% Quiz element colors
\definecolor{quiz-question-color1}{RGB}{225,243,248}  % Light blue background
\definecolor{quiz-question-color2}{RGB}{17,158,199}   % Blue border
\definecolor{quiz-answer-color1}{RGB}{250,234,241}    % Light pink background
\definecolor{quiz-answer-color2}{RGB}{152,14,90}      % Magenta border

% =============================================================================
% LIST FORMATTING
% =============================================================================
% Tighter list spacing for academic style
\def\tightlist{}
\setlist{itemsep=1pt, parsep=1pt, topsep=0pt,after={\vspace{0.3\baselineskip}}}
\let\tightlist\relax

\makeatletter
\@ifpackageloaded{framed}{}{\usepackage{framed}}
\@ifpackageloaded{fancyvrb}{}{\usepackage{fancyvrb}}
\makeatother

\makeatletter
%New float "codelisting" has been updated
\AtBeginDocument{%
\floatstyle{ruled}
\newfloat{codelisting}{!htb}{lop}
\floatname{codelisting}{Listing}
\floatplacement{codelisting}{!htb}
\captionsetup[codelisting]{labelfont={bf,ninept},labelformat=mylabel,
  singlelinecheck=false,width=\linewidth,labelsep=none,font={ninept}}%
\renewenvironment{snugshade}{%
   \def\OuterFrameSep{3pt}%
   \def\FrameCommand{\fboxsep=5pt\colorbox{shadecolor}}%
   \MakeFramed{\advance\hsize-\width\FrameRestore}%
   \leftskip 0.5em \rightskip 0.5em%
   \small% decrease font size
   }{\endMakeFramed}%
}
\makeatother

%The space before and after the verbatim environment "Highlighting" has been reduced
\fvset{listparameters=\setlength{\topsep}{0pt}\setlength{\partopsep}{0pt}}
\DefineVerbatimEnvironment{Highlighting}{Verbatim}{framesep=0mm,commandchars=\\\{\}}

\makeatletter
\renewcommand\fs@ruled{\def\@fs@cfont{\bfseries}\let\@fs@capt\floatc@ruled
\def\@fs@pre{\hrule height.8pt depth0pt \kern2pt}%
\def\@fs@post{\kern2pt\hrule\relax}%
\def\@fs@mid{\kern2pt\hrule\kern1pt}%space between float and caption
\let\@fs@iftopcapt\iftrue}
\makeatother


% =============================================================================
% HYPHENATION RULES
% =============================================================================
% Explicit hyphenation points for technical terms to avoid bad breaks
\hyphenation{
  light-weight
  light-weight-ed
  de-vel-op-ment
  un-der-stand-ing
  mod-els
  prin-ci-ples
  ex-per-tise
  com-pli-cat-ed
  blue-print
  per‧for‧mance
  com-mu-ni-ca-tion
  par-a-digms
  hy-per-ten-sion
  a-chieved
}

% =============================================================================
% CODE LISTING CONFIGURATION
% =============================================================================
% Settings for code blocks using listings package
\lstset{
breaklines=true,              % Automatic line wrapping
breakatwhitespace=true,       % Break at whitespace only
basicstyle=\ttfamily,         % Monospace font
frame=none,                   % No frame around code
keepspaces=true,              % Preserve spaces
showspaces=false,             % Don't show space characters
showtabs=false,               % Don't show tab characters
columns=flexible,             % Flexible column width
belowskip=0pt,               % Minimal spacing
aboveskip=0pt
}

% =============================================================================
% PAGE GEOMETRY
% =============================================================================
% MIT Press trim size: 7" x 10" (per publisher specifications)
% This is a standard academic textbook format providing good readability
% for technical content with figures and code blocks.
% Wide outer margin accommodates sidenotes/margin notes.
\geometry{
  paperwidth=7in,
  paperheight=10in,
  top=0.875in,
  bottom=0.875in,
  inner=0.875in,              % Inner margin (binding side)
  outer=1.75in,               % Outer margin (includes space for sidenotes)
  footskip=30pt,
  marginparwidth=1.25in,      % Width for margin notes
  twoside                     % Different left/right pages
}

% =============================================================================
% SIDENOTE STYLING
% =============================================================================
% Custom sidenote design with crimson vertical bar
\renewcommand{\thefootnote}{\textcolor{crimson}{\arabic{footnote}}}

% Save original sidenote command
\makeatletter
\@ifundefined{oldsidenote}{
  \let\oldsidenote\sidenote%
}{}
\makeatother

% Redefine sidenote with vertical crimson bar
\renewcommand{\sidenote}[1]{%
  \oldsidenote{%
    \noindent
    \color{crimson!100}                        % Crimson vertical line
    \raisebox{0em}{%
      \rule{0.5pt}{1.5em}                      % Thin vertical line
    }
    \hspace{0.3em}                             % Space after line
    \color{black}                              % Reset text color
    \footnotesize #1                           % Sidenote content
  }%
}

% =============================================================================
% FLOAT HANDLING
% =============================================================================
% Patch LaTeX's output routine to handle float overflow gracefully
% The "Float(s) lost" error occurs in \@doclearpage when \@currlist is not empty
% This patch silently clears pending floats that can't be placed
\makeatletter
\let\orig@doclearpage\@doclearpage
\def\@doclearpage{%
  \ifx\@currlist\@empty\else
    \global\let\@currlist\@empty
    \typeout{Warning: Floats cleared to prevent overflow}%
  \fi
  \orig@doclearpage
}
\makeatother

% Additional safety for structural commands
\let\originalbackmatter\backmatter
\renewcommand{\backmatter}{%
  \clearpage%
  \originalbackmatter%
}

\let\originalfrontmatter\frontmatter
\renewcommand{\frontmatter}{%
  \clearpage%
  \originalfrontmatter%
}

\let\originalmainmatter\mainmatter
\renewcommand{\mainmatter}{%
  \clearpage%
  \originalmainmatter%
}

% =============================================================================
% PAGE HEADERS AND FOOTERS
% =============================================================================
% Ensure chapters use fancy page style (not plain)
\patchcmd{\chapter}{\thispagestyle{plain}}{\thispagestyle{fancy}}{}{}

% Main page style with crimson headers
\pagestyle{fancy}
\fancyhf{}                                              % Clear all
\fancyhead[LE]{\small\color{crimson}\nouppercase{\rightmark}}  % Left even: section
\fancyhead[RO]{\color{crimson}\thepage}                 % Right odd: page number
\fancyhead[LO]{\small\color{crimson}\nouppercase{\leftmark}}   % Left odd: chapter
\fancyhead[RE]{\color{crimson}\thepage}                 % Right even: page number
\renewcommand{\headrulewidth}{0.4pt}                    % Thin header line
\renewcommand{\footrulewidth}{0pt}                      % No footer line

% Plain page style (for chapter openings)
\fancypagestyle{plain}{
  \fancyhf{}
  \fancyfoot[C]{\color{crimson}\thepage}                % Centered page number
  \renewcommand{\headrulewidth}{0pt}
  \renewcommand{\footrulewidth}{0pt}
}

% =============================================================================
% KOMA-SCRIPT FONT ADJUSTMENTS
% =============================================================================
% Apply crimson color to all heading levels
\addtokomafont{disposition}{\rmfamily\color{crimson}}
\addtokomafont{chapter}{\color{crimson}}
\addtokomafont{section}{\color{crimson}}
\addtokomafont{subsection}{\color{crimson}}

% =============================================================================
% ABSTRACT ENVIRONMENT
% =============================================================================
\newenvironment{abstract}{
  \chapter*{\abstractname}
  \addcontentsline{toc}{chapter}{\abstractname}
  \small
}{
  \clearpage
}

% =============================================================================
% HYPERLINK CONFIGURATION
% =============================================================================
% Crimson-colored links throughout, two-page PDF layout
\hypersetup{
  linkcolor=crimson,
  citecolor=crimson,
  urlcolor=crimson,
  pdfpagelayout=TwoPageRight,   % Two-page spread view
  pdfstartview=Fit               % Initial zoom fits page
}

% =============================================================================
% PART SUMMARY SYSTEM
% =============================================================================
% Allows adding descriptive text below part titles
\newcommand{\partsummary}{}     % Empty by default
\newif\ifhaspartsummary%
\haspartsummaryfalse%

\newcommand{\setpartsummary}[1]{%
  \renewcommand{\partsummary}{#1}%
  \haspartsummarytrue%
}

% Additional colors for part page backgrounds
\definecolor{BrownLL}{RGB}{233,222,220}
\definecolor{BlueDD}{RGB}{62,100,125}
\colorlet{BlueDD}{magenta}

% ===============================================================================
% PART STYLING SYSTEM
% ===============================================================================
%
% This system provides three distinct visual styles for book organization:
%
% 1. NUMBERED PARTS (\part{title}) - For main book sections
%    - Roman numerals (I, II, III, etc.) in top right corner
%    - Crimson title with horizontal lines above/below
%    - "Part I" label in sidebar
%    - Used for: foundations, principles, optimization, deployment, etc.
%
% 2. UNNUMBERED PARTS (\part*{title}) - For special sections like "Labs"
%    - Division-style geometric background (left side)
%    - No Roman numerals
%    - Used for: labs section
%
% 3. DIVISIONS (\division{title}) - For major book divisions
%    - Clean geometric background with centered title
%    - Used for: frontmatter, main_content, backmatter
%
% The Lua filter (inject-parts.lua) automatically routes parts by {key:xxx} commands
% to the appropriate LaTeX command based on the key name.
% ===============================================================================

% NUMBERED PARTS: Roman numeral styling for main book sections
\titleformat{\part}[display]
{\thispagestyle{empty}}{}{20pt}{
\begin{tikzpicture}[remember picture,overlay]
%%%
%%
\node[crimson,align=flush right,
inner sep=0,outer sep=0mm,draw=none,%
anchor=east,minimum height=31mm, text width=1.2\textwidth,
yshift=-30mm,font={%
\fontsize{98pt}{104}\selectfont\bfseries}]  (BG) at (current page text area.north east){\thepart};
%
\node[black,inner sep=0mm,draw=none,
anchor=mid,text width=1.2\textwidth,
 minimum height=35mm, align=right,
node distance=7mm,below=of BG,
font={\fontsize{30pt}{34}\selectfont}]
(BGG)  {\hyphenchar\font=-1 \color{black}\MakeUppercase {#1}};
\draw [crimson,line width=3pt] ([yshift=0mm]BGG.north west) -- ([yshift=0mm]BGG.north east);
\draw [crimson,line width=2pt] ([yshift=0mm]BGG.south west) -- ([yshift=0mm]BGG.south east);
%
\node[fill=crimson,text=white,rotate=90,%
anchor=south west,minimum height=15mm,
minimum width=40mm,font={%
\fontsize{20pt}{20}\selectfont\bfseries}](BP)  at
(current page text area.south east)
{{\sffamily Part}~\thepart};
%
\path[red](BP.north west)-|coordinate(PS)(BGG.south west);
%
% Part summary box commented out for cleaner design
% \ifhaspartsummary
% \node[inner sep=4pt,text width=0.7\textwidth,draw=none,fill=BrownLL!40,
% align=justify,font={\fontsize{9pt}{12}\selectfont},anchor=south west]
% at (PS) {\partsummary};
% \fi
\end{tikzpicture}
}[]

\renewcommand{\thepart}{\Roman{part}}

% UNNUMBERED PARTS: Division-style background for special sections
\titleformat{name=\part,numberless}[display]
{\thispagestyle{empty}}{}{20pt}{
\begin{tikzpicture}[remember picture,overlay]
%%%
\coordinate(S1)at([yshift=-200mm]current page.north west);
\draw[draw=none,fill=BlueDD!7](S1)--++(45:16)coordinate(S2)-
|(S2|-current page.north west)--(current page.north west)coordinate(S3)--(S1);
%
\coordinate(E1)at([yshift=-98mm]current page.north west);
\draw[draw=none,fill=BlueDD!15](E1)--(current page.north west)coordinate(E2)
--++(0:98mm)coordinate(E3)--(E1);
%
\coordinate(D1)at([yshift=15mm]current page.south west);
\draw[draw=none,fill=BlueDD!40,opacity=0.5](D1)--++(45:5.5)coordinate(D2)
-|(D2|-current page.north west)--(current page.north west)coordinate(D3)--(D1);
%%%%
\path[red](S2)-|(S2-|current page.east)coordinate(SS2);
%PART
\node[crimson,align=flush right,inner sep=0,outer sep=0mm,draw=none,anchor=south,
font={\fontsize{48pt}{48}\selectfont\bfseries}]  (BG) at ($(S2)!0.5!(SS2)$){\hphantom{Part}};
%%%
\path[green]([yshift=15mm]D2)-|coordinate(TPD)(BG.south east);
\node[inner sep=0mm,draw=none,anchor=south east,%text width=0.9\textwidth,
align=right,font={\fontsize{40pt}{40}\selectfont}]
(BGG) at (TPD)  {\color{crimson}\MakeUppercase {#1}};%\MakeUppercase {}
\end{tikzpicture}
}

% Define \numberedpart command for numbered parts
\newcommand{\numberedpart}[1]{%
\FloatBarrier%  % Flush all pending floats before part break
\clearpage
\thispagestyle{empty}
\stepcounter{part}%
\begin{tikzpicture}[remember picture,overlay]
%%%
%%
\node[crimson,align=flush right,
inner sep=0,outer sep=0mm,draw=none,%
anchor=east,minimum height=31mm, text width=1.2\textwidth,
yshift=-30mm,font={%
\fontsize{98pt}{104}\selectfont\bfseries}]  (BG) at (current page text area.north east){\thepart};
%
\node[black,inner sep=0mm,draw=none,
anchor=mid,text width=1.2\textwidth,
 minimum height=35mm, align=right,
node distance=7mm,below=of BG,
font={\fontsize{30pt}{34}\selectfont}]
(BGG)  {\hyphenchar\font=-1 \color{black}\MakeUppercase {#1}};
\draw [crimson,line width=3pt] ([yshift=0mm]BGG.north west) -- ([yshift=0mm]BGG.north east);
\draw [crimson,line width=2pt] ([yshift=0mm]BGG.south west) -- ([yshift=0mm]BGG.south east);
%
\node[fill=crimson,text=white,rotate=90,%
anchor=south west,minimum height=15mm,
minimum width=40mm,font={%
\fontsize{20pt}{20}\selectfont\bfseries}](BP)  at
(current page text area.south east)
{{\sffamily Part}~\thepart};
%
\path[red](BP.north west)-|coordinate(PS)(BGG.south west);
%
% Part summary box commented out for cleaner design
% \ifhaspartsummary
% \node[inner sep=4pt,text width=0.7\textwidth,draw=none,fill=BrownLL!40,
% align=justify,font={\fontsize{9pt}{12}\selectfont},anchor=south west]
% at (PS) {\partsummary};
% \fi
\end{tikzpicture}
\clearpage
}



% DIVISIONS: Clean geometric styling with subtle tech elements
% Used for frontmatter, main_content, and backmatter divisions
\newcommand{\division}[1]{%
\FloatBarrier%  % Flush all pending floats before division break
\clearpage
\thispagestyle{empty}
\begin{tikzpicture}[remember picture,overlay]

% Clean geometric background (original design)
\coordinate(S1)at([yshift=-200mm]current page.north west);
\draw[draw=none,fill=BlueDD!7](S1)--++(45:16)coordinate(S2)-
|(S2|-current page.north west)--(current page.north west)coordinate(S3)--(S1);

\coordinate(E1)at([yshift=-98mm]current page.north west);
\draw[draw=none,fill=BlueDD!15](E1)--(current page.north west)coordinate(E2)
--++(0:98mm)coordinate(E3)--(E1);

\coordinate(D1)at([yshift=15mm]current page.south west);
\draw[draw=none,fill=BlueDD!40,opacity=0.5](D1)--++(45:5.5)coordinate(D2)
-|(D2|-current page.north west)--(current page.north west)coordinate(D3)--(D1);

% Subtle tech elements - positioned in white areas for better visibility
% Upper right white area - more visible
\draw[crimson!40, line width=0.8pt] ([xshift=140mm,yshift=-60mm]current page.north west) -- ++(40mm,0);
\draw[crimson!40, line width=0.8pt] ([xshift=150mm,yshift=-70mm]current page.north west) -- ++(30mm,0);
\draw[crimson!35, line width=0.7pt] ([xshift=160mm,yshift=-60mm]current page.north west) -- ++(0,-15mm);
\draw[crimson!35, line width=0.7pt] ([xshift=170mm,yshift=-70mm]current page.north west) -- ++(0,10mm);

% Circuit nodes - upper right
\fill[crimson!50] ([xshift=160mm,yshift=-60mm]current page.north west) circle (1.5mm);
\fill[white] ([xshift=160mm,yshift=-60mm]current page.north west) circle (0.8mm);
\fill[crimson!50] ([xshift=170mm,yshift=-70mm]current page.north west) circle (1.3mm);
\fill[white] ([xshift=170mm,yshift=-70mm]current page.north west) circle (0.6mm);

% Lower right white area - enhanced visibility
\draw[crimson!45, line width=0.9pt] ([xshift=140mm,yshift=-190mm]current page.north west) -- ++(45mm,0);
\draw[crimson!45, line width=0.9pt] ([xshift=150mm,yshift=-200mm]current page.north west) -- ++(35mm,0);
\draw[crimson!40, line width=0.8pt] ([xshift=160mm,yshift=-190mm]current page.north west) -- ++(0,-20mm);
\draw[crimson!40, line width=0.8pt] ([xshift=170mm,yshift=-200mm]current page.north west) -- ++(0,15mm);

% Additional connecting lines in lower right
\draw[crimson!35, line width=0.7pt] ([xshift=130mm,yshift=-180mm]current page.north west) -- ++(25mm,0);
\draw[crimson!35, line width=0.7pt] ([xshift=145mm,yshift=-180mm]current page.north west) -- ++(0,-25mm);

% Circuit nodes - lower right (more prominent)
\fill[crimson!55] ([xshift=160mm,yshift=-190mm]current page.north west) circle (1.6mm);
\fill[white] ([xshift=160mm,yshift=-190mm]current page.north west) circle (0.9mm);
\fill[crimson!55] ([xshift=170mm,yshift=-200mm]current page.north west) circle (1.4mm);
\fill[white] ([xshift=170mm,yshift=-200mm]current page.north west) circle (0.7mm);
\fill[crimson!50] ([xshift=145mm,yshift=-180mm]current page.north west) circle (1.2mm);
\fill[white] ([xshift=145mm,yshift=-180mm]current page.north west) circle (0.6mm);

% Title positioned in center - clean and readable
\node[inner sep=0mm,draw=none,anchor=center,text width=0.8\textwidth,
align=center,font={\fontsize{40pt}{40}\selectfont}]
(BGG) at (current page.center)  {\color{crimson}\MakeUppercase {#1}};

\end{tikzpicture}
\clearpage
}

% LAB DIVISIONS: Circuit-style neural network design for lab sections
% Used specifically for lab platform sections (arduino, xiao, grove, etc.)
\newcommand{\labdivision}[1]{%
\FloatBarrier%  % Flush all pending floats before lab division break
\clearpage
\thispagestyle{empty}
\begin{tikzpicture}[remember picture,overlay]
% Circuit background with subtle gradient
\coordinate(S1)at([yshift=-200mm]current page.north west);
\draw[draw=none,fill=BlueDD!5](S1)--++(45:16)coordinate(S2)-
|(S2|-current page.north west)--(current page.north west)coordinate(S3)--(S1);

% TOP AREA: Circuit lines in upper white space
\draw[crimson!50, line width=1.5pt] ([xshift=30mm,yshift=-40mm]current page.north west) -- ++(60mm,0);
\draw[crimson!40, line width=1pt] ([xshift=120mm,yshift=-50mm]current page.north west) -- ++(50mm,0);
\draw[crimson!50, line width=1.5pt] ([xshift=40mm,yshift=-70mm]current page.north west) -- ++(40mm,0);

% Connecting lines in top area
\draw[crimson!30, line width=1pt] ([xshift=60mm,yshift=-40mm]current page.north west) -- ++(0,-20mm);
\draw[crimson!30, line width=1pt] ([xshift=145mm,yshift=-50mm]current page.north west) -- ++(0,10mm);

% Neural nodes in top area
\fill[crimson!70] ([xshift=60mm,yshift=-40mm]current page.north west) circle (2.5mm);
\fill[white] ([xshift=60mm,yshift=-40mm]current page.north west) circle (1.5mm);
\fill[crimson!60] ([xshift=145mm,yshift=-50mm]current page.north west) circle (2mm);
\fill[white] ([xshift=145mm,yshift=-50mm]current page.north west) circle (1mm);
\fill[crimson!80] ([xshift=80mm,yshift=-70mm]current page.north west) circle (2mm);
\fill[white] ([xshift=80mm,yshift=-70mm]current page.north west) circle (1mm);

% BOTTOM AREA: Circuit lines in lower white space
\draw[crimson!50, line width=1.5pt] ([xshift=20mm,yshift=-200mm]current page.north west) -- ++(70mm,0);
\draw[crimson!40, line width=1pt] ([xshift=110mm,yshift=-210mm]current page.north west) -- ++(60mm,0);
\draw[crimson!50, line width=1.5pt] ([xshift=35mm,yshift=-230mm]current page.north west) -- ++(45mm,0);

% Connecting lines in bottom area
\draw[crimson!30, line width=1pt] ([xshift=55mm,yshift=-200mm]current page.north west) -- ++(0,-20mm);
\draw[crimson!30, line width=1pt] ([xshift=140mm,yshift=-210mm]current page.north west) -- ++(0,15mm);

% Neural nodes in bottom area
\fill[crimson!70] ([xshift=55mm,yshift=-200mm]current page.north west) circle (2.5mm);
\fill[white] ([xshift=55mm,yshift=-200mm]current page.north west) circle (1.5mm);
\fill[crimson!60] ([xshift=140mm,yshift=-210mm]current page.north west) circle (2mm);
\fill[white] ([xshift=140mm,yshift=-210mm]current page.north west) circle (1mm);
\fill[crimson!80] ([xshift=80mm,yshift=-230mm]current page.north west) circle (2mm);
\fill[white] ([xshift=80mm,yshift=-230mm]current page.north west) circle (1mm);

% SIDE AREAS: Subtle circuit elements on left and right edges
\draw[crimson!30, line width=1pt] ([xshift=15mm,yshift=-120mm]current page.north west) -- ++(20mm,0);
\draw[crimson!30, line width=1pt] ([xshift=175mm,yshift=-130mm]current page.north west) -- ++(15mm,0);
\fill[crimson!50] ([xshift=25mm,yshift=-120mm]current page.north west) circle (1.5mm);
\fill[white] ([xshift=25mm,yshift=-120mm]current page.north west) circle (0.8mm);
\fill[crimson!50] ([xshift=185mm,yshift=-130mm]current page.north west) circle (1.5mm);
\fill[white] ([xshift=185mm,yshift=-130mm]current page.north west) circle (0.8mm);

% Title positioned in center - CLEAN AREA
\node[inner sep=0mm,draw=none,anchor=center,text width=0.8\textwidth,
align=center,font={\fontsize{44pt}{44}\selectfont\bfseries}]
(BGG) at (current page.center)  {\color{crimson}\MakeUppercase {#1}};

\end{tikzpicture}
\clearpage
}

% Define \lab command for lab styling (different visual treatment)
\newcommand{\lab}[1]{%
\begin{tikzpicture}[remember picture,overlay]
%%%
% Different background pattern for labs
\coordinate(S1)at([yshift=-200mm]current page.north west);
\draw[draw=none,fill=BlueDD!15](S1)--++(45:16)coordinate(S2)-
|(S2|-current page.north west)--(current page.north west)coordinate(S3)--(S1);
%
\coordinate(E1)at([yshift=-98mm]current page.north west);
\draw[draw=none,fill=BlueDD!25](E1)--(current page.north west)coordinate(E2)
--++(0:98mm)coordinate(E3)--(E1);
%
\coordinate(D1)at([yshift=15mm]current page.south west);
\draw[draw=none,fill=BlueDD!60,opacity=0.7](D1)--++(45:5.5)coordinate(D2)
-|(D2|-current page.north west)--(current page.north west)coordinate(D3)--(D1);
%%%%
\path[red](S2)-|(S2-|current page.east)coordinate(SS2);
%LAB - Different styling
\node[crimson,align=flush right,inner sep=0,outer sep=0mm,draw=none,anchor=south,
font={\fontsize{48pt}{48}\selectfont\bfseries}]  (BG) at ($(S2)!0.5!(SS2)$){\hphantom{Workshop}};
%%%
\path[green]([yshift=15mm]D2)-|coordinate(TPD)(BG.south east);
\node[inner sep=0mm,draw=none,anchor=south east,%text width=0.9\textwidth,
align=right,font={\fontsize{40pt}{40}\selectfont}]
(BGG) at (TPD)  {\color{crimson}\MakeUppercase {#1}};%\MakeUppercase {}
\end{tikzpicture}
\thispagestyle{empty}
\clearpage
}

% =============================================================================
% SECTION FORMATTING
% =============================================================================
% All section levels use crimson color and are ragged right

% Section (Large, bold, crimson)
\titleformat{\section}
  {\normalfont\Large\bfseries\color{crimson}\raggedright}
  {\thesection}
  {0.5em}
  {#1}
\titlespacing*{\section}{0pc}{14pt plus 4pt minus 4pt}{6pt plus 2pt minus 2pt}[0pc]

% Subsection (large, bold, crimson)
\titleformat{\subsection}
  {\normalfont\large\bfseries\color{crimson}\raggedright}
  {\thesubsection}
  {0.5em}
  {#1}
\titlespacing*{\subsection}{0pc}{12pt plus 4pt minus 4pt}{5pt plus 1pt minus 2pt}[0pc]

% Subsubsection (normal size, bold, crimson)
\titleformat{\subsubsection}
  {\normalfont\normalsize\bfseries\color{crimson}\raggedright}
  {\thesubsubsection}
  {0.5em}
  {#1}
\titlespacing*{\subsubsection}{0pc}{12pt plus 4pt minus 4pt}{5pt plus 1pt minus 2pt}[0pc]

% Paragraph (run-in, bold, crimson, ends with period)
\titleformat{\paragraph}[runin]
  {\normalfont\normalsize\bfseries\color{crimson}}
  {\theparagraph}
  {0.5em}
  {#1}
  [\textbf{.}]
  \titlespacing*{\paragraph}{0pc}{6pt plus 2pt minus 2pt}{0.5em}[0pc]

% Subparagraph (run-in, italic, crimson, ends with period)
\titleformat{\subparagraph}[runin]
  {\normalfont\normalsize\itshape\color{crimson}}
  {\thesubparagraph}
  {0.5em}
  {#1}
  [\textbf{.}]
  \titlespacing*{\subparagraph}{0pc}{6pt plus 2pt minus 2pt}{0.5em}[0pc]

% =============================================================================
% CHAPTER FORMATTING
% =============================================================================
% Numbered chapters: "Chapter X" prefix, huge crimson title
\titleformat{\chapter}[display]
  {\normalfont\huge\bfseries\color{crimson}}
  {\chaptername\ \thechapter}
  {20pt}
  {\Huge #1}
  []

% Unnumbered chapters: no prefix, huge crimson title
\titleformat{name=\chapter,numberless}
  {\normalfont\huge\bfseries\color{crimson}}
  {}
  {0pt}
  {\Huge #1}
  []

\renewcommand{\chaptername}{Chapter}
% =============================================================================
% TABLE OF CONTENTS FORMATTING
% =============================================================================
\setcounter{tocdepth}{2}                      % Show chapters, sections, subsections

% TOC spacing adjustments for number widths and indentation
\setlength{\cftchapnumwidth}{2em}             % Chapter number width
\setlength{\cftsecnumwidth}{2.75em}           % Section number width
\setlength{\cftsubsecnumwidth}{3.25em}        % Subsection number width
\setlength{\cftsubsubsecnumwidth}{4em}        % Subsubsection number width
\setlength{\cftsubsecindent}{4.25em}          % Subsection indent
\setlength{\cftsubsubsecindent}{7.5em}        % Subsubsection indent

% Chapter entries in TOC: bold crimson with "Chapter" prefix
\renewcommand{\cftchapfont}{\bfseries\color{crimson}}
\renewcommand{\cftchappresnum}{\color{crimson}Chapter~}

% Custom formatting for division entries (styled like parts)
\newcommand{\divisionchapter}[1]{%
  \addvspace{12pt}%
  \noindent\hfil\bfseries\color{crimson}#1\hfil\par%
  \addvspace{6pt}%
}

% Adjust TOC spacing for "Chapter" prefix
\newlength{\xtraspace}
\settowidth{\xtraspace}{\cftchappresnum\cftchapaftersnum}
\addtolength{\cftchapnumwidth}{\xtraspace}

% Unnumbered chapters with TOC entry
\newcommand{\likechapter}[1]{%
    \chapter*{#1}
    \addcontentsline{toc}{chapter}{\textcolor{crimson}{#1}}
}

% =============================================================================
% PAGE NUMBERING SYSTEM
% =============================================================================
% Implements traditional book numbering:
% - Roman numerals (i, ii, iii...) for frontmatter
% - Arabic numerals (1, 2, 3...) for mainmatter
% Automatically switches at first numbered chapter
\makeatletter
\newif\if@firstnumbered%
\@firstnumberedtrue%
\newif\if@firstunnumbered%
\@firstunnumberedtrue%

\newcounter{lastRomanPage}
\setcounter{lastRomanPage}{1}

% Start document with Roman numerals (frontmatter)
\AtBeginDocument{
  \pagenumbering{roman}
  \renewcommand{\thepage}{\roman{page}}
}

% Intercept chapter command
\let\old@chapter\chapter%
\renewcommand{\chapter}{%
  \@ifstar{\unnumbered@chapter}{\numbered@chapter}%
}

% Numbered chapters: switch to Arabic on first occurrence
\newcommand{\numbered@chapter}[1]{%
  \if@firstnumbered%
    \cleardoublepage%
    \setcounter{lastRomanPage}{\value{page}}%
    \pagenumbering{arabic}%
    \@firstnumberedfalse%
  \else
    \setcounter{page}{\value{page}}%
  \fi
  \setcounter{sidenote}{1}                    % Reset footnote counter per chapter
  \old@chapter{#1}%
}

% Unnumbered chapters: stay in Roman numerals
\newcommand{\unnumbered@chapter}[1]{%
  \if@firstunnumbered%
    \clearpage
    \setcounter{lastRomanPage}{\value{page}}%
    \pagenumbering{roman}%
    \@firstunnumberedfalse%
  \fi
  \setcounter{sidenote}{1}
  \old@chapter*{#1}%
}
\makeatother

% =============================================================================
% TABLE SIZING AND SPACING
% =============================================================================
% Make tables slightly smaller to fit more content
\AtBeginEnvironment{longtable}{\scriptsize}

% Increase vertical spacing in table cells (default is 1.0)
\renewcommand{\arraystretch}{1.3}

% Prefer placing tables at the top of pages
\makeatletter
\renewcommand{\fps@table}{t}  % Default placement: top of page
\makeatother

% =============================================================================
% LONGTABLE PAGE BREAKING FIXES (Windows compatibility)
% =============================================================================
% Prevent "Infinite glue shrinkage" errors on Windows LaTeX builds
% by giving longtable more flexibility in page breaking

% Allow more flexible page breaking (vs strict \flushbottom)
\raggedbottom

% Process more rows before attempting page break (default is 20)
\setcounter{LTchunksize}{50}

% Add extra stretch for longtable environments specifically
\AtBeginEnvironment{longtable}{%
  \setlength{\emergencystretch}{3em}%
  \setlength{\parskip}{0pt plus 1pt}%
}

% =============================================================================
% TABLE STYLING - Clean tables with crimson borders
% =============================================================================
% Professional table appearance with:
% - Clean white background (no colored rows)
% - Crimson-colored borders
% - Good spacing for readability
%
% Note: Headers are automatically bolded by Quarto when using **text** in source
\usepackage{booktabs}      % Professional table rules (\toprule, \midrule, \bottomrule)
\usepackage{colortbl}      % For colored borders (\arrayrulecolor)

% Global table styling - crimson borders
\setlength{\arrayrulewidth}{0.5pt}          % Thinner borders than default
%\arrayrulecolor{crimson}                    % Crimson borders matching brand

\setcounter{chapter}{0}
\usepackage{needspace}
\let\Needspace\needspace
\makeatletter
\@ifpackageloaded{float}{}{\usepackage{float}}
\floatstyle{plain}
\@ifundefined{c@chapter}{\newfloat{vid}{h}{lovid}}{\newfloat{vid}{h}{lovid}[chapter]}
\floatname{vid}{Video}
\newcommand*\listofvids{\listof{vid}{List of Videos}}
\makeatother
\makeatletter
\@ifpackageloaded{tcolorbox}{}{\usepackage[skins,breakable]{tcolorbox}}
\@ifpackageloaded{fontawesome5}{}{\usepackage{fontawesome5}}
\definecolor{quarto-callout-color}{HTML}{909090}
\definecolor{quarto-callout-note-color}{HTML}{0758E5}
\definecolor{quarto-callout-important-color}{HTML}{CC1914}
\definecolor{quarto-callout-warning-color}{HTML}{EB9113}
\definecolor{quarto-callout-tip-color}{HTML}{00A047}
\definecolor{quarto-callout-caution-color}{HTML}{FC5300}
\definecolor{quarto-callout-color-frame}{HTML}{acacac}
\definecolor{quarto-callout-note-color-frame}{HTML}{4582ec}
\definecolor{quarto-callout-important-color-frame}{HTML}{d9534f}
\definecolor{quarto-callout-warning-color-frame}{HTML}{f0ad4e}
\definecolor{quarto-callout-tip-color-frame}{HTML}{02b875}
\definecolor{quarto-callout-caution-color-frame}{HTML}{fd7e14}
\makeatother
\makeatletter
\@ifpackageloaded{bookmark}{}{\usepackage{bookmark}}
\makeatother
\makeatletter
\@ifpackageloaded{caption}{}{\usepackage{caption}}
\AtBeginDocument{%
\ifdefined\contentsname
  \renewcommand*\contentsname{Table of contents}
\else
  \newcommand\contentsname{Table of contents}
\fi
\ifdefined\listfigurename
  \renewcommand*\listfigurename{List of Figures}
\else
  \newcommand\listfigurename{List of Figures}
\fi
\ifdefined\listtablename
  \renewcommand*\listtablename{List of Tables}
\else
  \newcommand\listtablename{List of Tables}
\fi
\ifdefined\figurename
  \renewcommand*\figurename{Figure}
\else
  \newcommand\figurename{Figure}
\fi
\ifdefined\tablename
  \renewcommand*\tablename{Table}
\else
  \newcommand\tablename{Table}
\fi
}
\@ifpackageloaded{float}{}{\usepackage{float}}
\floatstyle{ruled}
\@ifundefined{c@chapter}{\newfloat{codelisting}{h}{lop}}{\newfloat{codelisting}{h}{lop}[chapter]}
\floatname{codelisting}{Listing}
\newcommand*\listoflistings{\listof{codelisting}{List of Listings}}
\makeatother
\makeatletter
\makeatother
\makeatletter
\@ifpackageloaded{caption}{}{\usepackage{caption}}
\@ifpackageloaded{subcaption}{}{\usepackage{subcaption}}
\makeatother
\newcommand{\fbxIconPath}{assets/images/icons/callouts}
\newcommand{\fbxIconFormat}{pdf}
\makeatletter
\@ifpackageloaded{tcolorbox}{}{\usepackage[many]{tcolorbox}}
\makeatother
%%%% ---foldboxy preamble ----- %%%%%

% Load xstring for string manipulation
\RequirePackage{xstring}

% Icon path and format configuration - can be overridden in filter-metadata
\providecommand{\fbxIconPath}{assets/images/icons/callouts}
\providecommand{\fbxIconFormat}{pdf}

% Helper command to include icon with hyphen-to-underscore conversion
% This ensures consistency: callout-quiz-question -> callout_quiz_question
\newcommand{\fbxIncludeIcon}[2]{%
  \StrSubstitute{#1}{-}{_}[\fbxIconName]%
  \includegraphics[width=#2]{\fbxIconPath/icon_\fbxIconName.\fbxIconFormat}%
}

% Legacy fallback colors (keep for compatibility)
\definecolor{fbx-default-color1}{HTML}{c7c7d0}
\definecolor{fbx-default-color2}{HTML}{a3a3aa}
\definecolor{fbox-color1}{HTML}{c7c7d0}
\definecolor{fbox-color2}{HTML}{a3a3aa}

% arguments: #1 typelabelnummer: #2 titel: #3
\newenvironment{fbx}[3]{%
\begin{tcolorbox}[
  enhanced,
  breakable,
  %fontupper=\fontsize{8pt}{10pt}\selectfont,  % 95% of body text (10pt -> 9.5pt)
  before skip=8pt,  % space above box (increased)
  after skip=8pt,   % space below box (increased)
  attach boxed title to top*={xshift=0pt},
  boxed title style={
  %fuzzy shadow={1pt}{-1pt}{0mm}{0.1mm}{gray},
  arc=1.5pt,
  rounded corners=north,
  sharp corners=south,
  top=6pt,          % Adjusted for ~40px equivalent height
  bottom=5pt,       % Adjusted for ~40px equivalent height
  overlay={
      \node [left,outer sep=0em, black,draw=none,anchor=west,
        rectangle,fill=none,inner sep=0pt]
        at ([xshift=4mm]frame.west) {\fbxIncludeIcon{#1}{4.2mm}};
    },
  },
  colframe=#1-color2,             % Border color (auto-generated from YAML)
  colbacktitle=#1-color1,         % Background color (auto-generated from YAML)
  colback=white,
  coltitle=black,
  titlerule=0mm,
  toprule=0.5pt,
  bottomrule=0.5pt,
  leftrule=2.2pt,
  rightrule=0.5pt,
  outer arc=1.5pt,
  arc=1.5pt,
  left=0.5em,       % increased left padding
  bottomtitle=1.5mm, % increased title bottom margin
  toptitle=1.5mm,    % increased title top margin
  title=\hspace{2.5em}\protect#2\hspace{0.2em}\protect#3, % Protect parameters
  extras middle and last={top=4pt} % increased continuation spacing
]}
{\end{tcolorbox}}


% boxed environment with right border
\newenvironment{fbxSimple}[3]{\begin{tcolorbox}[
  enhanced,
  breakable,
  %fontupper=\fontsize{8pt}{10pt}\selectfont,  % 95% of body text (10pt -> 9.5pt)
  before skip=8pt,  % space above box (increased)
  after skip=8pt,   % space below box (increased)
  attach boxed title to top*={xshift=0pt},
  boxed title style={
  %fuzzy shadow={1pt}{-1pt}{0mm}{0.1mm}{gray},
  arc=1.5pt,
  rounded corners=north,
  sharp corners=south,
  top=6pt,          % Adjusted for ~40px equivalent height
  bottom=5pt,       % Adjusted for ~40px equivalent height
  overlay={
      \node [left,outer sep=0em, black,draw=none,anchor=west,
        rectangle,fill=none,inner sep=0pt]
        at ([xshift=3mm]frame.west) {\fbxIncludeIcon{#1}{4.2mm}};
    },
  },
  colframe=#1-color2,             % Border color (auto-generated from YAML)
  colbacktitle=#1-color1,         % Background color (auto-generated from YAML)
  colback=white,
  coltitle=black,
  titlerule=0mm,
  toprule=0.5pt,
  bottomrule=0.5pt,
  leftrule=2.2pt,
  rightrule=0.5pt,
  outer arc=1.5pt,
  arc=1.5pt,
  left=0.5em,       % increased left padding
  bottomtitle=1.5mm, % increased title bottom margin
  toptitle=1.5mm,    % increased title top margin
  title=\hspace{2.5em}\protect#2\hspace{0.2em}\protect#3, % Protect parameters
  boxsep=1pt,
  extras first={bottom=0pt},
  extras last={top=0pt,bottom=-4pt},
  overlay first={
    \draw[line width=1pt,white] ([xshift=2.2pt]frame.south west)-- ([xshift=-0.5pt]frame.south east);
  },
  overlay last={
    \draw[line width=1pt,white] ([xshift=2.2pt]frame.north west)-- ([xshift=-0.5pt]frame.north east);
   }
]}
{\end{tcolorbox}}

%%%% --- end foldboxy preamble ----- %%%%%
%%==== colors from yaml ===%
\definecolor{callout-checkpoint-color1}{HTML}{E8F5E9}
\definecolor{callout-checkpoint-color2}{HTML}{2E7D32}
\definecolor{callout-lighthouse-color1}{HTML}{FDF8E6}
\definecolor{callout-lighthouse-color2}{HTML}{B8860B}
\definecolor{callout-perspective-color1}{HTML}{F7F8FA}
\definecolor{callout-perspective-color2}{HTML}{4A5568}
\definecolor{callout-example-color1}{HTML}{F0F8F6}
\definecolor{callout-example-color2}{HTML}{148F77}
\definecolor{callout-quiz-question-color1}{HTML}{F0F0F8}
\definecolor{callout-quiz-question-color2}{HTML}{5B4B8A}
\definecolor{callout-colab-color1}{HTML}{FFF5E6}
\definecolor{callout-colab-color2}{HTML}{FF6B35}
\definecolor{callout-notebook-color1}{HTML}{F2F7FF}
\definecolor{callout-notebook-color2}{HTML}{2C5282}
\definecolor{callout-quiz-answer-color1}{HTML}{E8F2EA}
\definecolor{callout-quiz-answer-color2}{HTML}{4a7c59}
\definecolor{callout-definition-color1}{HTML}{F0F4F8}
\definecolor{callout-definition-color2}{HTML}{1B4F72}
\definecolor{callout-chapter-connection-color1}{HTML}{FDF2F7}
\definecolor{callout-chapter-connection-color2}{HTML}{A51C30}
\definecolor{callout-code-color1}{HTML}{F2F4F8}
\definecolor{callout-code-color2}{HTML}{D1D7E0}
\definecolor{callout-resource-exercises-color1}{HTML}{E0F2F1}
\definecolor{callout-resource-exercises-color2}{HTML}{20B2AA}
\definecolor{callout-resource-videos-color1}{HTML}{E0F2F1}
\definecolor{callout-resource-videos-color2}{HTML}{20B2AA}
\definecolor{callout-resource-slides-color1}{HTML}{E0F2F1}
\definecolor{callout-resource-slides-color2}{HTML}{20B2AA}
%=============%

\usepackage{hyphenat}
\usepackage{ifthen}
\usepackage{calc}
\usepackage{calculator}



\usepackage{graphicx}
\usepackage{geometry}
\usepackage{afterpage}
\usepackage{tikz}
\usetikzlibrary{calc}
\usetikzlibrary{fadings}
\usepackage[pagecolor=none]{pagecolor}


% Set the titlepage font families







% Set the coverpage font families

\usepackage{bookmark}
\IfFileExists{xurl.sty}{\usepackage{xurl}}{} % add URL line breaks if available
\urlstyle{same}
\hypersetup{
  pdftitle={Machine Learning Systems},
  pdfauthor={Vijay Janapa Reddi},
  colorlinks=true,
  linkcolor={Maroon},
  filecolor={Maroon},
  citecolor={Blue},
  urlcolor={Blue},
  pdfcreator={LaTeX via pandoc}}


\title{Machine Learning Systems}
\usepackage{etoolbox}
\makeatletter
\providecommand{\subtitle}[1]{% add subtitle to \maketitle
  \apptocmd{\@title}{\par {\large #1 \par}}{}{}
}
\makeatother
\subtitle{Volume I: Introduction}
\author{Vijay Janapa Reddi}
\date{January 28, 2026}
\begin{document}
%%%%% begin titlepage extension code

  \begin{frontmatter}

\begin{titlepage}
% This is a combination of Pandoc templating and LaTeX
% Pandoc templating https://pandoc.org/MANUAL.html#templates
% See the README for help

\thispagestyle{empty}

\newgeometry{top=-100in}

% Page color

\newcommand{\coverauthorstyle}[1]{{\fontsize{20}{24.0}\selectfont
{#1}}}

\begin{tikzpicture}[remember picture, overlay, inner sep=0pt, outer sep=0pt]

\tikzfading[name=fadeout, inner color=transparent!0,outer color=transparent!100]
\tikzfading[name=fadein, inner color=transparent!100,outer color=transparent!0]
\node[anchor=south west, rotate=0, opacity=1] at ($(current page.south west)+(0.225\paperwidth, 9)$) {
\includegraphics[width=\paperwidth, keepaspectratio]{assets/images/covers/cover-image-transparent-vol1.png}};

% Title
\newcommand{\titlelocationleft}{0.075\paperwidth}
\newcommand{\titlelocationbottom}{0.4\paperwidth}
\newcommand{\titlealign}{left}

\begin{scope}{%
\fontsize{52}{62.4}\selectfont
\node[anchor=north
west, align=left, rotate=0] (Title1) at ($(current page.south west)+(\titlelocationleft,\titlelocationbottom)$)  [text width = 0.9\paperwidth]  {{\nohyphens{Machine
Learning Systems}}};
}
\end{scope}

% Author
\newcommand{\authorlocationleft}{.925\paperwidth}
\newcommand{\authorlocationbottom}{0.150\paperwidth}
\newcommand{\authoralign}{right}

\begin{scope}
{%
\fontsize{20}{24.0}\selectfont
\node[anchor=north
east, align=right, rotate=0] (Author1) at ($(current page.south west)+(\authorlocationleft,\authorlocationbottom)$)  [text width = 6in]  {\coverauthorstyle{Vijay\\Janapa
Reddi\\}};
}
\end{scope}

% Footer
\newcommand{\footerlocationleft}{0.075\paperwidth}
\newcommand{\footerlocationbottom}{0.475\paperwidth}
\newcommand{\footerlocationalign}{left}

\begin{scope}
{%
\fontsize{25}{30.0}\selectfont
 \node[anchor=north west, align=left, rotate=0] (Footer1) at %
($(current page.south west)+(\footerlocationleft,\footerlocationbottom)$)  [text width = 0.9\paperwidth]  {{\nohyphens{Volume
I: Introduction}}};
}
\end{scope}

\end{tikzpicture}
\clearpage
\restoregeometry
%%% TITLE PAGE START

% Set up alignment commands
%Page
\newcommand{\titlepagepagealign}{
\ifthenelse{\equal{left}{right}}{\raggedleft}{}
\ifthenelse{\equal{left}{center}}{\centering}{}
\ifthenelse{\equal{left}{left}}{\raggedright}{}
}


\newcommand{\titleandsubtitle}{
% Title and subtitle
{{\huge{\bfseries{\nohyphens{Machine Learning Systems}}}}\par
}%

\vspace{\betweentitlesubtitle}
{
{\large{\textit{\nohyphens{Volume I: Introduction}}}}\par
}}
\newcommand{\titlepagetitleblock}{
\titleandsubtitle
}

\newcommand{\authorstyle}[1]{{\large{#1}}}

\newcommand{\affiliationstyle}[1]{{\large{#1}}}

\newcommand{\titlepageauthorblock}{
{\authorstyle{\nohyphens{Vijay Janapa
Reddi}{\textsuperscript{1}}\textsuperscript{,}{\textsuperscript{,*}}}}}

\newcommand{\titlepageaffiliationblock}{
\hangindent=1em
\hangafter=1
{\affiliationstyle{
{1}.~Harvard University


\vspace{1\baselineskip}
* \textit{Correspondence:}~Vijay Janapa Reddi~vj@eecs.harvard.edu
}}
}
\newcommand{\headerstyled}{%
{}
}
\newcommand{\footerstyled}{%
{\large{}}
}
\newcommand{\datestyled}{%
{January 28, 2026}
}


\newcommand{\titlepageheaderblock}{\headerstyled}

\newcommand{\titlepagefooterblock}{
\footerstyled
}

\newcommand{\titlepagedateblock}{
\datestyled
}

%set up blocks so user can specify order
\newcommand{\titleblock}{\newlength{\betweentitlesubtitle}
\setlength{\betweentitlesubtitle}{0.05\textheight}
{

{\titlepagetitleblock}
}

\vspace{4\baselineskip}
}

\newcommand{\authorblock}{{\titlepageauthorblock}

\vspace{2\baselineskip}
}

\newcommand{\affiliationblock}{{\titlepageaffiliationblock}

\vspace{0pt}
}

\newcommand{\logoblock}{}

\newcommand{\footerblock}{}

\newcommand{\dateblock}{{\titlepagedateblock}

\vspace{0pt}
}

\newcommand{\headerblock}{}

\thispagestyle{empty} % no page numbers on titlepages


\newcommand{\vrulecode}{\textcolor{black}{\rule{\vrulewidth}{\textheight}}}
\newlength{\vrulewidth}
\setlength{\vrulewidth}{2pt}
\newlength{\B}
\setlength{\B}{\ifdim\vrulewidth > 0pt 0.05\textwidth\else 0pt\fi}
\newlength{\minipagewidth}
\ifthenelse{\equal{left}{left} \OR \equal{left}{right} }
{% True case
\setlength{\minipagewidth}{\textwidth - \vrulewidth - \B - 0.1\textwidth}
}{
\setlength{\minipagewidth}{\textwidth - 2\vrulewidth - 2\B - 0.1\textwidth}
}
\ifthenelse{\equal{left}{left} \OR \equal{left}{leftright}}
{% True case
\raggedleft % needed for the minipage to work
\vrulecode
\hspace{\B}
}{%
\raggedright % else it is right only and width is not 0
}
% [position of box][box height][inner position]{width}
% [s] means stretch out vertically; assuming there is a vfill
\begin{minipage}[b][\textheight][s]{\minipagewidth}
\titlepagepagealign
\titleblock

Prof.~Vijay Janapa Reddi

School of Engineering and Applied Sciences

Harvard University

\vspace{80mm}

With heartfelt gratitude to the community for their invaluable
contributions and steadfast support.

\vfill

January 28, 2026

\vfill
\par

\end{minipage}\ifthenelse{\equal{left}{right} \OR \equal{left}{leftright} }{
\hspace{\B}
\vrulecode}{}
\clearpage
%%% TITLE PAGE END
\end{titlepage}
\setcounter{page}{1}
\end{frontmatter}

%%%%% end titlepage extension code

\renewcommand*\contentsname{Table of contents}
{
\hypersetup{linkcolor=}
\setcounter{tocdepth}{2}
\tableofcontents
}

\mainmatter
\bookmarksetup{startatroot}

\chapter*{Welcome to Volume I}\label{welcome-to-volume-i}
\addcontentsline{toc}{chapter}{Welcome to Volume I}

\markboth{Welcome to Volume I}{Welcome to Volume I}

\section*{What You Will Learn}\label{what-you-will-learn}
\addcontentsline{toc}{section}{What You Will Learn}

\markright{What You Will Learn}

Volume I progresses through four stages:

\begin{itemize}
\tightlist
\item
  \textbf{Part I: Foundations} --- Build your conceptual foundation with
  mental models that underpin all effective systems work.
\item
  \textbf{Part II: Build} --- Engineer complete workflows from data
  pipelines through training infrastructure.
\item
  \textbf{Part III: Optimize} --- Transform theoretical understanding
  into systems that run efficiently in resource-constrained
  environments.
\item
  \textbf{Part IV: Deploy} --- Navigate serving, operations, and
  responsible engineering practices.
\end{itemize}

\section*{Prerequisites}\label{prerequisites}
\addcontentsline{toc}{section}{Prerequisites}

\markright{Prerequisites}

This volume assumes:

\begin{itemize}
\tightlist
\item
  \textbf{Programming proficiency} in Python with familiarity in NumPy
\item
  \textbf{Mathematics foundations} in linear algebra, calculus, and
  probability at the undergraduate level
\item
  Prior ML experience is helpful but not required;
  \textbf{?@sec-deep-learning-systems-foundations} provides essential
  background
\end{itemize}

\section*{Support Our Mission}\label{support-our-mission}
\addcontentsline{toc}{section}{Support Our Mission}

\markright{Support Our Mission}

\section*{Continue Your Journey}\label{continue-your-journey}
\addcontentsline{toc}{section}{Continue Your Journey}

\markright{Continue Your Journey}

\section*{Listen to the AI Podcast}\label{listen-to-the-ai-podcast}
\addcontentsline{toc}{section}{Listen to the AI Podcast}

\markright{Listen to the AI Podcast}

\section*{Want to Help Out?}\label{want-to-help-out}
\addcontentsline{toc}{section}{Want to Help Out?}

\markright{Want to Help Out?}

This is a collaborative project, and your input matters. If you'd like
to contribute, check out our
\href{https://github.com/harvard-edge/cs249r_book/blob/dev/docs/contribute.md}{contribution
guidelines}. Feedback, corrections, and new ideas are welcome. Simply
file a GitHub
\href{https://github.com/harvard-edge/cs249r_book/issues}{issue}.

\bookmarksetup{startatroot}

\chapter{Mathematical Foundations}\label{mathematical-foundations}

\bookmarksetup{startatroot}

\chapter{Mathematical Foundations}\label{sec-mathematical-foundations}

This appendix provides the mathematical tools you will need to reason
about ML system performance. Rather than an exhaustive reference, think
of this as a practitioner's toolkit: the concepts that come up again and
again when debugging slow training jobs, optimizing inference latency,
or choosing between hardware configurations.

\section*{Quick Reference}\label{quick-reference}
\addcontentsline{toc}{section}{Quick Reference}

\markright{Quick Reference}

Before diving in, Table~\ref{tbl-appendix-overview} provides a map of
what this appendix covers and where each concept appears in the main
text.

\begin{longtable}[]{@{}
  >{\raggedright\arraybackslash}p{(\linewidth - 4\tabcolsep) * \real{0.1985}}
  >{\raggedright\arraybackslash}p{(\linewidth - 4\tabcolsep) * \real{0.3309}}
  >{\raggedright\arraybackslash}p{(\linewidth - 4\tabcolsep) * \real{0.4632}}@{}}
\caption{\textbf{Appendix Quick Reference}: Each mathematical concept in
this appendix, what engineering decisions it informs, and where it
appears in the main text.}\label{tbl-appendix-overview}\tabularnewline
\toprule\noalign{}
\begin{minipage}[b]{\linewidth}\raggedright
\textbf{Topic}
\end{minipage} & \begin{minipage}[b]{\linewidth}\raggedright
\textbf{What It Helps You Do}
\end{minipage} & \begin{minipage}[b]{\linewidth}\raggedright
\textbf{Where It Appears}
\end{minipage} \\
\midrule\noalign{}
\endfirsthead
\toprule\noalign{}
\begin{minipage}[b]{\linewidth}\raggedright
\textbf{Topic}
\end{minipage} & \begin{minipage}[b]{\linewidth}\raggedright
\textbf{What It Helps You Do}
\end{minipage} & \begin{minipage}[b]{\linewidth}\raggedright
\textbf{Where It Appears}
\end{minipage} \\
\midrule\noalign{}
\endhead
\bottomrule\noalign{}
\endlastfoot
\textbf{Roofline Model} & Determine if a workload is memory-bound or
compute-bound & \textbf{?@sec-ai-acceleration},
\textbf{?@sec-ai-training}, \textbf{?@sec-model-serving-systems} \\
\textbf{Amdahl's Law} & Calculate maximum speedup from parallelization &
\textbf{?@sec-ai-acceleration}, \textbf{?@sec-ai-training} \\
\textbf{Gustafson's Law} & Understand weak scaling for large models &
\textbf{?@sec-ai-training} \\
\textbf{Little's Law} & Size serving infrastructure for target QPS &
\textbf{?@sec-model-serving-systems} \\
\textbf{Memory Hierarchy} & Optimize data movement and cache usage &
\textbf{?@sec-ai-acceleration}, \textbf{?@sec-model-compression} \\
\textbf{Numerical Formats} & Choose precision for training vs inference
& \textbf{?@sec-model-compression},
\textbf{?@sec-model-serving-systems} \\
\textbf{GEMM Operations} & Understand the core computation in neural
networks & \textbf{?@sec-deep-learning-systems-foundations},
\textbf{?@sec-ai-acceleration} \\
\textbf{Backpropagation} & Debug gradient issues and memory usage &
\textbf{?@sec-deep-learning-systems-foundations},
\textbf{?@sec-ai-training} \\
\textbf{Sparse Formats (CSR)} & Work with recommendation systems and
pruned models & \textbf{?@sec-model-compression} \\
\textbf{Computational Graphs} & Understand compiler optimizations &
\textbf{?@sec-ai-frameworks}, \textbf{?@sec-ai-acceleration} \\
\end{longtable}

\begin{center}\rule{0.5\linewidth}{0.5pt}\end{center}

\section{Performance Analysis
Models}\label{sec-mathematical-foundations-performance-analysis-models-3abc}

\begin{tcolorbox}[enhanced jigsaw, colbacktitle=quarto-callout-tip-color!10!white, left=2mm, coltitle=black, toprule=.15mm, rightrule=.15mm, colback=white, breakable, colframe=quarto-callout-tip-color-frame, bottomtitle=1mm, opacityback=0, bottomrule=.15mm, title=\textcolor{quarto-callout-tip-color}{\faLightbulb}\hspace{0.5em}{Why This Matters}, leftrule=.75mm, arc=.35mm, titlerule=0mm, opacitybacktitle=0.6, toptitle=1mm]

You have trained a model that achieves good accuracy, but inference
takes 200ms when your SLA requires 50ms. Where do you start? Performance
analysis models give you a systematic way to diagnose whether you are
limited by computation, memory bandwidth, or something else entirely.
Without these tools, optimization becomes guesswork.

\end{tcolorbox}

The models in this section form the foundation of quantitative systems
thinking. Master them, and you will be able to look at almost any ML
workload and predict where the bottlenecks lie.

\subsection{The Roofline
Model}\label{sec-mathematical-foundations-roofline-model-32fb}

The Roofline Model (\citeproc{ref-williams2009roofline}{Williams,
Waterman, and Patterson 2009}) answers a deceptively simple question:
\emph{how fast can this workload possibly run on this hardware?} The
answer depends on whether you run out of compute or memory bandwidth
first.

Every operation has an \textbf{arithmetic intensity}: the ratio of
computations performed to bytes moved from memory. Matrix multiplication
has high arithmetic intensity because you can reuse each loaded element
many times. Element-wise operations like ReLU have low intensity because
you load a number, do one operation, and write it back.

\begin{figure}[htb]

\centering{

\pandocbounded{\includegraphics[keepaspectratio]{index_files/mediabag/d056e3cba1e2389a96f44fafaedbc47de20b3d75.pdf}}

}

\caption{\label{fig-roofline}\textbf{The Roofline Model}: Performance
ceiling for a hypothetical accelerator. The sloped line represents
memory bandwidth limits; the horizontal line represents peak compute.
Every workload can be plotted on this diagram to determine its
optimization strategy.}

\end{figure}%

\textbf{How to read a roofline plot.} Figure~\ref{fig-roofline} shows
the roofline for a hypothetical accelerator. The key to reading this
diagram is the \textbf{ridge point}, the corner where the two lines
meet. Start there and work outward:

\begin{enumerate}
\def\labelenumi{\arabic{enumi}.}
\item
  \textbf{Find the ridge point.} This tells you the hardware's balance
  between compute and memory. For this accelerator, the ridge point
  occurs at a specific arithmetic intensity (marked on the x-axis).
  Calculate yours using Equation~\ref{eq-ridge-point} below.
\item
  \textbf{Plot your workload's arithmetic intensity.} Compute
  Equation~\ref{eq-arithmetic-intensity} for your operation. Draw a
  vertical line up from that x-value until it hits the roofline.
\item
  \textbf{Read off the regime.} If your vertical line hits the sloped
  portion, you are \textbf{memory-bound}. If it hits the flat portion,
  you are \textbf{compute-bound}. This distinction determines your
  entire optimization strategy.
\end{enumerate}

Now examine where Workload A sits on the plot. It falls on the sloped
portion, meaning it operates in the memory-bound regime. What does this
mean practically? Every byte of additional memory bandwidth would
directly translate to more FLOP/s. Buying a faster GPU would not help
much because the compute units are already waiting for data. Instead,
you should focus on data reuse: can you tile the computation to keep
data in cache? Can you fuse this operation with adjacent operations to
avoid writing intermediate results to HBM?

Contrast this with Workload B on the flat portion. This workload is
already hitting peak compute. The memory system is feeding data fast
enough; the bottleneck is raw arithmetic throughput. Adding more memory
bandwidth would not help. The only paths to better performance are: (1)
a faster processor, (2) algorithmic improvements that reduce total
FLOPs, or (3) mixed-precision arithmetic that doubles throughput.

The power of this diagram is that once you learn to read it, you can
interpret ANY roofline plot for any hardware. The specific numbers
change (an A100 has a different ridge point than an H100), but the
interpretation process is identical.

The key equations formalize these observations. Arithmetic intensity is:

\begin{equation}\phantomsection\label{eq-arithmetic-intensity}{ \text{Arithmetic Intensity} = \frac{\text{FLOPs}}{\text{Bytes Accessed}} }\end{equation}

In the \textbf{memory-bound region} (the sloped portion), performance
scales with bandwidth (Equation~\ref{eq-memory-bound}):

\begin{equation}\phantomsection\label{eq-memory-bound}{ \text{Attainable Performance} = \text{Bandwidth} \times \text{Arithmetic Intensity} }\end{equation}

In the \textbf{compute-bound region} (the flat portion), you hit the
hardware ceiling (Equation~\ref{eq-compute-bound}):

\begin{equation}\phantomsection\label{eq-compute-bound}{ \text{Attainable Performance} = \text{Peak FLOP/s} }\end{equation}

The transition point, called the \textbf{ridge point}, characterizes the
hardware's balance between compute and memory:

\begin{equation}\phantomsection\label{eq-ridge-point}{ \text{Ridge Point} = \frac{\text{Peak FLOP/s}}{\text{Memory Bandwidth}} }\end{equation}

\subsubsection{A Concrete
Example}\label{sec-mathematical-foundations-concrete-example-b803}

Consider an NVIDIA A100 GPU with FP16 Tensor Core performance of 312
TFLOP/s and HBM2e bandwidth of 2.0 TB/s. The ridge point is
\(312 / 2.0 = 156\) FLOP/byte.

Now compare two common operations:

\textbf{GEMM (Matrix Multiplication)}: For two \(4096 \times 4096\)
matrices, arithmetic intensity is approximately \(1365\) FLOP/byte.
Since \(1365 > 156\), this operation is compute-bound. You are using the
hardware efficiently.

\textbf{ReLU (Element-wise)}: For a \(4096 \times 4096\) tensor,
intensity is approximately \(0.25\) op/byte. Since \(0.25 \ll 156\),
this operation is severely memory-bound, achieving only about \(0.16\%\)
of peak TFLOP/s. The hardware is mostly waiting for data.

This explains why modern frameworks fuse operations: combining ReLU with
the preceding MatMul avoids writing intermediate results to memory,
effectively increasing arithmetic intensity.

\subsection{Amdahl's Law and Gustafson's
Law}\label{sec-mathematical-foundations-amdahls-law-gustafsons-law-ce70}

You have access to a cluster with 64 GPUs. How much faster will training
be compared to a single GPU? The answer is almost never ``64 times
faster,'' and these two laws explain why.

\textbf{Amdahl's Law} (\citeproc{ref-amdahl1967validity}{Amdahl 1967})
captures the fundamental limit of parallelization
(Equation~\ref{eq-amdahl}). If a fraction \(p\) of your workload can be
parallelized, the maximum speedup with \(n\) processors is:

\begin{equation}\phantomsection\label{eq-amdahl}{ \text{Speedup}(n) = \frac{1}{(1-p) + \frac{p}{n}} }\end{equation}

As \(n \to \infty\), speedup approaches \(\frac{1}{1-p}\). If 10\% of
your training iteration is serial (data loading on CPU, gradient
aggregation), the maximum possible speedup is 10x. Adding more GPUs
beyond that point yields diminishing returns.

This sounds pessimistic, but \textbf{Gustafson's Law}
(\citeproc{ref-gustafson1988reevaluating}{Gustafson 1988}) offers a more
optimistic view for ML workloads (Equation~\ref{eq-gustafson}). In
practice, when you get more hardware, you often train larger models or
use larger batches rather than just trying to finish faster. Under this
``weak scaling'' model, speedup becomes:

\begin{equation}\phantomsection\label{eq-gustafson}{ \text{Speedup}(n) = (1-p) + p \times n }\end{equation}

This linear scaling better describes large-scale ML training, where
increased resources enable training models that would otherwise be
impossible, not just faster training of the same model.

\subsection{Little's
Law}\label{sec-mathematical-foundations-littles-law-a361}

Shifting from training to serving, \textbf{Little's Law}
(\citeproc{ref-little1961proof}{Little 1961}) is the essential tool for
capacity planning (Equation~\ref{eq-little}). It relates the average
number of items in a system (\(L\)), the arrival rate (\(\lambda\)), and
the average wait time (\(W\)):

\begin{equation}\phantomsection\label{eq-little}{ L = \lambda \times W }\end{equation}

For ML serving: to sustain 1,000 queries per second (QPS) with 20ms
average latency, your system must handle \(1000 \times 0.02 = 20\)
concurrent requests. This tells you how many inference workers you need
and helps you right-size your infrastructure.

\begin{center}\rule{0.5\linewidth}{0.5pt}\end{center}

\section{Computer Architecture
Essentials}\label{sec-mathematical-foundations-computer-architecture-essentials-59df}

\begin{tcolorbox}[enhanced jigsaw, colbacktitle=quarto-callout-tip-color!10!white, left=2mm, coltitle=black, toprule=.15mm, rightrule=.15mm, colback=white, breakable, colframe=quarto-callout-tip-color-frame, bottomtitle=1mm, opacityback=0, bottomrule=.15mm, title=\textcolor{quarto-callout-tip-color}{\faLightbulb}\hspace{0.5em}{Why This Matters}, leftrule=.75mm, arc=.35mm, titlerule=0mm, opacitybacktitle=0.6, toptitle=1mm]

Your model runs 100x slower on CPU than GPU, but why? Understanding the
memory hierarchy explains not just why GPUs are faster, but also why
some optimizations (like batching) help dramatically while others (like
buying more RAM) barely matter. The concepts here underpin every
hardware decision you will make.

\end{tcolorbox}

The previous section gave you tools to analyze workloads. This section
explains the hardware constraints those analyses reveal.

\subsection{The Memory
Hierarchy}\label{sec-mathematical-foundations-memory-hierarchy-a996}

Computer systems use a hierarchy of memory technologies because no
single technology provides both high capacity and low latency.
Understanding this hierarchy explains why your model's memory access
patterns matter as much as its computational complexity.

\begin{figure}[htb]

\centering{

\pandocbounded{\includegraphics[keepaspectratio]{index_files/mediabag/9464b63716a5d3e17655ff0895abaa15ac86df24.pdf}}

}

\caption{\label{fig-memory-hierarchy}\textbf{The Memory Hierarchy}: Each
level trades off speed against capacity. The performance of your ML code
depends on which level your working data resides in during the critical
inner loops.}

\end{figure}%

\textbf{How to use this mental model.} Figure~\ref{fig-memory-hierarchy}
is not just a diagram to memorize; it is a tool for reasoning about code
performance. When analyzing any ML workload, ask: \emph{where does my
data live during the inner loop?}

Consider a concrete scenario. You are writing a custom attention kernel,
and your inner loop computes dot products between query and key vectors.
If those vectors fit in registers, each dot product takes about 1
nanosecond. If they must be fetched from HBM on every iteration, each
access takes roughly 100 nanoseconds. That is a 100x difference in
memory access time alone.

The latency numbers to internalize:

\begin{longtable}[]{@{}llll@{}}
\toprule\noalign{}
Level & Latency & Relative to Registers & Example Capacity \\
\midrule\noalign{}
\endhead
\bottomrule\noalign{}
\endlastfoot
Registers & \textasciitilde1 ns & 1x & \textasciitilde20 MB (A100
total) \\
L1 Cache & \textasciitilde4 ns & 4x & 128 KB per SM \\
L2 Cache & \textasciitilde20 ns & 20x & 40 MB (A100) \\
HBM & \textasciitilde100 ns & 100x & 80 GB (A100) \\
SSD & \textasciitilde100 us & 100,000x & TB scale \\
\end{longtable}

Now apply this to real code patterns:

\textbf{Data loading}: If your training loop reads each batch from SSD,
you pay 100 microseconds per access. At 1,000 batches per epoch, that is
100 milliseconds of I/O per epoch, which may dominate training time for
small models. This is why data loaders prefetch batches into DRAM while
the GPU computes.

\textbf{Activation memory}: During backpropagation, you need
intermediate activations stored during the forward pass. If these exceed
HBM capacity, they spill to CPU memory or SSD. A single access to CPU
memory over PCIe takes roughly 1 microsecond. If your backward pass
touches millions of activations, the slowdown is catastrophic. This is
why gradient checkpointing exists: it trades compute (recomputing
activations) for memory (not storing them).

\textbf{Weight updates}: During training, you read weights from HBM,
compute gradients, and write updated weights back. If you can keep the
optimizer state in HBM rather than swapping to CPU, you avoid the 10x
latency penalty of crossing the PCIe bus.

The design principle this diagram teaches: \textbf{every technique that
keeps data higher in the pyramid directly improves performance}. Tiling
keeps working sets in cache. Operator fusion avoids writing
intermediates to HBM. Batching amortizes the cost of loading weights
from HBM across multiple inputs. Once you internalize this hierarchy,
optimization strategies become intuitive rather than mysterious.

\subsection{Bandwidth
vs.~Latency}\label{sec-mathematical-foundations-bandwidth-vs-latency-ccd9}

Bandwidth (throughput) and latency (delay) are distinct constraints that
matter in different situations. Total transfer time follows:

\[ T = L + \frac{B}{BW} \]

where \(L\) is latency, \(B\) is the data size, and \(BW\) is bandwidth.

For small transfers (e.g., KV cache lookups in autoregressive
generation), latency dominates. For large transfers (e.g., loading model
checkpoints), bandwidth dominates. Understanding which regime you are in
determines whether you should optimize for lower latency or higher
throughput.

\begin{center}\rule{0.5\linewidth}{0.5pt}\end{center}

\section{Numerical
Representations}\label{sec-mathematical-foundations-numerical-representations-5a20}

\begin{tcolorbox}[enhanced jigsaw, colbacktitle=quarto-callout-tip-color!10!white, left=2mm, coltitle=black, toprule=.15mm, rightrule=.15mm, colback=white, breakable, colframe=quarto-callout-tip-color-frame, bottomtitle=1mm, opacityback=0, bottomrule=.15mm, title=\textcolor{quarto-callout-tip-color}{\faLightbulb}\hspace{0.5em}{Why This Matters}, leftrule=.75mm, arc=.35mm, titlerule=0mm, opacitybacktitle=0.6, toptitle=1mm]

Your production model runs at 50 QPS in FP32 but your target is 200 QPS.
Switching to INT8 could get you there, but will accuracy suffer?
Understanding numerical formats lets you make this trade-off
quantitatively rather than hoping for the best.

\end{tcolorbox}

Neural networks are remarkably tolerant of reduced numerical precision.
This section explains the formats you will encounter and their
trade-offs.

\subsection{Floating-Point Format
Comparison}\label{sec-mathematical-foundations-floatingpoint-format-comparison-67c3}

The IEEE 754 standard and its AI-specific derivatives define different
trade-offs between dynamic range (the span of representable values) and
precision (how finely you can represent values within that range).
Table~\ref{tbl-numerical-formats} summarizes the key formats and their
use cases.

\begin{longtable}[]{@{}
  >{\raggedright\arraybackslash}p{(\linewidth - 10\tabcolsep) * \real{0.1066}}
  >{\raggedleft\arraybackslash}p{(\linewidth - 10\tabcolsep) * \real{0.0902}}
  >{\raggedleft\arraybackslash}p{(\linewidth - 10\tabcolsep) * \real{0.1230}}
  >{\raggedleft\arraybackslash}p{(\linewidth - 10\tabcolsep) * \real{0.1230}}
  >{\raggedleft\arraybackslash}p{(\linewidth - 10\tabcolsep) * \real{0.1639}}
  >{\raggedright\arraybackslash}p{(\linewidth - 10\tabcolsep) * \real{0.3607}}@{}}
\caption{\textbf{Numerical Format Comparison}: Each format trades off
precision, dynamic range, memory footprint, and compute throughput. BF16
has emerged as the preferred training format because it matches FP32's
range while using half the
memory.}\label{tbl-numerical-formats}\tabularnewline
\toprule\noalign{}
\begin{minipage}[b]{\linewidth}\raggedright
\textbf{Format}
\end{minipage} & \begin{minipage}[b]{\linewidth}\raggedleft
\textbf{Bits}
\end{minipage} & \begin{minipage}[b]{\linewidth}\raggedleft
\textbf{Exponent}
\end{minipage} & \begin{minipage}[b]{\linewidth}\raggedleft
\textbf{Mantissa}
\end{minipage} & \begin{minipage}[b]{\linewidth}\raggedleft
\textbf{Dynamic Range}
\end{minipage} & \begin{minipage}[b]{\linewidth}\raggedright
\textbf{Typical Use Case}
\end{minipage} \\
\midrule\noalign{}
\endfirsthead
\toprule\noalign{}
\begin{minipage}[b]{\linewidth}\raggedright
\textbf{Format}
\end{minipage} & \begin{minipage}[b]{\linewidth}\raggedleft
\textbf{Bits}
\end{minipage} & \begin{minipage}[b]{\linewidth}\raggedleft
\textbf{Exponent}
\end{minipage} & \begin{minipage}[b]{\linewidth}\raggedleft
\textbf{Mantissa}
\end{minipage} & \begin{minipage}[b]{\linewidth}\raggedleft
\textbf{Dynamic Range}
\end{minipage} & \begin{minipage}[b]{\linewidth}\raggedright
\textbf{Typical Use Case}
\end{minipage} \\
\midrule\noalign{}
\endhead
\bottomrule\noalign{}
\endlastfoot
\textbf{FP32} & 32 & 8 & 23 & \textasciitilde{}\(10^{-38}\) to
\(10^{38}\) & Training (full precision), reference inference \\
\textbf{FP16} & 16 & 5 & 10 & \textasciitilde{}\(10^{-5}\) to
\(6.5 \times
10^{4}\) & Training with loss scaling, inference \\
\textbf{BF16} & 16 & 8 & 7 & Same as FP32 & Training (preferred), avoids
loss scaling \\
\textbf{FP8} & 8 & 4 or 5 & 3 or 2 & Varies & Inference on newest
hardware (H100+) \\
\textbf{INT8} & 8 & N/A & N/A & -128 to 127 & Inference after
quantization \\
\end{longtable}

\textbf{Brain Float 16 (BF16)} deserves special attention. It matches
FP32's 8-bit exponent (preserving dynamic range) while truncating the
mantissa to 7 bits. This avoids the gradient underflow problems that
plague FP16 training, eliminating the need for complex loss scaling.
Most modern training uses BF16 for this reason.

\subsection{Integer
Quantization}\label{sec-mathematical-foundations-integer-quantization-78a5}

Quantization maps continuous floating-point values to discrete integers,
typically INT8. The key challenge is choosing how to map the
floating-point range to integers.

\textbf{Symmetric quantization} centers the mapping at zero:
\[ x_{int} = \text{round}\left(\frac{x}{\alpha} \times 127\right) \]

where \(\alpha\) is the scale factor (typically the maximum absolute
value). This works well for weight distributions centered around zero.

\textbf{Asymmetric quantization} handles distributions that are not
centered (common after ReLU, which produces only non-negative values) by
adding a zero-point offset:
\[ x_{int} = \text{round}\left(\frac{x - z}{\alpha} \times 255\right) \]

The choice between symmetric and asymmetric quantization depends on your
tensor's distribution and has measurable accuracy implications.

\begin{center}\rule{0.5\linewidth}{0.5pt}\end{center}

\section{Linear Algebra for Neural
Networks}\label{sec-mathematical-foundations-linear-algebra-neural-networks-2550}

\begin{tcolorbox}[enhanced jigsaw, colbacktitle=quarto-callout-tip-color!10!white, left=2mm, coltitle=black, toprule=.15mm, rightrule=.15mm, colback=white, breakable, colframe=quarto-callout-tip-color-frame, bottomtitle=1mm, opacityback=0, bottomrule=.15mm, title=\textcolor{quarto-callout-tip-color}{\faLightbulb}\hspace{0.5em}{Why This Matters}, leftrule=.75mm, arc=.35mm, titlerule=0mm, opacitybacktitle=0.6, toptitle=1mm]

Every neural network, regardless of architecture, spends most of its
time doing matrix multiplication. Understanding GEMM performance
characteristics explains why batch size affects throughput, why certain
layer dimensions are ``better'' than others, and how to interpret
profiler output.

\end{tcolorbox}

\subsection{Tensor Operations and
Notation}\label{sec-mathematical-foundations-tensor-operations-notation-0d7d}

We use \textbf{Einstein summation} notation throughout this book because
it makes complex operations explicit. Matrix multiplication \(C = AB\)
becomes:

\[ C_{ij} = \sum_k A_{ik} B_{kj} \]

Or in einsum notation: \texttt{ik,kj-\textgreater{}ij}. This notation
extends naturally to the multi-dimensional operations in attention
mechanisms. For example, batched multi-head attention is
\texttt{bhid,bhjd-\textgreater{}bhij} (batch, head, sequence indices).

\subsection{General Matrix Multiply
(GEMM)}\label{sec-mathematical-foundations-general-matrix-multiply-gemm-c421}

GEMM is the computational workhorse of deep learning. For matrices of
size \(M \times K\) and \(K \times N\), GEMM performs \(2MNK\)
floating-point operations (multiply-accumulate counts as two
operations).

The arithmetic intensity of GEMM scales linearly with matrix dimension.
For square \(n \times n\) matrices, intensity is approximately \(n/3\)
FLOP/byte. This explains several important phenomena:

\begin{itemize}
\tightlist
\item
  \textbf{Larger batches improve efficiency}: Batching increases the
  effective matrix dimensions, pushing workloads toward the
  compute-bound region of the roofline.
\item
  \textbf{Power-of-two dimensions help}: Hardware tensor cores are
  optimized for specific tile sizes (typically 16x16 or 32x32).
  Dimensions that align with these sizes avoid padding overhead.
\item
  \textbf{Small matrices are inefficient}: A 64x64 GEMM may achieve only
  10\% of peak throughput because it cannot fully utilize the hardware.
\end{itemize}

\subsection{Memory Layouts and
Performance}\label{sec-mathematical-foundations-memory-layouts-performance-76b7}

Data layout in memory (row-major vs.~column-major) directly affects
cache efficiency. When iterating over a matrix, accessing contiguous
memory locations is dramatically faster than strided access. The
difference can be 10x to 100x in effective bandwidth.

A common optimization pattern: transpose tensors once before repeated
operations to ensure contiguous access in the hot loop. The one-time
transpose cost is amortized across many subsequent operations.

\begin{center}\rule{0.5\linewidth}{0.5pt}\end{center}

\section{Calculus for
Backpropagation}\label{sec-mathematical-foundations-calculus-backpropagation-7555}

\begin{tcolorbox}[enhanced jigsaw, colbacktitle=quarto-callout-tip-color!10!white, left=2mm, coltitle=black, toprule=.15mm, rightrule=.15mm, colback=white, breakable, colframe=quarto-callout-tip-color-frame, bottomtitle=1mm, opacityback=0, bottomrule=.15mm, title=\textcolor{quarto-callout-tip-color}{\faLightbulb}\hspace{0.5em}{Why This Matters}, leftrule=.75mm, arc=.35mm, titlerule=0mm, opacitybacktitle=0.6, toptitle=1mm]

When training fails (loss goes to NaN, gradients explode, memory runs
out), understanding what backpropagation actually does helps you
diagnose the problem. This section gives you the mental model to reason
about gradient flow and memory usage during training.

\end{tcolorbox}

\subsection{The Chain Rule in Deep
Networks}\label{sec-mathematical-foundations-chain-rule-deep-networks-cd3f}

For a composed function \(y = f(g(x))\), the derivative is:

\[ \frac{dy}{dx} = \frac{dy}{dg} \cdot \frac{dg}{dx} \]

In deep networks, this chain extends across many layers. Each layer
contributes a local Jacobian that multiplies during the backward pass.
When these multiplied terms consistently exceed 1.0, gradients explode;
when they consistently fall below 1.0, gradients vanish.

\subsection{The Backpropagation
Algorithm}\label{sec-mathematical-foundations-backpropagation-algorithm-9fc1}

Backpropagation implements the chain rule efficiently through two
passes: forward to compute outputs, backward to compute gradients.

\begin{figure}[htb]

\centering{

\pandocbounded{\includegraphics[keepaspectratio]{index_files/mediabag/af28c041d4f4415f6ee5cab832d3c4fbb20033d8.pdf}}

}

\caption{\label{fig-backprop-graph}\textbf{Backpropagation Computational
Graph}: A two-layer network showing the forward pass (black arrows) and
backward pass (red dashed arrows). Each node caches values during the
forward pass that are reused during the backward pass.}

\end{figure}%

\textbf{How to trace the computation.} Figure~\ref{fig-backprop-graph}
shows a simple two-layer network. Practice tracing both passes to
understand what happens during training:

\textbf{Forward pass (black arrows, left to right)}: Start at \(x\),
your input. Multiply by \(W_1\) to get hidden activation \(h\). Cache
\(h\) because you will need it later. Multiply \(h\) by \(W_2\) to get
output \(y\). Cache \(y\). Compare \(y\) to the target label to compute
loss \(L\).

At this point, you have computed the loss and your memory contains: the
input \(x\), the cached activation \(h\), the cached output \(y\), and
the loss \(L\). For a large model, these cached activations dominate
memory usage.

\textbf{Backward pass (red arrows, right to left)}: Now trace backward
from \(L\). The loss function tells you
\(\frac{\partial L}{\partial y}\), the gradient of loss with respect to
your prediction. This is where the error signal enters the network.

To compute \(\frac{\partial L}{\partial W_2}\), you need to know how
\(W_2\) affected \(y\). That requires the cached value of \(h\). The
chain rule gives you:
\(\frac{\partial L}{\partial W_2} = \frac{\partial L}{\partial y} \cdot h^T\).

To continue backward to \(W_1\), you need
\(\frac{\partial L}{\partial h}\), then multiply by the cached input
\(x\). Each step backward requires the activations cached during the
forward pass.

\textbf{Why this matters for memory.} The cached activations at each
layer cannot be freed until the backward pass reaches that layer. For a
100-layer network, you store 100 layers of activations simultaneously.
For a transformer processing a 4096-token sequence with hidden dimension
4096, a single layer's activations consume \(4096 \times 4096 \times 4\)
bytes \(\approx 67\) MB in FP32. Multiply by 100 layers: 6.7 GB just for
activations, often exceeding the memory for weights themselves.

\textbf{Gradient checkpointing} addresses this by not caching all
activations. Instead, you cache every \(k\)th layer's activations and
recompute the intermediate ones during the backward pass. This trades
compute (recomputing activations) for memory (not storing them). For a
100-layer network with checkpointing every 10 layers, you store only 10
activations instead of 100, reducing activation memory by 10x at the
cost of roughly 33\% more forward-pass compute.

\subsection{Automatic
Differentiation}\label{sec-mathematical-foundations-automatic-differentiation-74a5}

Modern frameworks use \textbf{reverse-mode automatic differentiation},
which computes gradients for all \(N\) parameters in a single backward
pass. This is why training (which needs gradients) has similar compute
cost to inference (which does not): one forward pass plus one backward
pass, where the backward pass has roughly the same cost as the forward
pass.

Forward-mode autodiff, by contrast, would require \(N\) passes to
compute all gradients. This would make training prohibitively expensive.

\begin{center}\rule{0.5\linewidth}{0.5pt}\end{center}

\section{Sparse Matrix
Formats}\label{sec-mathematical-foundations-sparse-matrix-formats-7bd3}

\begin{tcolorbox}[enhanced jigsaw, colbacktitle=quarto-callout-tip-color!10!white, left=2mm, coltitle=black, toprule=.15mm, rightrule=.15mm, colback=white, breakable, colframe=quarto-callout-tip-color-frame, bottomtitle=1mm, opacityback=0, bottomrule=.15mm, title=\textcolor{quarto-callout-tip-color}{\faLightbulb}\hspace{0.5em}{Why This Matters}, leftrule=.75mm, arc=.35mm, titlerule=0mm, opacitybacktitle=0.6, toptitle=1mm]

Recommendation systems and pruned models contain mostly zeros. Storing
and computing with dense matrices wastes both memory and compute.
Understanding sparse formats lets you work efficiently with these common
workloads.

\end{tcolorbox}

When most elements in a matrix are zero, specialized storage formats
avoid wasting memory on zeros and enable computations that skip them
entirely.

The \textbf{Compressed Sparse Row (CSR)} format uses three arrays:

\begin{itemize}
\tightlist
\item
  \texttt{Values}: The non-zero elements, stored in row order
\item
  \texttt{Col\_Idx}: The column index of each non-zero element
\item
  \texttt{Row\_Ptr}: The starting position in \texttt{Values} for each
  row (length = num\_rows + 1)
\end{itemize}

\begin{figure}[htb]

\centering{

\pandocbounded{\includegraphics[keepaspectratio]{index_files/mediabag/f8fe621f6fe14f083dc819eb55687063572231ca.pdf}}

}

\caption{\label{fig-csr-format}\textbf{Compressed Sparse Row (CSR)}: A
\(4 \times 4\) matrix with only 4 non-zero values. CSR stores these 4
values plus indexing overhead, rather than all 16 elements.}

\end{figure}%

\textbf{How to read a CSR representation.} Figure~\ref{fig-csr-format}
shows a \(4 \times 4\) sparse matrix and its CSR encoding. The skill you
need is reconstructing the matrix from the three arrays. Work through
this step by step:

\textbf{Finding all values in a specific row.} Suppose you want row 0.
Look at \texttt{Row\_Ptr{[}0{]}\ =\ 0} and
\texttt{Row\_Ptr{[}1{]}\ =\ 2}. This tells you row 0's non-zeros are at
indices 0 through 1 (inclusive) in the \texttt{Values} array. Check
\texttt{Values{[}0:2{]}\ =\ {[}5,\ 8{]}} and
\texttt{Col\_Idx{[}0:2{]}\ =\ {[}0,\ 2{]}}. So row 0 has value 5 in
column 0 and value 8 in column 2.

\textbf{Finding a specific element.} To find the element at row 1,
column 1: First, \texttt{Row\_Ptr{[}1{]}\ =\ 2} and
\texttt{Row\_Ptr{[}2{]}\ =\ 3}, so row 1 has one non-zero at index 2 in
\texttt{Values}. Check \texttt{Col\_Idx{[}2{]}\ =\ 1}. Since we are
looking for column 1 and this matches, the value is
\texttt{Values{[}2{]}\ =\ 3}. If the column did not appear in
\texttt{Col\_Idx} for this row, the element would be zero.

\textbf{Detecting empty rows.} Look at row 2:
\texttt{Row\_Ptr{[}2{]}\ =\ 3} and \texttt{Row\_Ptr{[}3{]}\ =\ 3}. Since
these are equal, row 2 has no non-zeros. The entire row is zeros.

\textbf{Why this matters for ML.} Recommendation systems use embedding
tables where each user or item maps to a sparse high-dimensional vector.
A typical embedding table might have millions of users but each user
only interacts with hundreds of items. Storing this as a dense matrix
wastes 99.99\% of memory on zeros. CSR stores only the non-zero
interactions.

Pruned neural networks exhibit similar sparsity. After pruning 90\% of
weights, you have 10x fewer values to store and 10x fewer
multiplications to perform, but only if your storage format and compute
kernels exploit sparsity. Dense GEMM on a 90\%-sparse matrix does all
the work for zeros that contribute nothing to the result.

For a matrix with \(N\) total elements and \(K\) non-zeros, CSR uses
\(O(K)\) storage instead of \(O(N)\). The crossover where CSR becomes
more efficient than dense storage depends on the sparsity pattern, but
typically occurs around 90\% sparsity.

\begin{center}\rule{0.5\linewidth}{0.5pt}\end{center}

\section{Computational Graphs and
Optimization}\label{sec-mathematical-foundations-computational-graphs-optimization-bade}

\begin{tcolorbox}[enhanced jigsaw, colbacktitle=quarto-callout-tip-color!10!white, left=2mm, coltitle=black, toprule=.15mm, rightrule=.15mm, colback=white, breakable, colframe=quarto-callout-tip-color-frame, bottomtitle=1mm, opacityback=0, bottomrule=.15mm, title=\textcolor{quarto-callout-tip-color}{\faLightbulb}\hspace{0.5em}{Why This Matters}, leftrule=.75mm, arc=.35mm, titlerule=0mm, opacitybacktitle=0.6, toptitle=1mm]

When you call \texttt{torch.compile()} or use TensorRT, the framework
transforms your model into an optimized computational graph.
Understanding what these transformations do helps you write code that
compilers can optimize effectively and debug cases where optimization
fails.

\end{tcolorbox}

ML compilers represent models as directed acyclic graphs (DAGs) where
nodes are operations and edges are data dependencies. This
representation enables hardware-independent optimizations.

The most impactful optimization is \textbf{operator fusion}. Consider a
sequence like Conv followed by BatchNorm followed by ReLU. Without
fusion, each operation reads its input from memory, computes, and writes
its output back to memory. With fusion, all three operations execute as
a single kernel: read input once, compute all three operations in
registers, write final output once.

This fusion eliminates two round trips to HBM memory. Given the memory
wall discussed earlier, the speedup from fusion often exceeds 2x for
memory-bound operation sequences.

\textbf{Static Single Assignment (SSA)} form simplifies the dataflow
analysis needed to identify fusion opportunities. In SSA, each variable
is assigned exactly once, making data dependencies explicit and enabling
safe code transformations.

\begin{center}\rule{0.5\linewidth}{0.5pt}\end{center}

\section*{Summary}\label{summary}
\addcontentsline{toc}{section}{Summary}

\markright{Summary}

The concepts in this appendix share a common theme:
\textbf{understanding the relationship between computation and data
movement}. The Roofline Model quantifies this relationship. The memory
hierarchy explains why it matters. Numerical formats and sparse
representations reduce data movement. GEMM and computational graphs are
the contexts where these trade-offs play out.

When you encounter a slow ML system, reach for these tools. Is the
workload memory-bound or compute-bound? Where does the data live, and
how often must it move? What precision does the application actually
require? Answering these questions systematically, rather than guessing,
is what separates effective ML systems engineering from trial and error.

\phantomsection\label{refs}
\begin{CSLReferences}{1}{0}
\bibitem[\citeproctext]{ref-amdahl1967validity}
Amdahl, Gene M. 1967. {``Validity of the Single Processor Approach to
Achieving Large Scale Computing Capabilities.''} In \emph{Proceedings of
the April 18-20, 1967, Spring Joint Computer Conference}, 483--85. AFIPS
'67 (Spring). New York, NY, USA: ACM.
\url{https://doi.org/10.1145/1465482.1465560}.

\bibitem[\citeproctext]{ref-gustafson1988reevaluating}
Gustafson, John L. 1988. {``Reevaluating {Amdahl's} Law.''}
\emph{Communications of the ACM} 31 (5): 532--33.
\url{https://doi.org/10.1145/42411.42415}.

\bibitem[\citeproctext]{ref-little1961proof}
Little, John D. C. 1961. {``A Proof for the Queuing Formula:
{\(L = \lambda W\)}.''} \emph{Operations Research} 9 (3): 383--87.
\url{https://doi.org/10.1287/opre.9.3.383}.

\bibitem[\citeproctext]{ref-williams2009roofline}
Williams, Samuel, Andrew Waterman, and David Patterson. 2009.
{``Roofline: An Insightful Visual Performance Model for Multicore
Architectures.''} \emph{Communications of the ACM} 52 (4): 65--76.
\url{https://doi.org/10.1145/1498765.1498785}.

\end{CSLReferences}


\backmatter


\end{document}
