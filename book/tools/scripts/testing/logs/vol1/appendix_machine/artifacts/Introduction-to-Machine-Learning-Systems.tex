% Options for packages loaded elsewhere
% Options for packages loaded elsewhere
\PassOptionsToPackage{unicode,linktoc=all,pdfwindowui,pdfpagemode=FullScreen,pdfpagelayout=TwoPageRight}{hyperref}
\PassOptionsToPackage{hyphens}{url}
\PassOptionsToPackage{dvipsnames,svgnames,x11names}{xcolor}
%
\documentclass[
  9pt,
  letterpaper,
  abstract,
  titlepage]{scrbook}
\usepackage{xcolor}
\usepackage{amsmath,amssymb}
\setcounter{secnumdepth}{3}
\usepackage{iftex}
\ifPDFTeX
  \usepackage[T1]{fontenc}
  \usepackage[utf8]{inputenc}
  \usepackage{textcomp} % provide euro and other symbols
\else % if luatex or xetex
  \usepackage{unicode-math} % this also loads fontspec
  \defaultfontfeatures{Scale=MatchLowercase}
  \defaultfontfeatures[\rmfamily]{Ligatures=TeX,Scale=1}
\fi
\usepackage{lmodern}
\ifPDFTeX\else
  % xetex/luatex font selection
\fi
% Use upquote if available, for straight quotes in verbatim environments
\IfFileExists{upquote.sty}{\usepackage{upquote}}{}
\IfFileExists{microtype.sty}{% use microtype if available
  \usepackage[]{microtype}
  \UseMicrotypeSet[protrusion]{basicmath} % disable protrusion for tt fonts
}{}
% Make \paragraph and \subparagraph free-standing
\makeatletter
\ifx\paragraph\undefined\else
  \let\oldparagraph\paragraph
  \renewcommand{\paragraph}{
    \@ifstar
      \xxxParagraphStar
      \xxxParagraphNoStar
  }
  \newcommand{\xxxParagraphStar}[1]{\oldparagraph*{#1}\mbox{}}
  \newcommand{\xxxParagraphNoStar}[1]{\oldparagraph{#1}\mbox{}}
\fi
\ifx\subparagraph\undefined\else
  \let\oldsubparagraph\subparagraph
  \renewcommand{\subparagraph}{
    \@ifstar
      \xxxSubParagraphStar
      \xxxSubParagraphNoStar
  }
  \newcommand{\xxxSubParagraphStar}[1]{\oldsubparagraph*{#1}\mbox{}}
  \newcommand{\xxxSubParagraphNoStar}[1]{\oldsubparagraph{#1}\mbox{}}
\fi
\makeatother


\providecommand{\tightlist}{%
  \setlength{\itemsep}{0pt}\setlength{\parskip}{0pt}}\usepackage{longtable,booktabs,array}
\usepackage{calc} % for calculating minipage widths
% Correct order of tables after \paragraph or \subparagraph
\usepackage{etoolbox}
\makeatletter
\patchcmd\longtable{\par}{\if@noskipsec\mbox{}\fi\par}{}{}
\makeatother
% Allow footnotes in longtable head/foot
\IfFileExists{footnotehyper.sty}{\usepackage{footnotehyper}}{\usepackage{footnote}}
\makesavenoteenv{longtable}
\usepackage{graphicx}
\makeatletter
\newsavebox\pandoc@box
\newcommand*\pandocbounded[1]{% scales image to fit in text height/width
  \sbox\pandoc@box{#1}%
  \Gscale@div\@tempa{\textheight}{\dimexpr\ht\pandoc@box+\dp\pandoc@box\relax}%
  \Gscale@div\@tempb{\linewidth}{\wd\pandoc@box}%
  \ifdim\@tempb\p@<\@tempa\p@\let\@tempa\@tempb\fi% select the smaller of both
  \ifdim\@tempa\p@<\p@\scalebox{\@tempa}{\usebox\pandoc@box}%
  \else\usebox{\pandoc@box}%
  \fi%
}
% Set default figure placement to htbp
\def\fps@figure{htbp}
\makeatother
% definitions for citeproc citations
\NewDocumentCommand\citeproctext{}{}
\NewDocumentCommand\citeproc{mm}{%
  \begingroup\def\citeproctext{#2}\cite{#1}\endgroup}
\makeatletter
 % allow citations to break across lines
 \let\@cite@ofmt\@firstofone
 % avoid brackets around text for \cite:
 \def\@biblabel#1{}
 \def\@cite#1#2{{#1\if@tempswa , #2\fi}}
\makeatother
\newlength{\cslhangindent}
\setlength{\cslhangindent}{1.5em}
\newlength{\csllabelwidth}
\setlength{\csllabelwidth}{3em}
\newenvironment{CSLReferences}[2] % #1 hanging-indent, #2 entry-spacing
 {\begin{list}{}{%
  \setlength{\itemindent}{0pt}
  \setlength{\leftmargin}{0pt}
  \setlength{\parsep}{0pt}
  % turn on hanging indent if param 1 is 1
  \ifodd #1
   \setlength{\leftmargin}{\cslhangindent}
   \setlength{\itemindent}{-1\cslhangindent}
  \fi
  % set entry spacing
  \setlength{\itemsep}{#2\baselineskip}}}
 {\end{list}}
\usepackage{calc}
\newcommand{\CSLBlock}[1]{\hfill\break\parbox[t]{\linewidth}{\strut\ignorespaces#1\strut}}
\newcommand{\CSLLeftMargin}[1]{\parbox[t]{\csllabelwidth}{\strut#1\strut}}
\newcommand{\CSLRightInline}[1]{\parbox[t]{\linewidth - \csllabelwidth}{\strut#1\strut}}
\newcommand{\CSLIndent}[1]{\hspace{\cslhangindent}#1}

% =============================================================================
% LATEX HEADER CONFIGURATION FOR MLSYSBOOK PDF
% =============================================================================
% This file contains all LaTeX package imports, custom commands, and styling
% definitions for the PDF output of the Machine Learning Systems textbook.
%
% Key Features:
% - Harvard crimson branding throughout
% - Custom part/chapter/section styling
% - Professional table formatting with colored headers
% - Margin notes with custom styling
% - TikZ-based part dividers
% - Page numbering (Roman for frontmatter, Arabic for mainmatter)
%
% Note: This file is included via _quarto-pdf.yml and affects PDF output only.
% HTML/EPUB styling is handled separately via CSS files.
% =============================================================================

% =============================================================================
% PACKAGE IMPORTS
% =============================================================================

% Layout and positioning
% \usepackage[outercaption, ragged]{sidecap}  % Commented out to make figure captions inline instead of in margin
\usepackage{adjustbox}      % Adjusting box dimensions
\usepackage{afterpage}      % Execute commands after page break
\usepackage{morefloats}     % Increase number of floats
\usepackage{array}          % Enhanced table column formatting
\usepackage{atbegshi}       % Insert content at page beginning
%\usepackage{changepage}     % Change page dimensions mid-document
\usepackage{emptypage}      % Clear headers/footers on empty pages

% Language and text
\usepackage[english]{babel} % English language support
\usepackage{microtype}      % Improved typography and hyphenation

% Captions and floats
\usepackage{caption}
% Caption styling configuration
%\captionsetup[table]{belowskip=5pt}
\captionsetup{format=plain}
\DeclareCaptionLabelFormat{mylabel}{#1
#2:\hspace{1.0ex}}
\DeclareCaptionFont{ninept}{\fontsize{7pt}{8}\selectfont #1}

% Figure captions: Small font, bold label, ragged right
\captionsetup[figure]{labelfont={bf,ninept},labelsep=space,
belowskip=2pt,aboveskip=6pt,labelformat=mylabel,
justification=raggedright,singlelinecheck=false,font={ninept}}

% Table captions: Small font, bold label, ragged right
\captionsetup[table]{belowskip=6pt,labelfont={bf,ninept},labelsep=none,
labelformat=mylabel,justification=raggedright,singlelinecheck=false,font={ninept}}

% Typography fine-tuning
\emergencystretch=5pt       % Allow extra stretch to avoid overfull boxes

% Utility packages
\usepackage{etoolbox}       % For patching commands and environments

% Page layout and headers
\usepackage{fancyhdr}       % Custom headers and footers
\usepackage{geometry}       % Page dimensions and margins

% Graphics and figures
\usepackage{graphicx}       % Include graphics
\usepackage{float}          % Improved float placement
\usepackage[skins,breakable]{tcolorbox} % Coloured and framed text boxes
\tcbset{before upper=\setlength{\parskip}{3pt}}

% Tables
\usepackage{longtable}      % Multi-page tables

% Fonts and typography
\usepackage{fontspec}       % Font selection for LuaLaTeX
\usepackage{mathptmx}       % Times-like math fonts
\usepackage{newpxtext}      % Palatino-like font for body text

% Colors and visual elements
\usepackage[dvipsnames]{xcolor}  % Extended color support
\usepackage{tikz}           % Programmatic graphics
\usetikzlibrary{positioning}
\usetikzlibrary{calc}
\usepackage{tikzpagenodes}  % TikZ positioning relative to page

% Code listings
\usepackage{listings}       % Code highlighting

% Hyperlinks
\usepackage{hyperref}       % Clickable links in PDF

% Conditional logic
\usepackage{ifthen}         % If-then-else commands

% Math symbols
\usepackage{amsmath}        % AMS math extensions
\usepackage{amssymb}        % AMS math symbols
\usepackage{latexsym}       % Additional LaTeX symbols
\usepackage{pifont}         % Zapf Dingbats symbols
\providecommand{\blacklozenge}{\ding{117}}  % Black diamond symbol

% Lists
\usepackage{enumitem}       % Customizable lists

% Margin notes and sidenotes
\usepackage{marginfix}      % Fixes margin note overflow
\usepackage{marginnote}     % Margin notes
\usepackage{sidenotes}      % Academic-style sidenotes
\renewcommand\raggedrightmarginnote{\sloppy}
\renewcommand\raggedleftmarginnote{\sloppy}

% Typography improvements
\usepackage{ragged2e}       % Better ragged text
\usepackage[all]{nowidow}   % Prevent widows and orphans
\usepackage{needspace}      % Ensure minimum space on page

% Section formatting
\usepackage[explicit]{titlesec}  % Custom section titles
\usepackage{tocloft}        % Table of contents formatting

% QR codes and icons
\usepackage{fontawesome5}   % Font Awesome icons
\usepackage{qrcode}         % QR code generation
\qrset{link, height=15mm}

% =============================================================================
% FLOAT CONFIGURATION
% =============================================================================
% Allow more floats per page to handle figure-heavy chapters
\extrafloats{200}
\setcounter{topnumber}{12}       % Max floats at top of page
\setcounter{bottomnumber}{12}    % Max floats at bottom of page
\setcounter{totalnumber}{24}     % Max floats per page
\setcounter{dbltopnumber}{8}     % Max floats at top of two-column page
\renewcommand{\topfraction}{.95}  % Max fraction of page for top floats
\renewcommand{\bottomfraction}{.95}
\renewcommand{\textfraction}{.05}  % Min fraction of page for text
\renewcommand{\floatpagefraction}{.7}  % Min fraction of float page
\renewcommand{\dbltopfraction}{.95}

% Prevent "Float(s) lost" errors by flushing floats more aggressively
\usepackage{placeins}  % Provides \FloatBarrier

% =============================================================================
% COLOR DEFINITIONS
% =============================================================================
% Harvard crimson - primary brand color used throughout
\definecolor{crimson}{HTML}{A51C30}

% Quiz element colors
\definecolor{quiz-question-color1}{RGB}{225,243,248}  % Light blue background
\definecolor{quiz-question-color2}{RGB}{17,158,199}   % Blue border
\definecolor{quiz-answer-color1}{RGB}{250,234,241}    % Light pink background
\definecolor{quiz-answer-color2}{RGB}{152,14,90}      % Magenta border

% =============================================================================
% LIST FORMATTING
% =============================================================================
% Tighter list spacing for academic style
\def\tightlist{}
\setlist{itemsep=1pt, parsep=1pt, topsep=0pt,after={\vspace{0.3\baselineskip}}}
\let\tightlist\relax

\makeatletter
\@ifpackageloaded{framed}{}{\usepackage{framed}}
\@ifpackageloaded{fancyvrb}{}{\usepackage{fancyvrb}}
\makeatother

\makeatletter
%New float "codelisting" has been updated
\AtBeginDocument{%
\floatstyle{ruled}
\newfloat{codelisting}{!htb}{lop}
\floatname{codelisting}{Listing}
\floatplacement{codelisting}{!htb}
\captionsetup[codelisting]{labelfont={bf,ninept},labelformat=mylabel,
  singlelinecheck=false,width=\linewidth,labelsep=none,font={ninept}}%
\renewenvironment{snugshade}{%
   \def\OuterFrameSep{3pt}%
   \def\FrameCommand{\fboxsep=5pt\colorbox{shadecolor}}%
   \MakeFramed{\advance\hsize-\width\FrameRestore}%
   \leftskip 0.5em \rightskip 0.5em%
   \small% decrease font size
   }{\endMakeFramed}%
}
\makeatother

%The space before and after the verbatim environment "Highlighting" has been reduced
\fvset{listparameters=\setlength{\topsep}{0pt}\setlength{\partopsep}{0pt}}
\DefineVerbatimEnvironment{Highlighting}{Verbatim}{framesep=0mm,commandchars=\\\{\}}

\makeatletter
\renewcommand\fs@ruled{\def\@fs@cfont{\bfseries}\let\@fs@capt\floatc@ruled
\def\@fs@pre{\hrule height.8pt depth0pt \kern2pt}%
\def\@fs@post{\kern2pt\hrule\relax}%
\def\@fs@mid{\kern2pt\hrule\kern1pt}%space between float and caption
\let\@fs@iftopcapt\iftrue}
\makeatother


% =============================================================================
% HYPHENATION RULES
% =============================================================================
% Explicit hyphenation points for technical terms to avoid bad breaks
\hyphenation{
  light-weight
  light-weight-ed
  de-vel-op-ment
  un-der-stand-ing
  mod-els
  prin-ci-ples
  ex-per-tise
  com-pli-cat-ed
  blue-print
  per‧for‧mance
  com-mu-ni-ca-tion
  par-a-digms
  hy-per-ten-sion
  a-chieved
}

% =============================================================================
% CODE LISTING CONFIGURATION
% =============================================================================
% Settings for code blocks using listings package
\lstset{
breaklines=true,              % Automatic line wrapping
breakatwhitespace=true,       % Break at whitespace only
basicstyle=\ttfamily,         % Monospace font
frame=none,                   % No frame around code
keepspaces=true,              % Preserve spaces
showspaces=false,             % Don't show space characters
showtabs=false,               % Don't show tab characters
columns=flexible,             % Flexible column width
belowskip=0pt,               % Minimal spacing
aboveskip=0pt
}

% =============================================================================
% PAGE GEOMETRY
% =============================================================================
% MIT Press trim size: 7" x 10" (per publisher specifications)
% This is a standard academic textbook format providing good readability
% for technical content with figures and code blocks.
% Wide outer margin accommodates sidenotes/margin notes.
\geometry{
  paperwidth=7in,
  paperheight=10in,
  top=0.875in,
  bottom=0.875in,
  inner=0.875in,              % Inner margin (binding side)
  outer=1.75in,               % Outer margin (includes space for sidenotes)
  footskip=30pt,
  marginparwidth=1.25in,      % Width for margin notes
  twoside                     % Different left/right pages
}

% =============================================================================
% SIDENOTE STYLING
% =============================================================================
% Custom sidenote design with crimson vertical bar
\renewcommand{\thefootnote}{\textcolor{crimson}{\arabic{footnote}}}

% Save original sidenote command
\makeatletter
\@ifundefined{oldsidenote}{
  \let\oldsidenote\sidenote%
}{}
\makeatother

% Redefine sidenote with vertical crimson bar
\renewcommand{\sidenote}[1]{%
  \oldsidenote{%
    \noindent
    \color{crimson!100}                        % Crimson vertical line
    \raisebox{0em}{%
      \rule{0.5pt}{1.5em}                      % Thin vertical line
    }
    \hspace{0.3em}                             % Space after line
    \color{black}                              % Reset text color
    \footnotesize #1                           % Sidenote content
  }%
}

% =============================================================================
% FLOAT HANDLING
% =============================================================================
% Patch LaTeX's output routine to handle float overflow gracefully
% The "Float(s) lost" error occurs in \@doclearpage when \@currlist is not empty
% This patch silently clears pending floats that can't be placed
\makeatletter
\let\orig@doclearpage\@doclearpage
\def\@doclearpage{%
  \ifx\@currlist\@empty\else
    \global\let\@currlist\@empty
    \typeout{Warning: Floats cleared to prevent overflow}%
  \fi
  \orig@doclearpage
}
\makeatother

% Additional safety for structural commands
\let\originalbackmatter\backmatter
\renewcommand{\backmatter}{%
  \clearpage%
  \originalbackmatter%
}

\let\originalfrontmatter\frontmatter
\renewcommand{\frontmatter}{%
  \clearpage%
  \originalfrontmatter%
}

\let\originalmainmatter\mainmatter
\renewcommand{\mainmatter}{%
  \clearpage%
  \originalmainmatter%
}

% =============================================================================
% PAGE HEADERS AND FOOTERS
% =============================================================================
% Ensure chapters use fancy page style (not plain)
\patchcmd{\chapter}{\thispagestyle{plain}}{\thispagestyle{fancy}}{}{}

% Main page style with crimson headers
\pagestyle{fancy}
\fancyhf{}                                              % Clear all
\fancyhead[LE]{\small\color{crimson}\nouppercase{\rightmark}}  % Left even: section
\fancyhead[RO]{\color{crimson}\thepage}                 % Right odd: page number
\fancyhead[LO]{\small\color{crimson}\nouppercase{\leftmark}}   % Left odd: chapter
\fancyhead[RE]{\color{crimson}\thepage}                 % Right even: page number
\renewcommand{\headrulewidth}{0.4pt}                    % Thin header line
\renewcommand{\footrulewidth}{0pt}                      % No footer line

% Plain page style (for chapter openings)
\fancypagestyle{plain}{
  \fancyhf{}
  \fancyfoot[C]{\color{crimson}\thepage}                % Centered page number
  \renewcommand{\headrulewidth}{0pt}
  \renewcommand{\footrulewidth}{0pt}
}

% =============================================================================
% KOMA-SCRIPT FONT ADJUSTMENTS
% =============================================================================
% Apply crimson color to all heading levels
\addtokomafont{disposition}{\rmfamily\color{crimson}}
\addtokomafont{chapter}{\color{crimson}}
\addtokomafont{section}{\color{crimson}}
\addtokomafont{subsection}{\color{crimson}}

% =============================================================================
% ABSTRACT ENVIRONMENT
% =============================================================================
\newenvironment{abstract}{
  \chapter*{\abstractname}
  \addcontentsline{toc}{chapter}{\abstractname}
  \small
}{
  \clearpage
}

% =============================================================================
% HYPERLINK CONFIGURATION
% =============================================================================
% Crimson-colored links throughout, two-page PDF layout
\hypersetup{
  linkcolor=crimson,
  citecolor=crimson,
  urlcolor=crimson,
  pdfpagelayout=TwoPageRight,   % Two-page spread view
  pdfstartview=Fit               % Initial zoom fits page
}

% =============================================================================
% PART SUMMARY SYSTEM
% =============================================================================
% Allows adding descriptive text below part titles
\newcommand{\partsummary}{}     % Empty by default
\newif\ifhaspartsummary%
\haspartsummaryfalse%

\newcommand{\setpartsummary}[1]{%
  \renewcommand{\partsummary}{#1}%
  \haspartsummarytrue%
}

% Additional colors for part page backgrounds
\definecolor{BrownLL}{RGB}{233,222,220}
\definecolor{BlueDD}{RGB}{62,100,125}
\colorlet{BlueDD}{magenta}

% ===============================================================================
% PART STYLING SYSTEM
% ===============================================================================
%
% This system provides three distinct visual styles for book organization:
%
% 1. NUMBERED PARTS (\part{title}) - For main book sections
%    - Roman numerals (I, II, III, etc.) in top right corner
%    - Crimson title with horizontal lines above/below
%    - "Part I" label in sidebar
%    - Used for: foundations, principles, optimization, deployment, etc.
%
% 2. UNNUMBERED PARTS (\part*{title}) - For special sections like "Labs"
%    - Division-style geometric background (left side)
%    - No Roman numerals
%    - Used for: labs section
%
% 3. DIVISIONS (\division{title}) - For major book divisions
%    - Clean geometric background with centered title
%    - Used for: frontmatter, main_content, backmatter
%
% The Lua filter (inject-parts.lua) automatically routes parts by {key:xxx} commands
% to the appropriate LaTeX command based on the key name.
% ===============================================================================

% NUMBERED PARTS: Roman numeral styling for main book sections
\titleformat{\part}[display]
{\thispagestyle{empty}}{}{20pt}{
\begin{tikzpicture}[remember picture,overlay]
%%%
%%
\node[crimson,align=flush right,
inner sep=0,outer sep=0mm,draw=none,%
anchor=east,minimum height=31mm, text width=1.2\textwidth,
yshift=-30mm,font={%
\fontsize{98pt}{104}\selectfont\bfseries}]  (BG) at (current page text area.north east){\thepart};
%
\node[black,inner sep=0mm,draw=none,
anchor=mid,text width=1.2\textwidth,
 minimum height=35mm, align=right,
node distance=7mm,below=of BG,
font={\fontsize{30pt}{34}\selectfont}]
(BGG)  {\hyphenchar\font=-1 \color{black}\MakeUppercase {#1}};
\draw [crimson,line width=3pt] ([yshift=0mm]BGG.north west) -- ([yshift=0mm]BGG.north east);
\draw [crimson,line width=2pt] ([yshift=0mm]BGG.south west) -- ([yshift=0mm]BGG.south east);
%
\node[fill=crimson,text=white,rotate=90,%
anchor=south west,minimum height=15mm,
minimum width=40mm,font={%
\fontsize{20pt}{20}\selectfont\bfseries}](BP)  at
(current page text area.south east)
{{\sffamily Part}~\thepart};
%
\path[red](BP.north west)-|coordinate(PS)(BGG.south west);
%
% Part summary box commented out for cleaner design
% \ifhaspartsummary
% \node[inner sep=4pt,text width=0.7\textwidth,draw=none,fill=BrownLL!40,
% align=justify,font={\fontsize{9pt}{12}\selectfont},anchor=south west]
% at (PS) {\partsummary};
% \fi
\end{tikzpicture}
}[]

\renewcommand{\thepart}{\Roman{part}}

% UNNUMBERED PARTS: Division-style background for special sections
\titleformat{name=\part,numberless}[display]
{\thispagestyle{empty}}{}{20pt}{
\begin{tikzpicture}[remember picture,overlay]
%%%
\coordinate(S1)at([yshift=-200mm]current page.north west);
\draw[draw=none,fill=BlueDD!7](S1)--++(45:16)coordinate(S2)-
|(S2|-current page.north west)--(current page.north west)coordinate(S3)--(S1);
%
\coordinate(E1)at([yshift=-98mm]current page.north west);
\draw[draw=none,fill=BlueDD!15](E1)--(current page.north west)coordinate(E2)
--++(0:98mm)coordinate(E3)--(E1);
%
\coordinate(D1)at([yshift=15mm]current page.south west);
\draw[draw=none,fill=BlueDD!40,opacity=0.5](D1)--++(45:5.5)coordinate(D2)
-|(D2|-current page.north west)--(current page.north west)coordinate(D3)--(D1);
%%%%
\path[red](S2)-|(S2-|current page.east)coordinate(SS2);
%PART
\node[crimson,align=flush right,inner sep=0,outer sep=0mm,draw=none,anchor=south,
font={\fontsize{48pt}{48}\selectfont\bfseries}]  (BG) at ($(S2)!0.5!(SS2)$){\hphantom{Part}};
%%%
\path[green]([yshift=15mm]D2)-|coordinate(TPD)(BG.south east);
\node[inner sep=0mm,draw=none,anchor=south east,%text width=0.9\textwidth,
align=right,font={\fontsize{40pt}{40}\selectfont}]
(BGG) at (TPD)  {\color{crimson}\MakeUppercase {#1}};%\MakeUppercase {}
\end{tikzpicture}
}

% Define \numberedpart command for numbered parts
\newcommand{\numberedpart}[1]{%
\FloatBarrier%  % Flush all pending floats before part break
\clearpage
\thispagestyle{empty}
\stepcounter{part}%
\begin{tikzpicture}[remember picture,overlay]
%%%
%%
\node[crimson,align=flush right,
inner sep=0,outer sep=0mm,draw=none,%
anchor=east,minimum height=31mm, text width=1.2\textwidth,
yshift=-30mm,font={%
\fontsize{98pt}{104}\selectfont\bfseries}]  (BG) at (current page text area.north east){\thepart};
%
\node[black,inner sep=0mm,draw=none,
anchor=mid,text width=1.2\textwidth,
 minimum height=35mm, align=right,
node distance=7mm,below=of BG,
font={\fontsize{30pt}{34}\selectfont}]
(BGG)  {\hyphenchar\font=-1 \color{black}\MakeUppercase {#1}};
\draw [crimson,line width=3pt] ([yshift=0mm]BGG.north west) -- ([yshift=0mm]BGG.north east);
\draw [crimson,line width=2pt] ([yshift=0mm]BGG.south west) -- ([yshift=0mm]BGG.south east);
%
\node[fill=crimson,text=white,rotate=90,%
anchor=south west,minimum height=15mm,
minimum width=40mm,font={%
\fontsize{20pt}{20}\selectfont\bfseries}](BP)  at
(current page text area.south east)
{{\sffamily Part}~\thepart};
%
\path[red](BP.north west)-|coordinate(PS)(BGG.south west);
%
% Part summary box commented out for cleaner design
% \ifhaspartsummary
% \node[inner sep=4pt,text width=0.7\textwidth,draw=none,fill=BrownLL!40,
% align=justify,font={\fontsize{9pt}{12}\selectfont},anchor=south west]
% at (PS) {\partsummary};
% \fi
\end{tikzpicture}
\clearpage
}



% DIVISIONS: Clean geometric styling with subtle tech elements
% Used for frontmatter, main_content, and backmatter divisions
\newcommand{\division}[1]{%
\FloatBarrier%  % Flush all pending floats before division break
\clearpage
\thispagestyle{empty}
\begin{tikzpicture}[remember picture,overlay]

% Clean geometric background (original design)
\coordinate(S1)at([yshift=-200mm]current page.north west);
\draw[draw=none,fill=BlueDD!7](S1)--++(45:16)coordinate(S2)-
|(S2|-current page.north west)--(current page.north west)coordinate(S3)--(S1);

\coordinate(E1)at([yshift=-98mm]current page.north west);
\draw[draw=none,fill=BlueDD!15](E1)--(current page.north west)coordinate(E2)
--++(0:98mm)coordinate(E3)--(E1);

\coordinate(D1)at([yshift=15mm]current page.south west);
\draw[draw=none,fill=BlueDD!40,opacity=0.5](D1)--++(45:5.5)coordinate(D2)
-|(D2|-current page.north west)--(current page.north west)coordinate(D3)--(D1);

% Subtle tech elements - positioned in white areas for better visibility
% Upper right white area - more visible
\draw[crimson!40, line width=0.8pt] ([xshift=140mm,yshift=-60mm]current page.north west) -- ++(40mm,0);
\draw[crimson!40, line width=0.8pt] ([xshift=150mm,yshift=-70mm]current page.north west) -- ++(30mm,0);
\draw[crimson!35, line width=0.7pt] ([xshift=160mm,yshift=-60mm]current page.north west) -- ++(0,-15mm);
\draw[crimson!35, line width=0.7pt] ([xshift=170mm,yshift=-70mm]current page.north west) -- ++(0,10mm);

% Circuit nodes - upper right
\fill[crimson!50] ([xshift=160mm,yshift=-60mm]current page.north west) circle (1.5mm);
\fill[white] ([xshift=160mm,yshift=-60mm]current page.north west) circle (0.8mm);
\fill[crimson!50] ([xshift=170mm,yshift=-70mm]current page.north west) circle (1.3mm);
\fill[white] ([xshift=170mm,yshift=-70mm]current page.north west) circle (0.6mm);

% Lower right white area - enhanced visibility
\draw[crimson!45, line width=0.9pt] ([xshift=140mm,yshift=-190mm]current page.north west) -- ++(45mm,0);
\draw[crimson!45, line width=0.9pt] ([xshift=150mm,yshift=-200mm]current page.north west) -- ++(35mm,0);
\draw[crimson!40, line width=0.8pt] ([xshift=160mm,yshift=-190mm]current page.north west) -- ++(0,-20mm);
\draw[crimson!40, line width=0.8pt] ([xshift=170mm,yshift=-200mm]current page.north west) -- ++(0,15mm);

% Additional connecting lines in lower right
\draw[crimson!35, line width=0.7pt] ([xshift=130mm,yshift=-180mm]current page.north west) -- ++(25mm,0);
\draw[crimson!35, line width=0.7pt] ([xshift=145mm,yshift=-180mm]current page.north west) -- ++(0,-25mm);

% Circuit nodes - lower right (more prominent)
\fill[crimson!55] ([xshift=160mm,yshift=-190mm]current page.north west) circle (1.6mm);
\fill[white] ([xshift=160mm,yshift=-190mm]current page.north west) circle (0.9mm);
\fill[crimson!55] ([xshift=170mm,yshift=-200mm]current page.north west) circle (1.4mm);
\fill[white] ([xshift=170mm,yshift=-200mm]current page.north west) circle (0.7mm);
\fill[crimson!50] ([xshift=145mm,yshift=-180mm]current page.north west) circle (1.2mm);
\fill[white] ([xshift=145mm,yshift=-180mm]current page.north west) circle (0.6mm);

% Title positioned in center - clean and readable
\node[inner sep=0mm,draw=none,anchor=center,text width=0.8\textwidth,
align=center,font={\fontsize{40pt}{40}\selectfont}]
(BGG) at (current page.center)  {\color{crimson}\MakeUppercase {#1}};

\end{tikzpicture}
\clearpage
}

% LAB DIVISIONS: Circuit-style neural network design for lab sections
% Used specifically for lab platform sections (arduino, xiao, grove, etc.)
\newcommand{\labdivision}[1]{%
\FloatBarrier%  % Flush all pending floats before lab division break
\clearpage
\thispagestyle{empty}
\begin{tikzpicture}[remember picture,overlay]
% Circuit background with subtle gradient
\coordinate(S1)at([yshift=-200mm]current page.north west);
\draw[draw=none,fill=BlueDD!5](S1)--++(45:16)coordinate(S2)-
|(S2|-current page.north west)--(current page.north west)coordinate(S3)--(S1);

% TOP AREA: Circuit lines in upper white space
\draw[crimson!50, line width=1.5pt] ([xshift=30mm,yshift=-40mm]current page.north west) -- ++(60mm,0);
\draw[crimson!40, line width=1pt] ([xshift=120mm,yshift=-50mm]current page.north west) -- ++(50mm,0);
\draw[crimson!50, line width=1.5pt] ([xshift=40mm,yshift=-70mm]current page.north west) -- ++(40mm,0);

% Connecting lines in top area
\draw[crimson!30, line width=1pt] ([xshift=60mm,yshift=-40mm]current page.north west) -- ++(0,-20mm);
\draw[crimson!30, line width=1pt] ([xshift=145mm,yshift=-50mm]current page.north west) -- ++(0,10mm);

% Neural nodes in top area
\fill[crimson!70] ([xshift=60mm,yshift=-40mm]current page.north west) circle (2.5mm);
\fill[white] ([xshift=60mm,yshift=-40mm]current page.north west) circle (1.5mm);
\fill[crimson!60] ([xshift=145mm,yshift=-50mm]current page.north west) circle (2mm);
\fill[white] ([xshift=145mm,yshift=-50mm]current page.north west) circle (1mm);
\fill[crimson!80] ([xshift=80mm,yshift=-70mm]current page.north west) circle (2mm);
\fill[white] ([xshift=80mm,yshift=-70mm]current page.north west) circle (1mm);

% BOTTOM AREA: Circuit lines in lower white space
\draw[crimson!50, line width=1.5pt] ([xshift=20mm,yshift=-200mm]current page.north west) -- ++(70mm,0);
\draw[crimson!40, line width=1pt] ([xshift=110mm,yshift=-210mm]current page.north west) -- ++(60mm,0);
\draw[crimson!50, line width=1.5pt] ([xshift=35mm,yshift=-230mm]current page.north west) -- ++(45mm,0);

% Connecting lines in bottom area
\draw[crimson!30, line width=1pt] ([xshift=55mm,yshift=-200mm]current page.north west) -- ++(0,-20mm);
\draw[crimson!30, line width=1pt] ([xshift=140mm,yshift=-210mm]current page.north west) -- ++(0,15mm);

% Neural nodes in bottom area
\fill[crimson!70] ([xshift=55mm,yshift=-200mm]current page.north west) circle (2.5mm);
\fill[white] ([xshift=55mm,yshift=-200mm]current page.north west) circle (1.5mm);
\fill[crimson!60] ([xshift=140mm,yshift=-210mm]current page.north west) circle (2mm);
\fill[white] ([xshift=140mm,yshift=-210mm]current page.north west) circle (1mm);
\fill[crimson!80] ([xshift=80mm,yshift=-230mm]current page.north west) circle (2mm);
\fill[white] ([xshift=80mm,yshift=-230mm]current page.north west) circle (1mm);

% SIDE AREAS: Subtle circuit elements on left and right edges
\draw[crimson!30, line width=1pt] ([xshift=15mm,yshift=-120mm]current page.north west) -- ++(20mm,0);
\draw[crimson!30, line width=1pt] ([xshift=175mm,yshift=-130mm]current page.north west) -- ++(15mm,0);
\fill[crimson!50] ([xshift=25mm,yshift=-120mm]current page.north west) circle (1.5mm);
\fill[white] ([xshift=25mm,yshift=-120mm]current page.north west) circle (0.8mm);
\fill[crimson!50] ([xshift=185mm,yshift=-130mm]current page.north west) circle (1.5mm);
\fill[white] ([xshift=185mm,yshift=-130mm]current page.north west) circle (0.8mm);

% Title positioned in center - CLEAN AREA
\node[inner sep=0mm,draw=none,anchor=center,text width=0.8\textwidth,
align=center,font={\fontsize{44pt}{44}\selectfont\bfseries}]
(BGG) at (current page.center)  {\color{crimson}\MakeUppercase {#1}};

\end{tikzpicture}
\clearpage
}

% Define \lab command for lab styling (different visual treatment)
\newcommand{\lab}[1]{%
\begin{tikzpicture}[remember picture,overlay]
%%%
% Different background pattern for labs
\coordinate(S1)at([yshift=-200mm]current page.north west);
\draw[draw=none,fill=BlueDD!15](S1)--++(45:16)coordinate(S2)-
|(S2|-current page.north west)--(current page.north west)coordinate(S3)--(S1);
%
\coordinate(E1)at([yshift=-98mm]current page.north west);
\draw[draw=none,fill=BlueDD!25](E1)--(current page.north west)coordinate(E2)
--++(0:98mm)coordinate(E3)--(E1);
%
\coordinate(D1)at([yshift=15mm]current page.south west);
\draw[draw=none,fill=BlueDD!60,opacity=0.7](D1)--++(45:5.5)coordinate(D2)
-|(D2|-current page.north west)--(current page.north west)coordinate(D3)--(D1);
%%%%
\path[red](S2)-|(S2-|current page.east)coordinate(SS2);
%LAB - Different styling
\node[crimson,align=flush right,inner sep=0,outer sep=0mm,draw=none,anchor=south,
font={\fontsize{48pt}{48}\selectfont\bfseries}]  (BG) at ($(S2)!0.5!(SS2)$){\hphantom{Workshop}};
%%%
\path[green]([yshift=15mm]D2)-|coordinate(TPD)(BG.south east);
\node[inner sep=0mm,draw=none,anchor=south east,%text width=0.9\textwidth,
align=right,font={\fontsize{40pt}{40}\selectfont}]
(BGG) at (TPD)  {\color{crimson}\MakeUppercase {#1}};%\MakeUppercase {}
\end{tikzpicture}
\thispagestyle{empty}
\clearpage
}

% =============================================================================
% SECTION FORMATTING
% =============================================================================
% All section levels use crimson color and are ragged right

% Section (Large, bold, crimson)
\titleformat{\section}
  {\normalfont\Large\bfseries\color{crimson}\raggedright}
  {\thesection}
  {0.5em}
  {#1}
\titlespacing*{\section}{0pc}{14pt plus 4pt minus 4pt}{6pt plus 2pt minus 2pt}[0pc]

% Subsection (large, bold, crimson)
\titleformat{\subsection}
  {\normalfont\large\bfseries\color{crimson}\raggedright}
  {\thesubsection}
  {0.5em}
  {#1}
\titlespacing*{\subsection}{0pc}{12pt plus 4pt minus 4pt}{5pt plus 1pt minus 2pt}[0pc]

% Subsubsection (normal size, bold, crimson)
\titleformat{\subsubsection}
  {\normalfont\normalsize\bfseries\color{crimson}\raggedright}
  {\thesubsubsection}
  {0.5em}
  {#1}
\titlespacing*{\subsubsection}{0pc}{12pt plus 4pt minus 4pt}{5pt plus 1pt minus 2pt}[0pc]

% Paragraph (run-in, bold, crimson, ends with period)
\titleformat{\paragraph}[runin]
  {\normalfont\normalsize\bfseries\color{crimson}}
  {\theparagraph}
  {0.5em}
  {#1}
  [\textbf{.}]
  \titlespacing*{\paragraph}{0pc}{6pt plus 2pt minus 2pt}{0.5em}[0pc]

% Subparagraph (run-in, italic, crimson, ends with period)
\titleformat{\subparagraph}[runin]
  {\normalfont\normalsize\itshape\color{crimson}}
  {\thesubparagraph}
  {0.5em}
  {#1}
  [\textbf{.}]
  \titlespacing*{\subparagraph}{0pc}{6pt plus 2pt minus 2pt}{0.5em}[0pc]

% =============================================================================
% CHAPTER FORMATTING
% =============================================================================
% Numbered chapters: "Chapter X" prefix, huge crimson title
\titleformat{\chapter}[display]
  {\normalfont\huge\bfseries\color{crimson}}
  {\chaptername\ \thechapter}
  {20pt}
  {\Huge #1}
  []

% Unnumbered chapters: no prefix, huge crimson title
\titleformat{name=\chapter,numberless}
  {\normalfont\huge\bfseries\color{crimson}}
  {}
  {0pt}
  {\Huge #1}
  []

\renewcommand{\chaptername}{Chapter}
% =============================================================================
% TABLE OF CONTENTS FORMATTING
% =============================================================================
\setcounter{tocdepth}{2}                      % Show chapters, sections, subsections

% TOC spacing adjustments for number widths and indentation
\setlength{\cftchapnumwidth}{2em}             % Chapter number width
\setlength{\cftsecnumwidth}{2.75em}           % Section number width
\setlength{\cftsubsecnumwidth}{3.25em}        % Subsection number width
\setlength{\cftsubsubsecnumwidth}{4em}        % Subsubsection number width
\setlength{\cftsubsecindent}{4.25em}          % Subsection indent
\setlength{\cftsubsubsecindent}{7.5em}        % Subsubsection indent

% Chapter entries in TOC: bold crimson with "Chapter" prefix
\renewcommand{\cftchapfont}{\bfseries\color{crimson}}
\renewcommand{\cftchappresnum}{\color{crimson}Chapter~}

% Custom formatting for division entries (styled like parts)
\newcommand{\divisionchapter}[1]{%
  \addvspace{12pt}%
  \noindent\hfil\bfseries\color{crimson}#1\hfil\par%
  \addvspace{6pt}%
}

% Adjust TOC spacing for "Chapter" prefix
\newlength{\xtraspace}
\settowidth{\xtraspace}{\cftchappresnum\cftchapaftersnum}
\addtolength{\cftchapnumwidth}{\xtraspace}

% Unnumbered chapters with TOC entry
\newcommand{\likechapter}[1]{%
    \chapter*{#1}
    \addcontentsline{toc}{chapter}{\textcolor{crimson}{#1}}
}

% =============================================================================
% PAGE NUMBERING SYSTEM
% =============================================================================
% Implements traditional book numbering:
% - Roman numerals (i, ii, iii...) for frontmatter
% - Arabic numerals (1, 2, 3...) for mainmatter
% Automatically switches at first numbered chapter
\makeatletter
\newif\if@firstnumbered%
\@firstnumberedtrue%
\newif\if@firstunnumbered%
\@firstunnumberedtrue%

\newcounter{lastRomanPage}
\setcounter{lastRomanPage}{1}

% Start document with Roman numerals (frontmatter)
\AtBeginDocument{
  \pagenumbering{roman}
  \renewcommand{\thepage}{\roman{page}}
}

% Intercept chapter command
\let\old@chapter\chapter%
\renewcommand{\chapter}{%
  \@ifstar{\unnumbered@chapter}{\numbered@chapter}%
}

% Numbered chapters: switch to Arabic on first occurrence
\newcommand{\numbered@chapter}[1]{%
  \if@firstnumbered%
    \cleardoublepage%
    \setcounter{lastRomanPage}{\value{page}}%
    \pagenumbering{arabic}%
    \@firstnumberedfalse%
  \else
    \setcounter{page}{\value{page}}%
  \fi
  \setcounter{sidenote}{1}                    % Reset footnote counter per chapter
  \old@chapter{#1}%
}

% Unnumbered chapters: stay in Roman numerals
\newcommand{\unnumbered@chapter}[1]{%
  \if@firstunnumbered%
    \clearpage
    \setcounter{lastRomanPage}{\value{page}}%
    \pagenumbering{roman}%
    \@firstunnumberedfalse%
  \fi
  \setcounter{sidenote}{1}
  \old@chapter*{#1}%
}
\makeatother

% =============================================================================
% TABLE SIZING AND SPACING
% =============================================================================
% Make tables slightly smaller to fit more content
\AtBeginEnvironment{longtable}{\scriptsize}

% Increase vertical spacing in table cells (default is 1.0)
\renewcommand{\arraystretch}{1.3}

% Prefer placing figures and tables at the top of pages
\makeatletter
\renewcommand{\fps@figure}{t}  % Default placement: top of page
\renewcommand{\fps@table}{t}   % Default placement: top of page
\makeatother

% =============================================================================
% LONGTABLE PAGE BREAKING FIXES (Windows compatibility)
% =============================================================================
% Prevent "Infinite glue shrinkage" errors on Windows LaTeX builds
% by giving longtable more flexibility in page breaking

% Allow more flexible page breaking (vs strict \flushbottom)
\raggedbottom

% Process more rows before attempting page break (default is 20)
\setcounter{LTchunksize}{50}

% Add extra stretch for longtable environments specifically
\AtBeginEnvironment{longtable}{%
  \setlength{\emergencystretch}{3em}%
  \setlength{\parskip}{0pt plus 1pt}%
}

% =============================================================================
% TABLE STYLING - Clean tables with crimson borders
% =============================================================================
% Professional table appearance with:
% - Clean white background (no colored rows)
% - Crimson-colored borders
% - Good spacing for readability
%
% Note: Headers are automatically bolded by Quarto when using **text** in source
\usepackage{booktabs}      % Professional table rules (\toprule, \midrule, \bottomrule)
\usepackage{colortbl}      % For colored borders (\arrayrulecolor)

% Global table styling - crimson borders
\setlength{\arrayrulewidth}{0.5pt}          % Thinner borders than default
%\arrayrulecolor{crimson}                    % Crimson borders matching brand

\setcounter{chapter}{0}

% =============================================================================
% DROP CAPS (Lettrine)
% =============================================================================
% Decorative large first letter at chapter openings, following the tradition
% of Hennessy & Patterson and other MIT Press textbooks.
% Usage in QMD: \lettrine{T}{he first sentence...}
\usepackage{lettrine}
\renewcommand{\LettrineFontHook}{\color{crimson}\bfseries}
\setcounter{DefaultLines}{3}          % Drop cap spans 3 lines
\renewcommand{\DefaultLoversize}{0.1} % Slight oversize for visual weight
\renewcommand{\DefaultLraise}{0}      % No vertical shift
\setlength{\DefaultNindent}{0.5em}    % Indent of continuation text
\setlength{\DefaultSlope}{0pt}        % No slope on continuation

% =============================================================================
% RUNNING HEADERS — Truncation Safety
% =============================================================================
% Long chapter/section titles can overflow the header. These marks truncate
% gracefully so headers stay within the text block.
\renewcommand{\chaptermark}[1]{%
  \markboth{\thechapter.\ #1}{}}
\renewcommand{\sectionmark}[1]{%
  \markright{\thesection\ #1}}

% =============================================================================
% EPIGRAPH ENVIRONMENT
% =============================================================================
% For chapter-opening quotations. Renders as right-aligned italic block
% with attribution in small caps below.
% Usage: \epigraph{Quote text}{Author Name, \textit{Source}}
\newcommand{\bookepigraph}[2]{%
  \vspace{1em}%
  \begin{flushright}%
    \begin{minipage}{0.75\textwidth}%
      \raggedleft\itshape\small #1\\[0.5em]%
      \normalfont\small --- #2%
    \end{minipage}%
  \end{flushright}%
  \vspace{1.5em}%
}

% =============================================================================
% THUMB INDEX TABS
% =============================================================================
% Colored tabs on the outer page edge for quick chapter navigation.
% Each Part gets a different vertical position; chapters within a Part
% share the same tab position. Visible when flipping through the book.
\newcounter{thumbindex}
\setcounter{thumbindex}{0}
\newlength{\thumbtabheight}
\setlength{\thumbtabheight}{16mm}     % Height of each tab
\newlength{\thumbtabwidth}
\setlength{\thumbtabwidth}{8mm}       % Width protruding from edge
\newlength{\thumbtabgap}
\setlength{\thumbtabgap}{1mm}         % Gap between tabs

% Advance to next thumb tab position (call at each \part)
\newcommand{\nextthumb}{%
  \stepcounter{thumbindex}%
}

% Draw the thumb tab on every page (placed in header via fancyhdr)
\newcommand{\drawthumb}{%
  \ifnum\value{thumbindex}>0%
    \begin{tikzpicture}[remember picture,overlay]
      \pgfmathsetmacro{\thumboffset}{%
        20 + (\value{thumbindex}-1) * (16 + 1)}  % mm from top
      \ifodd\value{page}%
        % Odd pages: tab on right edge
        \fill[crimson!80]
          ([yshift=-\thumboffset mm]current page.north east)
          rectangle +(-\thumbtabwidth, -\thumbtabheight);
        \node[white,font=\tiny\bfseries,rotate=90]
          at ([yshift=-\thumboffset mm - 0.5\thumbtabheight,
               xshift=-0.5\thumbtabwidth]current page.north east)
          {\Roman{thumbindex}};
      \else
        % Even pages: tab on left edge
        \fill[crimson!80]
          ([yshift=-\thumboffset mm]current page.north west)
          rectangle +(\thumbtabwidth, -\thumbtabheight);
        \node[white,font=\tiny\bfseries,rotate=-90]
          at ([yshift=-\thumboffset mm - 0.5\thumbtabheight,
               xshift=0.5\thumbtabwidth]current page.north west)
          {\Roman{thumbindex}};
      \fi
    \end{tikzpicture}%
  \fi
}

% Hook into fancyhdr to draw thumb on every content page
\AddToHook{shipout/foreground}{%
  \drawthumb%
}

% =============================================================================
% CROP / BLEED MARKS
% =============================================================================
% For final print submission, uncomment the line below to add crop marks.
% MIT Press production will advise on exact requirements.
% \usepackage[cam,center,width=7.5in,height=10.5in]{crop}

% =============================================================================
% PDF/A ARCHIVAL COMPLIANCE
% =============================================================================
% MIT Press increasingly requires PDF/A for long-term preservation.
% This embeds all fonts and removes transparency.
% Note: pdfx must be loaded early; if it conflicts with hyperref,
% MIT Press production can handle the conversion post-build.
% Uncomment when ready for final submission:
% \usepackage[a-3u]{pdfx}

% =============================================================================
% ENHANCED WIDOW / ORPHAN CONTROL
% =============================================================================
% Prevent single lines at top/bottom of pages and breaks before equations
\clubpenalty=10000          % No orphans (single first line at bottom)
\widowpenalty=10000         % No widows (single last line at top)
\displaywidowpenalty=10000  % No widow before display math
\predisplaypenalty=10000    % No page break just before display math
\postdisplaypenalty=0       % Allow break after display math (natural)
\usepackage{needspace}
\let\Needspace\needspace
\makeatletter
\@ifpackageloaded{float}{}{\usepackage{float}}
\floatstyle{plain}
\@ifundefined{c@chapter}{\newfloat{vid}{h}{lovid}}{\newfloat{vid}{h}{lovid}[chapter]}
\floatname{vid}{Video}
\newcommand*\listofvids{\listof{vid}{List of Videos}}
\makeatother
\makeatletter
\@ifpackageloaded{tcolorbox}{}{\usepackage[skins,breakable]{tcolorbox}}
\@ifpackageloaded{fontawesome5}{}{\usepackage{fontawesome5}}
\definecolor{quarto-callout-color}{HTML}{909090}
\definecolor{quarto-callout-note-color}{HTML}{0758E5}
\definecolor{quarto-callout-important-color}{HTML}{CC1914}
\definecolor{quarto-callout-warning-color}{HTML}{EB9113}
\definecolor{quarto-callout-tip-color}{HTML}{00A047}
\definecolor{quarto-callout-caution-color}{HTML}{FC5300}
\definecolor{quarto-callout-color-frame}{HTML}{acacac}
\definecolor{quarto-callout-note-color-frame}{HTML}{4582ec}
\definecolor{quarto-callout-important-color-frame}{HTML}{d9534f}
\definecolor{quarto-callout-warning-color-frame}{HTML}{f0ad4e}
\definecolor{quarto-callout-tip-color-frame}{HTML}{02b875}
\definecolor{quarto-callout-caution-color-frame}{HTML}{fd7e14}
\makeatother
\makeatletter
\@ifpackageloaded{bookmark}{}{\usepackage{bookmark}}
\makeatother
\makeatletter
\@ifpackageloaded{caption}{}{\usepackage{caption}}
\AtBeginDocument{%
\ifdefined\contentsname
  \renewcommand*\contentsname{Table of contents}
\else
  \newcommand\contentsname{Table of contents}
\fi
\ifdefined\listfigurename
  \renewcommand*\listfigurename{List of Figures}
\else
  \newcommand\listfigurename{List of Figures}
\fi
\ifdefined\listtablename
  \renewcommand*\listtablename{List of Tables}
\else
  \newcommand\listtablename{List of Tables}
\fi
\ifdefined\figurename
  \renewcommand*\figurename{Figure}
\else
  \newcommand\figurename{Figure}
\fi
\ifdefined\tablename
  \renewcommand*\tablename{Table}
\else
  \newcommand\tablename{Table}
\fi
}
\@ifpackageloaded{float}{}{\usepackage{float}}
\floatstyle{ruled}
\@ifundefined{c@chapter}{\newfloat{codelisting}{h}{lop}}{\newfloat{codelisting}{h}{lop}[chapter]}
\floatname{codelisting}{Listing}
\newcommand*\listoflistings{\listof{codelisting}{List of Listings}}
\makeatother
\makeatletter
\makeatother
\makeatletter
\@ifpackageloaded{caption}{}{\usepackage{caption}}
\@ifpackageloaded{subcaption}{}{\usepackage{subcaption}}
\makeatother
\newcommand{\fbxIconPath}{assets/images/icons/callouts}
\newcommand{\fbxIconFormat}{pdf}
\makeatletter
\@ifpackageloaded{tcolorbox}{}{\usepackage[many]{tcolorbox}}
\makeatother
%%%% ---foldboxy preamble ----- %%%%%

% Load xstring for string manipulation
\RequirePackage{xstring}

% Icon path and format configuration - can be overridden in filter-metadata
\providecommand{\fbxIconPath}{assets/images/icons/callouts}
\providecommand{\fbxIconFormat}{pdf}

% Helper command to include icon with hyphen-to-underscore conversion
% This ensures consistency: callout-quiz-question -> callout_quiz_question
\newcommand{\fbxIncludeIcon}[2]{%
  \StrSubstitute{#1}{-}{_}[\fbxIconName]%
  \includegraphics[width=#2]{\fbxIconPath/icon_\fbxIconName.\fbxIconFormat}%
}

% Legacy fallback colors (keep for compatibility)
\definecolor{fbx-default-color1}{HTML}{c7c7d0}
\definecolor{fbx-default-color2}{HTML}{a3a3aa}
\definecolor{fbox-color1}{HTML}{c7c7d0}
\definecolor{fbox-color2}{HTML}{a3a3aa}

% arguments: #1 typelabelnummer: #2 titel: #3
\newenvironment{fbx}[3]{%
\begin{tcolorbox}[
  enhanced,
  breakable,
  %fontupper=\fontsize{8pt}{10pt}\selectfont,  % 95% of body text (10pt -> 9.5pt)
  before skip=8pt,  % space above box (increased)
  after skip=8pt,   % space below box (increased)
  attach boxed title to top*={xshift=0pt},
  boxed title style={
  %fuzzy shadow={1pt}{-1pt}{0mm}{0.1mm}{gray},
  arc=1.5pt,
  rounded corners=north,
  sharp corners=south,
  top=6pt,          % Adjusted for ~40px equivalent height
  bottom=5pt,       % Adjusted for ~40px equivalent height
  overlay={
      \node [left,outer sep=0em, black,draw=none,anchor=west,
        rectangle,fill=none,inner sep=0pt]
        at ([xshift=4mm]frame.west) {\fbxIncludeIcon{#1}{4.2mm}};
    },
  },
  colframe=#1-color2,             % Border color (auto-generated from YAML)
  colbacktitle=#1-color1,         % Background color (auto-generated from YAML)
  colback=white,
  coltitle=black,
  titlerule=0mm,
  toprule=0.5pt,
  bottomrule=0.5pt,
  leftrule=2.2pt,
  rightrule=0.5pt,
  outer arc=1.5pt,
  arc=1.5pt,
  left=0.5em,       % increased left padding
  bottomtitle=1.5mm, % increased title bottom margin
  toptitle=1.5mm,    % increased title top margin
  title=\hspace{2.5em}\protect#2\hspace{0.5em}\protect#3, % Protect parameters
  extras middle and last={top=4pt} % increased continuation spacing
]}
{\end{tcolorbox}}


% boxed environment with right border
\newenvironment{fbxSimple}[3]{\begin{tcolorbox}[
  enhanced,
  breakable,
  %fontupper=\fontsize{8pt}{10pt}\selectfont,  % 95% of body text (10pt -> 9.5pt)
  before skip=8pt,  % space above box (increased)
  after skip=8pt,   % space below box (increased)
  attach boxed title to top*={xshift=0pt},
  boxed title style={
  %fuzzy shadow={1pt}{-1pt}{0mm}{0.1mm}{gray},
  arc=1.5pt,
  rounded corners=north,
  sharp corners=south,
  top=6pt,          % Adjusted for ~40px equivalent height
  bottom=5pt,       % Adjusted for ~40px equivalent height
  overlay={
      \node [left,outer sep=0em, black,draw=none,anchor=west,
        rectangle,fill=none,inner sep=0pt]
        at ([xshift=3mm]frame.west) {\fbxIncludeIcon{#1}{4.2mm}};
    },
  },
  colframe=#1-color2,             % Border color (auto-generated from YAML)
  colbacktitle=#1-color1,         % Background color (auto-generated from YAML)
  colback=white,
  coltitle=black,
  titlerule=0mm,
  toprule=0.5pt,
  bottomrule=0.5pt,
  leftrule=2.2pt,
  rightrule=0.5pt,
  outer arc=1.5pt,
  arc=1.5pt,
  left=0.5em,       % increased left padding
  bottomtitle=1.5mm, % increased title bottom margin
  toptitle=1.5mm,    % increased title top margin
  title=\hspace{2.5em}\protect#2\hspace{0.5em}\protect#3, % Protect parameters
  boxsep=1pt,
  extras first={bottom=0pt},
  extras last={top=0pt,bottom=-4pt},
  overlay first={
    \draw[line width=1pt,white] ([xshift=2.2pt]frame.south west)-- ([xshift=-0.5pt]frame.south east);
  },
  overlay last={
    \draw[line width=1pt,white] ([xshift=2.2pt]frame.north west)-- ([xshift=-0.5pt]frame.north east);
   }
]}
{\end{tcolorbox}}

%%%% --- end foldboxy preamble ----- %%%%%
%%==== colors from yaml ===%
\definecolor{callout-quiz-answer-color1}{HTML}{E8F2EA}
\definecolor{callout-quiz-answer-color2}{HTML}{4a7c59}
\definecolor{callout-resource-exercises-color1}{HTML}{E0F2F1}
\definecolor{callout-resource-exercises-color2}{HTML}{20B2AA}
\definecolor{callout-example-color1}{HTML}{F0F8F6}
\definecolor{callout-example-color2}{HTML}{148F77}
\definecolor{callout-quiz-question-color1}{HTML}{F0F0F8}
\definecolor{callout-quiz-question-color2}{HTML}{5B4B8A}
\definecolor{callout-perspective-color1}{HTML}{F7F8FA}
\definecolor{callout-perspective-color2}{HTML}{4A5568}
\definecolor{callout-code-color1}{HTML}{F2F4F8}
\definecolor{callout-code-color2}{HTML}{D1D7E0}
\definecolor{callout-colab-color1}{HTML}{FFF5E6}
\definecolor{callout-colab-color2}{HTML}{FF6B35}
\definecolor{callout-resource-videos-color1}{HTML}{E0F2F1}
\definecolor{callout-resource-videos-color2}{HTML}{20B2AA}
\definecolor{callout-resource-slides-color1}{HTML}{E0F2F1}
\definecolor{callout-resource-slides-color2}{HTML}{20B2AA}
\definecolor{callout-chapter-connection-color1}{HTML}{FDF2F7}
\definecolor{callout-chapter-connection-color2}{HTML}{A51C30}
\definecolor{callout-definition-color1}{HTML}{F0F4F8}
\definecolor{callout-definition-color2}{HTML}{1B4F72}
\definecolor{callout-principle-color1}{HTML}{F3F2FA}
\definecolor{callout-principle-color2}{HTML}{3D3B8E}
\definecolor{callout-checkpoint-color1}{HTML}{E8F5E9}
\definecolor{callout-checkpoint-color2}{HTML}{2E7D32}
\definecolor{callout-theorem-color1}{HTML}{F5F0FF}
\definecolor{callout-theorem-color2}{HTML}{6B46C1}
\definecolor{callout-takeaways-color1}{HTML}{FDF2F7}
\definecolor{callout-takeaways-color2}{HTML}{BE185D}
\definecolor{callout-lighthouse-color1}{HTML}{FDF8E6}
\definecolor{callout-lighthouse-color2}{HTML}{B8860B}
\definecolor{callout-notebook-color1}{HTML}{F2F7FF}
\definecolor{callout-notebook-color2}{HTML}{2C5282}
%=============%

\usepackage{hyphenat}
\usepackage{ifthen}
\usepackage{calc}
\usepackage{calculator}



\usepackage{graphicx}
\usepackage{geometry}
\usepackage{afterpage}
\usepackage{tikz}
\usetikzlibrary{calc}
\usetikzlibrary{fadings}
\usepackage[pagecolor=none]{pagecolor}


% Set the titlepage font families







% Set the coverpage font families

\usepackage{bookmark}
\IfFileExists{xurl.sty}{\usepackage{xurl}}{} % add URL line breaks if available
\urlstyle{same}
\hypersetup{
  pdftitle={Introduction to Machine Learning Systems},
  pdfauthor={Vijay Janapa Reddi},
  colorlinks=true,
  linkcolor={Maroon},
  filecolor={Maroon},
  citecolor={Blue},
  urlcolor={Blue},
  pdfcreator={LaTeX via pandoc}}


\title{Introduction to Machine Learning Systems}
\author{Vijay Janapa Reddi}
\date{February 1, 2026}
\begin{document}
%%%%% begin titlepage extension code

  \begin{frontmatter}

\begin{titlepage}
% This is a combination of Pandoc templating and LaTeX
% Pandoc templating https://pandoc.org/MANUAL.html#templates
% See the README for help

\thispagestyle{empty}

\newgeometry{top=-100in}

% Page color

\newcommand{\coverauthorstyle}[1]{{\fontsize{20}{24.0}\selectfont
{#1}}}

\begin{tikzpicture}[remember picture, overlay, inner sep=0pt, outer sep=0pt]

\tikzfading[name=fadeout, inner color=transparent!0,outer color=transparent!100]
\tikzfading[name=fadein, inner color=transparent!100,outer color=transparent!0]
\node[anchor=south west, rotate=0, opacity=1] at ($(current page.south west)+(0.225\paperwidth, 9)$) {
\includegraphics[width=\paperwidth, keepaspectratio]{assets/images/covers/cover-image-transparent-vol1.png}};

% Title
\newcommand{\titlelocationleft}{0.075\paperwidth}
\newcommand{\titlelocationbottom}{0.4\paperwidth}
\newcommand{\titlealign}{left}

\begin{scope}{%
\fontsize{52}{62.4}\selectfont
\node[anchor=north
west, align=left, rotate=0] (Title1) at ($(current page.south west)+(\titlelocationleft,\titlelocationbottom)$)  [text width = 0.9\paperwidth]  {{\nohyphens{Machine
Learning Systems}}};
}
\end{scope}

% Author
\newcommand{\authorlocationleft}{.925\paperwidth}
\newcommand{\authorlocationbottom}{0.175\paperwidth}
\newcommand{\authoralign}{right}

\begin{scope}
{%
\fontsize{20}{24.0}\selectfont
\node[anchor=north
east, align=right, rotate=0] (Author1) at ($(current page.south west)+(\authorlocationleft,\authorlocationbottom)$)  [text width = 6in]  {\coverauthorstyle{Vijay
Janapa Reddi\\}};
}
\end{scope}

% Footer
\newcommand{\footerlocationleft}{0.075\paperwidth}
\newcommand{\footerlocationbottom}{0.475\paperwidth}
\newcommand{\footerlocationalign}{left}

\begin{scope}
{%
\fontsize{25}{30.0}\selectfont
 \node[anchor=north west, align=left, rotate=0] (Footer1) at %
($(current page.south west)+(\footerlocationleft,\footerlocationbottom)$)  [text width = 0.9\paperwidth]  {{\nohyphens{Introduction
to}}};
}
\end{scope}

\end{tikzpicture}
\clearpage
\restoregeometry
%%% TITLE PAGE START

% Set up alignment commands
%Page
\newcommand{\titlepagepagealign}{
\ifthenelse{\equal{left}{right}}{\raggedleft}{}
\ifthenelse{\equal{left}{center}}{\centering}{}
\ifthenelse{\equal{left}{left}}{\raggedright}{}
}


\newcommand{\titleandsubtitle}{
% Title and subtitle
{{\huge{\bfseries{\nohyphens{Introduction to Machine Learning
Systems}}}}\par
}%
}
\newcommand{\titlepagetitleblock}{
\titleandsubtitle
}

\newcommand{\authorstyle}[1]{{\large{#1}}}

\newcommand{\affiliationstyle}[1]{{\large{#1}}}

\newcommand{\titlepageauthorblock}{
{\authorstyle{\nohyphens{Vijay Janapa
Reddi}{\textsuperscript{1}}\textsuperscript{,}{\textsuperscript{,*}}}}}

\newcommand{\titlepageaffiliationblock}{
\hangindent=1em
\hangafter=1
{\affiliationstyle{
{1}.~Harvard University


\vspace{1\baselineskip}
* \textit{Correspondence:}~Vijay Janapa Reddi~vj@eecs.harvard.edu
}}
}
\newcommand{\headerstyled}{%
{}
}
\newcommand{\footerstyled}{%
{\large{}}
}
\newcommand{\datestyled}{%
{February 1, 2026}
}


\newcommand{\titlepageheaderblock}{\headerstyled}

\newcommand{\titlepagefooterblock}{
\footerstyled
}

\newcommand{\titlepagedateblock}{
\datestyled
}

%set up blocks so user can specify order
\newcommand{\titleblock}{{

{\titlepagetitleblock}
}

\vspace{4\baselineskip}
}

\newcommand{\authorblock}{{\titlepageauthorblock}

\vspace{2\baselineskip}
}

\newcommand{\affiliationblock}{{\titlepageaffiliationblock}

\vspace{0pt}
}

\newcommand{\logoblock}{}

\newcommand{\footerblock}{}

\newcommand{\dateblock}{{\titlepagedateblock}

\vspace{0pt}
}

\newcommand{\headerblock}{}

\thispagestyle{empty} % no page numbers on titlepages


\newcommand{\vrulecode}{\textcolor{black}{\rule{\vrulewidth}{\textheight}}}
\newlength{\vrulewidth}
\setlength{\vrulewidth}{2pt}
\newlength{\B}
\setlength{\B}{\ifdim\vrulewidth > 0pt 0.05\textwidth\else 0pt\fi}
\newlength{\minipagewidth}
\ifthenelse{\equal{left}{left} \OR \equal{left}{right} }
{% True case
\setlength{\minipagewidth}{\textwidth - \vrulewidth - \B - 0.1\textwidth}
}{
\setlength{\minipagewidth}{\textwidth - 2\vrulewidth - 2\B - 0.1\textwidth}
}
\ifthenelse{\equal{left}{left} \OR \equal{left}{leftright}}
{% True case
\raggedleft % needed for the minipage to work
\vrulecode
\hspace{\B}
}{%
\raggedright % else it is right only and width is not 0
}
% [position of box][box height][inner position]{width}
% [s] means stretch out vertically; assuming there is a vfill
\begin{minipage}[b][\textheight][s]{\minipagewidth}
\titlepagepagealign
\titleblock

Prof.~Vijay Janapa Reddi

School of Engineering and Applied Sciences

Harvard University

\vspace{80mm}

With heartfelt gratitude to the community for their invaluable
contributions and steadfast support.

\vfill

February 1, 2026

\vfill
\par

\end{minipage}\ifthenelse{\equal{left}{right} \OR \equal{left}{leftright} }{
\hspace{\B}
\vrulecode}{}
\clearpage
%%% TITLE PAGE END
\end{titlepage}
\setcounter{page}{1}
\end{frontmatter}

%%%%% end titlepage extension code

% =============================================================================
% HALF-TITLE PAGE (Volume I)
% =============================================================================
% Standard academic book sequence: half-title -> blank -> title page -> copyright
% The half-title shows only the book title -- no author, no publisher, no date.
\thispagestyle{empty}
\begin{center}
\vspace*{0.3\textheight}
{\fontsize{24pt}{28pt}\selectfont\bfseries\color{crimson} Introduction to\\[0.4em] Machine Learning Systems}\\[2em]
{\large\itshape Volume~I}
\vfill
\end{center}
\clearpage
\thispagestyle{empty}\null\clearpage  % Blank verso (back of half-title)

\renewcommand*\contentsname{Table of contents}
{
\hypersetup{linkcolor=}
\setcounter{tocdepth}{2}
\tableofcontents
}
\listoffigures
\listoftables

\mainmatter
\bookmarksetup{startatroot}

\chapter*{Welcome to Volume I}\label{welcome-to-volume-i}
\addcontentsline{toc}{chapter}{Welcome to Volume I}

\markboth{Welcome to Volume I}{Welcome to Volume I}

\bookmarksetup{startatroot}

\chapter{Machine Foundations}\label{machine-foundations}

This appendix covers the hardware and physical constraints that govern
ML system performance. From performance modeling to memory hierarchies
to numerical precision, these are the ``speed limits'' that every ML
engineer must understand. The concepts here underpin the hardware
acceleration strategies in \textbf{?@sec-ai-acceleration}, the training
optimizations in \textbf{?@sec-ai-training}, and the serving
architectures in \textbf{?@sec-model-serving-systems}.

\section{The Physics of
Computing}\label{sec-system-foundations-physics-computing-b6a4}

\phantomsection\label{callout-perspectiveux2a-1.1}
\begin{fbx}{callout-perspective}{Systems Perspective:}{Why This Matters}
\phantomsection\label{callout-perspective*-1.1}
You have trained a model that achieves good accuracy, but inference
takes 200ms when your SLA requires 50ms. Where do you start? Performance
analysis models give you a systematic way to diagnose whether you are
limited by computation, memory bandwidth, or something else entirely.
Without these tools, optimization is guesswork.

\end{fbx}

The models in this section form the foundation of quantitative systems
thinking. They define the ``speed limits'' set by physics and hardware
design.

\subsection{The Roofline
Model}\label{sec-system-foundations-roofline-model-5f7c}

The Roofline Model (\citeproc{ref-williams2009roofline}{Williams,
Waterman, and Patterson 2009}) answers a deceptively simple question:
\emph{how fast can this workload possibly run on this hardware?} The
answer depends on whether you run out of compute or memory bandwidth
first.

Every operation has an \textbf{arithmetic intensity}: the ratio of
computations performed to bytes moved from memory. Matrix multiplication
has high arithmetic intensity because you can reuse each loaded element
many times. Element-wise operations like ReLU have low intensity because
you load a number, do one operation, and write it back.
Figure~\ref{fig-roofline} illustrates how workloads are bounded by
either memory bandwidth or compute throughput.

\begin{figure}[htb]

\centering{

\pandocbounded{\includegraphics[keepaspectratio]{index_files/mediabag/945ce59d735da2b0104c82f0f01eb33e64678c59.pdf}}

}

\caption{\label{fig-roofline}\textbf{The Roofline Model}: Performance
ceiling for a hypothetical accelerator. The sloped line represents
memory bandwidth limits; the horizontal line represents peak compute.
Every workload can be plotted on this diagram to determine its
optimization strategy.}

\end{figure}%

The \textbf{ridge point} determines the hardware's balance. If your
workload's intensity is below this point, you are \textbf{memory-bound}
(sloped region). If it is above, you are \textbf{compute-bound} (flat
region).

\[ \text{Arithmetic Intensity} = \frac{\text{FLOPs}}{\text{Bytes Accessed}} \]

\[ \text{Ridge Point} = \frac{\text{Peak FLOP/s}}{\text{Memory Bandwidth}} \]

\subsubsection{A Concrete Example: The A100
Analysis}\label{sec-system-foundations-concrete-example-a100-analysis-0bb9}

Consider an NVIDIA A100 GPU with FP16 Tensor Core performance of 312
TFLOP/s and HBM2e bandwidth of 2.0 TB/s. The ridge point is 312 / 2.0 =
156 FLOP/byte.

Now compare two common operations:

\textbf{GEMM (Matrix Multiplication)}: For two \(4096 \times 4096\)
matrices, arithmetic intensity is approximately \(1365\) FLOP/byte.
Since \(1365 > 156\), this operation is compute-bound. You are using the
hardware efficiently.

\textbf{ReLU (Element-wise)}: For a \(4096 \times 4096\) tensor,
intensity is approximately \(0.25\) op/byte. Since \(0.25 \ll 156\),
this operation is severely memory-bound, achieving only about \(0.16\%\)
of peak TFLOP/s. The hardware is mostly waiting for data.

This explains why modern frameworks fuse operations: combining ReLU with
the preceding MatMul avoids writing intermediate results to memory,
effectively increasing arithmetic intensity.

\subsection{Amdahl's Law and Gustafson's
Law}\label{sec-system-foundations-amdahls-law-gustafsons-law-2f5c}

Parallelization is the primary tool for scaling ML, but its limits
depend on \emph{how} you scale. These two laws frame the fundamental
tension in parallel computing. \textbf{Amdahl's Law} is the pessimist's
view, governing how much faster a \emph{fixed} task can run (optimizing
latency). \textbf{Gustafson's Law} is the optimist's view, governing how
much \emph{more} work we can do in the same time (optimizing
throughput).

\subsubsection{Strong Scaling (Amdahl's
Law)}\label{strong-scaling-amdahls-law}

\textbf{Strong scaling} answers the question: \emph{If I add more
processors to a fixed-size problem, how much faster will it run?}

\textbf{Amdahl's Law} (\citeproc{ref-amdahl1967validity}{Amdahl 1967})
states that the speedup is limited by the serial portion of the
task.\sidenote{Gene Amdahl (1922--2015) was a legendary computer
architect at IBM, where he was the chief architect of the System/360. He
later founded Amdahl Corporation to compete with IBM in the mainframe
market. } If a fraction \(s\) of your task is serial (cannot be
parallelized) and \(p = 1-s\) is parallelizable, the maximum speedup
with \(n\) processors is:

\[ \text{Speedup}(n) = \frac{1}{s + \frac{1-s}{n}} \]

As \(n \to \infty\), the term \(\frac{1-s}{n} \to 0\), and the speedup
converges to \(1/s\).

\textbf{Example}: If 5\% of your training step is serial overhead (e.g.,
Python GIL, kernel launch latency) and 95\% is parallelizable matrix
math:

\begin{itemize}
\tightlist
\item
  With \(n=1\), speedup is 1.
\item
  With \(n=8\), speedup is
  \(\frac{1}{0.05 + 0.95/8} \approx 5\.9\times\).
\item
  With \(n=\infty\), speedup is capped at \(1/0.05 = 20\times\).
\end{itemize}

No matter how many GPUs you buy, you cannot make this fixed workload run
faster than 20x.

\subsubsection{Weak Scaling (Gustafson's
Law)}\label{weak-scaling-gustafsons-law}

\textbf{Weak scaling} answers the question: \emph{If I add more
processors, how much larger of a problem can I solve in the same amount
of time?}

This is the reality of Large Language Models. We don't use 1,000 GPUs to
train GPT-4 on a laptop-sized dataset in milliseconds; we use them to
train on a dataset 1,000x larger in reasonable time.

\textbf{Gustafson's Law}
(\citeproc{ref-gustafson1988reevaluating}{Gustafson 1988}) models this
``scaled speedup'':\sidenote{John Gustafson is a computer scientist
known for his work in parallel computing and for introducing the Unum
(universal number) format. His law was a direct response to the
perceived ``limits'' of Amdahl's Law when applied to massive scale. }

\[ \text{Scaled Speedup}(n) = n - s(n - 1) \]

Here, the parallel part of the workload grows linearly with \(n\), while
the serial part \(s\) remains fixed.

\textbf{Example}: Using the same 5\% serial overhead (\(s=0.05\)):

\begin{itemize}
\tightlist
\item
  With \(n=1\), speedup is 1.
\item
  With \(n=8\), Scaled Speedup is
  \(8 - 0.05(7) = 8 - 0\.35 = 7\.65\times\).
\item
  With \(n=1000\), Scaled Speedup is
  \(1000 - 0.05(999) \approx 950\times\).
\end{itemize}

In weak scaling, efficiency remains high because the useful work
(training the model) scales up to dwarf the fixed overheads.

\phantomsection\label{callout-notebookux2a-1.2}
\begin{fbx}{callout-notebook}{AI Engineer’s Notebook:}{Napkin Math: The Training Time Equation}
\phantomsection\label{callout-notebook*-1.2}
Just as classical architecture has an ``Iron Law'' of performance, Large
Language Model training has a fundamental governing equation. To
estimate training time \(T\):

\[ T \approx \frac{6 \cdot P \cdot D}{N \cdot X \cdot U} \]

Where: * \textbf{\(6\)}: The factor deriving from the forward pass
(\(2PD\)) and backward pass (\(4PD\)) FLOPs per token. * \textbf{\(P\)}:
Number of model parameters. * \textbf{\(D\)}: Number of training tokens.
* \textbf{\(N\)}: Number of accelerators (GPUs). * \textbf{\(X\)}: Peak
FLOP/s of one accelerator. * \textbf{\(U\)}: Model FLOPS Utilization
(MFU), typically 30\%--50\%.

\textbf{Example}: Training a \textbf{1B parameter} model on \textbf{20B
tokens} using \textbf{1 A100} (312 TFLOPS) at \textbf{40\% utilization}.
\[ \text{Total FLOPs} = 6 \times 10^9 \times 20 \times 10^9 = 1\.2 \\\\\\\\times 10^\{20\} \text{ FLOPs} \]
\[ \text{Throughput} = 1 \times (312 \times 10^{12}) \times 0.40 \approx 1\.25 \\\\\\\\times 10^\{14\} \text{ FLOP/s} \]
\[ T = \frac{1\.2 \\\\\\\\times 10^\{20\}}{1\.25 \\\\\\\\times 10^\{14\}} \approx 961538 \text{ seconds} \approx 16026 \text{ minutes} \]

\end{fbx}

\subsection{Little's Law}\label{sec-system-foundations-littles-law-9c4c}

For capacity planning in inference systems, \textbf{Little's Law}
(\citeproc{ref-little1961proof}{Little 1961}) relates concurrency
(\(L\)), arrival rate (\(\lambda\)), and latency (\(W\)):\sidenote{John
Little is an Institute Professor at MIT and a pioneer in the field of
operations research. His law, proved in 1961, is fundamental to queuing
theory and is used across fields from manufacturing to computer network
analysis. }

\[ L = \lambda \times W \]

\textbf{Example}: To sustain 1,000 queries per second (QPS) with 50ms
average latency, your system must support \(1000 \times 0.05 = 50\)
concurrent requests.

\textbf{Implication for Memory}: This tells you exactly how to size your
inference worker pools. If serving one request requires 1 GB of
temporary memory (KV cache, activations), handling 50 concurrent
requests requires 50 GB of memory. If your GPU only has 24 GB, you are
physically limited to 24 concurrent requests. Your maximum throughput is
capped at \(L/W = 24 / 0.05 = 480\) QPS, regardless of how many requests
arrive.

\section{Computer Architecture
Essentials}\label{sec-system-foundations-computer-architecture-essentials-bb18}

While physics sets the theoretical speed limits, \textbf{Computer
Architecture} defines the actual machinery we use to approach them. The
gap between theoretical peaks (Roofline) and realized performance often
lies in how well we utilize the memory hierarchy.

Understanding computer architecture is fundamental to optimizing ML
systems. The physical constraints of hardware---memory capacity,
bandwidth, and latency---determine what is achievable in practice. This
section covers the memory hierarchy that governs data access patterns
and the distinction between bandwidth and latency that shapes system
design decisions.

To design these systems, we must move beyond abstract concepts and look
at the actual ``speed limits'' of modern silicon.

\subsection{Latencies Every Programmer Should Know (2025
Edition)}\label{latencies-every-programmer-should-know-2025-edition}

The first step in systems intuition is understanding the cost of
distance. To write efficient kernels or distributed algorithms, you must
have an intuitive sense of how long the processor waits for data. If
accessing a register is like picking up a pencil from your desk,
fetching from HBM is walking across the office, and fetching from disk
is flying to the moon.

\begin{longtable}[]{@{}
  >{\raggedright\arraybackslash}p{(\linewidth - 6\tabcolsep) * \real{0.2596}}
  >{\raggedright\arraybackslash}p{(\linewidth - 6\tabcolsep) * \real{0.2596}}
  >{\raggedleft\arraybackslash}p{(\linewidth - 6\tabcolsep) * \real{0.2115}}
  >{\raggedleft\arraybackslash}p{(\linewidth - 6\tabcolsep) * \real{0.2500}}@{}}
\caption{\textbf{The Latency Hierarchy.} Access times for modern AI
hardware. Note the massive jump from SRAM (Cache) to HBM. Any kernel
that misses cache pays a heavy
penalty.}\label{tbl-latency-numbers}\tabularnewline
\toprule\noalign{}
\begin{minipage}[b]{\linewidth}\raggedright
\textbf{Component}
\end{minipage} & \begin{minipage}[b]{\linewidth}\raggedright
\textbf{Latency (ns)}
\end{minipage} & \begin{minipage}[b]{\linewidth}\raggedleft
\textbf{Cycles (Approx)}
\end{minipage} & \begin{minipage}[b]{\linewidth}\raggedleft
\textbf{Relative ``Distance''}
\end{minipage} \\
\midrule\noalign{}
\endfirsthead
\toprule\noalign{}
\begin{minipage}[b]{\linewidth}\raggedright
\textbf{Component}
\end{minipage} & \begin{minipage}[b]{\linewidth}\raggedright
\textbf{Latency (ns)}
\end{minipage} & \begin{minipage}[b]{\linewidth}\raggedleft
\textbf{Cycles (Approx)}
\end{minipage} & \begin{minipage}[b]{\linewidth}\raggedleft
\textbf{Relative ``Distance''}
\end{minipage} \\
\midrule\noalign{}
\endhead
\bottomrule\noalign{}
\endlastfoot
\textbf{L1 Cache / Register} \textbf{L2 Cache} \textbf{HBM3 (GPU
Memory)} \textbf{NVLink (GPU-GPU)} \textbf{PCIe (CPU-GPU)}
\textbf{InfiniBand (Network)} \textbf{SSD (NVMe)} &
\multicolumn{3}{>{\raggedright\arraybackslash}p{(\linewidth - 6\tabcolsep) * \real{0.7212} + 4\tabcolsep}@{}}{%
\textasciitilde1 ns \textbar{} 3-4 cycles \textbar{} 1 minute \textbar{}
\textasciitilde4 ns \textbar{} 12 cycles \textbar{} 4 minutes \textbar{}
\textasciitilde300 ns \textbar{} 1,000 cycles \textbar{} 5 hours
\textbar{} \textasciitilde500 ns \textbar{} 1,500 cycles \textbar{} 8
hours \textbar{} \textasciitilde1000 ns \textbar{} 3,000 cycles
\textbar{} 1 day \textbar{} \textasciitilde5000 ns \textbar{} 15,000
cycles \textbar{} 1 week \textbar{} \textasciitilde100000 ns \textbar{}
300,000 cycles \textbar{} 3 months \textbar{}} \\
\end{longtable}

\subsection{The AI Hardware Cheat Sheet (Modern
Reference)}\label{the-ai-hardware-cheat-sheet-modern-reference}

While latency tells us how long we wait for the \emph{first} byte,
bandwidth tells us how many bytes follow. Use these constants for
back-of-the-envelope ``Roofline'' calculations. These represent the
``standard units of compute'' for the 2025 era of machine learning.

\begin{longtable}[]{@{}
  >{\raggedright\arraybackslash}p{(\linewidth - 6\tabcolsep) * \real{0.1643}}
  >{\raggedleft\arraybackslash}p{(\linewidth - 6\tabcolsep) * \real{0.2786}}
  >{\raggedleft\arraybackslash}p{(\linewidth - 6\tabcolsep) * \real{0.2429}}
  >{\raggedright\arraybackslash}p{(\linewidth - 6\tabcolsep) * \real{0.3000}}@{}}
\caption{\textbf{Reference Specs (2025).} Key constants for quantitative
analysis. Always check specific datasheets, but these serve as standard
units of compute.}\label{tbl-hardware-cheatsheet}\tabularnewline
\toprule\noalign{}
\begin{minipage}[b]{\linewidth}\raggedright
\textbf{Spec}
\end{minipage} & \begin{minipage}[b]{\linewidth}\raggedleft
\textbf{NVIDIA H100 (SXM)}
\end{minipage} & \begin{minipage}[b]{\linewidth}\raggedleft
\textbf{Google TPU v5p}
\end{minipage} & \begin{minipage}[b]{\linewidth}\raggedright
\textbf{System Impact}
\end{minipage} \\
\midrule\noalign{}
\endfirsthead
\toprule\noalign{}
\begin{minipage}[b]{\linewidth}\raggedright
\textbf{Spec}
\end{minipage} & \begin{minipage}[b]{\linewidth}\raggedleft
\textbf{NVIDIA H100 (SXM)}
\end{minipage} & \begin{minipage}[b]{\linewidth}\raggedleft
\textbf{Google TPU v5p}
\end{minipage} & \begin{minipage}[b]{\linewidth}\raggedright
\textbf{System Impact}
\end{minipage} \\
\midrule\noalign{}
\endhead
\bottomrule\noalign{}
\endlastfoot
\textbf{FP16/BF16 Peak} \textbf{Memory Bandwidth} \textbf{HBM Capacity}
\textbf{L2/SRAM Cache} \textbf{Interconnect} &
\multicolumn{3}{>{\raggedleft\arraybackslash}p{(\linewidth - 6\tabcolsep) * \real{0.8214} + 4\tabcolsep}@{}}{%
989 TFLOPS \textbar{} 459 TFLOPS \textbar{} The ``Speed Limit''
(\(R_{peak}\)) \textbar{} 3.35 \textbar{} 2.76 \textbar{} The ``Width of
the Pipe'' (\(BW\)) \textbar{} 80 GB \textbar{} 95 GB \textbar{} Max
Model Size (\(P\)) / Batch Size (\(B\)) \textbar{} 50 MB \textbar{}
\textasciitilde100 MB \textbar{} Critical for Operator Fusion 900 GB/s
(NVLink) \textbar{} 1600 GB/s (ICI) \textbar{} Determines Model
Parallelism Scaling \textbar{}} \\
\end{longtable}

\subsection{Latencies Every Programmer Should Know (2025
Edition)}\label{latencies-every-programmer-should-know-2025-edition-1}

To write efficient systems, you must have an intuitive sense of
``relative distance.'' If accessing a register is picking up a pencil
from your desk, fetching from HBM is walking across the office, and
fetching from disk is flying to the moon.

\begin{longtable}[]{@{}
  >{\raggedright\arraybackslash}p{(\linewidth - 6\tabcolsep) * \real{0.2812}}
  >{\raggedleft\arraybackslash}p{(\linewidth - 6\tabcolsep) * \real{0.1979}}
  >{\raggedleft\arraybackslash}p{(\linewidth - 6\tabcolsep) * \real{0.2292}}
  >{\raggedleft\arraybackslash}p{(\linewidth - 6\tabcolsep) * \real{0.2708}}@{}}
\caption{\textbf{The Latency Hierarchy.} Access times for modern AI
hardware. Note the massive jump from SRAM (Cache) to HBM. Any kernel
that misses cache pays a heavy
penalty.}\label{tbl-latency-numbers}\tabularnewline
\toprule\noalign{}
\begin{minipage}[b]{\linewidth}\raggedright
\textbf{Component}
\end{minipage} & \begin{minipage}[b]{\linewidth}\raggedleft
\textbf{Latency (ns)}
\end{minipage} & \begin{minipage}[b]{\linewidth}\raggedleft
\textbf{Cycles (Approx)}
\end{minipage} & \begin{minipage}[b]{\linewidth}\raggedleft
\textbf{Relative ``Distance''}
\end{minipage} \\
\midrule\noalign{}
\endfirsthead
\toprule\noalign{}
\begin{minipage}[b]{\linewidth}\raggedright
\textbf{Component}
\end{minipage} & \begin{minipage}[b]{\linewidth}\raggedleft
\textbf{Latency (ns)}
\end{minipage} & \begin{minipage}[b]{\linewidth}\raggedleft
\textbf{Cycles (Approx)}
\end{minipage} & \begin{minipage}[b]{\linewidth}\raggedleft
\textbf{Relative ``Distance''}
\end{minipage} \\
\midrule\noalign{}
\endhead
\bottomrule\noalign{}
\endlastfoot
\textbf{L1 Cache / Register} \textbf{L2 Cache} \textbf{HBM3 (GPU
Memory)} \textbf{NVLink (GPU-GPU)} \textbf{PCIe (CPU-GPU)}
\textbf{InfiniBand (Network)} \textbf{SSD (NVMe)} & \textasciitilde1 ns
\textasciitilde4 ns \textasciitilde300 ns \textasciitilde500 ns
\textasciitilde1,000 ns \textasciitilde5,000 ns \textasciitilde100,000
ns & 3-4 cycles 12 cycles 1,000 cycles 1,500 cycles 3,000 cycles 15,000
cycles 300,000 cycles & 1 minute 4 minutes 5 hours 8 hours 1 day 1 week
3 months \\
\end{longtable}

\subsection{The AI Hardware Cheat Sheet (Modern
Reference)}\label{the-ai-hardware-cheat-sheet-modern-reference-1}

Use these constants for back-of-the-envelope ``Roofline'' calculations.

\begin{longtable}[]{@{}
  >{\raggedright\arraybackslash}p{(\linewidth - 6\tabcolsep) * \real{0.2054}}
  >{\raggedleft\arraybackslash}p{(\linewidth - 6\tabcolsep) * \real{0.2143}}
  >{\raggedleft\arraybackslash}p{(\linewidth - 6\tabcolsep) * \real{0.1875}}
  >{\raggedright\arraybackslash}p{(\linewidth - 6\tabcolsep) * \real{0.3750}}@{}}
\caption{\textbf{Reference Specs (2025).} Key constants for quantitative
analysis. Always check specific datasheets, but these serve as standard
units of compute.}\label{tbl-hardware-cheatsheet}\tabularnewline
\toprule\noalign{}
\begin{minipage}[b]{\linewidth}\raggedright
\textbf{Spec}
\end{minipage} & \begin{minipage}[b]{\linewidth}\raggedleft
\textbf{NVIDIA H100 (SXM)}
\end{minipage} & \begin{minipage}[b]{\linewidth}\raggedleft
\textbf{Google TPU v5p}
\end{minipage} & \begin{minipage}[b]{\linewidth}\raggedright
\textbf{System Impact}
\end{minipage} \\
\midrule\noalign{}
\endfirsthead
\toprule\noalign{}
\begin{minipage}[b]{\linewidth}\raggedright
\textbf{Spec}
\end{minipage} & \begin{minipage}[b]{\linewidth}\raggedleft
\textbf{NVIDIA H100 (SXM)}
\end{minipage} & \begin{minipage}[b]{\linewidth}\raggedleft
\textbf{Google TPU v5p}
\end{minipage} & \begin{minipage}[b]{\linewidth}\raggedright
\textbf{System Impact}
\end{minipage} \\
\midrule\noalign{}
\endhead
\bottomrule\noalign{}
\endlastfoot
\textbf{FP16/BF16 Peak} \textbf{Memory Bandwidth} \textbf{HBM Capacity}
\textbf{L2/SRAM Cache} \textbf{Interconnect} & 989 TFLOPS 3.35 TB/s 80
GB 50 MB 900 GB/s (NVLink) & 459 TFLOPS 2.76 TB/s 95 GB
\textasciitilde100 MB 1,600 GB/s (ICI) & The ``Speed Limit''
(\(R_{peak}\)) The ``Width of the Pipe'' (\(BW\)) Max Model Size (\(P\))
/ Batch Size (\(B\)) Critical for Operator Fusion Determines Model
Parallelism Scaling \\
\end{longtable}

\subsection{The Memory
Hierarchy}\label{sec-system-foundations-memory-hierarchy-674b}

Computer systems use a hierarchy because no single technology provides
both high capacity and low latency, as shown in
Figure~\ref{fig-memory-hierarchy}. Every technique that keeps data
higher in the pyramid (registers/cache) directly improves performance.

\begin{figure}[htb]

\centering{

\pandocbounded{\includegraphics[keepaspectratio]{index_files/mediabag/42154270e17703feee22bd46ab4e99b63a9bc610.pdf}}

}

\caption{\label{fig-memory-hierarchy}\textbf{The Memory Hierarchy}:
Performance depends on data proximity. Accessing HBM is
\textasciitilde100x slower than registers; accessing SSD is
\textasciitilde100,000x slower.}

\end{figure}%

The hierarchy's energy costs reveal why data movement dominates modern
system design.

\phantomsection\label{callout-notebookux2a-1.3}
\begin{fbx}{callout-notebook}{AI Engineer’s Notebook:}{Napkin Math: The High Cost of Data Movement}
\phantomsection\label{callout-notebook*-1.3}
Fetching a 32-bit value from DRAM costs roughly \textbf{1000\(\times\)
more energy} than performing a floating-point operation on it (e.g.,
\textasciitilde640 pJ vs \textasciitilde1 pJ). This ``Energy Wall''
means that maximizing \textbf{arithmetic intensity} (doing many ops per
loaded byte) is the only way to be energy efficient.

\end{fbx}

\subsection{Bandwidth
vs.~Latency}\label{sec-system-foundations-bandwidth-vs-latency-e155}

Bandwidth (throughput) and latency (delay) are distinct constraints.
Total transfer time follows:

\[ T = \text{Latency} + \frac{\text{Data Size}}{\text{Bandwidth}} \]

For small transfers (e.g., single-token inference), latency dominates.
For large transfers (e.g., loading weights), bandwidth dominates.

\textbf{Example}: Sending data over a 10 Gbps link with 10ms ping
(latency).

\begin{itemize}
\tightlist
\item
  \textbf{Latency-Bound (1KB Packet)}:

  \begin{itemize}
  \tightlist
  \item
    Transmission: \(1\text{KB} / 10\text{Gbps} \approx 0\.8 \mu s\).
  \item
    Total Time \(\approx 10\text{ms} + 0\.8\mu s \approx 10\text{ms}\).
  \item
    \emph{Result}: The bandwidth is irrelevant; the speed of light
    (ping) is the bottleneck.
  \end{itemize}
\item
  \textbf{Bandwidth-Bound (1GB Checkpoint)}:

  \begin{itemize}
  \tightlist
  \item
    Transmission: \(1\text{GB} / 10\text{Gbps} \approx 800\text{ms}\).
  \item
    Total Time \(\approx 10\text{ms} + 800\text{ms} = 810\text{ms}\).
  \item
    \emph{Result}: The ping is negligible; the pipe size is the
    bottleneck.
  \end{itemize}
\end{itemize}

\section{Numerical
Representations}\label{sec-system-foundations-numerical-representations-7b2f}

While statistics helps us understand data distributions,
\textbf{Numerical Representations} determine how we store the values
themselves. In ML systems, the choice of precision (FP32 vs.~BF16
vs.~INT8) is a direct trade-off between statistical fidelity and
hardware throughput.

\phantomsection\label{callout-perspectiveux2a-1.4}
\begin{fbx}{callout-perspective}{Systems Perspective:}{Why This Matters}
\phantomsection\label{callout-perspective*-1.4}
Your production model runs at 50 QPS in FP32 but your target is 200 QPS.
Switching to INT8 could get you there, but will accuracy suffer?
Understanding numerical formats lets you make this trade-off
quantitatively rather than hoping for the best.

\end{fbx}

Neural networks are remarkably tolerant of reduced numerical precision.
This section explains the formats you will encounter and their
trade-offs.

\subsection{Floating-Point Format
Comparison}\label{sec-system-foundations-floatingpoint-format-comparison-c861}

The IEEE 754 standard and its AI-specific derivatives define different
trade-offs between dynamic range (the span of representable values) and
precision (how finely you can represent values within that range).
Table~\ref{tbl-numerical-formats} summarizes the key formats and their
use cases.

\begin{longtable}[]{@{}
  >{\raggedright\arraybackslash}p{(\linewidth - 10\tabcolsep) * \real{0.1066}}
  >{\raggedleft\arraybackslash}p{(\linewidth - 10\tabcolsep) * \real{0.0902}}
  >{\raggedleft\arraybackslash}p{(\linewidth - 10\tabcolsep) * \real{0.1230}}
  >{\raggedleft\arraybackslash}p{(\linewidth - 10\tabcolsep) * \real{0.1230}}
  >{\raggedleft\arraybackslash}p{(\linewidth - 10\tabcolsep) * \real{0.1639}}
  >{\raggedright\arraybackslash}p{(\linewidth - 10\tabcolsep) * \real{0.3607}}@{}}
\caption{\textbf{Numerical Format Comparison}: Each format trades off
precision, dynamic range, memory footprint, and compute throughput. BF16
has emerged as the preferred training format because it matches FP32's
range while using half the
memory.}\label{tbl-numerical-formats}\tabularnewline
\toprule\noalign{}
\begin{minipage}[b]{\linewidth}\raggedright
\textbf{Format}
\end{minipage} & \begin{minipage}[b]{\linewidth}\raggedleft
\textbf{Bits}
\end{minipage} & \begin{minipage}[b]{\linewidth}\raggedleft
\textbf{Exponent}
\end{minipage} & \begin{minipage}[b]{\linewidth}\raggedleft
\textbf{Mantissa}
\end{minipage} & \begin{minipage}[b]{\linewidth}\raggedleft
\textbf{Dynamic Range}
\end{minipage} & \begin{minipage}[b]{\linewidth}\raggedright
\textbf{Typical Use Case}
\end{minipage} \\
\midrule\noalign{}
\endfirsthead
\toprule\noalign{}
\begin{minipage}[b]{\linewidth}\raggedright
\textbf{Format}
\end{minipage} & \begin{minipage}[b]{\linewidth}\raggedleft
\textbf{Bits}
\end{minipage} & \begin{minipage}[b]{\linewidth}\raggedleft
\textbf{Exponent}
\end{minipage} & \begin{minipage}[b]{\linewidth}\raggedleft
\textbf{Mantissa}
\end{minipage} & \begin{minipage}[b]{\linewidth}\raggedleft
\textbf{Dynamic Range}
\end{minipage} & \begin{minipage}[b]{\linewidth}\raggedright
\textbf{Typical Use Case}
\end{minipage} \\
\midrule\noalign{}
\endhead
\bottomrule\noalign{}
\endlastfoot
\textbf{FP32} & 32 & 8 & 23 & \textasciitilde{}\(10^{-38}\) to
\(10^{38}\) & Training (full precision), reference inference \\
\textbf{FP16} & 16 & 5 & 10 & \textasciitilde{}\(10^{-5}\) to
\(6.5 \times
10^{4}\) & Training with loss scaling, inference \\
\textbf{BF16} & 16 & 8 & 7 & Same as FP32 & Training (preferred), avoids
loss scaling \\
\textbf{FP8} & 8 & 4 or 5 & 3 or 2 & Varies & Inference on newest
hardware (H100+) \\
\textbf{INT8} & 8 & N/A & N/A & -128 to 127 & Inference after
quantization \\
\end{longtable}

\begin{figure}[htb]

\centering{

\pandocbounded{\includegraphics[keepaspectratio]{index_files/mediabag/1aab683e2d85755e41679cbc1e3628e9ba68d310.pdf}}

}

\caption{\label{fig-float-formats}\textbf{Numerical Format Bit Layouts}:
A visual comparison of bit allocations. Note how \textbf{BF16} (Brain
Float 16) preserves the 8-bit exponent of \textbf{FP32}, ensuring the
same dynamic range for training stability. \textbf{FP16} trades range
for precision, often requiring loss scaling to prevent underflow.}

\end{figure}%

Beyond bit width, the allocation of bits between exponent and mantissa
determines what range of values each format can represent.

\begin{tcolorbox}[enhanced jigsaw, titlerule=0mm, breakable, bottomtitle=1mm, coltitle=black, colframe=quarto-callout-note-color-frame, colback=white, left=2mm, opacitybacktitle=0.6, colbacktitle=quarto-callout-note-color!10!white, arc=.35mm, toprule=.15mm, leftrule=.75mm, title=\textcolor{quarto-callout-note-color}{\faInfo}\hspace{0.5em}{Systems Insight: The Dynamic Range Wall}, rightrule=.15mm, bottomrule=.15mm, toptitle=1mm, opacityback=0]

The choice of numerical format is a direct application of the
\textbf{Iron Law of ML Systems}. Reducing precision from FP32 to BF16 or
FP16 halves the \textbf{Data Movement} term in the denominator,
potentially doubling throughput on memory-bound workloads. However, the
\emph{type} of 16-bit format determines the engineering complexity:

\begin{itemize}
\tightlist
\item
  \textbf{Dynamic Range (The Exponent)}: BF16 preserves the 8-bit
  exponent of FP32. This means it can represent the same range of
  extremely large and extremely small values (gradients). This is why
  BF16 is the ``gold standard'' for stable training---it rarely suffers
  from numerical overflow or underflow.
\item
  \textbf{Precision (The Mantissa)}: FP16 has a larger 10-bit mantissa
  than BF16 (7 bits), offering higher precision for values within its
  range. But its 5-bit exponent is a major constraint; gradients often
  ``vanish'' to zero (underflow) because the exponent cannot represent
  them. To solve this, FP16 training requires \textbf{Loss Scaling}, an
  operational overhead where gradients are multiplied by a large
  constant to push them into the representable range.
\item
  \textbf{Energy Efficiency}: INT8 operations are significantly more
  energy-efficient than floating-point equivalents because they utilize
  simpler integer ALUs and require less silicon area. Moving to INT8 for
  inference is the primary lever for deploying LLMs on
  battery-constrained edge devices.
\end{itemize}

\end{tcolorbox}

\textbf{Brain Float 16 (BF16)} deserves special attention
(\citeproc{ref-wang_bfloat16_2019}{Wang and Kanwar 2019}). It matches
FP32's 8-bit exponent (preserving dynamic range) while truncating the
mantissa to 7 bits. This avoids the gradient underflow problems that
plague FP16 training, eliminating the need for complex loss scaling.
Most modern training uses BF16 for this reason.\sidenote{BF16 was
originally introduced with the Google TPUv2 and has since been adopted
by Intel, Arm, and NVIDIA (starting with Ampere architectures). }

\subsection{Integer
Quantization}\label{sec-system-foundations-integer-quantization-9d02}

Quantization maps continuous floating-point values to discrete integers,
typically INT8. The key challenge is choosing how to map the
floating-point range to integers.

\textbf{Symmetric quantization} centers the mapping at zero:
\[ x_{int} = \text{round}\left(\frac{x}{\alpha} \times 127\right) \]

where \(\alpha\) is the scale factor (typically the maximum absolute
value). This works well for weight distributions centered around zero.

\textbf{Asymmetric quantization} handles distributions that are not
centered (common after ReLU, which produces only non-negative values) by
adding a zero-point offset:
\[ x_{int} = \text{round}\left(\frac{x - z}{\alpha} \times 255\right) \]

The choice between symmetric and asymmetric quantization depends on your
tensor's distribution and has measurable accuracy implications.

\section{Fallacies and
Pitfalls}\label{sec-machine-foundations-fallacies-pitfalls}

\begin{tcolorbox}[enhanced jigsaw, titlerule=0mm, breakable, bottomtitle=1mm, coltitle=black, colframe=quarto-callout-warning-color-frame, colback=white, left=2mm, opacitybacktitle=0.6, colbacktitle=quarto-callout-warning-color!10!white, arc=.35mm, toprule=.15mm, leftrule=.75mm, title=\textcolor{quarto-callout-warning-color}{\faExclamationTriangle}\hspace{0.5em}{Fallacy: Doubling GPUs halves training time.}, rightrule=.15mm, bottomrule=.15mm, toptitle=1mm, opacityback=0]

\textbf{The Reality}: This assumes perfect \textbf{strong scaling}
(Amdahl's Law). In practice, communication overhead (all-reduce) grows
with \(N\), and batch size constraints may limit parallelism. At large
scale, you often hit diminishing returns unless you also scale the
problem size (Weak Scaling).

\end{tcolorbox}

A related misconception concerns numerical precision.

\begin{tcolorbox}[enhanced jigsaw, titlerule=0mm, breakable, bottomtitle=1mm, coltitle=black, colframe=quarto-callout-warning-color-frame, colback=white, left=2mm, opacitybacktitle=0.6, colbacktitle=quarto-callout-warning-color!10!white, arc=.35mm, toprule=.15mm, leftrule=.75mm, title=\textcolor{quarto-callout-warning-color}{\faExclamationTriangle}\hspace{0.5em}{Fallacy: Higher precision (FP32) is always better.}, rightrule=.15mm, bottomrule=.15mm, toptitle=1mm, opacityback=0]

\textbf{The Reality}: For deep learning, FP32 often hurts performance
without improving convergence. It consumes 2x memory bandwidth and
energy compared to BF16. Since neural networks are resilient to noise,
the extra mantissa bits in FP32 are often modeling random variance
rather than signal.

\end{tcolorbox}

\phantomsection\label{refs}
\begin{CSLReferences}{1}{0}
\bibitem[\citeproctext]{ref-amdahl1967validity}
Amdahl, Gene M. 1967. {``Validity of the Single Processor Approach to
Achieving Large Scale Computing Capabilities.''} In \emph{Proceedings of
the April 18-20, 1967, Spring Joint Computer Conference on - AFIPS '67
(Spring)}, 483. AFIPS '67 (Spring). New York, NY, USA: ACM Press.
\url{https://doi.org/10.1145/1465482.1465560}.

\bibitem[\citeproctext]{ref-gustafson1988reevaluating}
Gustafson, John L. 1988. {``Reevaluating Amdahl's Law.''}
\emph{Communications of the ACM} 31 (5): 532--33.
\url{https://doi.org/10.1145/42411.42415}.

\bibitem[\citeproctext]{ref-little1961proof}
Little, John D. C. 1961. {``A Proof for the Queuing Formula:
\textless I\textgreater l\textless/i\textgreater{} =
λ\textless i\textgreater w\textless/i\textgreater{}.''} \emph{Operations
Research} 9 (3): 383--87. \url{https://doi.org/10.1287/opre.9.3.383}.

\bibitem[\citeproctext]{ref-wang_bfloat16_2019}
Wang, Y., and P. Kanwar. 2019. {``BFloat16: The Secret to High
Performance on Cloud TPUs.''} \emph{Google Cloud Blog}.

\bibitem[\citeproctext]{ref-williams2009roofline}
Williams, Samuel, Andrew Waterman, and David Patterson. 2009.
{``Roofline: An Insightful Visual Performance Model for Multicore
Architectures.''} \emph{Communications of the ACM} 52 (4): 65--76.
\url{https://doi.org/10.1145/1498765.1498785}.

\end{CSLReferences}


\backmatter

\clearpage


\end{document}
