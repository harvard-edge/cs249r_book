% Options for packages loaded elsewhere
% Options for packages loaded elsewhere
\PassOptionsToPackage{unicode,linktoc=all,pdfwindowui,pdfpagemode=FullScreen,pdfpagelayout=TwoPageRight}{hyperref}
\PassOptionsToPackage{hyphens}{url}
\PassOptionsToPackage{dvipsnames,svgnames,x11names}{xcolor}
%
\documentclass[
  9pt,
  letterpaper,
  abstract,
  titlepage]{scrbook}
\usepackage{xcolor}
\usepackage[left=1in,marginparwidth=2.0666666666667in,textwidth=4.1333333333333in,marginparsep=0.3in]{geometry}
\usepackage{amsmath,amssymb}
\setcounter{secnumdepth}{3}
\usepackage{iftex}
\ifPDFTeX
  \usepackage[T1]{fontenc}
  \usepackage[utf8]{inputenc}
  \usepackage{textcomp} % provide euro and other symbols
\else % if luatex or xetex
  \usepackage{unicode-math} % this also loads fontspec
  \defaultfontfeatures{Scale=MatchLowercase}
  \defaultfontfeatures[\rmfamily]{Ligatures=TeX,Scale=1}
\fi
\usepackage{lmodern}
\ifPDFTeX\else
  % xetex/luatex font selection
\fi
% Use upquote if available, for straight quotes in verbatim environments
\IfFileExists{upquote.sty}{\usepackage{upquote}}{}
\IfFileExists{microtype.sty}{% use microtype if available
  \usepackage[]{microtype}
  \UseMicrotypeSet[protrusion]{basicmath} % disable protrusion for tt fonts
}{}
% Make \paragraph and \subparagraph free-standing
\makeatletter
\ifx\paragraph\undefined\else
  \let\oldparagraph\paragraph
  \renewcommand{\paragraph}{
    \@ifstar
      \xxxParagraphStar
      \xxxParagraphNoStar
  }
  \newcommand{\xxxParagraphStar}[1]{\oldparagraph*{#1}\mbox{}}
  \newcommand{\xxxParagraphNoStar}[1]{\oldparagraph{#1}\mbox{}}
\fi
\ifx\subparagraph\undefined\else
  \let\oldsubparagraph\subparagraph
  \renewcommand{\subparagraph}{
    \@ifstar
      \xxxSubParagraphStar
      \xxxSubParagraphNoStar
  }
  \newcommand{\xxxSubParagraphStar}[1]{\oldsubparagraph*{#1}\mbox{}}
  \newcommand{\xxxSubParagraphNoStar}[1]{\oldsubparagraph{#1}\mbox{}}
\fi
\makeatother


\providecommand{\tightlist}{%
  \setlength{\itemsep}{0pt}\setlength{\parskip}{0pt}}\usepackage{longtable,booktabs,array}
\usepackage{calc} % for calculating minipage widths
% Correct order of tables after \paragraph or \subparagraph
\usepackage{etoolbox}
\makeatletter
\patchcmd\longtable{\par}{\if@noskipsec\mbox{}\fi\par}{}{}
\makeatother
% Allow footnotes in longtable head/foot
\IfFileExists{footnotehyper.sty}{\usepackage{footnotehyper}}{\usepackage{footnote}}
\makesavenoteenv{longtable}
\usepackage{graphicx}
\makeatletter
\newsavebox\pandoc@box
\newcommand*\pandocbounded[1]{% scales image to fit in text height/width
  \sbox\pandoc@box{#1}%
  \Gscale@div\@tempa{\textheight}{\dimexpr\ht\pandoc@box+\dp\pandoc@box\relax}%
  \Gscale@div\@tempb{\linewidth}{\wd\pandoc@box}%
  \ifdim\@tempb\p@<\@tempa\p@\let\@tempa\@tempb\fi% select the smaller of both
  \ifdim\@tempa\p@<\p@\scalebox{\@tempa}{\usebox\pandoc@box}%
  \else\usebox{\pandoc@box}%
  \fi%
}
% Set default figure placement to htbp
\def\fps@figure{htbp}
\makeatother
% definitions for citeproc citations
\NewDocumentCommand\citeproctext{}{}
\NewDocumentCommand\citeproc{mm}{%
  \begingroup\def\citeproctext{#2}\cite{#1}\endgroup}
\makeatletter
 % allow citations to break across lines
 \let\@cite@ofmt\@firstofone
 % avoid brackets around text for \cite:
 \def\@biblabel#1{}
 \def\@cite#1#2{{#1\if@tempswa , #2\fi}}
\makeatother
\newlength{\cslhangindent}
\setlength{\cslhangindent}{1.5em}
\newlength{\csllabelwidth}
\setlength{\csllabelwidth}{3em}
\newenvironment{CSLReferences}[2] % #1 hanging-indent, #2 entry-spacing
 {\begin{list}{}{%
  \setlength{\itemindent}{0pt}
  \setlength{\leftmargin}{0pt}
  \setlength{\parsep}{0pt}
  % turn on hanging indent if param 1 is 1
  \ifodd #1
   \setlength{\leftmargin}{\cslhangindent}
   \setlength{\itemindent}{-1\cslhangindent}
  \fi
  % set entry spacing
  \setlength{\itemsep}{#2\baselineskip}}}
 {\end{list}}
\usepackage{calc}
\newcommand{\CSLBlock}[1]{\hfill\break\parbox[t]{\linewidth}{\strut\ignorespaces#1\strut}}
\newcommand{\CSLLeftMargin}[1]{\parbox[t]{\csllabelwidth}{\strut#1\strut}}
\newcommand{\CSLRightInline}[1]{\parbox[t]{\linewidth - \csllabelwidth}{\strut#1\strut}}
\newcommand{\CSLIndent}[1]{\hspace{\cslhangindent}#1}

% =============================================================================
% LATEX HEADER CONFIGURATION FOR MLSYSBOOK PDF
% =============================================================================
% This file contains all LaTeX package imports, custom commands, and styling
% definitions for the PDF output of the Machine Learning Systems textbook.
%
% Key Features:
% - Harvard crimson branding throughout
% - Custom part/chapter/section styling
% - Professional table formatting with colored headers
% - Margin notes with custom styling
% - TikZ-based part dividers
% - Page numbering (Roman for frontmatter, Arabic for mainmatter)
%
% Note: This file is included via _quarto-pdf.yml and affects PDF output only.
% HTML/EPUB styling is handled separately via CSS files.
% =============================================================================

% =============================================================================
% PACKAGE IMPORTS
% =============================================================================

% Layout and positioning
% \usepackage[outercaption, ragged]{sidecap}  % Commented out to make figure captions inline instead of in margin
\usepackage{adjustbox}      % Adjusting box dimensions
\usepackage{afterpage}      % Execute commands after page break
\usepackage{morefloats}     % Increase number of floats
\usepackage{array}          % Enhanced table column formatting
\usepackage{atbegshi}       % Insert content at page beginning
%\usepackage{changepage}     % Change page dimensions mid-document
\usepackage{emptypage}      % Clear headers/footers on empty pages

% Language and text
\usepackage[english]{babel} % English language support
\usepackage{microtype}      % Improved typography and hyphenation

% Captions and floats
\usepackage{caption}
% Caption styling configuration
%\captionsetup[table]{belowskip=5pt}
\captionsetup{format=plain}
\DeclareCaptionLabelFormat{mylabel}{#1
#2:\hspace{1.0ex}}
\DeclareCaptionFont{ninept}{\fontsize{7pt}{8}\selectfont #1}

% Figure captions: Small font, bold label, ragged right
\captionsetup[figure]{labelfont={bf,ninept},labelsep=space,
belowskip=2pt,aboveskip=6pt,labelformat=mylabel,
justification=raggedright,singlelinecheck=false,font={ninept}}

% Table captions: Small font, bold label, ragged right
\captionsetup[table]{belowskip=6pt,labelfont={bf,ninept},labelsep=none,
labelformat=mylabel,justification=raggedright,singlelinecheck=false,font={ninept}}

% Typography fine-tuning
\emergencystretch=5pt       % Allow extra stretch to avoid overfull boxes

% Utility packages
\usepackage{etoolbox}       % For patching commands and environments

% Page layout and headers
\usepackage{fancyhdr}       % Custom headers and footers
\usepackage{geometry}       % Page dimensions and margins

% Graphics and figures
\usepackage{graphicx}       % Include graphics
\usepackage{float}          % Improved float placement
\usepackage[skins,breakable]{tcolorbox} % Coloured and framed text boxes
\tcbset{before upper=\setlength{\parskip}{3pt}}

% Tables
\usepackage{longtable}      % Multi-page tables

% Fonts and typography
\usepackage{fontspec}       % Font selection for LuaLaTeX
\usepackage{mathptmx}       % Times-like math fonts
\usepackage{newpxtext}      % Palatino-like font for body text

% Colors and visual elements
\usepackage[dvipsnames]{xcolor}  % Extended color support
\usepackage{tikz}           % Programmatic graphics
\usetikzlibrary{positioning}
\usetikzlibrary{calc}
\usepackage{tikzpagenodes}  % TikZ positioning relative to page

% Code listings
\usepackage{listings}       % Code highlighting

% Hyperlinks
\usepackage{hyperref}       % Clickable links in PDF

% Conditional logic
\usepackage{ifthen}         % If-then-else commands

% Math symbols
\usepackage{amsmath}        % AMS math extensions
\usepackage{amssymb}        % AMS math symbols
\usepackage{latexsym}       % Additional LaTeX symbols
\usepackage{pifont}         % Zapf Dingbats symbols
\providecommand{\blacklozenge}{\ding{117}}  % Black diamond symbol

% Lists
\usepackage{enumitem}       % Customizable lists

% Margin notes and sidenotes
\usepackage{marginfix}      % Fixes margin note overflow
\usepackage{marginnote}     % Margin notes
\usepackage{sidenotes}      % Academic-style sidenotes
\renewcommand\raggedrightmarginnote{\sloppy}
\renewcommand\raggedleftmarginnote{\sloppy}

% Typography improvements
\usepackage{ragged2e}       % Better ragged text
\usepackage[all]{nowidow}   % Prevent widows and orphans
\usepackage{needspace}      % Ensure minimum space on page

% Section formatting
\usepackage[explicit]{titlesec}  % Custom section titles
\usepackage{tocloft}        % Table of contents formatting

% QR codes and icons
\usepackage{fontawesome5}   % Font Awesome icons
\usepackage{qrcode}         % QR code generation
\qrset{link, height=15mm}

% =============================================================================
% FLOAT CONFIGURATION
% =============================================================================
% Allow more floats per page to handle figure-heavy chapters
\extrafloats{200}
\setcounter{topnumber}{12}       % Max floats at top of page
\setcounter{bottomnumber}{12}    % Max floats at bottom of page
\setcounter{totalnumber}{24}     % Max floats per page
\setcounter{dbltopnumber}{8}     % Max floats at top of two-column page
\renewcommand{\topfraction}{.95}  % Max fraction of page for top floats
\renewcommand{\bottomfraction}{.95}
\renewcommand{\textfraction}{.05}  % Min fraction of page for text
\renewcommand{\floatpagefraction}{.7}  % Min fraction of float page
\renewcommand{\dbltopfraction}{.95}

% Prevent "Float(s) lost" errors by flushing floats more aggressively
\usepackage{placeins}  % Provides \FloatBarrier

% =============================================================================
% COLOR DEFINITIONS
% =============================================================================
% Harvard crimson - primary brand color used throughout
\definecolor{crimson}{HTML}{A51C30}

% Quiz element colors
\definecolor{quiz-question-color1}{RGB}{225,243,248}  % Light blue background
\definecolor{quiz-question-color2}{RGB}{17,158,199}   % Blue border
\definecolor{quiz-answer-color1}{RGB}{250,234,241}    % Light pink background
\definecolor{quiz-answer-color2}{RGB}{152,14,90}      % Magenta border

% =============================================================================
% LIST FORMATTING
% =============================================================================
% Tighter list spacing for academic style
\def\tightlist{}
\setlist{itemsep=1pt, parsep=1pt, topsep=0pt,after={\vspace{0.3\baselineskip}}}
\let\tightlist\relax

\makeatletter
\@ifpackageloaded{framed}{}{\usepackage{framed}}
\@ifpackageloaded{fancyvrb}{}{\usepackage{fancyvrb}}
\makeatother

\makeatletter
%New float "codelisting" has been updated
\AtBeginDocument{%
\floatstyle{ruled}
\newfloat{codelisting}{!htb}{lop}
\floatname{codelisting}{Listing}
\floatplacement{codelisting}{!htb}
\captionsetup[codelisting]{labelfont={bf,ninept},labelformat=mylabel,
  singlelinecheck=false,width=\linewidth,labelsep=none,font={ninept}}%
\renewenvironment{snugshade}{%
   \def\OuterFrameSep{3pt}%
   \def\FrameCommand{\fboxsep=5pt\colorbox{shadecolor}}%
   \MakeFramed{\advance\hsize-\width\FrameRestore}%
   \leftskip 0.5em \rightskip 0.5em%
   \small% decrease font size
   }{\endMakeFramed}%
}
\makeatother

%The space before and after the verbatim environment "Highlighting" has been reduced
\fvset{listparameters=\setlength{\topsep}{0pt}\setlength{\partopsep}{0pt}}
\DefineVerbatimEnvironment{Highlighting}{Verbatim}{framesep=0mm,commandchars=\\\{\}}

\makeatletter
\renewcommand\fs@ruled{\def\@fs@cfont{\bfseries}\let\@fs@capt\floatc@ruled
\def\@fs@pre{\hrule height.8pt depth0pt \kern2pt}%
\def\@fs@post{\kern2pt\hrule\relax}%
\def\@fs@mid{\kern2pt\hrule\kern1pt}%space between float and caption
\let\@fs@iftopcapt\iftrue}
\makeatother


% =============================================================================
% HYPHENATION RULES
% =============================================================================
% Explicit hyphenation points for technical terms to avoid bad breaks
\hyphenation{
  light-weight
  light-weight-ed
  de-vel-op-ment
  un-der-stand-ing
  mod-els
  prin-ci-ples
  ex-per-tise
  com-pli-cat-ed
  blue-print
  per‧for‧mance
  com-mu-ni-ca-tion
  par-a-digms
  hy-per-ten-sion
  a-chieved
}

% =============================================================================
% CODE LISTING CONFIGURATION
% =============================================================================
% Settings for code blocks using listings package
\lstset{
breaklines=true,              % Automatic line wrapping
breakatwhitespace=true,       % Break at whitespace only
basicstyle=\ttfamily,         % Monospace font
frame=none,                   % No frame around code
keepspaces=true,              % Preserve spaces
showspaces=false,             % Don't show space characters
showtabs=false,               % Don't show tab characters
columns=flexible,             % Flexible column width
belowskip=0pt,               % Minimal spacing
aboveskip=0pt
}

% =============================================================================
% PAGE GEOMETRY
% =============================================================================
% MIT Press trim size: 7" x 10" (per publisher specifications)
% This is a standard academic textbook format providing good readability
% for technical content with figures and code blocks.
% Wide outer margin accommodates sidenotes/margin notes.
\geometry{
  paperwidth=7in,
  paperheight=10in,
  top=0.875in,
  bottom=0.875in,
  inner=0.875in,              % Inner margin (binding side)
  outer=1.75in,               % Outer margin (includes space for sidenotes)
  footskip=30pt,
  marginparwidth=1.25in,      % Width for margin notes
  twoside                     % Different left/right pages
}

% =============================================================================
% SIDENOTE STYLING
% =============================================================================
% Custom sidenote design with crimson vertical bar
\renewcommand{\thefootnote}{\textcolor{crimson}{\arabic{footnote}}}

% Save original sidenote command
\makeatletter
\@ifundefined{oldsidenote}{
  \let\oldsidenote\sidenote%
}{}
\makeatother

% Redefine sidenote with vertical crimson bar
\renewcommand{\sidenote}[1]{%
  \oldsidenote{%
    \noindent
    \color{crimson!100}                        % Crimson vertical line
    \raisebox{0em}{%
      \rule{0.5pt}{1.5em}                      % Thin vertical line
    }
    \hspace{0.3em}                             % Space after line
    \color{black}                              % Reset text color
    \footnotesize #1                           % Sidenote content
  }%
}

% =============================================================================
% FLOAT HANDLING
% =============================================================================
% Patch LaTeX's output routine to handle float overflow gracefully
% The "Float(s) lost" error occurs in \@doclearpage when \@currlist is not empty
% This patch silently clears pending floats that can't be placed
\makeatletter
\let\orig@doclearpage\@doclearpage
\def\@doclearpage{%
  \ifx\@currlist\@empty\else
    \global\let\@currlist\@empty
    \typeout{Warning: Floats cleared to prevent overflow}%
  \fi
  \orig@doclearpage
}
\makeatother

% Additional safety for structural commands
\let\originalbackmatter\backmatter
\renewcommand{\backmatter}{%
  \clearpage%
  \originalbackmatter%
}

\let\originalfrontmatter\frontmatter
\renewcommand{\frontmatter}{%
  \clearpage%
  \originalfrontmatter%
}

\let\originalmainmatter\mainmatter
\renewcommand{\mainmatter}{%
  \clearpage%
  \originalmainmatter%
}

% =============================================================================
% PAGE HEADERS AND FOOTERS
% =============================================================================
% Ensure chapters use fancy page style (not plain)
\patchcmd{\chapter}{\thispagestyle{plain}}{\thispagestyle{fancy}}{}{}

% Main page style with crimson headers
\pagestyle{fancy}
\fancyhf{}                                              % Clear all
\fancyhead[LE]{\small\color{crimson}\nouppercase{\rightmark}}  % Left even: section
\fancyhead[RO]{\color{crimson}\thepage}                 % Right odd: page number
\fancyhead[LO]{\small\color{crimson}\nouppercase{\leftmark}}   % Left odd: chapter
\fancyhead[RE]{\color{crimson}\thepage}                 % Right even: page number
\renewcommand{\headrulewidth}{0.4pt}                    % Thin header line
\renewcommand{\footrulewidth}{0pt}                      % No footer line

% Plain page style (for chapter openings)
\fancypagestyle{plain}{
  \fancyhf{}
  \fancyfoot[C]{\color{crimson}\thepage}                % Centered page number
  \renewcommand{\headrulewidth}{0pt}
  \renewcommand{\footrulewidth}{0pt}
}

% =============================================================================
% KOMA-SCRIPT FONT ADJUSTMENTS
% =============================================================================
% Apply crimson color to all heading levels
\addtokomafont{disposition}{\rmfamily\color{crimson}}
\addtokomafont{chapter}{\color{crimson}}
\addtokomafont{section}{\color{crimson}}
\addtokomafont{subsection}{\color{crimson}}

% =============================================================================
% ABSTRACT ENVIRONMENT
% =============================================================================
\newenvironment{abstract}{
  \chapter*{\abstractname}
  \addcontentsline{toc}{chapter}{\abstractname}
  \small
}{
  \clearpage
}

% =============================================================================
% HYPERLINK CONFIGURATION
% =============================================================================
% Crimson-colored links throughout, two-page PDF layout
\hypersetup{
  linkcolor=crimson,
  citecolor=crimson,
  urlcolor=crimson,
  pdfpagelayout=TwoPageRight,   % Two-page spread view
  pdfstartview=Fit               % Initial zoom fits page
}

% =============================================================================
% PART SUMMARY SYSTEM
% =============================================================================
% Allows adding descriptive text below part titles
\newcommand{\partsummary}{}     % Empty by default
\newif\ifhaspartsummary%
\haspartsummaryfalse%

\newcommand{\setpartsummary}[1]{%
  \renewcommand{\partsummary}{#1}%
  \haspartsummarytrue%
}

% Additional colors for part page backgrounds
\definecolor{BrownLL}{RGB}{233,222,220}
\definecolor{BlueDD}{RGB}{62,100,125}
\colorlet{BlueDD}{magenta}

% ===============================================================================
% PART STYLING SYSTEM
% ===============================================================================
%
% This system provides three distinct visual styles for book organization:
%
% 1. NUMBERED PARTS (\part{title}) - For main book sections
%    - Roman numerals (I, II, III, etc.) in top right corner
%    - Crimson title with horizontal lines above/below
%    - "Part I" label in sidebar
%    - Used for: foundations, principles, optimization, deployment, etc.
%
% 2. UNNUMBERED PARTS (\part*{title}) - For special sections like "Labs"
%    - Division-style geometric background (left side)
%    - No Roman numerals
%    - Used for: labs section
%
% 3. DIVISIONS (\division{title}) - For major book divisions
%    - Clean geometric background with centered title
%    - Used for: frontmatter, main_content, backmatter
%
% The Lua filter (inject-parts.lua) automatically routes parts by {key:xxx} commands
% to the appropriate LaTeX command based on the key name.
% ===============================================================================

% NUMBERED PARTS: Roman numeral styling for main book sections
\titleformat{\part}[display]
{\thispagestyle{empty}}{}{20pt}{
\begin{tikzpicture}[remember picture,overlay]
%%%
%%
\node[crimson,align=flush right,
inner sep=0,outer sep=0mm,draw=none,%
anchor=east,minimum height=31mm, text width=1.2\textwidth,
yshift=-30mm,font={%
\fontsize{98pt}{104}\selectfont\bfseries}]  (BG) at (current page text area.north east){\thepart};
%
\node[black,inner sep=0mm,draw=none,
anchor=mid,text width=1.2\textwidth,
 minimum height=35mm, align=right,
node distance=7mm,below=of BG,
font={\fontsize{30pt}{34}\selectfont}]
(BGG)  {\hyphenchar\font=-1 \color{black}\MakeUppercase {#1}};
\draw [crimson,line width=3pt] ([yshift=0mm]BGG.north west) -- ([yshift=0mm]BGG.north east);
\draw [crimson,line width=2pt] ([yshift=0mm]BGG.south west) -- ([yshift=0mm]BGG.south east);
%
\node[fill=crimson,text=white,rotate=90,%
anchor=south west,minimum height=15mm,
minimum width=40mm,font={%
\fontsize{20pt}{20}\selectfont\bfseries}](BP)  at
(current page text area.south east)
{{\sffamily Part}~\thepart};
%
\path[red](BP.north west)-|coordinate(PS)(BGG.south west);
%
% Part summary box commented out for cleaner design
% \ifhaspartsummary
% \node[inner sep=4pt,text width=0.7\textwidth,draw=none,fill=BrownLL!40,
% align=justify,font={\fontsize{9pt}{12}\selectfont},anchor=south west]
% at (PS) {\partsummary};
% \fi
\end{tikzpicture}
}[]

\renewcommand{\thepart}{\Roman{part}}

% UNNUMBERED PARTS: Division-style background for special sections
\titleformat{name=\part,numberless}[display]
{\thispagestyle{empty}}{}{20pt}{
\begin{tikzpicture}[remember picture,overlay]
%%%
\coordinate(S1)at([yshift=-200mm]current page.north west);
\draw[draw=none,fill=BlueDD!7](S1)--++(45:16)coordinate(S2)-
|(S2|-current page.north west)--(current page.north west)coordinate(S3)--(S1);
%
\coordinate(E1)at([yshift=-98mm]current page.north west);
\draw[draw=none,fill=BlueDD!15](E1)--(current page.north west)coordinate(E2)
--++(0:98mm)coordinate(E3)--(E1);
%
\coordinate(D1)at([yshift=15mm]current page.south west);
\draw[draw=none,fill=BlueDD!40,opacity=0.5](D1)--++(45:5.5)coordinate(D2)
-|(D2|-current page.north west)--(current page.north west)coordinate(D3)--(D1);
%%%%
\path[red](S2)-|(S2-|current page.east)coordinate(SS2);
%PART
\node[crimson,align=flush right,inner sep=0,outer sep=0mm,draw=none,anchor=south,
font={\fontsize{48pt}{48}\selectfont\bfseries}]  (BG) at ($(S2)!0.5!(SS2)$){\hphantom{Part}};
%%%
\path[green]([yshift=15mm]D2)-|coordinate(TPD)(BG.south east);
\node[inner sep=0mm,draw=none,anchor=south east,%text width=0.9\textwidth,
align=right,font={\fontsize{40pt}{40}\selectfont}]
(BGG) at (TPD)  {\color{crimson}\MakeUppercase {#1}};%\MakeUppercase {}
\end{tikzpicture}
}

% Define \numberedpart command for numbered parts
\newcommand{\numberedpart}[1]{%
\FloatBarrier%  % Flush all pending floats before part break
\clearpage
\thispagestyle{empty}
\stepcounter{part}%
\begin{tikzpicture}[remember picture,overlay]
%%%
%%
\node[crimson,align=flush right,
inner sep=0,outer sep=0mm,draw=none,%
anchor=east,minimum height=31mm, text width=1.2\textwidth,
yshift=-30mm,font={%
\fontsize{98pt}{104}\selectfont\bfseries}]  (BG) at (current page text area.north east){\thepart};
%
\node[black,inner sep=0mm,draw=none,
anchor=mid,text width=1.2\textwidth,
 minimum height=35mm, align=right,
node distance=7mm,below=of BG,
font={\fontsize{30pt}{34}\selectfont}]
(BGG)  {\hyphenchar\font=-1 \color{black}\MakeUppercase {#1}};
\draw [crimson,line width=3pt] ([yshift=0mm]BGG.north west) -- ([yshift=0mm]BGG.north east);
\draw [crimson,line width=2pt] ([yshift=0mm]BGG.south west) -- ([yshift=0mm]BGG.south east);
%
\node[fill=crimson,text=white,rotate=90,%
anchor=south west,minimum height=15mm,
minimum width=40mm,font={%
\fontsize{20pt}{20}\selectfont\bfseries}](BP)  at
(current page text area.south east)
{{\sffamily Part}~\thepart};
%
\path[red](BP.north west)-|coordinate(PS)(BGG.south west);
%
% Part summary box commented out for cleaner design
% \ifhaspartsummary
% \node[inner sep=4pt,text width=0.7\textwidth,draw=none,fill=BrownLL!40,
% align=justify,font={\fontsize{9pt}{12}\selectfont},anchor=south west]
% at (PS) {\partsummary};
% \fi
\end{tikzpicture}
\clearpage
}



% DIVISIONS: Clean geometric styling with subtle tech elements
% Used for frontmatter, main_content, and backmatter divisions
\newcommand{\division}[1]{%
\FloatBarrier%  % Flush all pending floats before division break
\clearpage
\thispagestyle{empty}
\begin{tikzpicture}[remember picture,overlay]

% Clean geometric background (original design)
\coordinate(S1)at([yshift=-200mm]current page.north west);
\draw[draw=none,fill=BlueDD!7](S1)--++(45:16)coordinate(S2)-
|(S2|-current page.north west)--(current page.north west)coordinate(S3)--(S1);

\coordinate(E1)at([yshift=-98mm]current page.north west);
\draw[draw=none,fill=BlueDD!15](E1)--(current page.north west)coordinate(E2)
--++(0:98mm)coordinate(E3)--(E1);

\coordinate(D1)at([yshift=15mm]current page.south west);
\draw[draw=none,fill=BlueDD!40,opacity=0.5](D1)--++(45:5.5)coordinate(D2)
-|(D2|-current page.north west)--(current page.north west)coordinate(D3)--(D1);

% Subtle tech elements - positioned in white areas for better visibility
% Upper right white area - more visible
\draw[crimson!40, line width=0.8pt] ([xshift=140mm,yshift=-60mm]current page.north west) -- ++(40mm,0);
\draw[crimson!40, line width=0.8pt] ([xshift=150mm,yshift=-70mm]current page.north west) -- ++(30mm,0);
\draw[crimson!35, line width=0.7pt] ([xshift=160mm,yshift=-60mm]current page.north west) -- ++(0,-15mm);
\draw[crimson!35, line width=0.7pt] ([xshift=170mm,yshift=-70mm]current page.north west) -- ++(0,10mm);

% Circuit nodes - upper right
\fill[crimson!50] ([xshift=160mm,yshift=-60mm]current page.north west) circle (1.5mm);
\fill[white] ([xshift=160mm,yshift=-60mm]current page.north west) circle (0.8mm);
\fill[crimson!50] ([xshift=170mm,yshift=-70mm]current page.north west) circle (1.3mm);
\fill[white] ([xshift=170mm,yshift=-70mm]current page.north west) circle (0.6mm);

% Lower right white area - enhanced visibility
\draw[crimson!45, line width=0.9pt] ([xshift=140mm,yshift=-190mm]current page.north west) -- ++(45mm,0);
\draw[crimson!45, line width=0.9pt] ([xshift=150mm,yshift=-200mm]current page.north west) -- ++(35mm,0);
\draw[crimson!40, line width=0.8pt] ([xshift=160mm,yshift=-190mm]current page.north west) -- ++(0,-20mm);
\draw[crimson!40, line width=0.8pt] ([xshift=170mm,yshift=-200mm]current page.north west) -- ++(0,15mm);

% Additional connecting lines in lower right
\draw[crimson!35, line width=0.7pt] ([xshift=130mm,yshift=-180mm]current page.north west) -- ++(25mm,0);
\draw[crimson!35, line width=0.7pt] ([xshift=145mm,yshift=-180mm]current page.north west) -- ++(0,-25mm);

% Circuit nodes - lower right (more prominent)
\fill[crimson!55] ([xshift=160mm,yshift=-190mm]current page.north west) circle (1.6mm);
\fill[white] ([xshift=160mm,yshift=-190mm]current page.north west) circle (0.9mm);
\fill[crimson!55] ([xshift=170mm,yshift=-200mm]current page.north west) circle (1.4mm);
\fill[white] ([xshift=170mm,yshift=-200mm]current page.north west) circle (0.7mm);
\fill[crimson!50] ([xshift=145mm,yshift=-180mm]current page.north west) circle (1.2mm);
\fill[white] ([xshift=145mm,yshift=-180mm]current page.north west) circle (0.6mm);

% Title positioned in center - clean and readable
\node[inner sep=0mm,draw=none,anchor=center,text width=0.8\textwidth,
align=center,font={\fontsize{40pt}{40}\selectfont}]
(BGG) at (current page.center)  {\color{crimson}\MakeUppercase {#1}};

\end{tikzpicture}
\clearpage
}

% LAB DIVISIONS: Circuit-style neural network design for lab sections
% Used specifically for lab platform sections (arduino, xiao, grove, etc.)
\newcommand{\labdivision}[1]{%
\FloatBarrier%  % Flush all pending floats before lab division break
\clearpage
\thispagestyle{empty}
\begin{tikzpicture}[remember picture,overlay]
% Circuit background with subtle gradient
\coordinate(S1)at([yshift=-200mm]current page.north west);
\draw[draw=none,fill=BlueDD!5](S1)--++(45:16)coordinate(S2)-
|(S2|-current page.north west)--(current page.north west)coordinate(S3)--(S1);

% TOP AREA: Circuit lines in upper white space
\draw[crimson!50, line width=1.5pt] ([xshift=30mm,yshift=-40mm]current page.north west) -- ++(60mm,0);
\draw[crimson!40, line width=1pt] ([xshift=120mm,yshift=-50mm]current page.north west) -- ++(50mm,0);
\draw[crimson!50, line width=1.5pt] ([xshift=40mm,yshift=-70mm]current page.north west) -- ++(40mm,0);

% Connecting lines in top area
\draw[crimson!30, line width=1pt] ([xshift=60mm,yshift=-40mm]current page.north west) -- ++(0,-20mm);
\draw[crimson!30, line width=1pt] ([xshift=145mm,yshift=-50mm]current page.north west) -- ++(0,10mm);

% Neural nodes in top area
\fill[crimson!70] ([xshift=60mm,yshift=-40mm]current page.north west) circle (2.5mm);
\fill[white] ([xshift=60mm,yshift=-40mm]current page.north west) circle (1.5mm);
\fill[crimson!60] ([xshift=145mm,yshift=-50mm]current page.north west) circle (2mm);
\fill[white] ([xshift=145mm,yshift=-50mm]current page.north west) circle (1mm);
\fill[crimson!80] ([xshift=80mm,yshift=-70mm]current page.north west) circle (2mm);
\fill[white] ([xshift=80mm,yshift=-70mm]current page.north west) circle (1mm);

% BOTTOM AREA: Circuit lines in lower white space
\draw[crimson!50, line width=1.5pt] ([xshift=20mm,yshift=-200mm]current page.north west) -- ++(70mm,0);
\draw[crimson!40, line width=1pt] ([xshift=110mm,yshift=-210mm]current page.north west) -- ++(60mm,0);
\draw[crimson!50, line width=1.5pt] ([xshift=35mm,yshift=-230mm]current page.north west) -- ++(45mm,0);

% Connecting lines in bottom area
\draw[crimson!30, line width=1pt] ([xshift=55mm,yshift=-200mm]current page.north west) -- ++(0,-20mm);
\draw[crimson!30, line width=1pt] ([xshift=140mm,yshift=-210mm]current page.north west) -- ++(0,15mm);

% Neural nodes in bottom area
\fill[crimson!70] ([xshift=55mm,yshift=-200mm]current page.north west) circle (2.5mm);
\fill[white] ([xshift=55mm,yshift=-200mm]current page.north west) circle (1.5mm);
\fill[crimson!60] ([xshift=140mm,yshift=-210mm]current page.north west) circle (2mm);
\fill[white] ([xshift=140mm,yshift=-210mm]current page.north west) circle (1mm);
\fill[crimson!80] ([xshift=80mm,yshift=-230mm]current page.north west) circle (2mm);
\fill[white] ([xshift=80mm,yshift=-230mm]current page.north west) circle (1mm);

% SIDE AREAS: Subtle circuit elements on left and right edges
\draw[crimson!30, line width=1pt] ([xshift=15mm,yshift=-120mm]current page.north west) -- ++(20mm,0);
\draw[crimson!30, line width=1pt] ([xshift=175mm,yshift=-130mm]current page.north west) -- ++(15mm,0);
\fill[crimson!50] ([xshift=25mm,yshift=-120mm]current page.north west) circle (1.5mm);
\fill[white] ([xshift=25mm,yshift=-120mm]current page.north west) circle (0.8mm);
\fill[crimson!50] ([xshift=185mm,yshift=-130mm]current page.north west) circle (1.5mm);
\fill[white] ([xshift=185mm,yshift=-130mm]current page.north west) circle (0.8mm);

% Title positioned in center - CLEAN AREA
\node[inner sep=0mm,draw=none,anchor=center,text width=0.8\textwidth,
align=center,font={\fontsize{44pt}{44}\selectfont\bfseries}]
(BGG) at (current page.center)  {\color{crimson}\MakeUppercase {#1}};

\end{tikzpicture}
\clearpage
}

% Define \lab command for lab styling (different visual treatment)
\newcommand{\lab}[1]{%
\begin{tikzpicture}[remember picture,overlay]
%%%
% Different background pattern for labs
\coordinate(S1)at([yshift=-200mm]current page.north west);
\draw[draw=none,fill=BlueDD!15](S1)--++(45:16)coordinate(S2)-
|(S2|-current page.north west)--(current page.north west)coordinate(S3)--(S1);
%
\coordinate(E1)at([yshift=-98mm]current page.north west);
\draw[draw=none,fill=BlueDD!25](E1)--(current page.north west)coordinate(E2)
--++(0:98mm)coordinate(E3)--(E1);
%
\coordinate(D1)at([yshift=15mm]current page.south west);
\draw[draw=none,fill=BlueDD!60,opacity=0.7](D1)--++(45:5.5)coordinate(D2)
-|(D2|-current page.north west)--(current page.north west)coordinate(D3)--(D1);
%%%%
\path[red](S2)-|(S2-|current page.east)coordinate(SS2);
%LAB - Different styling
\node[crimson,align=flush right,inner sep=0,outer sep=0mm,draw=none,anchor=south,
font={\fontsize{48pt}{48}\selectfont\bfseries}]  (BG) at ($(S2)!0.5!(SS2)$){\hphantom{Workshop}};
%%%
\path[green]([yshift=15mm]D2)-|coordinate(TPD)(BG.south east);
\node[inner sep=0mm,draw=none,anchor=south east,%text width=0.9\textwidth,
align=right,font={\fontsize{40pt}{40}\selectfont}]
(BGG) at (TPD)  {\color{crimson}\MakeUppercase {#1}};%\MakeUppercase {}
\end{tikzpicture}
\thispagestyle{empty}
\clearpage
}

% =============================================================================
% SECTION FORMATTING
% =============================================================================
% All section levels use crimson color and are ragged right

% Section (Large, bold, crimson)
\titleformat{\section}
  {\normalfont\Large\bfseries\color{crimson}\raggedright}
  {\thesection}
  {0.5em}
  {#1}
\titlespacing*{\section}{0pc}{14pt plus 4pt minus 4pt}{6pt plus 2pt minus 2pt}[0pc]

% Subsection (large, bold, crimson)
\titleformat{\subsection}
  {\normalfont\large\bfseries\color{crimson}\raggedright}
  {\thesubsection}
  {0.5em}
  {#1}
\titlespacing*{\subsection}{0pc}{12pt plus 4pt minus 4pt}{5pt plus 1pt minus 2pt}[0pc]

% Subsubsection (normal size, bold, crimson)
\titleformat{\subsubsection}
  {\normalfont\normalsize\bfseries\color{crimson}\raggedright}
  {\thesubsubsection}
  {0.5em}
  {#1}
\titlespacing*{\subsubsection}{0pc}{12pt plus 4pt minus 4pt}{5pt plus 1pt minus 2pt}[0pc]

% Paragraph (run-in, bold, crimson, ends with period)
\titleformat{\paragraph}[runin]
  {\normalfont\normalsize\bfseries\color{crimson}}
  {\theparagraph}
  {0.5em}
  {#1}
  [\textbf{.}]
  \titlespacing*{\paragraph}{0pc}{6pt plus 2pt minus 2pt}{0.5em}[0pc]

% Subparagraph (run-in, italic, crimson, ends with period)
\titleformat{\subparagraph}[runin]
  {\normalfont\normalsize\itshape\color{crimson}}
  {\thesubparagraph}
  {0.5em}
  {#1}
  [\textbf{.}]
  \titlespacing*{\subparagraph}{0pc}{6pt plus 2pt minus 2pt}{0.5em}[0pc]

% =============================================================================
% CHAPTER FORMATTING
% =============================================================================
% Numbered chapters: "Chapter X" prefix, huge crimson title
\titleformat{\chapter}[display]
  {\normalfont\huge\bfseries\color{crimson}}
  {\chaptername\ \thechapter}
  {20pt}
  {\Huge #1}
  []

% Unnumbered chapters: no prefix, huge crimson title
\titleformat{name=\chapter,numberless}
  {\normalfont\huge\bfseries\color{crimson}}
  {}
  {0pt}
  {\Huge #1}
  []

\renewcommand{\chaptername}{Chapter}
% =============================================================================
% TABLE OF CONTENTS FORMATTING
% =============================================================================
\setcounter{tocdepth}{2}                      % Show chapters, sections, subsections

% TOC spacing adjustments for number widths and indentation
\setlength{\cftchapnumwidth}{2em}             % Chapter number width
\setlength{\cftsecnumwidth}{2.75em}           % Section number width
\setlength{\cftsubsecnumwidth}{3.25em}        % Subsection number width
\setlength{\cftsubsubsecnumwidth}{4em}        % Subsubsection number width
\setlength{\cftsubsecindent}{4.25em}          % Subsection indent
\setlength{\cftsubsubsecindent}{7.5em}        % Subsubsection indent

% Chapter entries in TOC: bold crimson with "Chapter" prefix
\renewcommand{\cftchapfont}{\bfseries\color{crimson}}
\renewcommand{\cftchappresnum}{\color{crimson}Chapter~}

% Custom formatting for division entries (styled like parts)
\newcommand{\divisionchapter}[1]{%
  \addvspace{12pt}%
  \noindent\hfil\bfseries\color{crimson}#1\hfil\par%
  \addvspace{6pt}%
}

% Adjust TOC spacing for "Chapter" prefix
\newlength{\xtraspace}
\settowidth{\xtraspace}{\cftchappresnum\cftchapaftersnum}
\addtolength{\cftchapnumwidth}{\xtraspace}

% Unnumbered chapters with TOC entry
\newcommand{\likechapter}[1]{%
    \chapter*{#1}
    \addcontentsline{toc}{chapter}{\textcolor{crimson}{#1}}
}

% =============================================================================
% PAGE NUMBERING SYSTEM
% =============================================================================
% Implements traditional book numbering:
% - Roman numerals (i, ii, iii...) for frontmatter
% - Arabic numerals (1, 2, 3...) for mainmatter
% Automatically switches at first numbered chapter
\makeatletter
\newif\if@firstnumbered%
\@firstnumberedtrue%
\newif\if@firstunnumbered%
\@firstunnumberedtrue%

\newcounter{lastRomanPage}
\setcounter{lastRomanPage}{1}

% Start document with Roman numerals (frontmatter)
\AtBeginDocument{
  \pagenumbering{roman}
  \renewcommand{\thepage}{\roman{page}}
}

% Intercept chapter command
\let\old@chapter\chapter%
\renewcommand{\chapter}{%
  \@ifstar{\unnumbered@chapter}{\numbered@chapter}%
}

% Numbered chapters: switch to Arabic on first occurrence
\newcommand{\numbered@chapter}[1]{%
  \if@firstnumbered%
    \cleardoublepage%
    \setcounter{lastRomanPage}{\value{page}}%
    \pagenumbering{arabic}%
    \@firstnumberedfalse%
  \else
    \setcounter{page}{\value{page}}%
  \fi
  \setcounter{sidenote}{1}                    % Reset footnote counter per chapter
  \old@chapter{#1}%
}

% Unnumbered chapters: stay in Roman numerals
\newcommand{\unnumbered@chapter}[1]{%
  \if@firstunnumbered%
    \clearpage
    \setcounter{lastRomanPage}{\value{page}}%
    \pagenumbering{roman}%
    \@firstunnumberedfalse%
  \fi
  \setcounter{sidenote}{1}
  \old@chapter*{#1}%
}
\makeatother

% =============================================================================
% TABLE SIZING AND SPACING
% =============================================================================
% Make tables slightly smaller to fit more content
\AtBeginEnvironment{longtable}{\scriptsize}

% Increase vertical spacing in table cells (default is 1.0)
\renewcommand{\arraystretch}{1.5}

% Prefer placing figures and tables at the top of pages
\makeatletter
\renewcommand{\fps@figure}{t}  % Default placement: top of page
\renewcommand{\fps@table}{t}   % Default placement: top of page
\makeatother

% =============================================================================
% LONGTABLE PAGE BREAKING FIXES (Windows compatibility)
% =============================================================================
% Prevent "Infinite glue shrinkage" errors on Windows LaTeX builds
% by giving longtable more flexibility in page breaking

% Allow more flexible page breaking (vs strict \flushbottom)
\raggedbottom

% Process more rows before attempting page break (default is 20)
\setcounter{LTchunksize}{50}

% Add extra stretch for longtable environments specifically
\AtBeginEnvironment{longtable}{%
  \setlength{\emergencystretch}{3em}%
  \setlength{\parskip}{0pt plus 1pt}%
}

% =============================================================================
% TABLE STYLING - Clean tables with crimson borders
% =============================================================================
% Professional table appearance with:
% - Clean white background (no colored rows)
% - Crimson-colored borders
% - Good spacing for readability
%
% Note: Headers are automatically bolded by Quarto when using **text** in source
\usepackage{booktabs}      % Professional table rules (\toprule, \midrule, \bottomrule)
\usepackage{colortbl}      % For colored borders (\arrayrulecolor)

% Global table styling - crimson borders
\setlength{\arrayrulewidth}{0.5pt}          % Thinner borders than default
%\arrayrulecolor{crimson}                    % Crimson borders matching brand

\setcounter{chapter}{0}

% =============================================================================
% DROP CAPS (Lettrine)
% =============================================================================
% Decorative large first letter at chapter openings, following the tradition
% of Hennessy & Patterson and other MIT Press textbooks.
% Usage in QMD: \lettrine{T}{he first sentence...}
\usepackage{lettrine}
\renewcommand{\LettrineFontHook}{\color{crimson}\bfseries}
\setcounter{DefaultLines}{3}          % Drop cap spans 3 lines
\renewcommand{\DefaultLoversize}{0.1} % Slight oversize for visual weight
\renewcommand{\DefaultLraise}{0}      % No vertical shift
\setlength{\DefaultNindent}{0.5em}    % Indent of continuation text
\setlength{\DefaultSlope}{0pt}        % No slope on continuation

% =============================================================================
% RUNNING HEADERS — Truncation Safety
% =============================================================================
% Long chapter/section titles can overflow the header. These marks truncate
% gracefully so headers stay within the text block.
\renewcommand{\chaptermark}[1]{%
  \markboth{\thechapter.\ #1}{}}
\renewcommand{\sectionmark}[1]{%
  \markright{\thesection\ #1}}

% =============================================================================
% EPIGRAPH ENVIRONMENT
% =============================================================================
% For chapter-opening quotations. Renders as right-aligned italic block
% with attribution in small caps below.
% Usage: \epigraph{Quote text}{Author Name, \textit{Source}}
\newcommand{\bookepigraph}[2]{%
  \vspace{1em}%
  \begin{flushright}%
    \begin{minipage}{0.75\textwidth}%
      \raggedleft\itshape\small #1\\[0.5em]%
      \normalfont\small --- #2%
    \end{minipage}%
  \end{flushright}%
  \vspace{1.5em}%
}

% =============================================================================
% THUMB INDEX TABS
% =============================================================================
% Colored tabs on the outer page edge for quick chapter navigation.
% Each Part gets a different vertical position; chapters within a Part
% share the same tab position. Visible when flipping through the book.
\newcounter{thumbindex}
\setcounter{thumbindex}{0}
\newlength{\thumbtabheight}
\setlength{\thumbtabheight}{16mm}     % Height of each tab
\newlength{\thumbtabwidth}
\setlength{\thumbtabwidth}{8mm}       % Width protruding from edge
\newlength{\thumbtabgap}
\setlength{\thumbtabgap}{1mm}         % Gap between tabs

% Advance to next thumb tab position (call at each \part)
\newcommand{\nextthumb}{%
  \stepcounter{thumbindex}%
}

% Draw the thumb tab on every page (placed in header via fancyhdr)
\newcommand{\drawthumb}{%
  \ifnum\value{thumbindex}>0%
    \begin{tikzpicture}[remember picture,overlay]
      \pgfmathsetmacro{\thumboffset}{%
        20 + (\value{thumbindex}-1) * (16 + 1)}  % mm from top
      \ifodd\value{page}%
        % Odd pages: tab on right edge
        \fill[crimson!80]
          ([yshift=-\thumboffset mm]current page.north east)
          rectangle +(-\thumbtabwidth, -\thumbtabheight);
        \node[white,font=\tiny\bfseries,rotate=90]
          at ([yshift=-\thumboffset mm - 0.5\thumbtabheight,
               xshift=-0.5\thumbtabwidth]current page.north east)
          {\Roman{thumbindex}};
      \else
        % Even pages: tab on left edge
        \fill[crimson!80]
          ([yshift=-\thumboffset mm]current page.north west)
          rectangle +(\thumbtabwidth, -\thumbtabheight);
        \node[white,font=\tiny\bfseries,rotate=-90]
          at ([yshift=-\thumboffset mm - 0.5\thumbtabheight,
               xshift=0.5\thumbtabwidth]current page.north west)
          {\Roman{thumbindex}};
      \fi
    \end{tikzpicture}%
  \fi
}

% Hook into fancyhdr to draw thumb on every content page
\AddToHook{shipout/foreground}{%
  \drawthumb%
}

% =============================================================================
% CROP / BLEED MARKS
% =============================================================================
% For final print submission, uncomment the line below to add crop marks.
% MIT Press production will advise on exact requirements.
% \usepackage[cam,center,width=7.5in,height=10.5in]{crop}

% =============================================================================
% PDF/A ARCHIVAL COMPLIANCE
% =============================================================================
% MIT Press increasingly requires PDF/A for long-term preservation.
% This embeds all fonts and removes transparency.
% Note: pdfx must be loaded early; if it conflicts with hyperref,
% MIT Press production can handle the conversion post-build.
% Uncomment when ready for final submission:
% \usepackage[a-3u]{pdfx}

% =============================================================================
% ENHANCED WIDOW / ORPHAN CONTROL
% =============================================================================
% Prevent single lines at top/bottom of pages and breaks before equations
\clubpenalty=10000          % No orphans (single first line at bottom)
\widowpenalty=10000         % No widows (single last line at top)
\displaywidowpenalty=10000  % No widow before display math
\predisplaypenalty=10000    % No page break just before display math
\postdisplaypenalty=0       % Allow break after display math (natural)
\usepackage{needspace}
\let\Needspace\needspace
\makeatletter
\@ifpackageloaded{float}{}{\usepackage{float}}
\floatstyle{plain}
\@ifundefined{c@chapter}{\newfloat{vid}{h}{lovid}}{\newfloat{vid}{h}{lovid}[chapter]}
\floatname{vid}{Video}
\newcommand*\listofvids{\listof{vid}{List of Videos}}
\makeatother
\makeatletter
\@ifpackageloaded{tcolorbox}{}{\usepackage[skins,breakable]{tcolorbox}}
\@ifpackageloaded{fontawesome5}{}{\usepackage{fontawesome5}}
\definecolor{quarto-callout-color}{HTML}{909090}
\definecolor{quarto-callout-note-color}{HTML}{0758E5}
\definecolor{quarto-callout-important-color}{HTML}{CC1914}
\definecolor{quarto-callout-warning-color}{HTML}{EB9113}
\definecolor{quarto-callout-tip-color}{HTML}{00A047}
\definecolor{quarto-callout-caution-color}{HTML}{FC5300}
\definecolor{quarto-callout-color-frame}{HTML}{acacac}
\definecolor{quarto-callout-note-color-frame}{HTML}{4582ec}
\definecolor{quarto-callout-important-color-frame}{HTML}{d9534f}
\definecolor{quarto-callout-warning-color-frame}{HTML}{f0ad4e}
\definecolor{quarto-callout-tip-color-frame}{HTML}{02b875}
\definecolor{quarto-callout-caution-color-frame}{HTML}{fd7e14}
\makeatother
\makeatletter
\@ifpackageloaded{bookmark}{}{\usepackage{bookmark}}
\makeatother
\makeatletter
\@ifpackageloaded{caption}{}{\usepackage{caption}}
\AtBeginDocument{%
\ifdefined\contentsname
  \renewcommand*\contentsname{Table of contents}
\else
  \newcommand\contentsname{Table of contents}
\fi
\ifdefined\listfigurename
  \renewcommand*\listfigurename{List of Figures}
\else
  \newcommand\listfigurename{List of Figures}
\fi
\ifdefined\listtablename
  \renewcommand*\listtablename{List of Tables}
\else
  \newcommand\listtablename{List of Tables}
\fi
\ifdefined\figurename
  \renewcommand*\figurename{Figure}
\else
  \newcommand\figurename{Figure}
\fi
\ifdefined\tablename
  \renewcommand*\tablename{Table}
\else
  \newcommand\tablename{Table}
\fi
}
\@ifpackageloaded{float}{}{\usepackage{float}}
\floatstyle{ruled}
\@ifundefined{c@chapter}{\newfloat{codelisting}{h}{lop}}{\newfloat{codelisting}{h}{lop}[chapter]}
\floatname{codelisting}{Listing}
\newcommand*\listoflistings{\listof{codelisting}{List of Listings}}
\makeatother
\makeatletter
\makeatother
\makeatletter
\@ifpackageloaded{caption}{}{\usepackage{caption}}
\@ifpackageloaded{subcaption}{}{\usepackage{subcaption}}
\makeatother
\makeatletter
\@ifpackageloaded{sidenotes}{}{\usepackage{sidenotes}}
\@ifpackageloaded{marginnote}{}{\usepackage{marginnote}}
\makeatother
\newcommand{\fbxIconPath}{assets/images/icons/callouts}
\newcommand{\fbxIconFormat}{pdf}
\makeatletter
\@ifpackageloaded{tcolorbox}{}{\usepackage[many]{tcolorbox}}
\makeatother
%%%% ---foldboxy preamble ----- %%%%%

% Load xstring for string manipulation
\RequirePackage{xstring}

% Icon path and format configuration - can be overridden in filter-metadata
\providecommand{\fbxIconPath}{assets/images/icons/callouts}
\providecommand{\fbxIconFormat}{pdf}

% Helper command to include icon with hyphen-to-underscore conversion
% This ensures consistency: callout-quiz-question -> callout_quiz_question
\newcommand{\fbxIncludeIcon}[2]{%
  \StrSubstitute{#1}{-}{_}[\fbxIconName]%
  \includegraphics[width=#2]{\fbxIconPath/icon_\fbxIconName.\fbxIconFormat}%
}

% Legacy fallback colors (keep for compatibility)
\definecolor{fbx-default-color1}{HTML}{c7c7d0}
\definecolor{fbx-default-color2}{HTML}{a3a3aa}
\definecolor{fbox-color1}{HTML}{c7c7d0}
\definecolor{fbox-color2}{HTML}{a3a3aa}

% arguments: #1 typelabelnummer: #2 titel: #3
\newenvironment{fbx}[3]{%
\begin{tcolorbox}[
  enhanced,
  breakable,
  %fontupper=\fontsize{8pt}{10pt}\selectfont,  % 95% of body text (10pt -> 9.5pt)
  before skip=8pt,  % space above box (increased)
  after skip=8pt,   % space below box (increased)
  attach boxed title to top*={xshift=0pt},
  boxed title style={
  %fuzzy shadow={1pt}{-1pt}{0mm}{0.1mm}{gray},
  arc=1.5pt,
  rounded corners=north,
  sharp corners=south,
  top=6pt,          % Adjusted for ~40px equivalent height
  bottom=5pt,       % Adjusted for ~40px equivalent height
  overlay={
      \node [left,outer sep=0em, black,draw=none,anchor=west,
        rectangle,fill=none,inner sep=0pt]
        at ([xshift=4mm]frame.west) {\fbxIncludeIcon{#1}{4.2mm}};
    },
  },
  colframe=#1-color2,             % Border color (auto-generated from YAML)
  colbacktitle=#1-color1,         % Background color (auto-generated from YAML)
  colback=white,
  coltitle=black,
  titlerule=0mm,
  toprule=0.5pt,
  bottomrule=0.5pt,
  leftrule=2.2pt,
  rightrule=0.5pt,
  outer arc=1.5pt,
  arc=1.5pt,
  left=0.5em,       % increased left padding
  bottomtitle=1.5mm, % increased title bottom margin
  toptitle=1.5mm,    % increased title top margin
  title=\hspace{2.5em}\protect#2\hspace{0.5em}\protect#3, % Protect parameters
  extras middle and last={top=4pt} % increased continuation spacing
]}
{\end{tcolorbox}}


% boxed environment with right border
\newenvironment{fbxSimple}[3]{\begin{tcolorbox}[
  enhanced,
  breakable,
  %fontupper=\fontsize{8pt}{10pt}\selectfont,  % 95% of body text (10pt -> 9.5pt)
  before skip=8pt,  % space above box (increased)
  after skip=8pt,   % space below box (increased)
  attach boxed title to top*={xshift=0pt},
  boxed title style={
  %fuzzy shadow={1pt}{-1pt}{0mm}{0.1mm}{gray},
  arc=1.5pt,
  rounded corners=north,
  sharp corners=south,
  top=6pt,          % Adjusted for ~40px equivalent height
  bottom=5pt,       % Adjusted for ~40px equivalent height
  overlay={
      \node [left,outer sep=0em, black,draw=none,anchor=west,
        rectangle,fill=none,inner sep=0pt]
        at ([xshift=3mm]frame.west) {\fbxIncludeIcon{#1}{4.2mm}};
    },
  },
  colframe=#1-color2,             % Border color (auto-generated from YAML)
  colbacktitle=#1-color1,         % Background color (auto-generated from YAML)
  colback=white,
  coltitle=black,
  titlerule=0mm,
  toprule=0.5pt,
  bottomrule=0.5pt,
  leftrule=2.2pt,
  rightrule=0.5pt,
  outer arc=1.5pt,
  arc=1.5pt,
  left=0.5em,       % increased left padding
  bottomtitle=1.5mm, % increased title bottom margin
  toptitle=1.5mm,    % increased title top margin
  title=\hspace{2.5em}\protect#2\hspace{0.5em}\protect#3, % Protect parameters
  boxsep=1pt,
  extras first={bottom=0pt},
  extras last={top=0pt,bottom=-4pt},
  overlay first={
    \draw[line width=1pt,white] ([xshift=2.2pt]frame.south west)-- ([xshift=-0.5pt]frame.south east);
  },
  overlay last={
    \draw[line width=1pt,white] ([xshift=2.2pt]frame.north west)-- ([xshift=-0.5pt]frame.north east);
   }
]}
{\end{tcolorbox}}

%%%% --- end foldboxy preamble ----- %%%%%
%%==== colors from yaml ===%
\definecolor{callout-chapter-connection-color1}{HTML}{FDF2F7}
\definecolor{callout-chapter-connection-color2}{HTML}{A51C30}
\definecolor{callout-definition-color1}{HTML}{F0F4F8}
\definecolor{callout-definition-color2}{HTML}{1B4F72}
\definecolor{callout-theorem-color1}{HTML}{F5F0FF}
\definecolor{callout-theorem-color2}{HTML}{6B46C1}
\definecolor{callout-resource-exercises-color1}{HTML}{E0F2F1}
\definecolor{callout-resource-exercises-color2}{HTML}{20B2AA}
\definecolor{callout-notebook-color1}{HTML}{F2F7FF}
\definecolor{callout-notebook-color2}{HTML}{2C5282}
\definecolor{callout-principle-color1}{HTML}{F3F2FA}
\definecolor{callout-principle-color2}{HTML}{3D3B8E}
\definecolor{callout-checkpoint-color1}{HTML}{E8F5E9}
\definecolor{callout-checkpoint-color2}{HTML}{2E7D32}
\definecolor{callout-lighthouse-color1}{HTML}{FDF8E6}
\definecolor{callout-lighthouse-color2}{HTML}{B8860B}
\definecolor{callout-quiz-answer-color1}{HTML}{E8F2EA}
\definecolor{callout-quiz-answer-color2}{HTML}{4a7c59}
\definecolor{callout-example-color1}{HTML}{F0F8F6}
\definecolor{callout-example-color2}{HTML}{148F77}
\definecolor{callout-code-color1}{HTML}{F2F4F8}
\definecolor{callout-code-color2}{HTML}{D1D7E0}
\definecolor{callout-takeaways-color1}{HTML}{FDF2F7}
\definecolor{callout-takeaways-color2}{HTML}{BE185D}
\definecolor{callout-resource-slides-color1}{HTML}{E0F2F1}
\definecolor{callout-resource-slides-color2}{HTML}{20B2AA}
\definecolor{callout-perspective-color1}{HTML}{F7F8FA}
\definecolor{callout-perspective-color2}{HTML}{4A5568}
\definecolor{callout-resource-videos-color1}{HTML}{E0F2F1}
\definecolor{callout-resource-videos-color2}{HTML}{20B2AA}
\definecolor{callout-quiz-question-color1}{HTML}{F0F0F8}
\definecolor{callout-quiz-question-color2}{HTML}{5B4B8A}
\definecolor{callout-colab-color1}{HTML}{FFF5E6}
\definecolor{callout-colab-color2}{HTML}{FF6B35}
%=============%

\usepackage{hyphenat}
\usepackage{ifthen}
\usepackage{calc}
\usepackage{calculator}



\usepackage{graphicx}
\usepackage{geometry}
\usepackage{afterpage}
\usepackage{tikz}
\usetikzlibrary{calc}
\usetikzlibrary{fadings}
\usepackage[pagecolor=none]{pagecolor}


% Set the titlepage font families







% Set the coverpage font families

\usepackage{bookmark}
\IfFileExists{xurl.sty}{\usepackage{xurl}}{} % add URL line breaks if available
\urlstyle{same}
\hypersetup{
  pdftitle={Introduction to Machine Learning Systems},
  pdfauthor={Vijay Janapa Reddi},
  colorlinks=true,
  linkcolor={Maroon},
  filecolor={Maroon},
  citecolor={Blue},
  urlcolor={Blue},
  pdfcreator={LaTeX via pandoc}}


\title{Introduction to Machine Learning Systems}
\author{Vijay Janapa Reddi}
\date{February 1, 2026}
\begin{document}
%%%%% begin titlepage extension code

  \begin{frontmatter}

\begin{titlepage}
% This is a combination of Pandoc templating and LaTeX
% Pandoc templating https://pandoc.org/MANUAL.html#templates
% See the README for help

\thispagestyle{empty}

\newgeometry{top=-100in}

% Page color

\newcommand{\coverauthorstyle}[1]{{\fontsize{20}{24.0}\selectfont
{#1}}}

\begin{tikzpicture}[remember picture, overlay, inner sep=0pt, outer sep=0pt]

\tikzfading[name=fadeout, inner color=transparent!0,outer color=transparent!100]
\tikzfading[name=fadein, inner color=transparent!100,outer color=transparent!0]
\node[anchor=south west, rotate=0, opacity=1] at ($(current page.south west)+(0.225\paperwidth, 9)$) {
\includegraphics[width=\paperwidth, keepaspectratio]{assets/images/covers/cover-image-transparent-vol1.png}};

% Title
\newcommand{\titlelocationleft}{0.075\paperwidth}
\newcommand{\titlelocationbottom}{0.4\paperwidth}
\newcommand{\titlealign}{left}

\begin{scope}{%
\fontsize{52}{62.4}\selectfont
\node[anchor=north
west, align=left, rotate=0] (Title1) at ($(current page.south west)+(\titlelocationleft,\titlelocationbottom)$)  [text width = 0.9\paperwidth]  {{\nohyphens{Machine
Learning Systems}}};
}
\end{scope}

% Author
\newcommand{\authorlocationleft}{.925\paperwidth}
\newcommand{\authorlocationbottom}{0.175\paperwidth}
\newcommand{\authoralign}{right}

\begin{scope}
{%
\fontsize{20}{24.0}\selectfont
\node[anchor=north
east, align=right, rotate=0] (Author1) at ($(current page.south west)+(\authorlocationleft,\authorlocationbottom)$)  [text width = 6in]  {\coverauthorstyle{Vijay
Janapa Reddi\\}};
}
\end{scope}

% Footer
\newcommand{\footerlocationleft}{0.075\paperwidth}
\newcommand{\footerlocationbottom}{0.475\paperwidth}
\newcommand{\footerlocationalign}{left}

\begin{scope}
{%
\fontsize{25}{30.0}\selectfont
 \node[anchor=north west, align=left, rotate=0] (Footer1) at %
($(current page.south west)+(\footerlocationleft,\footerlocationbottom)$)  [text width = 0.9\paperwidth]  {{\nohyphens{Introduction
to}}};
}
\end{scope}

\end{tikzpicture}
\clearpage
\restoregeometry
%%% TITLE PAGE START

% Set up alignment commands
%Page
\newcommand{\titlepagepagealign}{
\ifthenelse{\equal{left}{right}}{\raggedleft}{}
\ifthenelse{\equal{left}{center}}{\centering}{}
\ifthenelse{\equal{left}{left}}{\raggedright}{}
}


\newcommand{\titleandsubtitle}{
% Title and subtitle
{{\huge{\bfseries{\nohyphens{Introduction to Machine Learning
Systems}}}}\par
}%
}
\newcommand{\titlepagetitleblock}{
\titleandsubtitle
}

\newcommand{\authorstyle}[1]{{\large{#1}}}

\newcommand{\affiliationstyle}[1]{{\large{#1}}}

\newcommand{\titlepageauthorblock}{
{\authorstyle{\nohyphens{Vijay Janapa
Reddi}{\textsuperscript{1}}\textsuperscript{,}{\textsuperscript{,*}}}}}

\newcommand{\titlepageaffiliationblock}{
\hangindent=1em
\hangafter=1
{\affiliationstyle{
{1}.~Harvard University


\vspace{1\baselineskip}
* \textit{Correspondence:}~Vijay Janapa Reddi~vj@eecs.harvard.edu
}}
}
\newcommand{\headerstyled}{%
{}
}
\newcommand{\footerstyled}{%
{\large{}}
}
\newcommand{\datestyled}{%
{February 1, 2026}
}


\newcommand{\titlepageheaderblock}{\headerstyled}

\newcommand{\titlepagefooterblock}{
\footerstyled
}

\newcommand{\titlepagedateblock}{
\datestyled
}

%set up blocks so user can specify order
\newcommand{\titleblock}{{

{\titlepagetitleblock}
}

\vspace{4\baselineskip}
}

\newcommand{\authorblock}{{\titlepageauthorblock}

\vspace{2\baselineskip}
}

\newcommand{\affiliationblock}{{\titlepageaffiliationblock}

\vspace{0pt}
}

\newcommand{\logoblock}{}

\newcommand{\footerblock}{}

\newcommand{\dateblock}{{\titlepagedateblock}

\vspace{0pt}
}

\newcommand{\headerblock}{}

\thispagestyle{empty} % no page numbers on titlepages


\newcommand{\vrulecode}{\textcolor{black}{\rule{\vrulewidth}{\textheight}}}
\newlength{\vrulewidth}
\setlength{\vrulewidth}{2pt}
\newlength{\B}
\setlength{\B}{\ifdim\vrulewidth > 0pt 0.05\textwidth\else 0pt\fi}
\newlength{\minipagewidth}
\ifthenelse{\equal{left}{left} \OR \equal{left}{right} }
{% True case
\setlength{\minipagewidth}{\textwidth - \vrulewidth - \B - 0.1\textwidth}
}{
\setlength{\minipagewidth}{\textwidth - 2\vrulewidth - 2\B - 0.1\textwidth}
}
\ifthenelse{\equal{left}{left} \OR \equal{left}{leftright}}
{% True case
\raggedleft % needed for the minipage to work
\vrulecode
\hspace{\B}
}{%
\raggedright % else it is right only and width is not 0
}
% [position of box][box height][inner position]{width}
% [s] means stretch out vertically; assuming there is a vfill
\begin{minipage}[b][\textheight][s]{\minipagewidth}
\titlepagepagealign
\titleblock

Prof.~Vijay Janapa Reddi

School of Engineering and Applied Sciences

Harvard University

\vspace{80mm}

With heartfelt gratitude to the community for their invaluable
contributions and steadfast support.

\vfill

February 1, 2026

\vfill
\par

\end{minipage}\ifthenelse{\equal{left}{right} \OR \equal{left}{leftright} }{
\hspace{\B}
\vrulecode}{}
\clearpage
%%% TITLE PAGE END
\end{titlepage}
\setcounter{page}{1}
\end{frontmatter}

%%%%% end titlepage extension code

% =============================================================================
% HALF-TITLE PAGE (Volume I)
% =============================================================================
% Standard academic book sequence: half-title -> blank -> title page -> copyright
% The half-title shows only the book title -- no author, no publisher, no date.
\thispagestyle{empty}
\begin{center}
\vspace*{0.3\textheight}
{\fontsize{24pt}{28pt}\selectfont\bfseries\color{crimson} Introduction to\\[0.4em] Machine Learning Systems}\\[2em]
{\large\itshape Volume~I}
\vfill
\end{center}
\clearpage
\thispagestyle{empty}\null\clearpage  % Blank verso (back of half-title)

\renewcommand*\contentsname{Table of contents}
{
\hypersetup{linkcolor=}
\setcounter{tocdepth}{2}
\tableofcontents
}
\listoffigures
\listoftables

\mainmatter
\bookmarksetup{startatroot}

\chapter*{Welcome to Volume I}\label{welcome-to-volume-i}
\addcontentsline{toc}{chapter}{Welcome to Volume I}

\markboth{Welcome to Volume I}{Welcome to Volume I}

\bookmarksetup{startatroot}

\chapter{Conclusion}\label{sec-conclusion}

\marginnote{\begin{footnotesize}

\emph{DALL·E 3 Prompt: An image depicting a concluding chapter of an ML
systems book, open to a two-page spread. The pages summarize key
concepts such as neural networks, model architectures, hardware
acceleration, and MLOps. One page features a diagram of a neural network
and different model architectures, while the other page shows
illustrations of hardware components for acceleration and MLOps
workflows. The background includes subtle elements like circuit patterns
and data points to reinforce the technological theme. The colors are
professional and clean, with an emphasis on clarity and understanding.}

\end{footnotesize}}

\noindent
\pandocbounded{\includegraphics[keepaspectratio]{contents/vol1/conclusion/images/png/cover_conclusion.png}}

\section*{Purpose}\label{purpose}
\addcontentsline{toc}{section}{Purpose}

\markright{Purpose}

\emph{Why does building machine learning systems require synthesizing
principles from across the entire engineering stack rather than
mastering individual components in isolation?}

Data pipelines, training, compression, acceleration, serving,
operations, responsible deployment---each addresses a specific
challenge. But the systems that actually work in production are not
collections of independently optimized components but integrated wholes
where decisions in one domain propagate constraints to every other. A
model architecture choice determines memory requirements that constrain
hardware selection, which influences quantization strategy, which
affects accuracy, which feeds back to architecture design. An engineer
who optimizes training without considering serving builds models that
cannot be deployed. An engineer who selects hardware without
understanding workload characteristics wastes money on capabilities that
will never be used. An engineer who ignores operational requirements
builds systems that work in demos but fail in production. The discipline
of ML systems engineering is the discipline of seeing these
connections---understanding that every choice opens some paths and
closes others, that \emph{optimization in isolation produces local
maxima that are global failures}, and that the principles governing good
decisions transcend the specific technologies that implement them.

\begin{tcolorbox}[enhanced jigsaw, opacityback=0, colbacktitle=quarto-callout-tip-color!10!white, colback=white, title=\textcolor{quarto-callout-tip-color}{\faLightbulb}\hspace{0.5em}{Learning Objectives}, titlerule=0mm, bottomrule=.15mm, opacitybacktitle=0.6, colframe=quarto-callout-tip-color-frame, breakable, bottomtitle=1mm, leftrule=.75mm, arc=.35mm, left=2mm, coltitle=black, toprule=.15mm, toptitle=1mm, rightrule=.15mm]

\begin{itemize}
\tightlist
\item
  Synthesize the twelve quantitative invariants introduced across Parts
  I through IV into an integrated framework governed by the Conservation
  of Complexity
\item
  Analyze how these invariants manifest across technical foundations,
  performance at scale, and production reality
\item
  Assess how data pipelines, training, model architectures, hardware
  acceleration, and operations interconnect in integrated ML systems
\item
  Evaluate trade-offs between deployment contexts by applying multiple
  principles to assess scalability, efficiency, and reliability
\item
  Critique how technical choices in efficiency, security, and
  sustainability affect democratization, accessibility, and
  environmental impact
\item
  Formulate strategies for applying systems thinking to emerging
  challenges in robust AI, compound systems, and artificial general
  intelligence
\end{itemize}

\end{tcolorbox}

\section{Synthesizing ML Systems Engineering: From Components to
Intelligence}\label{sec-conclusion-synthesizing-ml-systems-engineering-components-intelligence-29b5}

The Introduction to this volume posed a foundational question:
\emph{why} does building machine learning systems require engineering
principles fundamentally different from those governing traditional
software? Every chapter since has answered a piece of that question, and
the answer is deeper than any single component could reveal.

This volume began with a simple mathematical formula: the \textbf{Iron
Law of ML Systems} (\textbf{?@sec-silicon-contract}). At the time, the
terms \textbf{Data Movement}, \textbf{Compute}, and \textbf{Overhead}
may have seemed abstract. Today, they are \emph{your} primary
engineering levers. You have mastered the quantitative analysis of
systems that seemed opaque at the start. You now understand that
building intelligence is not just about writing algorithms; it is about
honoring the \textbf{Silicon Contract}, the \emph{physical and economic
agreement} between the model and the machine.
\textbf{?@sec-ai-acceleration} equipped you to calculate arithmetic
intensity and identify whether your workloads are memory-bound or
compute-bound, transforming vague performance intuitions into
quantitative engineering decisions.

This quantitative foundation reflects a broader truth: contemporary
artificial intelligence\sidenote{\textbf{Artificial Intelligence
(Systems Perspective)}: Intelligence emerging from integrated systems
rather than individual algorithms. Modern AI applications combine data
pipelines (often processing very large corpora), distributed training
(coordinating large accelerator fleets), efficient inference (serving
production traffic), security measures (preventing attacks), and
governance frameworks (ensuring safety). Success depends on systems
engineering excellence across all components. } achievements require
careful integration of interacting components, unifying computational
theory with engineering practice. This systems perspective places
machine learning within the same engineering tradition that built
reliable computers, where transformative capabilities arise from
coordinating many parts together. The Transformer architectures
(\citeproc{ref-vaswani2017attention}{Vaswani et al. 2025}) enabling
large language models exemplify this principle. Their practical utility
depends on integrating mathematical foundations with distributed
training infrastructure, algorithmic optimization techniques, and robust
operational frameworks.

\subsection{The System is the
Model}\label{sec-conclusion-system-model-0103}

We often speak of the ``model'' as the weights file, the 500MB blob of
floating-point numbers. In a production environment, however, the
weights are just one component of the true model.

The \textbf{True Model} is the sum of:

\begin{itemize}
\tightlist
\item
  The \textbf{Data Pipeline} that defines what the model sees.
\item
  The \textbf{Training Infrastructure} that determines what it learns.
\item
  The \textbf{Serving System} that decides how it interacts with the
  world.
\item
  The \textbf{Monitoring Loop} that keeps it tethered to reality.
\end{itemize}

When you optimize the system, you improve the model. When you
\emph{neglect} the system, you \emph{degrade} the model. Systems
Engineering is not a wrapper around ML; it is the implementation of ML.
The system \emph{is} the model.

\phantomsection\label{callout-checkpointux2a-1.1}
\begin{fbx}{callout-checkpoint}{Checkpoint:}{Systems Thinking}
\phantomsection\label{callout-checkpoint*-1.1}

An ML system is greater than the sum of its parts.

\textbf{The Integration}

\begin{itemize}
\tightlist
\item[$\square$]
  \textbf{Dependencies}: Do you understand how a change in the data
  pipeline affects the model's latency?
\item[$\square$]
  \textbf{Feedback Loops}: Have you mapped how the model's predictions
  influence its future training data?
\end{itemize}

\textbf{The Holism}

\begin{itemize}
\tightlist
\item[$\square$]
  \textbf{End-to-End}: Can you trace a user request from the UI, through
  the network, preprocessing, model, postprocessing, and back to the UI?
\end{itemize}

\end{fbx}

This insight has guided our exploration throughout this volume. You now
have theoretical understanding and the conceptual foundation for
professional application. \emph{How} do we translate this understanding
into practice? We need principles: distilled patterns that apply
regardless of which framework you use, which hardware you target, or
which domain you serve.

Before articulating these invariants, let us revisit the journey that
revealed them. This principle---that system boundaries define model
capabilities---manifests throughout the ML development journey.

\subsection{The Lighthouse
Journey}\label{sec-conclusion-lighthouse-journey-c7ee}

Throughout this volume, five Lighthouse Archetypes have served as our
systems detectives, revealing how different workloads expose different
bottlenecks:

\begin{itemize}
\tightlist
\item
  \textbf{ResNet-50} taught us compute-bound optimization: how batch
  size transforms memory-bound inference into compute-bound throughput,
  and why pruning achieves different speedups on different hardware.
\item
  \textbf{GPT-2/Llama} exposed the memory bandwidth wall: why attention
  is memory-bound, how KV-caches dominate serving costs, and why model
  parallelism becomes necessary at scale.
\item
  \textbf{MobileNetV2} demonstrated efficiency under constraint:
  depthwise separable convolutions trading parameters for latency,
  quantization enabling deployment on mobile NPUs, and the Pareto
  frontier between accuracy and power.
\item
  \textbf{DLRM} revealed the embedding table challenge: memory capacity
  as the binding constraint, the unique demands of recommendation
  systems, and why sparse operations behave differently than dense
  matrix multiplication.
\item
  \textbf{Keyword Spotting (KWS)} brought us to the extreme edge:
  sub-megabyte models running on microcontrollers, always-on inference
  under microwatt power budgets, and the TinyML frontier where every
  byte matters.
\end{itemize}

These five workloads span the full deployment spectrum from datacenter
to microcontroller. Together, they have probed every bottleneck and
tested every optimization strategy. The systems thinking you developed
by following these Lighthouses across chapters, from architecture design
through training, optimization, and deployment, is precisely the
integrated perspective that distinguishes ML systems engineering from
isolated algorithm development.

Table~\ref{tbl-lighthouse-journey-mobilenet} traces this journey for a
single model, MobileNetV2, demonstrating how every chapter's principles
converge on a single engineering artifact.

\begin{longtable}[]{@{}
  >{\raggedright\arraybackslash}p{(\linewidth - 4\tabcolsep) * \real{0.3333}}
  >{\raggedright\arraybackslash}p{(\linewidth - 4\tabcolsep) * \real{0.3333}}
  >{\raggedright\arraybackslash}p{(\linewidth - 4\tabcolsep) * \real{0.3333}}@{}}
\caption{\textbf{The Lighthouse Journey (MobileNetV2)}: Tracing one
model through the entire systems stack reveals how decisions in one
domain (e.g., architecture) propagate constraints and opportunities to
every other domain (e.g., hardware acceleration and
monitoring).}\label{tbl-lighthouse-journey-mobilenet}\tabularnewline
\toprule\noalign{}
\begin{minipage}[b]{\linewidth}\raggedright
\textbf{Journey Phase}
\end{minipage} & \begin{minipage}[b]{\linewidth}\raggedright
\textbf{System Lens}
\end{minipage} & \begin{minipage}[b]{\linewidth}\raggedright
\textbf{MobileNetV2 Implementation}
\end{minipage} \\
\midrule\noalign{}
\endfirsthead
\toprule\noalign{}
\begin{minipage}[b]{\linewidth}\raggedright
\textbf{Journey Phase}
\end{minipage} & \begin{minipage}[b]{\linewidth}\raggedright
\textbf{System Lens}
\end{minipage} & \begin{minipage}[b]{\linewidth}\raggedright
\textbf{MobileNetV2 Implementation}
\end{minipage} \\
\midrule\noalign{}
\endhead
\bottomrule\noalign{}
\endlastfoot
\textbf{Foundations (\textbf{?@sec-introduction})} & The AI Triad (DAM)
& Bounded by \textbf{Machine} constraints (Battery/Thermal) \\
\textbf{Architecture (\textbf{?@sec-dnn-architectures})} & Algorithmic
Efficiency & \textbf{Depthwise Separable Convolutions}: 8-9x reduction
in Ops/FLOPs vs ResNet-50 \\
\textbf{Training (\textbf{?@sec-ai-training})} & Throughput vs Latency &
Optimized for \textbf{Single-Stream} throughput; training requires data
augmentation for robustness \\
\textbf{Compression (\textbf{?@sec-model-compression})} & Navigating the
Pareto Frontier & \textbf{INT8 Quantization}: 4x memory reduction with
minimal accuracy loss (\textless1\%) \\
\textbf{Acceleration (\textbf{?@sec-ai-acceleration})} & Honoring the
Silicon Contract & Mapping kernels to \textbf{Mobile NPUs} (e.g., Apple
Neural Engine) to maximize hardware utilization \\
\textbf{Serving (\textbf{?@sec-model-serving-systems})} & Respecting the
Latency Budget & \textbf{P99 \textless{} 50ms} constraint; optimizing
preprocessing (resize/normalize) to avoid CPU bottlenecks \\
\textbf{Operations (\textbf{?@sec-machine-learning-operations-mlops})} &
Managing System Entropy & \textbf{Drift Monitoring}: Detecting accuracy
decay across heterogeneous device populations and lighting conditions \\
\end{longtable}

\emph{What} patterns emerged from this journey? \emph{What} quantitative
invariants transcend the specific technologies and will guide you
regardless of which framework you use or which hardware you target?

\section{The Twelve Quantitative
Invariants}\label{sec-conclusion-twelve-invariants}

Throughout this volume, each Part introduced quantitative principles
that govern ML system behavior. These are not rules of thumb or best
practices that evolve with fashion. They are invariants, constraints
rooted in physics, information theory, and statistics, that hold
regardless of which framework you use or which hardware you target.
Table~\ref{tbl-twelve-principles} collects all twelve in one place,
organized by the four Parts that revealed them.

\begin{longtable}[]{@{}
  >{\raggedright\arraybackslash}p{(\linewidth - 8\tabcolsep) * \real{0.2000}}
  >{\raggedright\arraybackslash}p{(\linewidth - 8\tabcolsep) * \real{0.2000}}
  >{\raggedright\arraybackslash}p{(\linewidth - 8\tabcolsep) * \real{0.2000}}
  >{\raggedright\arraybackslash}p{(\linewidth - 8\tabcolsep) * \real{0.2000}}
  >{\raggedright\arraybackslash}p{(\linewidth - 8\tabcolsep) * \real{0.2000}}@{}}
\caption{\textbf{The Twelve Quantitative Invariants of ML Systems
Engineering.} Each invariant was introduced in the Part where its
governing constraint first becomes visible. Together, they form the
complete analytical framework for reasoning about ML system design,
optimization, and deployment. The meta-principle that unifies them all
is the Conservation of Complexity: you cannot destroy complexity, only
move it between Data, Algorithm, and
Machine.}\label{tbl-twelve-principles}\tabularnewline
\toprule\noalign{}
\begin{minipage}[b]{\linewidth}\raggedright
\textbf{\#}
\end{minipage} & \begin{minipage}[b]{\linewidth}\raggedright
\textbf{Principle}
\end{minipage} & \begin{minipage}[b]{\linewidth}\raggedright
\textbf{Part}
\end{minipage} & \begin{minipage}[b]{\linewidth}\raggedright
\textbf{Core Equation / Statement}
\end{minipage} & \begin{minipage}[b]{\linewidth}\raggedright
\textbf{What It Predicts}
\end{minipage} \\
\midrule\noalign{}
\endfirsthead
\toprule\noalign{}
\begin{minipage}[b]{\linewidth}\raggedright
\textbf{\#}
\end{minipage} & \begin{minipage}[b]{\linewidth}\raggedright
\textbf{Principle}
\end{minipage} & \begin{minipage}[b]{\linewidth}\raggedright
\textbf{Part}
\end{minipage} & \begin{minipage}[b]{\linewidth}\raggedright
\textbf{Core Equation / Statement}
\end{minipage} & \begin{minipage}[b]{\linewidth}\raggedright
\textbf{What It Predicts}
\end{minipage} \\
\midrule\noalign{}
\endhead
\bottomrule\noalign{}
\endlastfoot
1 & Data as Code Invariant & I: Foundations & System Behavior
\(\approx f(\text{Data})\) & Changing data changes the program \\
2 & Data Gravity Invariant & I: Foundations &
\(C_{move}(D) \gg C_{move}(\text{Compute})\) & Move compute to data, not
data to compute \\
3 & Iron Law of ML Systems & II: Build &
\(L = (D_{move} + \text{Compute}\) \(+ \text{Overhead})/\eta\) & Every
optimization pulls one of three levers; reducing one may inflate
another \\
4 & Silicon Contract & II: Build & Every architecture bets on which
hardware resource it saturates & Mismatched hardware wastes money;
matched hardware unlocks peak throughput \\
5 & Pareto Frontier & III: Optimize & Multi-objective optimization; no
free improvements & There is no universal optimum; every gain trades
against another metric \\
6 & Arithmetic Intensity Law & III: Optimize &
\(P = \min(P_{peak},\; I \times B_{mem})\) & Adding compute to a
memory-bound model yields zero gain \\
7 & Energy-Movement Invariant & III: Optimize &
\(E_{move} \gg E_{compute}\) (100-500x) & Data locality, not raw FLOPS,
drives efficiency \\
8 & Amdahl's Law & III: Optimize &
\(\text{Speedup} = 1 / ((1-p) + p/s)\) & The serial fraction caps all
parallelism gains \\
9 & Verification Gap & IV: Deploy & \(P(f(X) \approx Y) > 1 - \epsilon\)
& ML testing is statistical; you bound error, not prove correctness \\
10 & Statistical Drift Invariant & IV: Deploy &
\(\text{Acc}(t) \approx \text{Acc}_0 - \lambda \cdot D(P_t \Vert P_0)\)
& Models decay without code changes; the world drifts away from training
data \\
11 & Training-Serving Skew Law & IV: Deploy &
\(\Delta\text{Acc} \geq \mathbb{E}[\lvert f_{serve}(x) - f_{train}(x)\rvert]\)
& Even subtle preprocessing differences silently degrade accuracy \\
12 & Latency Budget Invariant & IV: Deploy & P99 is the hard constraint;
throughput is optimized within it & Throughput is optimized within the
latency envelope, never at its expense \\
\end{longtable}

These twelve invariants are not independent axioms. They form an
integrated framework held together by a single meta-principle: the
\textbf{Conservation of Complexity}. You cannot destroy complexity in an
ML system; you can only move it between Data, Algorithm, and Machine.
Every invariant in Table~\ref{tbl-twelve-principles} quantifies a
specific consequence of where complexity currently resides. The
following sections trace how each Part's invariants connect to the
Lighthouse archetypes and to each other.

\subsection*{Foundations: Where Complexity Originates (Invariants
1-2)}\label{sec-conclusion-foundations-invariants}
\addcontentsline{toc}{subsection}{Foundations: Where Complexity
Originates (Invariants 1-2)}

The Data as Code Invariant (1) and the Data Gravity Invariant (2)
establish that data is simultaneously the logical program and the
physical anchor of every ML system. ResNet-50 and GPT-2 both illustrate
this duality: their capabilities are determined entirely by \emph{what}
they were trained on, and the datasets that produced them are far more
expensive to move than the models themselves. DLRM makes the point even
more forcefully, since its embedding tables are so large that the system
architecture must be designed around \emph{where} the data physically
resides. These two invariants explain \emph{why}
\textbf{?@sec-data-engineering-ml} devoted an entire chapter to treating
data with the same engineering rigor as source code, and \emph{why} the
``compute-to-data'' pattern recurs in every deployment context from
cloud to edge.

\subsection*{Build: How Complexity Becomes Computation (Invariants
3-4)}\label{sec-conclusion-build-invariants}
\addcontentsline{toc}{subsection}{Build: How Complexity Becomes
Computation (Invariants 3-4)}

The Iron Law (3) and the Silicon Contract (4) govern every decision in
constructing an ML system. The Iron Law decomposes latency into three
competing terms, and the Silicon Contract determines \emph{which} term
dominates for a given architecture-hardware pair. ResNet-50 saturates
floating-point compute on tensor cores (compute-bound). Llama saturates
memory bandwidth during autoregressive decoding (bandwidth-bound). DLRM
saturates memory capacity for its embedding tables (capacity-bound).
MobileNetV2 deliberately reshapes its computation to fit within mobile
NPU constraints. Each Lighthouse archetype represents a different bet
under the Silicon Contract, and \textbf{?@sec-ai-training} showed that
training time reduces only when engineers optimize the dominant term in
the Iron Law rather than distributing effort uniformly.

\subsection*{Optimize: How Constraints Shape Trade-offs (Invariants
5-8)}\label{sec-conclusion-optimize-invariants}
\addcontentsline{toc}{subsection}{Optimize: How Constraints Shape
Trade-offs (Invariants 5-8)}

The four optimization invariants form a tightly coupled diagnostic
chain. The Pareto Frontier (5) establishes that no free improvements
exist: quantization trades precision for bandwidth, pruning trades
capacity for speed, and distillation trades training compute for
inference efficiency. The Arithmetic Intensity Law (6) diagnoses
\emph{which} resource is the bottleneck, revealing whether optimization
should target compute or memory. The Energy-Movement Invariant (7)
explains \emph{why} data locality dominates efficiency, since moving a
bit from DRAM costs 100 to 500 times more energy than computing on it.
Amdahl's Law (8) sets the ceiling on any parallelism gain, explaining
\emph{why} data loading and preprocessing become the ultimate
bottlenecks in highly optimized systems.

MobileNetV2 navigates all four simultaneously: depthwise separable
convolutions reshape the Pareto Frontier, quantization to INT8 exploits
the Arithmetic Intensity Law by fitting more operations per byte of
bandwidth, and the resulting energy savings respect the Energy-Movement
Invariant while Amdahl's Law explains \emph{why} the non-accelerable
preprocessing stage limits end-to-end speedup. The KWS Lighthouse pushes
these trade-offs to their extreme, where sub-megabyte models on
microcontrollers leave zero margin for waste on any axis.

\subsection*{Deploy: How Reality Defeats Assumptions (Invariants
9-12)}\label{sec-conclusion-deploy-invariants}
\addcontentsline{toc}{subsection}{Deploy: How Reality Defeats
Assumptions (Invariants 9-12)}

The deployment invariants address a category of failure that the first
eight invariants cannot prevent: the system works correctly on the bench
but degrades silently in production. The Verification Gap (9)
establishes that ML testing is fundamentally statistical; you bound
error rather than prove correctness. The Statistical Drift Invariant
(10) quantifies \emph{how} accuracy erodes as the world drifts from the
training distribution, even when no code changes. The Training-Serving
Skew Law (11) warns that even subtle differences between training and
serving code paths, a different image resize library, a float32 versus
float64 normalization, silently degrade accuracy. The Latency Budget
Invariant (12) constrains the entire serving architecture: P99 latency
is the hard constraint, and throughput is optimized within that
envelope, never at its expense.

These four invariants explain \emph{why}
\textbf{?@sec-machine-learning-operations-mlops} devoted extensive
attention to monitoring, drift detection, and feature stores. A DLRM
recommendation system that achieves excellent offline accuracy will lose
revenue if training-serving skew corrupts feature values in production
(Invariant 11) or if user behavior drifts seasonally without triggering
retraining (Invariant 10). GPT-2/Llama serving must respect the Latency
Budget (Invariant 12) through techniques like continuous batching and
speculative decoding, because a chatbot that responds in ten seconds is
a chatbot nobody uses.

\subsection*{The Integrated
Framework}\label{sec-conclusion-integrated-framework}
\addcontentsline{toc}{subsection}{The Integrated Framework}

The twelve invariants are not a checklist to be applied sequentially.
They form a web of mutual constraints unified by the Conservation of
Complexity. When you quantize a model (navigating the Pareto Frontier,
Invariant 5), you change its Silicon Contract (Invariant 4), which
shifts where it sits on the Arithmetic Intensity curve (Invariant 6),
which affects its energy profile (Invariant 7). When you deploy that
quantized model (respecting the Latency Budget, Invariant 12), you must
verify that the reduced precision did not introduce training-serving
skew (Invariant 11) and monitor for drift-induced accuracy loss
(Invariant 10) that your statistical tests can only bound (Invariant 9).
The Data Gravity Invariant (2) determines \emph{where} the model runs,
the Data as Code Invariant (1) determines \emph{what} it learned, the
Iron Law (3) determines \emph{how} fast it runs, and Amdahl's Law (8)
determines \emph{how} much faster it can ever run. Complexity is
conserved; the engineer's task is to allocate it wisely.

\begin{figure}[htb]
\centering
\begin{tikzpicture}[
    node distance=3.5cm,
    font=\small\usefont{T1}{phv}{m}{n},
    main/.style={rectangle, rounded corners=10pt, draw, ultra thick, minimum width=2.8cm, minimum height=1.4cm, align=center},
    inv/.style={font=\scriptsize\usefont{T1}{phv}{m}{n}, align=center, fill=white, fill opacity=0.9, text opacity=1, inner sep=2.5pt}
]

% Colors
\definecolor{GreenLine}{HTML}{008F45}
\definecolor{BlueLine}{HTML}{006395}
\definecolor{OrangeLine}{HTML}{E67817}
\definecolor{VioletLine}{HTML}{7E317B}

% Main Nodes
\node[main, draw=GreenLine, fill=GreenLine!10] (Data) at (-4.5, 0) {\normalsize \textbf{Foundations}\\\scriptsize (Data)};
\node[main, draw=BlueLine, fill=BlueLine!10] (Model) at (0, 3.5) {\normalsize \textbf{Build}\\\scriptsize (Model)};
\node[main, draw=OrangeLine, fill=OrangeLine!10] (Hardware) at (4.5, 0) {\normalsize \textbf{Optimize}\\\scriptsize (Hardware)};
\node[main, draw=VioletLine, fill=VioletLine!10] (Ops) at (0, -3.5) {\normalsize \textbf{Deploy}\\\scriptsize (Operations)};

% Center
\node[circle, draw=gray, dashed, ultra thick, align=center, inner sep=10pt, fill=gray!5] (Center) at (0,0) {\small \textbf{Conservation}\\\small \textbf{of}\\\small \textbf{Complexity}};

% Paths with refined curves and labels
\draw[->, line width=2.5pt, GreenLine, bend left=40] (Data) to node[inv, pos=0.5] {\textbf{1. Data as Code}\\\textbf{2. Data Gravity}} (Model);
\draw[->, line width=2.5pt, BlueLine, bend left=40] (Model) to node[inv, pos=0.5] {\textbf{3. Iron Law}\\\textbf{4. Silicon Contract}} (Hardware);
\draw[->, line width=2.5pt, OrangeLine, bend left=40] (Hardware) to node[inv, pos=0.5] {\textbf{5. Pareto Frontier}\\\textbf{6. Arith. Intensity}\\\textbf{7. Energy-Movement}\\\textbf{8. Amdahl's Law}} (Ops);
\draw[->, line width=2.5pt, VioletLine, bend left=40] (Ops) to node[inv, pos=0.5] {\textbf{9. Verification Gap}\\\textbf{10. Stat. Drift}\\\textbf{11. Skew Law}\\\textbf{12. Latency Budget}} (Data);

\end{tikzpicture}
\caption{\textbf{The Cycle of ML Systems (The 12 Invariants)}: The complete systems engineering lifecycle. The meta-principle of \textit{Conservation of Complexity} (center) unifies the process: complexity is neither created nor destroyed, only shifted between Data, Model, Hardware, and Operations. Each transition is governed by specific quantitative invariants that constrain valid engineering decisions.}
\label{fig-invariants-cycle}
\end{figure}

This cycle (\textbf{?@fig-invariants-cycle}) visualizes the perpetual
flow of engineering complexity. Decisions in the \textbf{Build} phase
(governed by the \emph{Iron Law}) constrain the \textbf{Optimize} phase
(bounded by \emph{Arithmetic Intensity}). Operational realities like
\emph{Drift} and \emph{Skew} force feedback into the
\textbf{Foundations}, requiring new data to stabilize the system. The
engineer's role is to manage this flow, ensuring that complexity lands
where it can be handled most efficiently.

These twelve invariants provide a theoretical foundation, but their
value emerges through application. The following sections demonstrate
how these principles guide decisions across different AI domains.

\section{Principles in
Practice}\label{sec-conclusion-applying-principles-across-three-critical-domains-821a}

Throughout this volume, you have seen these twelve invariants manifest
across three areas that connect the Lighthouse archetypes to real
engineering decisions.

\textbf{Building Technical Foundations.} Data quality determines system
quality (\textbf{?@sec-data-engineering-ml}). The Data as Code Invariant
demands that datasets be versioned, tested, and debugged with the same
rigor as source code, which is why ``data is the new code''
(\citeproc{ref-karpathy2017software}{Karpathy 2017}) became a rallying
cry for production ML teams. Mathematical foundations
(\textbf{?@sec-deep-learning-systems-foundations}) established the
computational patterns that drive the Silicon Contract, while framework
selection (\textbf{?@sec-ai-frameworks}) illustrated its practical
consequence: the framework you choose constrains which deployment paths
remain open, because each framework makes different bets on graph
optimization, memory management, and hardware backend support.

\textbf{Engineering for Scale.} Training systems
(\textbf{?@sec-ai-training}) demonstrated the Iron Law in action: data
parallelism reduces the Compute term by distributing work, mixed
precision halves the Data Movement term by using FP16, and gradient
checkpointing trades recomputation for memory capacity. Model
compression (\textbf{?@sec-model-compression}) navigated the Pareto
Frontier directly, with pruning, quantization, and knowledge
distillation each trading one metric for another while the Arithmetic
Intensity Law diagnosed which trade-off would yield the greatest return
for a given hardware target.

\textbf{Navigating Production Reality.} The transition from training to
inference inverts optimization objectives: where training maximizes
throughput over days, inference optimizes latency per request in
milliseconds. The Latency Budget Invariant makes P99 the governing
constraint, and tracking tail latencies reveals that mean latency tells
little about user experience when one in a hundred users waits forty
times longer than average. MLOps\sidenote{\textbf{Machine Learning
Operations (MLOps)}: Engineering discipline applying DevOps principles
to ML systems. MLOps encompasses continuous integration, deployment,
monitoring, and governance at production scale. } orchestrates the full
system lifecycle, transforming the Statistical Drift Invariant and the
Training-Serving Skew Law from abstract equations into actionable
monitoring alerts and automated retraining triggers.

Beyond technical performance, \textbf{?@sec-responsible-engineering}
broadened the framework to include societal impact. The Verification Gap
demands monitoring not just for performance but for fairness violations:
tracking prediction distributions across demographic groups, detecting
bias amplification over time, and alerting on unexplained accuracy
disparities. The Statistical Drift Invariant applies equally to
demographic subgroup performance, where accuracy may degrade for
underrepresented populations even as aggregate metrics remain stable.
These connections reveal that responsible AI is an integral dimension of
systems engineering, not an afterthought but a first-class design
constraint governed by the same invariants that govern performance.

\section{Future Directions and Emerging
Opportunities}\label{sec-conclusion-future-directions-emerging-opportunities-337f}

The twelve invariants you have learned will guide future development
across three emerging frontiers: near-term deployment across diverse
contexts, building resilient systems for societal benefit, and
engineering the path toward artificial general intelligence. Each
frontier tests these invariants in new ways.

\subsection{Applying Principles to Emerging Deployment
Contexts}\label{sec-conclusion-applying-principles-emerging-deployment-contexts-f5eb}

As ML systems move beyond research labs, four deployment paradigms test
different combinations of our quantitative invariants: resource-abundant
cloud environments, resource-constrained edge and mobile devices,
emerging generative AI systems, and ultra-constrained TinyML and
embedded systems.

Cloud deployment prioritizes throughput and scalability, achieving high
GPU utilization through kernel fusion, mixed precision training, and
gradient compression. \textbf{?@sec-model-compression} and
\textbf{?@sec-ai-training} explored these techniques, demonstrating how
they combine to balance performance optimization with cost efficiency at
scale.

In contrast, mobile and edge systems face stringent power, memory, and
latency constraints that demand sophisticated hardware-software
co-design. \textbf{?@sec-model-compression} introduced efficiency
techniques including depthwise separable convolutions, neural
architecture search, and quantization that enable deployment on devices
with 100--1000x less computational power than data centers. Systems that
cannot run on billions of edge devices cannot achieve global impact,
making edge deployment essential for AI
democratization\sidenote{\textbf{AI Democratization}: Making AI
accessible beyond a small number of well-resourced organizations through
efficient systems engineering. Mobile-optimized models and cloud APIs
can widen access, but doing so sustainably requires systematic
optimization across hardware, algorithms, and infrastructure to maintain
quality at scale. }.

Generative AI systems apply the principles at unprecedented scale,
requiring novel approaches to autoregressive computation, dynamic model
partitioning, and speculative decoding\sidenote{\textbf{Speculative
Decoding}: Inference optimization where a smaller draft model generates
candidate tokens that a larger target model verifies in parallel. Since
autoregressive generation is memory-bound (each token requires loading
the full model), speculative decoding trades compute for latency: the
draft model proposes 4-8 tokens; the target verifies them in a single
forward pass. Achieves 2-3x speedup when draft acceptance rates exceed
70\%, making it essential for interactive LLM applications. } that
demonstrate how measurement, optimization, and co-design principles
adapt to emerging technologies pushing infrastructure boundaries.

At the opposite extreme, TinyML and embedded systems face kilobyte
memory budgets, milliwatt power envelopes, and decade-long deployment
lifecycles. Success in these contexts validates the full systems
engineering approach: careful measurement reveals actual bottlenecks,
hardware co-design maximizes efficiency, and planning for failure
ensures reliability despite severe resource limitations. Mobile
deployment constraints have driven breakthrough techniques like
MobileNets and EfficientNets that benefit all AI deployment contexts,
demonstrating how systems constraints catalyze algorithmic innovation.

These deployment contexts confirm the core insight: success depends on
applying the twelve quantitative invariants together rather than
pursuing isolated optimizations.

\subsection{Building Robust AI
Systems}\label{sec-conclusion-building-robust-ai-systems-443a}

Each deployment context we examined assumes systems will function
correctly. What happens when they do not? ML systems face unique failure
modes: distribution shifts degrade accuracy, adversarial inputs exploit
vulnerabilities, and edge cases reveal training data limitations.

The robustness challenge connects directly to two deployment invariants
established earlier. The Verification Gap Invariant (9) reminds us that
ML testing is fundamentally statistical: we bound error rates rather
than prove correctness, meaning some failures will inevitably reach
production. The Statistical Drift Invariant (10) guarantees that even a
perfectly tested system will degrade as the world changes around it,
making continuous monitoring essential rather than optional.

Robustness requires designing for failure from the ground up, combining
redundant hardware for fault tolerance, ensemble methods to reduce
single-point failures, and uncertainty quantification to enable graceful
degradation. As AI systems assume increasingly autonomous roles,
planning for failure becomes the difference between safe deployment and
catastrophic failure. Advanced treatments of these topics explore these
robustness techniques in depth, showing how failure planning scales to
distributed production systems.

\subsection{AI for Societal
Benefit}\label{sec-conclusion-ai-societal-benefit-3796}

Building robust systems is the prerequisite for deploying AI where it
can benefit society. A medical AI that fails unpredictably cannot be
trusted with patient care; an educational system that degrades under
load cannot serve the students who need it most. AI's transformative
potential across healthcare, climate science, education, and
accessibility represents domains where all twelve invariants converge,
and where robustness becomes not just an engineering virtue but an
ethical imperative. Climate modeling requires efficient inference;
medical AI demands explainable decisions and continuous monitoring;
educational technology needs privacy-preserving personalization at
global scale. These applications demonstrate that technical excellence
alone is insufficient; success requires interdisciplinary collaboration
among technologists, domain experts, policymakers, and affected
communities. Specialized studies examine these applications in detail,
showing how the principles you have learned apply to real-world societal
challenges.

\subsection{The Path to AGI}\label{sec-conclusion-path-agi-3fc8}

The most ambitious application of these invariants lies ahead:
engineering the path toward artificial general intelligence. Where
societal benefit applications require robustness within defined domains,
AGI demands systems that generalize across all cognitive tasks while
maintaining the reliability, efficiency, and safety that these
invariants ensure. The architectural approach most likely to achieve
this generalization is what researchers call \emph{compound AI systems}.

\phantomsection\label{callout-definitionux2a-1.2}
\begin{fbx}{callout-definition}{Definition:}{Artificial General Intelligence (AGI)}
\phantomsection\label{callout-definition*-1.2}
\textbf{\emph{Artificial General Intelligence (AGI)}} is a system
capable of \textbf{Universal Cognitive Generalization} at or above human
levels. Unlike \textbf{Narrow AI}, which is optimized for specific
domains, AGI requires the ability to \textbf{Transfer Learning} to novel
situations and reason about unfamiliar problems without specific
retraining.

\end{fbx}

Rather than monolithic models, the most promising path toward such
generalization involves modular architectures that compose specialized
capabilities.

\phantomsection\label{callout-definitionux2a-1.3}
\begin{fbx}{callout-definition}{Definition:}{Compound AI Systems}
\phantomsection\label{callout-definition*-1.3}
\textbf{\emph{Compound AI Systems}} are architectures that chain
multiple models and deterministic tools to achieve \textbf{Reliability}
exceeding their individual components. By decomposing monolithic tasks
into specialized steps (retrieval, reasoning, verification), they trade
\textbf{Latency} and \textbf{Complexity} for \textbf{Control} and
\textbf{Correctness}.

\end{fbx}

The compound AI systems framework provides the architectural blueprint
for advanced intelligence: modular components that can be updated
independently, specialized models optimized for specific tasks, and
decomposable architectures that enable interpretability and safety
through multiple validation layers. The engineering challenges ahead
require mastery across the full stack we have explored, from data
engineering and distributed training to model optimization and
operational infrastructure. These quantitative invariants, not
algorithmic breakthroughs alone, define the path toward artificial
general intelligence, what Hennessy and Patterson have called \emph{a
new golden age} for computer architecture.

\phantomsection\label{callout-perspectiveux2a-1.4}
\begin{fbx}{callout-perspective}{Systems Perspective:}{A New Golden Age}
\phantomsection\label{callout-perspective*-1.4}
\textbf{Engineering the Future}: Hennessy and Patterson
(\citeproc{ref-hennessy_patterson_2019}{Hennessy and Patterson 2019})
declared a \textbf{``New Golden Age for Computer Architecture,''} driven
by the realization that general-purpose processors can no longer sustain
the exponential growth required by AI. Reaching AGI will not be a matter
of writing a better loss function; it will be an epic systems
engineering challenge. It will require a thousand-fold improvement in
energy efficiency, exascale interconnects that operate with the
reliability of a single chip, and software stacks that can manage
trillions of parameters as fluidly as we manage kilobytes today. The
twelve invariants you have learned in this volume, from the \textbf{Iron
Law} and \textbf{Silicon Contract} to the \textbf{Statistical Drift
Invariant} and \textbf{Latency Budget}, are the blueprints for this new
era.

\end{fbx}

To put this in systems terms: achieving \(10^{17}\) FLOPS requires not
just faster chips but fundamentally new approaches to power delivery,
cooling, interconnects, and software coordination. These challenges
await engineers who can apply systems thinking to problems beyond
current imagination.

You are now among those engineers.

Whether or not AGI emerges, the systems principles established
throughout this volume will remain essential. The final sections examine
how these principles evolve to meet future challenges.

\section{Your Journey Forward: Engineering
Intelligence}\label{sec-conclusion-journey-forward-engineering-intelligence-fdd7}

This textbook began by presenting artificial intelligence as a
transformative force reshaping how we build software systems, defining
AI Engineering as the discipline of building \textbf{Stochastic Systems}
with \textbf{Deterministic Reliability}. You now possess the systems
engineering principles to fulfill this mandate. You have learned to
manage the stochastic nature of data through the \emph{Data as Code} and
\emph{Statistical Drift} invariants, while enforcing deterministic
reliability through the \emph{Iron Law}, \emph{Silicon Contract}, and
\emph{Latency Budget}. You have bridged the gap between Software 1.0's
explicit logic and Software 2.0's learned behaviors, mastering the
engineering rigor required to make probabilistic systems dependable.

Intelligence is a systems property that emerges from integrating
components rather than any single breakthrough. Consider GPT-4's success
(\citeproc{ref-openai2023gpt4}{OpenAI et al. 2023}): it required robust
data pipelines processing petabytes of text, distributed training
infrastructure\sidenote{\textbf{Distributed ML Systems}: Traditional
distributed systems principles (consensus, partitioning, replication)
extended for ML workloads. Training very large models can require
coordinating hundreds to thousands of accelerators, where network
topology and gradient synchronization become critical bottlenecks.
Unlike stateless web services, ML systems maintain massive shared state,
motivating techniques like gradient compression and asynchronous
updates. } coordinating thousands of GPUs, efficient architectures
leveraging attention mechanisms and mixture-of-experts, secure
deployment preventing prompt injection attacks\sidenote{\textbf{Prompt
Injection}: Security vulnerability where malicious input manipulates LLM
behavior by embedding instructions that override system prompts. Unlike
SQL injection (which exploits parsing boundaries), prompt injection
exploits the model's inability to distinguish user data from control
instructions. Defenses include input sanitization, output filtering, and
architectural separation between system and user contexts, but no
complete solution exists as of 2024. }, and responsible governance
implementing safety filters and usage policies.

\subsection{The Engineering
Responsibility}\label{sec-conclusion-engineering-responsibility-0348}

Before we look to the horizon of scale, we must ground ourselves in
responsibility. The systems integration perspective explains why ethical
considerations cannot be separated from technical ones. The same
principles that enable efficient systems also determine who can access
them, what harms they might cause, and what benefits they can provide.
The question confronting our generation is not whether artificial
general intelligence will arrive but whether it will be built well:
efficiently enough to democratize access beyond wealthy institutions,
securely enough to resist exploitation, sustainably enough to preserve
our planet, and responsibly enough to serve all humanity equitably.

The intelligent systems that will define the coming decades require your
engineering expertise: climate models predicting extreme weather,
medical AI diagnosing rare diseases, educational systems personalizing
learning, and assistive technologies serving billions. You now possess
the knowledge to build them, the principles to guide design, the
techniques to ensure efficiency, the frameworks to support safe
deployment, and the wisdom to deploy responsibly.

\subsection{The Next Horizon: The Machine Learning
Fleet}\label{sec-conclusion-next-horizon-machine-learning-fleet-c5f1}

The responsibility to build well extends beyond the single machine.
Every principle we have established, from measuring bottlenecks to
co-designing for hardware, was developed within the scope of a single
system. The systems that will define the next decade of AI, however,
operate at a scale where individual machines become components of
something far larger. That transition is not merely an increase in
quantity; it is a qualitative shift in the engineering challenges
involved.

This book has deliberately focused on \textbf{Mastering the ML Node}. We
established principles you can directly observe and experiment with on a
single system. Understanding bottlenecks on one machine (whether memory
bandwidth limitations, CPU-GPU data transfer overhead, or preprocessing
inefficiencies) enables recognition of when and why scaling becomes
necessary. You learned to calculate arithmetic intensity, optimize data
pipelines, and prune models to fit within strict constraints.

As we saw in \textbf{?@sec-ai-training}, however, even a perfectly
optimized node has a physical ceiling. To train the next generation of
foundation models or serve billions of users, we must leave the single
node behind. We must transition from optimizing the individual unit to
\textbf{Orchestrating the ML Fleet}.

This is the frontier of the \textbf{Warehouse-Scale Computer}. In this
regime, the datacenter is no longer a building that houses computers;
the datacenter \emph{is} the computer.

\begin{itemize}
\tightlist
\item
  \textbf{From Bus to Network:} The memory bandwidth constraints we
  studied in \textbf{?@sec-ai-acceleration} expand to become network
  topology challenges. The interconnects between racks become the new
  system bus.
\item
  \textbf{From Failure to Resilience:} Failure planning shifts from
  ``if'' to ``when.'' In a cluster of thousands of GPUs, mean time
  between failures drops to hours. The system must be designed to heal
  itself while computation continues.
\item
  \textbf{From Synchronization to Consensus:} Training shifts from a
  local loop to a distributed consensus problem, where gradient updates
  must be synchronized across a fleet without stalling the math.
\end{itemize}

The transition from Node to Fleet is a fundamental shift in physics. Yet
the foundation remains the same. The Iron Law still governs performance,
but the variables now span racks and zones. The DAM taxonomy still
applies, but the ``Machine'' is now a global infrastructure.

You have mastered the unit. You are now ready to build the collective.

The following points summarize the essential insights from this chapter:

\phantomsection\label{callout-takeawaysux2a-1.5}
\begin{fbx}{callout-takeaways}{Takeaways:}{Key Takeaways}
\phantomsection\label{callout-takeaways*-1.5}

\begin{itemize}
\tightlist
\item
  \textbf{Twelve quantitative invariants define ML systems engineering}:
  From the Data as Code Invariant through the Latency Budget Invariant,
  these principles quantify the constraints that govern every design
  decision, organized across Foundations (data physics), Build
  (computation physics), Optimize (efficiency physics), and Deploy
  (reliability physics).
\item
  \textbf{The Conservation of Complexity unifies all twelve}: You cannot
  destroy complexity in an ML system; you can only move it between Data,
  Algorithm, and Machine. Every invariant quantifies a specific
  consequence of where complexity currently resides.
\item
  \textbf{The system is the model}: The true model is data pipeline +
  training infrastructure + serving system + monitoring loop. Optimize
  the system to improve the model.
\item
  \textbf{Production ML requires continuous operation}: The Statistical
  Drift Invariant and Training-Serving Skew Law guarantee that models
  degrade without code changes. Monitor, measure, and adapt
  continuously.
\item
  \textbf{Technical excellence must combine with ethical commitment}:
  The Verification Gap and drift invariants apply equally to fairness
  metrics. Build systems that are efficient, accessible, sustainable,
  and beneficial.
\end{itemize}

\end{fbx}

The future of intelligence is not a destiny we will simply witness. It
is a system we must engineer. Go build it well.

\vspace{1cm}

\emph{Prof.~Vijay Janapa Reddi, Harvard University}

\FloatBarrier\clearpage

\addtocontents{toc}{\par\addvspace{12pt}\noindent\hfil\bfseries\color{crimson}References\color{black}\hfil\par\addvspace{6pt}}

\addtocontents{toc}{\par\noindent\hfil{\color{crimson}\rule{0.6\textwidth}{0.5pt}}\hfil\par\addvspace{6pt}}

\division{References}

\phantomsection\label{refs}
\begin{CSLReferences}{1}{0}
\bibitem[\citeproctext]{ref-hennessy_patterson_2019}
Hennessy, John L., and David A. Patterson. 2019. {``A New Golden Age for
Computer Architecture.''} \emph{Communications of the ACM} 62 (2):
48--60. \url{https://doi.org/10.1145/3282307}.

\bibitem[\citeproctext]{ref-karpathy2017software}
Karpathy, Andrej. 2017. {``Software 2.0.''} \emph{Medium}.
\url{https://karpathy.medium.com/software-2-0-a64152b37c35}.

\bibitem[\citeproctext]{ref-openai2023gpt4}
OpenAI, Josh Achiam, Steven Adler, Sandhini Agarwal, Lama Ahmad, Ilge
Akkaya, Florencia Leoni Aleman, et al. 2023. {``GPT-4 Technical
Report,''} March. \url{http://arxiv.org/abs/2303.08774v6}.

\bibitem[\citeproctext]{ref-vaswani2017attention}
Vaswani, Ashish, Noam Shazeer, Niki Parmar, Jakob Uszkoreit, Llion
Jones, Aidan N.Gomez, Lukasz Kaiser, and Illia Polosukhin. 2025.
{``Attention Is All You Need.''} Shenzhen Medical Academy of Research;
Translation. \url{https://doi.org/10.65215/ctdc8e75}.

\end{CSLReferences}


\backmatter

\clearpage


\end{document}
