% Options for packages loaded elsewhere
% Options for packages loaded elsewhere
\PassOptionsToPackage{unicode,linktoc=all,pdfwindowui,pdfpagemode=FullScreen,pdfpagelayout=TwoPageRight}{hyperref}
\PassOptionsToPackage{hyphens}{url}
\PassOptionsToPackage{dvipsnames,svgnames,x11names}{xcolor}
%
\documentclass[
  9pt,
  letterpaper,
  abstract,
  titlepage]{scrbook}
\usepackage{xcolor}
\usepackage{amsmath,amssymb}
\setcounter{secnumdepth}{3}
\usepackage{iftex}
\ifPDFTeX
  \usepackage[T1]{fontenc}
  \usepackage[utf8]{inputenc}
  \usepackage{textcomp} % provide euro and other symbols
\else % if luatex or xetex
  \usepackage{unicode-math} % this also loads fontspec
  \defaultfontfeatures{Scale=MatchLowercase}
  \defaultfontfeatures[\rmfamily]{Ligatures=TeX,Scale=1}
\fi
\usepackage{lmodern}
\ifPDFTeX\else
  % xetex/luatex font selection
\fi
% Use upquote if available, for straight quotes in verbatim environments
\IfFileExists{upquote.sty}{\usepackage{upquote}}{}
\IfFileExists{microtype.sty}{% use microtype if available
  \usepackage[]{microtype}
  \UseMicrotypeSet[protrusion]{basicmath} % disable protrusion for tt fonts
}{}
% Make \paragraph and \subparagraph free-standing
\makeatletter
\ifx\paragraph\undefined\else
  \let\oldparagraph\paragraph
  \renewcommand{\paragraph}{
    \@ifstar
      \xxxParagraphStar
      \xxxParagraphNoStar
  }
  \newcommand{\xxxParagraphStar}[1]{\oldparagraph*{#1}\mbox{}}
  \newcommand{\xxxParagraphNoStar}[1]{\oldparagraph{#1}\mbox{}}
\fi
\ifx\subparagraph\undefined\else
  \let\oldsubparagraph\subparagraph
  \renewcommand{\subparagraph}{
    \@ifstar
      \xxxSubParagraphStar
      \xxxSubParagraphNoStar
  }
  \newcommand{\xxxSubParagraphStar}[1]{\oldsubparagraph*{#1}\mbox{}}
  \newcommand{\xxxSubParagraphNoStar}[1]{\oldsubparagraph{#1}\mbox{}}
\fi
\makeatother


\providecommand{\tightlist}{%
  \setlength{\itemsep}{0pt}\setlength{\parskip}{0pt}}\usepackage{longtable,booktabs,array}
\usepackage{calc} % for calculating minipage widths
% Correct order of tables after \paragraph or \subparagraph
\usepackage{etoolbox}
\makeatletter
\patchcmd\longtable{\par}{\if@noskipsec\mbox{}\fi\par}{}{}
\makeatother
% Allow footnotes in longtable head/foot
\IfFileExists{footnotehyper.sty}{\usepackage{footnotehyper}}{\usepackage{footnote}}
\makesavenoteenv{longtable}
\usepackage{graphicx}
\makeatletter
\newsavebox\pandoc@box
\newcommand*\pandocbounded[1]{% scales image to fit in text height/width
  \sbox\pandoc@box{#1}%
  \Gscale@div\@tempa{\textheight}{\dimexpr\ht\pandoc@box+\dp\pandoc@box\relax}%
  \Gscale@div\@tempb{\linewidth}{\wd\pandoc@box}%
  \ifdim\@tempb\p@<\@tempa\p@\let\@tempa\@tempb\fi% select the smaller of both
  \ifdim\@tempa\p@<\p@\scalebox{\@tempa}{\usebox\pandoc@box}%
  \else\usebox{\pandoc@box}%
  \fi%
}
% Set default figure placement to htbp
\def\fps@figure{htbp}
\makeatother
% definitions for citeproc citations
\NewDocumentCommand\citeproctext{}{}
\NewDocumentCommand\citeproc{mm}{%
  \begingroup\def\citeproctext{#2}\cite{#1}\endgroup}
\makeatletter
 % allow citations to break across lines
 \let\@cite@ofmt\@firstofone
 % avoid brackets around text for \cite:
 \def\@biblabel#1{}
 \def\@cite#1#2{{#1\if@tempswa , #2\fi}}
\makeatother
\newlength{\cslhangindent}
\setlength{\cslhangindent}{1.5em}
\newlength{\csllabelwidth}
\setlength{\csllabelwidth}{3em}
\newenvironment{CSLReferences}[2] % #1 hanging-indent, #2 entry-spacing
 {\begin{list}{}{%
  \setlength{\itemindent}{0pt}
  \setlength{\leftmargin}{0pt}
  \setlength{\parsep}{0pt}
  % turn on hanging indent if param 1 is 1
  \ifodd #1
   \setlength{\leftmargin}{\cslhangindent}
   \setlength{\itemindent}{-1\cslhangindent}
  \fi
  % set entry spacing
  \setlength{\itemsep}{#2\baselineskip}}}
 {\end{list}}
\usepackage{calc}
\newcommand{\CSLBlock}[1]{\hfill\break\parbox[t]{\linewidth}{\strut\ignorespaces#1\strut}}
\newcommand{\CSLLeftMargin}[1]{\parbox[t]{\csllabelwidth}{\strut#1\strut}}
\newcommand{\CSLRightInline}[1]{\parbox[t]{\linewidth - \csllabelwidth}{\strut#1\strut}}
\newcommand{\CSLIndent}[1]{\hspace{\cslhangindent}#1}

% =============================================================================
% LATEX HEADER CONFIGURATION FOR MLSYSBOOK PDF
% =============================================================================
% This file contains all LaTeX package imports, custom commands, and styling
% definitions for the PDF output of the Machine Learning Systems textbook.
%
% Key Features:
% - Harvard crimson branding throughout
% - Custom part/chapter/section styling
% - Professional table formatting with colored headers
% - Margin notes with custom styling
% - TikZ-based part dividers
% - Page numbering (Roman for frontmatter, Arabic for mainmatter)
%
% Note: This file is included via _quarto-pdf.yml and affects PDF output only.
% HTML/EPUB styling is handled separately via CSS files.
% =============================================================================

% =============================================================================
% PACKAGE IMPORTS
% =============================================================================

% Layout and positioning
% \usepackage[outercaption, ragged]{sidecap}  % Commented out to make figure captions inline instead of in margin
\usepackage{adjustbox}      % Adjusting box dimensions
\usepackage{afterpage}      % Execute commands after page break
\usepackage{morefloats}     % Increase number of floats
\usepackage{array}          % Enhanced table column formatting
\usepackage{atbegshi}       % Insert content at page beginning
%\usepackage{changepage}     % Change page dimensions mid-document
\usepackage{emptypage}      % Clear headers/footers on empty pages

% Language and text
\usepackage[english]{babel} % English language support
\usepackage{microtype}      % Improved typography and hyphenation

% Captions and floats
\usepackage{caption}
% Caption styling configuration
%\captionsetup[table]{belowskip=5pt}
\captionsetup{format=plain}
\DeclareCaptionLabelFormat{mylabel}{#1
#2:\hspace{1.0ex}}
\DeclareCaptionFont{ninept}{\fontsize{7pt}{8}\selectfont #1}

% Figure captions: Small font, bold label, ragged right
\captionsetup[figure]{labelfont={bf,ninept},labelsep=space,
belowskip=2pt,aboveskip=6pt,labelformat=mylabel,
justification=raggedright,singlelinecheck=false,font={ninept}}

% Table captions: Small font, bold label, ragged right
\captionsetup[table]{belowskip=6pt,labelfont={bf,ninept},labelsep=none,
labelformat=mylabel,justification=raggedright,singlelinecheck=false,font={ninept}}

% Typography fine-tuning
\emergencystretch=5pt       % Allow extra stretch to avoid overfull boxes

% Utility packages
\usepackage{etoolbox}       % For patching commands and environments

% Page layout and headers
\usepackage{fancyhdr}       % Custom headers and footers
\usepackage{geometry}       % Page dimensions and margins

% Graphics and figures
\usepackage{graphicx}       % Include graphics
\usepackage{float}          % Improved float placement
\usepackage[skins,breakable]{tcolorbox} % Coloured and framed text boxes
\tcbset{before upper=\setlength{\parskip}{3pt}}

% Tables
\usepackage{longtable}      % Multi-page tables

% Fonts and typography
\usepackage{fontspec}       % Font selection for LuaLaTeX
\usepackage{mathptmx}       % Times-like math fonts
\usepackage{newpxtext}      % Palatino-like font for body text

% Colors and visual elements
\usepackage[dvipsnames]{xcolor}  % Extended color support
\usepackage{tikz}           % Programmatic graphics
\usetikzlibrary{positioning}
\usetikzlibrary{calc}
\usepackage{tikzpagenodes}  % TikZ positioning relative to page

% Code listings
\usepackage{listings}       % Code highlighting

% Hyperlinks
\usepackage{hyperref}       % Clickable links in PDF

% Conditional logic
\usepackage{ifthen}         % If-then-else commands

% Math symbols
\usepackage{amsmath}        % AMS math extensions
\usepackage{amssymb}        % AMS math symbols
\usepackage{latexsym}       % Additional LaTeX symbols
\usepackage{pifont}         % Zapf Dingbats symbols
\providecommand{\blacklozenge}{\ding{117}}  % Black diamond symbol

% Lists
\usepackage{enumitem}       % Customizable lists

% Margin notes and sidenotes
\usepackage{marginfix}      % Fixes margin note overflow
\usepackage{marginnote}     % Margin notes
\usepackage{sidenotes}      % Academic-style sidenotes
\renewcommand\raggedrightmarginnote{\sloppy}
\renewcommand\raggedleftmarginnote{\sloppy}

% Typography improvements
\usepackage{ragged2e}       % Better ragged text
\usepackage[all]{nowidow}   % Prevent widows and orphans
\usepackage{needspace}      % Ensure minimum space on page

% Section formatting
\usepackage[explicit]{titlesec}  % Custom section titles
\usepackage{tocloft}        % Table of contents formatting

% QR codes and icons
\usepackage{fontawesome5}   % Font Awesome icons
\usepackage{qrcode}         % QR code generation
\qrset{link, height=15mm}

% =============================================================================
% FLOAT CONFIGURATION
% =============================================================================
% Allow more floats per page to handle figure-heavy chapters
\extrafloats{200}
\setcounter{topnumber}{12}       % Max floats at top of page
\setcounter{bottomnumber}{12}    % Max floats at bottom of page
\setcounter{totalnumber}{24}     % Max floats per page
\setcounter{dbltopnumber}{8}     % Max floats at top of two-column page
\renewcommand{\topfraction}{.95}  % Max fraction of page for top floats
\renewcommand{\bottomfraction}{.95}
\renewcommand{\textfraction}{.05}  % Min fraction of page for text
\renewcommand{\floatpagefraction}{.7}  % Min fraction of float page
\renewcommand{\dbltopfraction}{.95}

% Prevent "Float(s) lost" errors by flushing floats more aggressively
\usepackage{placeins}  % Provides \FloatBarrier

% =============================================================================
% COLOR DEFINITIONS
% =============================================================================
% Harvard crimson - primary brand color used throughout
\definecolor{crimson}{HTML}{A51C30}

% Quiz element colors
\definecolor{quiz-question-color1}{RGB}{225,243,248}  % Light blue background
\definecolor{quiz-question-color2}{RGB}{17,158,199}   % Blue border
\definecolor{quiz-answer-color1}{RGB}{250,234,241}    % Light pink background
\definecolor{quiz-answer-color2}{RGB}{152,14,90}      % Magenta border

% =============================================================================
% LIST FORMATTING
% =============================================================================
% Tighter list spacing for academic style
\def\tightlist{}
\setlist{itemsep=1pt, parsep=1pt, topsep=0pt,after={\vspace{0.3\baselineskip}}}
\let\tightlist\relax

\makeatletter
\@ifpackageloaded{framed}{}{\usepackage{framed}}
\@ifpackageloaded{fancyvrb}{}{\usepackage{fancyvrb}}
\makeatother

\makeatletter
%New float "codelisting" has been updated
\AtBeginDocument{%
\floatstyle{ruled}
\newfloat{codelisting}{!htb}{lop}
\floatname{codelisting}{Listing}
\floatplacement{codelisting}{!htb}
\captionsetup[codelisting]{labelfont={bf,ninept},labelformat=mylabel,
  singlelinecheck=false,width=\linewidth,labelsep=none,font={ninept}}%
\renewenvironment{snugshade}{%
   \def\OuterFrameSep{3pt}%
   \def\FrameCommand{\fboxsep=5pt\colorbox{shadecolor}}%
   \MakeFramed{\advance\hsize-\width\FrameRestore}%
   \leftskip 0.5em \rightskip 0.5em%
   \small% decrease font size
   }{\endMakeFramed}%
}
\makeatother

%The space before and after the verbatim environment "Highlighting" has been reduced
\fvset{listparameters=\setlength{\topsep}{0pt}\setlength{\partopsep}{0pt}}
\DefineVerbatimEnvironment{Highlighting}{Verbatim}{framesep=0mm,commandchars=\\\{\}}

\makeatletter
\renewcommand\fs@ruled{\def\@fs@cfont{\bfseries}\let\@fs@capt\floatc@ruled
\def\@fs@pre{\hrule height.8pt depth0pt \kern2pt}%
\def\@fs@post{\kern2pt\hrule\relax}%
\def\@fs@mid{\kern2pt\hrule\kern1pt}%space between float and caption
\let\@fs@iftopcapt\iftrue}
\makeatother


% =============================================================================
% HYPHENATION RULES
% =============================================================================
% Explicit hyphenation points for technical terms to avoid bad breaks
\hyphenation{
  light-weight
  light-weight-ed
  de-vel-op-ment
  un-der-stand-ing
  mod-els
  prin-ci-ples
  ex-per-tise
  com-pli-cat-ed
  blue-print
  per‧for‧mance
  com-mu-ni-ca-tion
  par-a-digms
  hy-per-ten-sion
  a-chieved
}

% =============================================================================
% CODE LISTING CONFIGURATION
% =============================================================================
% Settings for code blocks using listings package
\lstset{
breaklines=true,              % Automatic line wrapping
breakatwhitespace=true,       % Break at whitespace only
basicstyle=\ttfamily,         % Monospace font
frame=none,                   % No frame around code
keepspaces=true,              % Preserve spaces
showspaces=false,             % Don't show space characters
showtabs=false,               % Don't show tab characters
columns=flexible,             % Flexible column width
belowskip=0pt,               % Minimal spacing
aboveskip=0pt
}

% =============================================================================
% PAGE GEOMETRY
% =============================================================================
% MIT Press trim size: 7" x 10" (per publisher specifications)
% This is a standard academic textbook format providing good readability
% for technical content with figures and code blocks.
% Wide outer margin accommodates sidenotes/margin notes.
\geometry{
  paperwidth=7in,
  paperheight=10in,
  top=0.875in,
  bottom=0.875in,
  inner=0.875in,              % Inner margin (binding side)
  outer=1.75in,               % Outer margin (includes space for sidenotes)
  footskip=30pt,
  marginparwidth=1.25in,      % Width for margin notes
  twoside                     % Different left/right pages
}

% =============================================================================
% SIDENOTE STYLING
% =============================================================================
% Custom sidenote design with crimson vertical bar
\renewcommand{\thefootnote}{\textcolor{crimson}{\arabic{footnote}}}

% Save original sidenote command
\makeatletter
\@ifundefined{oldsidenote}{
  \let\oldsidenote\sidenote%
}{}
\makeatother

% Redefine sidenote with vertical crimson bar
\renewcommand{\sidenote}[1]{%
  \oldsidenote{%
    \noindent
    \color{crimson!100}                        % Crimson vertical line
    \raisebox{0em}{%
      \rule{0.5pt}{1.5em}                      % Thin vertical line
    }
    \hspace{0.3em}                             % Space after line
    \color{black}                              % Reset text color
    \footnotesize #1                           % Sidenote content
  }%
}

% =============================================================================
% FLOAT HANDLING
% =============================================================================
% Patch LaTeX's output routine to handle float overflow gracefully
% The "Float(s) lost" error occurs in \@doclearpage when \@currlist is not empty
% This patch silently clears pending floats that can't be placed
\makeatletter
\let\orig@doclearpage\@doclearpage
\def\@doclearpage{%
  \ifx\@currlist\@empty\else
    \global\let\@currlist\@empty
    \typeout{Warning: Floats cleared to prevent overflow}%
  \fi
  \orig@doclearpage
}
\makeatother

% Additional safety for structural commands
\let\originalbackmatter\backmatter
\renewcommand{\backmatter}{%
  \clearpage%
  \originalbackmatter%
}

\let\originalfrontmatter\frontmatter
\renewcommand{\frontmatter}{%
  \clearpage%
  \originalfrontmatter%
}

\let\originalmainmatter\mainmatter
\renewcommand{\mainmatter}{%
  \clearpage%
  \originalmainmatter%
}

% =============================================================================
% PAGE HEADERS AND FOOTERS
% =============================================================================
% Ensure chapters use fancy page style (not plain)
\patchcmd{\chapter}{\thispagestyle{plain}}{\thispagestyle{fancy}}{}{}

% Main page style with crimson headers
\pagestyle{fancy}
\fancyhf{}                                              % Clear all
\fancyhead[LE]{\small\color{crimson}\nouppercase{\rightmark}}  % Left even: section
\fancyhead[RO]{\color{crimson}\thepage}                 % Right odd: page number
\fancyhead[LO]{\small\color{crimson}\nouppercase{\leftmark}}   % Left odd: chapter
\fancyhead[RE]{\color{crimson}\thepage}                 % Right even: page number
\renewcommand{\headrulewidth}{0.4pt}                    % Thin header line
\renewcommand{\footrulewidth}{0pt}                      % No footer line

% Plain page style (for chapter openings)
\fancypagestyle{plain}{
  \fancyhf{}
  \fancyfoot[C]{\color{crimson}\thepage}                % Centered page number
  \renewcommand{\headrulewidth}{0pt}
  \renewcommand{\footrulewidth}{0pt}
}

% =============================================================================
% KOMA-SCRIPT FONT ADJUSTMENTS
% =============================================================================
% Apply crimson color to all heading levels
\addtokomafont{disposition}{\rmfamily\color{crimson}}
\addtokomafont{chapter}{\color{crimson}}
\addtokomafont{section}{\color{crimson}}
\addtokomafont{subsection}{\color{crimson}}

% =============================================================================
% ABSTRACT ENVIRONMENT
% =============================================================================
\newenvironment{abstract}{
  \chapter*{\abstractname}
  \addcontentsline{toc}{chapter}{\abstractname}
  \small
}{
  \clearpage
}

% =============================================================================
% HYPERLINK CONFIGURATION
% =============================================================================
% Crimson-colored links throughout, two-page PDF layout
\hypersetup{
  linkcolor=crimson,
  citecolor=crimson,
  urlcolor=crimson,
  pdfpagelayout=TwoPageRight,   % Two-page spread view
  pdfstartview=Fit               % Initial zoom fits page
}

% =============================================================================
% PART SUMMARY SYSTEM
% =============================================================================
% Allows adding descriptive text below part titles
\newcommand{\partsummary}{}     % Empty by default
\newif\ifhaspartsummary%
\haspartsummaryfalse%

\newcommand{\setpartsummary}[1]{%
  \renewcommand{\partsummary}{#1}%
  \haspartsummarytrue%
}

% Additional colors for part page backgrounds
\definecolor{BrownLL}{RGB}{233,222,220}
\definecolor{BlueDD}{RGB}{62,100,125}
\colorlet{BlueDD}{magenta}

% ===============================================================================
% PART STYLING SYSTEM
% ===============================================================================
%
% This system provides three distinct visual styles for book organization:
%
% 1. NUMBERED PARTS (\part{title}) - For main book sections
%    - Roman numerals (I, II, III, etc.) in top right corner
%    - Crimson title with horizontal lines above/below
%    - "Part I" label in sidebar
%    - Used for: foundations, principles, optimization, deployment, etc.
%
% 2. UNNUMBERED PARTS (\part*{title}) - For special sections like "Labs"
%    - Division-style geometric background (left side)
%    - No Roman numerals
%    - Used for: labs section
%
% 3. DIVISIONS (\division{title}) - For major book divisions
%    - Clean geometric background with centered title
%    - Used for: frontmatter, main_content, backmatter
%
% The Lua filter (inject-parts.lua) automatically routes parts by {key:xxx} commands
% to the appropriate LaTeX command based on the key name.
% ===============================================================================

% NUMBERED PARTS: Roman numeral styling for main book sections
\titleformat{\part}[display]
{\thispagestyle{empty}}{}{20pt}{
\begin{tikzpicture}[remember picture,overlay]
%%%
%%
\node[crimson,align=flush right,
inner sep=0,outer sep=0mm,draw=none,%
anchor=east,minimum height=31mm, text width=1.2\textwidth,
yshift=-30mm,font={%
\fontsize{98pt}{104}\selectfont\bfseries}]  (BG) at (current page text area.north east){\thepart};
%
\node[black,inner sep=0mm,draw=none,
anchor=mid,text width=1.2\textwidth,
 minimum height=35mm, align=right,
node distance=7mm,below=of BG,
font={\fontsize{30pt}{34}\selectfont}]
(BGG)  {\hyphenchar\font=-1 \color{black}\MakeUppercase {#1}};
\draw [crimson,line width=3pt] ([yshift=0mm]BGG.north west) -- ([yshift=0mm]BGG.north east);
\draw [crimson,line width=2pt] ([yshift=0mm]BGG.south west) -- ([yshift=0mm]BGG.south east);
%
\node[fill=crimson,text=white,rotate=90,%
anchor=south west,minimum height=15mm,
minimum width=40mm,font={%
\fontsize{20pt}{20}\selectfont\bfseries}](BP)  at
(current page text area.south east)
{{\sffamily Part}~\thepart};
%
\path[red](BP.north west)-|coordinate(PS)(BGG.south west);
%
% Part summary box commented out for cleaner design
% \ifhaspartsummary
% \node[inner sep=4pt,text width=0.7\textwidth,draw=none,fill=BrownLL!40,
% align=justify,font={\fontsize{9pt}{12}\selectfont},anchor=south west]
% at (PS) {\partsummary};
% \fi
\end{tikzpicture}
}[]

\renewcommand{\thepart}{\Roman{part}}

% UNNUMBERED PARTS: Division-style background for special sections
\titleformat{name=\part,numberless}[display]
{\thispagestyle{empty}}{}{20pt}{
\begin{tikzpicture}[remember picture,overlay]
%%%
\coordinate(S1)at([yshift=-200mm]current page.north west);
\draw[draw=none,fill=BlueDD!7](S1)--++(45:16)coordinate(S2)-
|(S2|-current page.north west)--(current page.north west)coordinate(S3)--(S1);
%
\coordinate(E1)at([yshift=-98mm]current page.north west);
\draw[draw=none,fill=BlueDD!15](E1)--(current page.north west)coordinate(E2)
--++(0:98mm)coordinate(E3)--(E1);
%
\coordinate(D1)at([yshift=15mm]current page.south west);
\draw[draw=none,fill=BlueDD!40,opacity=0.5](D1)--++(45:5.5)coordinate(D2)
-|(D2|-current page.north west)--(current page.north west)coordinate(D3)--(D1);
%%%%
\path[red](S2)-|(S2-|current page.east)coordinate(SS2);
%PART
\node[crimson,align=flush right,inner sep=0,outer sep=0mm,draw=none,anchor=south,
font={\fontsize{48pt}{48}\selectfont\bfseries}]  (BG) at ($(S2)!0.5!(SS2)$){\hphantom{Part}};
%%%
\path[green]([yshift=15mm]D2)-|coordinate(TPD)(BG.south east);
\node[inner sep=0mm,draw=none,anchor=south east,%text width=0.9\textwidth,
align=right,font={\fontsize{40pt}{40}\selectfont}]
(BGG) at (TPD)  {\color{crimson}\MakeUppercase {#1}};%\MakeUppercase {}
\end{tikzpicture}
}

% Define \numberedpart command for numbered parts
\newcommand{\numberedpart}[1]{%
\FloatBarrier%  % Flush all pending floats before part break
\clearpage
\thispagestyle{empty}
\stepcounter{part}%
\begin{tikzpicture}[remember picture,overlay]
%%%
%%
\node[crimson,align=flush right,
inner sep=0,outer sep=0mm,draw=none,%
anchor=east,minimum height=31mm, text width=1.2\textwidth,
yshift=-30mm,font={%
\fontsize{98pt}{104}\selectfont\bfseries}]  (BG) at (current page text area.north east){\thepart};
%
\node[black,inner sep=0mm,draw=none,
anchor=mid,text width=1.2\textwidth,
 minimum height=35mm, align=right,
node distance=7mm,below=of BG,
font={\fontsize{30pt}{34}\selectfont}]
(BGG)  {\hyphenchar\font=-1 \color{black}\MakeUppercase {#1}};
\draw [crimson,line width=3pt] ([yshift=0mm]BGG.north west) -- ([yshift=0mm]BGG.north east);
\draw [crimson,line width=2pt] ([yshift=0mm]BGG.south west) -- ([yshift=0mm]BGG.south east);
%
\node[fill=crimson,text=white,rotate=90,%
anchor=south west,minimum height=15mm,
minimum width=40mm,font={%
\fontsize{20pt}{20}\selectfont\bfseries}](BP)  at
(current page text area.south east)
{{\sffamily Part}~\thepart};
%
\path[red](BP.north west)-|coordinate(PS)(BGG.south west);
%
% Part summary box commented out for cleaner design
% \ifhaspartsummary
% \node[inner sep=4pt,text width=0.7\textwidth,draw=none,fill=BrownLL!40,
% align=justify,font={\fontsize{9pt}{12}\selectfont},anchor=south west]
% at (PS) {\partsummary};
% \fi
\end{tikzpicture}
\clearpage
}



% DIVISIONS: Clean geometric styling with subtle tech elements
% Used for frontmatter, main_content, and backmatter divisions
\newcommand{\division}[1]{%
\FloatBarrier%  % Flush all pending floats before division break
\clearpage
\thispagestyle{empty}
\begin{tikzpicture}[remember picture,overlay]

% Clean geometric background (original design)
\coordinate(S1)at([yshift=-200mm]current page.north west);
\draw[draw=none,fill=BlueDD!7](S1)--++(45:16)coordinate(S2)-
|(S2|-current page.north west)--(current page.north west)coordinate(S3)--(S1);

\coordinate(E1)at([yshift=-98mm]current page.north west);
\draw[draw=none,fill=BlueDD!15](E1)--(current page.north west)coordinate(E2)
--++(0:98mm)coordinate(E3)--(E1);

\coordinate(D1)at([yshift=15mm]current page.south west);
\draw[draw=none,fill=BlueDD!40,opacity=0.5](D1)--++(45:5.5)coordinate(D2)
-|(D2|-current page.north west)--(current page.north west)coordinate(D3)--(D1);

% Subtle tech elements - positioned in white areas for better visibility
% Upper right white area - more visible
\draw[crimson!40, line width=0.8pt] ([xshift=140mm,yshift=-60mm]current page.north west) -- ++(40mm,0);
\draw[crimson!40, line width=0.8pt] ([xshift=150mm,yshift=-70mm]current page.north west) -- ++(30mm,0);
\draw[crimson!35, line width=0.7pt] ([xshift=160mm,yshift=-60mm]current page.north west) -- ++(0,-15mm);
\draw[crimson!35, line width=0.7pt] ([xshift=170mm,yshift=-70mm]current page.north west) -- ++(0,10mm);

% Circuit nodes - upper right
\fill[crimson!50] ([xshift=160mm,yshift=-60mm]current page.north west) circle (1.5mm);
\fill[white] ([xshift=160mm,yshift=-60mm]current page.north west) circle (0.8mm);
\fill[crimson!50] ([xshift=170mm,yshift=-70mm]current page.north west) circle (1.3mm);
\fill[white] ([xshift=170mm,yshift=-70mm]current page.north west) circle (0.6mm);

% Lower right white area - enhanced visibility
\draw[crimson!45, line width=0.9pt] ([xshift=140mm,yshift=-190mm]current page.north west) -- ++(45mm,0);
\draw[crimson!45, line width=0.9pt] ([xshift=150mm,yshift=-200mm]current page.north west) -- ++(35mm,0);
\draw[crimson!40, line width=0.8pt] ([xshift=160mm,yshift=-190mm]current page.north west) -- ++(0,-20mm);
\draw[crimson!40, line width=0.8pt] ([xshift=170mm,yshift=-200mm]current page.north west) -- ++(0,15mm);

% Additional connecting lines in lower right
\draw[crimson!35, line width=0.7pt] ([xshift=130mm,yshift=-180mm]current page.north west) -- ++(25mm,0);
\draw[crimson!35, line width=0.7pt] ([xshift=145mm,yshift=-180mm]current page.north west) -- ++(0,-25mm);

% Circuit nodes - lower right (more prominent)
\fill[crimson!55] ([xshift=160mm,yshift=-190mm]current page.north west) circle (1.6mm);
\fill[white] ([xshift=160mm,yshift=-190mm]current page.north west) circle (0.9mm);
\fill[crimson!55] ([xshift=170mm,yshift=-200mm]current page.north west) circle (1.4mm);
\fill[white] ([xshift=170mm,yshift=-200mm]current page.north west) circle (0.7mm);
\fill[crimson!50] ([xshift=145mm,yshift=-180mm]current page.north west) circle (1.2mm);
\fill[white] ([xshift=145mm,yshift=-180mm]current page.north west) circle (0.6mm);

% Title positioned in center - clean and readable
\node[inner sep=0mm,draw=none,anchor=center,text width=0.8\textwidth,
align=center,font={\fontsize{40pt}{40}\selectfont}]
(BGG) at (current page.center)  {\color{crimson}\MakeUppercase {#1}};

\end{tikzpicture}
\clearpage
}

% LAB DIVISIONS: Circuit-style neural network design for lab sections
% Used specifically for lab platform sections (arduino, xiao, grove, etc.)
\newcommand{\labdivision}[1]{%
\FloatBarrier%  % Flush all pending floats before lab division break
\clearpage
\thispagestyle{empty}
\begin{tikzpicture}[remember picture,overlay]
% Circuit background with subtle gradient
\coordinate(S1)at([yshift=-200mm]current page.north west);
\draw[draw=none,fill=BlueDD!5](S1)--++(45:16)coordinate(S2)-
|(S2|-current page.north west)--(current page.north west)coordinate(S3)--(S1);

% TOP AREA: Circuit lines in upper white space
\draw[crimson!50, line width=1.5pt] ([xshift=30mm,yshift=-40mm]current page.north west) -- ++(60mm,0);
\draw[crimson!40, line width=1pt] ([xshift=120mm,yshift=-50mm]current page.north west) -- ++(50mm,0);
\draw[crimson!50, line width=1.5pt] ([xshift=40mm,yshift=-70mm]current page.north west) -- ++(40mm,0);

% Connecting lines in top area
\draw[crimson!30, line width=1pt] ([xshift=60mm,yshift=-40mm]current page.north west) -- ++(0,-20mm);
\draw[crimson!30, line width=1pt] ([xshift=145mm,yshift=-50mm]current page.north west) -- ++(0,10mm);

% Neural nodes in top area
\fill[crimson!70] ([xshift=60mm,yshift=-40mm]current page.north west) circle (2.5mm);
\fill[white] ([xshift=60mm,yshift=-40mm]current page.north west) circle (1.5mm);
\fill[crimson!60] ([xshift=145mm,yshift=-50mm]current page.north west) circle (2mm);
\fill[white] ([xshift=145mm,yshift=-50mm]current page.north west) circle (1mm);
\fill[crimson!80] ([xshift=80mm,yshift=-70mm]current page.north west) circle (2mm);
\fill[white] ([xshift=80mm,yshift=-70mm]current page.north west) circle (1mm);

% BOTTOM AREA: Circuit lines in lower white space
\draw[crimson!50, line width=1.5pt] ([xshift=20mm,yshift=-200mm]current page.north west) -- ++(70mm,0);
\draw[crimson!40, line width=1pt] ([xshift=110mm,yshift=-210mm]current page.north west) -- ++(60mm,0);
\draw[crimson!50, line width=1.5pt] ([xshift=35mm,yshift=-230mm]current page.north west) -- ++(45mm,0);

% Connecting lines in bottom area
\draw[crimson!30, line width=1pt] ([xshift=55mm,yshift=-200mm]current page.north west) -- ++(0,-20mm);
\draw[crimson!30, line width=1pt] ([xshift=140mm,yshift=-210mm]current page.north west) -- ++(0,15mm);

% Neural nodes in bottom area
\fill[crimson!70] ([xshift=55mm,yshift=-200mm]current page.north west) circle (2.5mm);
\fill[white] ([xshift=55mm,yshift=-200mm]current page.north west) circle (1.5mm);
\fill[crimson!60] ([xshift=140mm,yshift=-210mm]current page.north west) circle (2mm);
\fill[white] ([xshift=140mm,yshift=-210mm]current page.north west) circle (1mm);
\fill[crimson!80] ([xshift=80mm,yshift=-230mm]current page.north west) circle (2mm);
\fill[white] ([xshift=80mm,yshift=-230mm]current page.north west) circle (1mm);

% SIDE AREAS: Subtle circuit elements on left and right edges
\draw[crimson!30, line width=1pt] ([xshift=15mm,yshift=-120mm]current page.north west) -- ++(20mm,0);
\draw[crimson!30, line width=1pt] ([xshift=175mm,yshift=-130mm]current page.north west) -- ++(15mm,0);
\fill[crimson!50] ([xshift=25mm,yshift=-120mm]current page.north west) circle (1.5mm);
\fill[white] ([xshift=25mm,yshift=-120mm]current page.north west) circle (0.8mm);
\fill[crimson!50] ([xshift=185mm,yshift=-130mm]current page.north west) circle (1.5mm);
\fill[white] ([xshift=185mm,yshift=-130mm]current page.north west) circle (0.8mm);

% Title positioned in center - CLEAN AREA
\node[inner sep=0mm,draw=none,anchor=center,text width=0.8\textwidth,
align=center,font={\fontsize{44pt}{44}\selectfont\bfseries}]
(BGG) at (current page.center)  {\color{crimson}\MakeUppercase {#1}};

\end{tikzpicture}
\clearpage
}

% Define \lab command for lab styling (different visual treatment)
\newcommand{\lab}[1]{%
\begin{tikzpicture}[remember picture,overlay]
%%%
% Different background pattern for labs
\coordinate(S1)at([yshift=-200mm]current page.north west);
\draw[draw=none,fill=BlueDD!15](S1)--++(45:16)coordinate(S2)-
|(S2|-current page.north west)--(current page.north west)coordinate(S3)--(S1);
%
\coordinate(E1)at([yshift=-98mm]current page.north west);
\draw[draw=none,fill=BlueDD!25](E1)--(current page.north west)coordinate(E2)
--++(0:98mm)coordinate(E3)--(E1);
%
\coordinate(D1)at([yshift=15mm]current page.south west);
\draw[draw=none,fill=BlueDD!60,opacity=0.7](D1)--++(45:5.5)coordinate(D2)
-|(D2|-current page.north west)--(current page.north west)coordinate(D3)--(D1);
%%%%
\path[red](S2)-|(S2-|current page.east)coordinate(SS2);
%LAB - Different styling
\node[crimson,align=flush right,inner sep=0,outer sep=0mm,draw=none,anchor=south,
font={\fontsize{48pt}{48}\selectfont\bfseries}]  (BG) at ($(S2)!0.5!(SS2)$){\hphantom{Workshop}};
%%%
\path[green]([yshift=15mm]D2)-|coordinate(TPD)(BG.south east);
\node[inner sep=0mm,draw=none,anchor=south east,%text width=0.9\textwidth,
align=right,font={\fontsize{40pt}{40}\selectfont}]
(BGG) at (TPD)  {\color{crimson}\MakeUppercase {#1}};%\MakeUppercase {}
\end{tikzpicture}
\thispagestyle{empty}
\clearpage
}

% =============================================================================
% SECTION FORMATTING
% =============================================================================
% All section levels use crimson color and are ragged right

% Section (Large, bold, crimson)
\titleformat{\section}
  {\normalfont\Large\bfseries\color{crimson}\raggedright}
  {\thesection}
  {0.5em}
  {#1}
\titlespacing*{\section}{0pc}{14pt plus 4pt minus 4pt}{6pt plus 2pt minus 2pt}[0pc]

% Subsection (large, bold, crimson)
\titleformat{\subsection}
  {\normalfont\large\bfseries\color{crimson}\raggedright}
  {\thesubsection}
  {0.5em}
  {#1}
\titlespacing*{\subsection}{0pc}{12pt plus 4pt minus 4pt}{5pt plus 1pt minus 2pt}[0pc]

% Subsubsection (normal size, bold, crimson)
\titleformat{\subsubsection}
  {\normalfont\normalsize\bfseries\color{crimson}\raggedright}
  {\thesubsubsection}
  {0.5em}
  {#1}
\titlespacing*{\subsubsection}{0pc}{12pt plus 4pt minus 4pt}{5pt plus 1pt minus 2pt}[0pc]

% Paragraph (run-in, bold, crimson, ends with period)
\titleformat{\paragraph}[runin]
  {\normalfont\normalsize\bfseries\color{crimson}}
  {\theparagraph}
  {0.5em}
  {#1}
  [\textbf{.}]
  \titlespacing*{\paragraph}{0pc}{6pt plus 2pt minus 2pt}{0.5em}[0pc]

% Subparagraph (run-in, italic, crimson, ends with period)
\titleformat{\subparagraph}[runin]
  {\normalfont\normalsize\itshape\color{crimson}}
  {\thesubparagraph}
  {0.5em}
  {#1}
  [\textbf{.}]
  \titlespacing*{\subparagraph}{0pc}{6pt plus 2pt minus 2pt}{0.5em}[0pc]

% =============================================================================
% CHAPTER FORMATTING
% =============================================================================
% Numbered chapters: "Chapter X" prefix, huge crimson title
\titleformat{\chapter}[display]
  {\normalfont\huge\bfseries\color{crimson}}
  {\chaptername\ \thechapter}
  {20pt}
  {\Huge #1}
  []

% Unnumbered chapters: no prefix, huge crimson title
\titleformat{name=\chapter,numberless}
  {\normalfont\huge\bfseries\color{crimson}}
  {}
  {0pt}
  {\Huge #1}
  []

\renewcommand{\chaptername}{Chapter}
% =============================================================================
% TABLE OF CONTENTS FORMATTING
% =============================================================================
\setcounter{tocdepth}{2}                      % Show chapters, sections, subsections

% TOC spacing adjustments for number widths and indentation
\setlength{\cftchapnumwidth}{2em}             % Chapter number width
\setlength{\cftsecnumwidth}{2.75em}           % Section number width
\setlength{\cftsubsecnumwidth}{3.25em}        % Subsection number width
\setlength{\cftsubsubsecnumwidth}{4em}        % Subsubsection number width
\setlength{\cftsubsecindent}{4.25em}          % Subsection indent
\setlength{\cftsubsubsecindent}{7.5em}        % Subsubsection indent

% Chapter entries in TOC: bold crimson with "Chapter" prefix
\renewcommand{\cftchapfont}{\bfseries\color{crimson}}
\renewcommand{\cftchappresnum}{\color{crimson}Chapter~}

% Custom formatting for division entries (styled like parts)
\newcommand{\divisionchapter}[1]{%
  \addvspace{12pt}%
  \noindent\hfil\bfseries\color{crimson}#1\hfil\par%
  \addvspace{6pt}%
}

% Adjust TOC spacing for "Chapter" prefix
\newlength{\xtraspace}
\settowidth{\xtraspace}{\cftchappresnum\cftchapaftersnum}
\addtolength{\cftchapnumwidth}{\xtraspace}

% Unnumbered chapters with TOC entry
\newcommand{\likechapter}[1]{%
    \chapter*{#1}
    \addcontentsline{toc}{chapter}{\textcolor{crimson}{#1}}
}

% =============================================================================
% PAGE NUMBERING SYSTEM
% =============================================================================
% Implements traditional book numbering:
% - Roman numerals (i, ii, iii...) for frontmatter
% - Arabic numerals (1, 2, 3...) for mainmatter
% Automatically switches at first numbered chapter
\makeatletter
\newif\if@firstnumbered%
\@firstnumberedtrue%
\newif\if@firstunnumbered%
\@firstunnumberedtrue%

\newcounter{lastRomanPage}
\setcounter{lastRomanPage}{1}

% Start document with Roman numerals (frontmatter)
\AtBeginDocument{
  \pagenumbering{roman}
  \renewcommand{\thepage}{\roman{page}}
}

% Intercept chapter command
\let\old@chapter\chapter%
\renewcommand{\chapter}{%
  \@ifstar{\unnumbered@chapter}{\numbered@chapter}%
}

% Numbered chapters: switch to Arabic on first occurrence
\newcommand{\numbered@chapter}[1]{%
  \if@firstnumbered%
    \cleardoublepage%
    \setcounter{lastRomanPage}{\value{page}}%
    \pagenumbering{arabic}%
    \@firstnumberedfalse%
  \else
    \setcounter{page}{\value{page}}%
  \fi
  \setcounter{sidenote}{1}                    % Reset footnote counter per chapter
  \old@chapter{#1}%
}

% Unnumbered chapters: stay in Roman numerals
\newcommand{\unnumbered@chapter}[1]{%
  \if@firstunnumbered%
    \clearpage
    \setcounter{lastRomanPage}{\value{page}}%
    \pagenumbering{roman}%
    \@firstunnumberedfalse%
  \fi
  \setcounter{sidenote}{1}
  \old@chapter*{#1}%
}
\makeatother

% =============================================================================
% TABLE SIZING AND SPACING
% =============================================================================
% Make tables slightly smaller to fit more content
\AtBeginEnvironment{longtable}{\scriptsize}

% Increase vertical spacing in table cells (default is 1.0)
\renewcommand{\arraystretch}{1.3}

% Prefer placing figures and tables at the top of pages
\makeatletter
\renewcommand{\fps@figure}{t}  % Default placement: top of page
\renewcommand{\fps@table}{t}   % Default placement: top of page
\makeatother

% =============================================================================
% LONGTABLE PAGE BREAKING FIXES (Windows compatibility)
% =============================================================================
% Prevent "Infinite glue shrinkage" errors on Windows LaTeX builds
% by giving longtable more flexibility in page breaking

% Allow more flexible page breaking (vs strict \flushbottom)
\raggedbottom

% Process more rows before attempting page break (default is 20)
\setcounter{LTchunksize}{50}

% Add extra stretch for longtable environments specifically
\AtBeginEnvironment{longtable}{%
  \setlength{\emergencystretch}{3em}%
  \setlength{\parskip}{0pt plus 1pt}%
}

% =============================================================================
% TABLE STYLING - Clean tables with crimson borders
% =============================================================================
% Professional table appearance with:
% - Clean white background (no colored rows)
% - Crimson-colored borders
% - Good spacing for readability
%
% Note: Headers are automatically bolded by Quarto when using **text** in source
\usepackage{booktabs}      % Professional table rules (\toprule, \midrule, \bottomrule)
\usepackage{colortbl}      % For colored borders (\arrayrulecolor)

% Global table styling - crimson borders
\setlength{\arrayrulewidth}{0.5pt}          % Thinner borders than default
%\arrayrulecolor{crimson}                    % Crimson borders matching brand

\setcounter{chapter}{0}

% =============================================================================
% DROP CAPS (Lettrine)
% =============================================================================
% Decorative large first letter at chapter openings, following the tradition
% of Hennessy & Patterson and other MIT Press textbooks.
% Usage in QMD: \lettrine{T}{he first sentence...}
\usepackage{lettrine}
\renewcommand{\LettrineFontHook}{\color{crimson}\bfseries}
\setcounter{DefaultLines}{3}          % Drop cap spans 3 lines
\renewcommand{\DefaultLoversize}{0.1} % Slight oversize for visual weight
\renewcommand{\DefaultLraise}{0}      % No vertical shift
\setlength{\DefaultNindent}{0.5em}    % Indent of continuation text
\setlength{\DefaultSlope}{0pt}        % No slope on continuation

% =============================================================================
% RUNNING HEADERS — Truncation Safety
% =============================================================================
% Long chapter/section titles can overflow the header. These marks truncate
% gracefully so headers stay within the text block.
\renewcommand{\chaptermark}[1]{%
  \markboth{\thechapter.\ #1}{}}
\renewcommand{\sectionmark}[1]{%
  \markright{\thesection\ #1}}

% =============================================================================
% EPIGRAPH ENVIRONMENT
% =============================================================================
% For chapter-opening quotations. Renders as right-aligned italic block
% with attribution in small caps below.
% Usage: \epigraph{Quote text}{Author Name, \textit{Source}}
\newcommand{\bookepigraph}[2]{%
  \vspace{1em}%
  \begin{flushright}%
    \begin{minipage}{0.75\textwidth}%
      \raggedleft\itshape\small #1\\[0.5em]%
      \normalfont\small --- #2%
    \end{minipage}%
  \end{flushright}%
  \vspace{1.5em}%
}

% =============================================================================
% THUMB INDEX TABS
% =============================================================================
% Colored tabs on the outer page edge for quick chapter navigation.
% Each Part gets a different vertical position; chapters within a Part
% share the same tab position. Visible when flipping through the book.
\newcounter{thumbindex}
\setcounter{thumbindex}{0}
\newlength{\thumbtabheight}
\setlength{\thumbtabheight}{16mm}     % Height of each tab
\newlength{\thumbtabwidth}
\setlength{\thumbtabwidth}{8mm}       % Width protruding from edge
\newlength{\thumbtabgap}
\setlength{\thumbtabgap}{1mm}         % Gap between tabs

% Advance to next thumb tab position (call at each \part)
\newcommand{\nextthumb}{%
  \stepcounter{thumbindex}%
}

% Draw the thumb tab on every page (placed in header via fancyhdr)
\newcommand{\drawthumb}{%
  \ifnum\value{thumbindex}>0%
    \begin{tikzpicture}[remember picture,overlay]
      \pgfmathsetmacro{\thumboffset}{%
        20 + (\value{thumbindex}-1) * (16 + 1)}  % mm from top
      \ifodd\value{page}%
        % Odd pages: tab on right edge
        \fill[crimson!80]
          ([yshift=-\thumboffset mm]current page.north east)
          rectangle +(-\thumbtabwidth, -\thumbtabheight);
        \node[white,font=\tiny\bfseries,rotate=90]
          at ([yshift=-\thumboffset mm - 0.5\thumbtabheight,
               xshift=-0.5\thumbtabwidth]current page.north east)
          {\Roman{thumbindex}};
      \else
        % Even pages: tab on left edge
        \fill[crimson!80]
          ([yshift=-\thumboffset mm]current page.north west)
          rectangle +(\thumbtabwidth, -\thumbtabheight);
        \node[white,font=\tiny\bfseries,rotate=-90]
          at ([yshift=-\thumboffset mm - 0.5\thumbtabheight,
               xshift=0.5\thumbtabwidth]current page.north west)
          {\Roman{thumbindex}};
      \fi
    \end{tikzpicture}%
  \fi
}

% Hook into fancyhdr to draw thumb on every content page
\AddToHook{shipout/foreground}{%
  \drawthumb%
}

% =============================================================================
% CROP / BLEED MARKS
% =============================================================================
% For final print submission, uncomment the line below to add crop marks.
% MIT Press production will advise on exact requirements.
% \usepackage[cam,center,width=7.5in,height=10.5in]{crop}

% =============================================================================
% PDF/A ARCHIVAL COMPLIANCE
% =============================================================================
% MIT Press increasingly requires PDF/A for long-term preservation.
% This embeds all fonts and removes transparency.
% Note: pdfx must be loaded early; if it conflicts with hyperref,
% MIT Press production can handle the conversion post-build.
% Uncomment when ready for final submission:
% \usepackage[a-3u]{pdfx}

% =============================================================================
% ENHANCED WIDOW / ORPHAN CONTROL
% =============================================================================
% Prevent single lines at top/bottom of pages and breaks before equations
\clubpenalty=10000          % No orphans (single first line at bottom)
\widowpenalty=10000         % No widows (single last line at top)
\displaywidowpenalty=10000  % No widow before display math
\predisplaypenalty=10000    % No page break just before display math
\postdisplaypenalty=0       % Allow break after display math (natural)
\usepackage{needspace}
\let\Needspace\needspace
\makeatletter
\@ifpackageloaded{float}{}{\usepackage{float}}
\floatstyle{plain}
\@ifundefined{c@chapter}{\newfloat{vid}{h}{lovid}}{\newfloat{vid}{h}{lovid}[chapter]}
\floatname{vid}{Video}
\newcommand*\listofvids{\listof{vid}{List of Videos}}
\makeatother
\makeatletter
\@ifpackageloaded{tcolorbox}{}{\usepackage[skins,breakable]{tcolorbox}}
\@ifpackageloaded{fontawesome5}{}{\usepackage{fontawesome5}}
\definecolor{quarto-callout-color}{HTML}{909090}
\definecolor{quarto-callout-note-color}{HTML}{0758E5}
\definecolor{quarto-callout-important-color}{HTML}{CC1914}
\definecolor{quarto-callout-warning-color}{HTML}{EB9113}
\definecolor{quarto-callout-tip-color}{HTML}{00A047}
\definecolor{quarto-callout-caution-color}{HTML}{FC5300}
\definecolor{quarto-callout-color-frame}{HTML}{acacac}
\definecolor{quarto-callout-note-color-frame}{HTML}{4582ec}
\definecolor{quarto-callout-important-color-frame}{HTML}{d9534f}
\definecolor{quarto-callout-warning-color-frame}{HTML}{f0ad4e}
\definecolor{quarto-callout-tip-color-frame}{HTML}{02b875}
\definecolor{quarto-callout-caution-color-frame}{HTML}{fd7e14}
\makeatother
\makeatletter
\@ifpackageloaded{bookmark}{}{\usepackage{bookmark}}
\makeatother
\makeatletter
\@ifpackageloaded{caption}{}{\usepackage{caption}}
\AtBeginDocument{%
\ifdefined\contentsname
  \renewcommand*\contentsname{Table of contents}
\else
  \newcommand\contentsname{Table of contents}
\fi
\ifdefined\listfigurename
  \renewcommand*\listfigurename{List of Figures}
\else
  \newcommand\listfigurename{List of Figures}
\fi
\ifdefined\listtablename
  \renewcommand*\listtablename{List of Tables}
\else
  \newcommand\listtablename{List of Tables}
\fi
\ifdefined\figurename
  \renewcommand*\figurename{Figure}
\else
  \newcommand\figurename{Figure}
\fi
\ifdefined\tablename
  \renewcommand*\tablename{Table}
\else
  \newcommand\tablename{Table}
\fi
}
\@ifpackageloaded{float}{}{\usepackage{float}}
\floatstyle{ruled}
\@ifundefined{c@chapter}{\newfloat{codelisting}{h}{lop}}{\newfloat{codelisting}{h}{lop}[chapter]}
\floatname{codelisting}{Listing}
\newcommand*\listoflistings{\listof{codelisting}{List of Listings}}
\makeatother
\makeatletter
\makeatother
\makeatletter
\@ifpackageloaded{caption}{}{\usepackage{caption}}
\@ifpackageloaded{subcaption}{}{\usepackage{subcaption}}
\makeatother
\newcommand{\fbxIconPath}{assets/images/icons/callouts}
\newcommand{\fbxIconFormat}{pdf}
\makeatletter
\@ifpackageloaded{tcolorbox}{}{\usepackage[many]{tcolorbox}}
\makeatother
%%%% ---foldboxy preamble ----- %%%%%

% Load xstring for string manipulation
\RequirePackage{xstring}

% Icon path and format configuration - can be overridden in filter-metadata
\providecommand{\fbxIconPath}{assets/images/icons/callouts}
\providecommand{\fbxIconFormat}{pdf}

% Helper command to include icon with hyphen-to-underscore conversion
% This ensures consistency: callout-quiz-question -> callout_quiz_question
\newcommand{\fbxIncludeIcon}[2]{%
  \StrSubstitute{#1}{-}{_}[\fbxIconName]%
  \includegraphics[width=#2]{\fbxIconPath/icon_\fbxIconName.\fbxIconFormat}%
}

% Legacy fallback colors (keep for compatibility)
\definecolor{fbx-default-color1}{HTML}{c7c7d0}
\definecolor{fbx-default-color2}{HTML}{a3a3aa}
\definecolor{fbox-color1}{HTML}{c7c7d0}
\definecolor{fbox-color2}{HTML}{a3a3aa}

% arguments: #1 typelabelnummer: #2 titel: #3
\newenvironment{fbx}[3]{%
\begin{tcolorbox}[
  enhanced,
  breakable,
  %fontupper=\fontsize{8pt}{10pt}\selectfont,  % 95% of body text (10pt -> 9.5pt)
  before skip=8pt,  % space above box (increased)
  after skip=8pt,   % space below box (increased)
  attach boxed title to top*={xshift=0pt},
  boxed title style={
  %fuzzy shadow={1pt}{-1pt}{0mm}{0.1mm}{gray},
  arc=1.5pt,
  rounded corners=north,
  sharp corners=south,
  top=6pt,          % Adjusted for ~40px equivalent height
  bottom=5pt,       % Adjusted for ~40px equivalent height
  overlay={
      \node [left,outer sep=0em, black,draw=none,anchor=west,
        rectangle,fill=none,inner sep=0pt]
        at ([xshift=4mm]frame.west) {\fbxIncludeIcon{#1}{4.2mm}};
    },
  },
  colframe=#1-color2,             % Border color (auto-generated from YAML)
  colbacktitle=#1-color1,         % Background color (auto-generated from YAML)
  colback=white,
  coltitle=black,
  titlerule=0mm,
  toprule=0.5pt,
  bottomrule=0.5pt,
  leftrule=2.2pt,
  rightrule=0.5pt,
  outer arc=1.5pt,
  arc=1.5pt,
  left=0.5em,       % increased left padding
  bottomtitle=1.5mm, % increased title bottom margin
  toptitle=1.5mm,    % increased title top margin
  title=\hspace{2.5em}\protect#2\hspace{0.5em}\protect#3, % Protect parameters
  extras middle and last={top=4pt} % increased continuation spacing
]}
{\end{tcolorbox}}


% boxed environment with right border
\newenvironment{fbxSimple}[3]{\begin{tcolorbox}[
  enhanced,
  breakable,
  %fontupper=\fontsize{8pt}{10pt}\selectfont,  % 95% of body text (10pt -> 9.5pt)
  before skip=8pt,  % space above box (increased)
  after skip=8pt,   % space below box (increased)
  attach boxed title to top*={xshift=0pt},
  boxed title style={
  %fuzzy shadow={1pt}{-1pt}{0mm}{0.1mm}{gray},
  arc=1.5pt,
  rounded corners=north,
  sharp corners=south,
  top=6pt,          % Adjusted for ~40px equivalent height
  bottom=5pt,       % Adjusted for ~40px equivalent height
  overlay={
      \node [left,outer sep=0em, black,draw=none,anchor=west,
        rectangle,fill=none,inner sep=0pt]
        at ([xshift=3mm]frame.west) {\fbxIncludeIcon{#1}{4.2mm}};
    },
  },
  colframe=#1-color2,             % Border color (auto-generated from YAML)
  colbacktitle=#1-color1,         % Background color (auto-generated from YAML)
  colback=white,
  coltitle=black,
  titlerule=0mm,
  toprule=0.5pt,
  bottomrule=0.5pt,
  leftrule=2.2pt,
  rightrule=0.5pt,
  outer arc=1.5pt,
  arc=1.5pt,
  left=0.5em,       % increased left padding
  bottomtitle=1.5mm, % increased title bottom margin
  toptitle=1.5mm,    % increased title top margin
  title=\hspace{2.5em}\protect#2\hspace{0.5em}\protect#3, % Protect parameters
  boxsep=1pt,
  extras first={bottom=0pt},
  extras last={top=0pt,bottom=-4pt},
  overlay first={
    \draw[line width=1pt,white] ([xshift=2.2pt]frame.south west)-- ([xshift=-0.5pt]frame.south east);
  },
  overlay last={
    \draw[line width=1pt,white] ([xshift=2.2pt]frame.north west)-- ([xshift=-0.5pt]frame.north east);
   }
]}
{\end{tcolorbox}}

%%%% --- end foldboxy preamble ----- %%%%%
%%==== colors from yaml ===%
\definecolor{callout-chapter-connection-color1}{HTML}{FDF2F7}
\definecolor{callout-chapter-connection-color2}{HTML}{A51C30}
\definecolor{callout-takeaways-color1}{HTML}{FDF2F7}
\definecolor{callout-takeaways-color2}{HTML}{BE185D}
\definecolor{callout-perspective-color1}{HTML}{F7F8FA}
\definecolor{callout-perspective-color2}{HTML}{4A5568}
\definecolor{callout-checkpoint-color1}{HTML}{E8F5E9}
\definecolor{callout-checkpoint-color2}{HTML}{2E7D32}
\definecolor{callout-code-color1}{HTML}{F2F4F8}
\definecolor{callout-code-color2}{HTML}{D1D7E0}
\definecolor{callout-notebook-color1}{HTML}{F2F7FF}
\definecolor{callout-notebook-color2}{HTML}{2C5282}
\definecolor{callout-lighthouse-color1}{HTML}{FDF8E6}
\definecolor{callout-lighthouse-color2}{HTML}{B8860B}
\definecolor{callout-quiz-question-color1}{HTML}{F0F0F8}
\definecolor{callout-quiz-question-color2}{HTML}{5B4B8A}
\definecolor{callout-colab-color1}{HTML}{FFF5E6}
\definecolor{callout-colab-color2}{HTML}{FF6B35}
\definecolor{callout-definition-color1}{HTML}{F0F4F8}
\definecolor{callout-definition-color2}{HTML}{1B4F72}
\definecolor{callout-theorem-color1}{HTML}{F5F0FF}
\definecolor{callout-theorem-color2}{HTML}{6B46C1}
\definecolor{callout-resource-slides-color1}{HTML}{E0F2F1}
\definecolor{callout-resource-slides-color2}{HTML}{20B2AA}
\definecolor{callout-resource-videos-color1}{HTML}{E0F2F1}
\definecolor{callout-resource-videos-color2}{HTML}{20B2AA}
\definecolor{callout-resource-exercises-color1}{HTML}{E0F2F1}
\definecolor{callout-resource-exercises-color2}{HTML}{20B2AA}
\definecolor{callout-example-color1}{HTML}{F0F8F6}
\definecolor{callout-example-color2}{HTML}{148F77}
\definecolor{callout-quiz-answer-color1}{HTML}{E8F2EA}
\definecolor{callout-quiz-answer-color2}{HTML}{4a7c59}
\definecolor{callout-principle-color1}{HTML}{F3F2FA}
\definecolor{callout-principle-color2}{HTML}{3D3B8E}
%=============%

\usepackage{hyphenat}
\usepackage{ifthen}
\usepackage{calc}
\usepackage{calculator}



\usepackage{graphicx}
\usepackage{geometry}
\usepackage{afterpage}
\usepackage{tikz}
\usetikzlibrary{calc}
\usetikzlibrary{fadings}
\usepackage[pagecolor=none]{pagecolor}


% Set the titlepage font families







% Set the coverpage font families

\usepackage{bookmark}
\IfFileExists{xurl.sty}{\usepackage{xurl}}{} % add URL line breaks if available
\urlstyle{same}
\hypersetup{
  pdftitle={Introduction to Machine Learning Systems},
  pdfauthor={Vijay Janapa Reddi},
  colorlinks=true,
  linkcolor={Maroon},
  filecolor={Maroon},
  citecolor={Blue},
  urlcolor={Blue},
  pdfcreator={LaTeX via pandoc}}


\title{Introduction to Machine Learning Systems}
\author{Vijay Janapa Reddi}
\date{February 1, 2026}
\begin{document}
%%%%% begin titlepage extension code

  \begin{frontmatter}

\begin{titlepage}
% This is a combination of Pandoc templating and LaTeX
% Pandoc templating https://pandoc.org/MANUAL.html#templates
% See the README for help

\thispagestyle{empty}

\newgeometry{top=-100in}

% Page color

\newcommand{\coverauthorstyle}[1]{{\fontsize{20}{24.0}\selectfont
{#1}}}

\begin{tikzpicture}[remember picture, overlay, inner sep=0pt, outer sep=0pt]

\tikzfading[name=fadeout, inner color=transparent!0,outer color=transparent!100]
\tikzfading[name=fadein, inner color=transparent!100,outer color=transparent!0]
\node[anchor=south west, rotate=0, opacity=1] at ($(current page.south west)+(0.225\paperwidth, 9)$) {
\includegraphics[width=\paperwidth, keepaspectratio]{assets/images/covers/cover-image-transparent-vol1.png}};

% Title
\newcommand{\titlelocationleft}{0.075\paperwidth}
\newcommand{\titlelocationbottom}{0.4\paperwidth}
\newcommand{\titlealign}{left}

\begin{scope}{%
\fontsize{52}{62.4}\selectfont
\node[anchor=north
west, align=left, rotate=0] (Title1) at ($(current page.south west)+(\titlelocationleft,\titlelocationbottom)$)  [text width = 0.9\paperwidth]  {{\nohyphens{Machine
Learning Systems}}};
}
\end{scope}

% Author
\newcommand{\authorlocationleft}{.925\paperwidth}
\newcommand{\authorlocationbottom}{0.175\paperwidth}
\newcommand{\authoralign}{right}

\begin{scope}
{%
\fontsize{20}{24.0}\selectfont
\node[anchor=north
east, align=right, rotate=0] (Author1) at ($(current page.south west)+(\authorlocationleft,\authorlocationbottom)$)  [text width = 6in]  {\coverauthorstyle{Vijay
Janapa Reddi\\}};
}
\end{scope}

% Footer
\newcommand{\footerlocationleft}{0.075\paperwidth}
\newcommand{\footerlocationbottom}{0.475\paperwidth}
\newcommand{\footerlocationalign}{left}

\begin{scope}
{%
\fontsize{25}{30.0}\selectfont
 \node[anchor=north west, align=left, rotate=0] (Footer1) at %
($(current page.south west)+(\footerlocationleft,\footerlocationbottom)$)  [text width = 0.9\paperwidth]  {{\nohyphens{Introduction
to}}};
}
\end{scope}

\end{tikzpicture}
\clearpage
\restoregeometry
%%% TITLE PAGE START

% Set up alignment commands
%Page
\newcommand{\titlepagepagealign}{
\ifthenelse{\equal{left}{right}}{\raggedleft}{}
\ifthenelse{\equal{left}{center}}{\centering}{}
\ifthenelse{\equal{left}{left}}{\raggedright}{}
}


\newcommand{\titleandsubtitle}{
% Title and subtitle
{{\huge{\bfseries{\nohyphens{Introduction to Machine Learning
Systems}}}}\par
}%
}
\newcommand{\titlepagetitleblock}{
\titleandsubtitle
}

\newcommand{\authorstyle}[1]{{\large{#1}}}

\newcommand{\affiliationstyle}[1]{{\large{#1}}}

\newcommand{\titlepageauthorblock}{
{\authorstyle{\nohyphens{Vijay Janapa
Reddi}{\textsuperscript{1}}\textsuperscript{,}{\textsuperscript{,*}}}}}

\newcommand{\titlepageaffiliationblock}{
\hangindent=1em
\hangafter=1
{\affiliationstyle{
{1}.~Harvard University


\vspace{1\baselineskip}
* \textit{Correspondence:}~Vijay Janapa Reddi~vj@eecs.harvard.edu
}}
}
\newcommand{\headerstyled}{%
{}
}
\newcommand{\footerstyled}{%
{\large{}}
}
\newcommand{\datestyled}{%
{February 1, 2026}
}


\newcommand{\titlepageheaderblock}{\headerstyled}

\newcommand{\titlepagefooterblock}{
\footerstyled
}

\newcommand{\titlepagedateblock}{
\datestyled
}

%set up blocks so user can specify order
\newcommand{\titleblock}{{

{\titlepagetitleblock}
}

\vspace{4\baselineskip}
}

\newcommand{\authorblock}{{\titlepageauthorblock}

\vspace{2\baselineskip}
}

\newcommand{\affiliationblock}{{\titlepageaffiliationblock}

\vspace{0pt}
}

\newcommand{\logoblock}{}

\newcommand{\footerblock}{}

\newcommand{\dateblock}{{\titlepagedateblock}

\vspace{0pt}
}

\newcommand{\headerblock}{}

\thispagestyle{empty} % no page numbers on titlepages


\newcommand{\vrulecode}{\textcolor{black}{\rule{\vrulewidth}{\textheight}}}
\newlength{\vrulewidth}
\setlength{\vrulewidth}{2pt}
\newlength{\B}
\setlength{\B}{\ifdim\vrulewidth > 0pt 0.05\textwidth\else 0pt\fi}
\newlength{\minipagewidth}
\ifthenelse{\equal{left}{left} \OR \equal{left}{right} }
{% True case
\setlength{\minipagewidth}{\textwidth - \vrulewidth - \B - 0.1\textwidth}
}{
\setlength{\minipagewidth}{\textwidth - 2\vrulewidth - 2\B - 0.1\textwidth}
}
\ifthenelse{\equal{left}{left} \OR \equal{left}{leftright}}
{% True case
\raggedleft % needed for the minipage to work
\vrulecode
\hspace{\B}
}{%
\raggedright % else it is right only and width is not 0
}
% [position of box][box height][inner position]{width}
% [s] means stretch out vertically; assuming there is a vfill
\begin{minipage}[b][\textheight][s]{\minipagewidth}
\titlepagepagealign
\titleblock

Prof.~Vijay Janapa Reddi

School of Engineering and Applied Sciences

Harvard University

\vspace{80mm}

With heartfelt gratitude to the community for their invaluable
contributions and steadfast support.

\vfill

February 1, 2026

\vfill
\par

\end{minipage}\ifthenelse{\equal{left}{right} \OR \equal{left}{leftright} }{
\hspace{\B}
\vrulecode}{}
\clearpage
%%% TITLE PAGE END
\end{titlepage}
\setcounter{page}{1}
\end{frontmatter}

%%%%% end titlepage extension code

% =============================================================================
% HALF-TITLE PAGE (Volume I)
% =============================================================================
% Standard academic book sequence: half-title -> blank -> title page -> copyright
% The half-title shows only the book title -- no author, no publisher, no date.
\thispagestyle{empty}
\begin{center}
\vspace*{0.3\textheight}
{\fontsize{24pt}{28pt}\selectfont\bfseries\color{crimson} Introduction to\\[0.4em] Machine Learning Systems}\\[2em]
{\large\itshape Volume~I}
\vfill
\end{center}
\clearpage
\thispagestyle{empty}\null\clearpage  % Blank verso (back of half-title)

\renewcommand*\contentsname{Table of contents}
{
\hypersetup{linkcolor=}
\setcounter{tocdepth}{2}
\tableofcontents
}
\listoffigures
\listoftables

\mainmatter
\bookmarksetup{startatroot}

\chapter*{Welcome to Volume I}\label{welcome-to-volume-i}
\addcontentsline{toc}{chapter}{Welcome to Volume I}

\markboth{Welcome to Volume I}{Welcome to Volume I}

\bookmarksetup{startatroot}

\chapter{The DAM Taxonomy}\label{sec-appendix-dam}

\begin{tcolorbox}[enhanced jigsaw, colframe=quarto-callout-note-color-frame, breakable, left=2mm, opacityback=0, colback=white, bottomrule=.15mm, arc=.35mm, rightrule=.15mm, toprule=.15mm, leftrule=.75mm]
\begin{minipage}[t]{5.5mm}
\textcolor{quarto-callout-note-color}{\faInfo}
\end{minipage}%
\begin{minipage}[t]{\textwidth - 5.5mm}

\vspace{-3mm}\textbf{Learning Objectives}\vspace{3mm}

By the end of this appendix, you will be able to:

\begin{itemize}
\tightlist
\item
  \textbf{Classify} any ML system bottleneck into one of three MECE
  categories: Data, Algorithm, or Machine.
\item
  \textbf{Apply} the Iron Law equation to quantitatively diagnose
  performance problems.
\item
  \textbf{Distinguish} between memory-bound and compute-bound workloads
  using Arithmetic Intensity.
\item
  \textbf{Select} appropriate profiling tools and optimization
  strategies for each DAM component.
\item
  \textbf{Evaluate} system health using the DAM Scorecard metrics (I/O
  Overhead, Active Params, MFU).
\end{itemize}

\end{minipage}%
\end{tcolorbox}

The \textbf{DAM Taxonomy} (Data, Algorithm, Machine) is the primary
diagnostic framework for ML systems engineering. It formalizes the
interdependence between information flow, mathematical logic, and
physical execution. When performance stalls or behavior degrades, this
framework enables practitioners to isolate the bottleneck to one of
three mutually exclusive and collectively exhaustive (MECE) components.

\section{Diagnostic Summary}\label{diagnostic-summary}

The taxonomy maps directly to the \textbf{Iron Law of ML Systems}, as
established in \textbf{?@sec-introduction}.
Table~\ref{tbl-dam-components-ref} summarizes the role, primary physical
constraint, and core optimization pathway for each component.

\begin{longtable}[]{@{}
  >{\raggedright\arraybackslash}p{(\linewidth - 6\tabcolsep) * \real{0.1613}}
  >{\raggedright\arraybackslash}p{(\linewidth - 6\tabcolsep) * \real{0.2339}}
  >{\raggedright\arraybackslash}p{(\linewidth - 6\tabcolsep) * \real{0.2097}}
  >{\raggedright\arraybackslash}p{(\linewidth - 6\tabcolsep) * \real{0.3790}}@{}}
\caption{\textbf{DAM Component Reference.} Role and constraints of Data,
Algorithm, and Machine components within the ML systems
stack.}\label{tbl-dam-components-ref}\tabularnewline
\toprule\noalign{}
\begin{minipage}[b]{\linewidth}\raggedright
\textbf{Component}
\end{minipage} & \begin{minipage}[b]{\linewidth}\raggedright
\textbf{Role}
\end{minipage} & \begin{minipage}[b]{\linewidth}\raggedright
\textbf{Physical Constraint}
\end{minipage} & \begin{minipage}[b]{\linewidth}\raggedright
\textbf{High-Leverage Optimization}
\end{minipage} \\
\midrule\noalign{}
\endfirsthead
\toprule\noalign{}
\begin{minipage}[b]{\linewidth}\raggedright
\textbf{Component}
\end{minipage} & \begin{minipage}[b]{\linewidth}\raggedright
\textbf{Role}
\end{minipage} & \begin{minipage}[b]{\linewidth}\raggedright
\textbf{Physical Constraint}
\end{minipage} & \begin{minipage}[b]{\linewidth}\raggedright
\textbf{High-Leverage Optimization}
\end{minipage} \\
\midrule\noalign{}
\endhead
\bottomrule\noalign{}
\endlastfoot
\textbf{Data (D)} \textbf{Algorithm (A)} \textbf{Machine (M)} &
\textbf{Information} (The Fuel) \textbf{Logic} (The Blueprint)
\textbf{Physics} (The Engine) & Bandwidth (\(BW\)) Operations (\(Ops\))
Throughput (\(R_{peak}\)) & Data Selection
(\textbf{?@sec-data-selection}) Model Compression
(\textbf{?@sec-model-compression}) Hardware Acceleration
(\textbf{?@sec-ai-acceleration}) \\
\end{longtable}

\section{The Iron Law Mapping}\label{the-iron-law-mapping}

The performance of any ML task is governed by the distribution of work
across the DAM components. The \textbf{Iron Law Mapping} reveals which
component's variables dominate the execution time:

\[ \text{Time} = \underbrace{ \frac{D_{vol}}{BW} }_{\text{Data (D)}} + \underbrace{ \frac{Ops}{R_{peak} \cdot \eta} }_{\text{Algorithm (A) / Machine (M)}} + \underbrace{ L_{lat} }_{\text{Overhead}} \]

This equation transforms performance debugging from a qualitative
guessing game into a quantitative engineering problem. Every bottleneck
hides in one of these terms. If your system is slow, it is because you
are moving too much data (\(D_{vol}\)), lacking bandwidth (\(BW\)),
executing too many operations (\(Ops\)), or failing to utilize your
hardware's peak capability (\(\eta\)). The levers below map specific
optimizations to the variable they improve.

\subsection{Component Levers}\label{component-levers}

\begin{itemize}
\tightlist
\item
  \textbf{Data Lever}: Reducing the volume of data (\(D_{vol}\)) through
  deduplication or curriculum learning, or increasing I/O bandwidth
  (\(BW\)).
\item
  \textbf{Algorithm Lever}: Reducing total arithmetic operations
  (\(Ops\)) through pruning, quantization, or architectural refinement.
\item
  \textbf{Machine Lever}: Increasing the denominator of the compute term
  by improving peak throughput (\(R_{peak}\)) or increasing the
  utilization factor (\(\eta\)) via kernel fusion.
\end{itemize}

\section{The Boundary: Arithmetic
Intensity}\label{the-boundary-arithmetic-intensity}

The boundary between \textbf{Data} (Memory-Bound) and \textbf{Machine}
(Compute-Bound) is not arbitrary; it is defined mathematically by the
\textbf{Arithmetic Intensity} (\(I\)) of the workload.

For a rigorous definition of Arithmetic Intensity and the
\textbf{Roofline Model}, see
\textbf{?@sec-system-foundations-roofline-model-5f7c}. Use that model to
quantitatively distinguish between Data and Machine bottlenecks before
applying the optimizations below.

\section{Rules of Thumb}\label{rules-of-thumb}

In the heat of a production outage, you rarely have time to solve the
full Iron Law equation. Instead, veteran systems engineers rely on these
quantitative heuristics to quickly narrow down the search space. Use
these thresholds as your first line of defense.

\begin{itemize}
\tightlist
\item
  \textbf{If GPU Utilization \textless{} 80\%}: You are likely
  \textbf{Data Bound} (or CPU bound). The accelerator is starving.
\item
  \textbf{If GPU Utilization \textgreater{} 95\%}: You are likely
  \textbf{Machine Bound}. The accelerator is fully saturated.
\item
  \textbf{If Batch Size is 1}: You are likely \textbf{Latency Bound}
  (Algorithm overhead dominates).
\item
  \textbf{If Arithmetic Intensity \textless{} 100 Ops/Byte}: You are
  likely \textbf{Memory Bound} (Data/Machine boundary).
\item
  \textbf{If System works in Dev but fails in Prod}: Suspect
  \textbf{Data Drift} (Data component).
\end{itemize}

\section{Anti-Patterns}\label{anti-patterns}

Diagnosing systems is often a process of elimination. Before diving into
complex kernel optimizations, ensure you aren't falling into one of
these common traps that waste engineering cycles.

\begin{itemize}
\tightlist
\item
  \textbf{The Hardware Crutch}: Buying faster GPUs (\textbf{Machine}) to
  fix a slow Python data loader (\textbf{Data}). The new GPUs will just
  idle faster.
\item
  \textbf{The Model Twiddle}: Changing neural architectures
  (\textbf{Algorithm}) when the bottleneck is actually network bandwidth
  or disk I/O.
\item
  \textbf{The Premature Optimizer}: writing custom CUDA kernels
  (\textbf{Machine}) before verifying if the Algorithm is simply doing
  too many unnecessary operations.
\end{itemize}

\section{Case Studies: DAM in the
Wild}\label{case-studies-dam-in-the-wild}

Theoretical constraints often manifest as confusing symptoms in
production. These real-world scenarios illustrate how to apply the
taxonomy.

\subsection{Case 1: The ``Data'' Bottleneck (The Starving
GPU)}\label{case-1-the-data-bottleneck-the-starving-gpu}

\textbf{Symptom}: You provision a massive A100 GPU instance to speed up
training, but your training time hardly improves. \texttt{nvidia-smi}
shows GPU utilization fluctuating between 10\% and 40\%.
\textbf{Diagnosis}: The \textbf{Data} component cannot supply the
\textbf{Machine} fast enough. You are I/O bound. \textbf{The Fix}: This
is not a model or hardware problem. You must optimize the ETL pipeline:
* Move from raw JPEGs (CPU decoding heavy) to TFRecords or WebDataset
(sequential reads). * Increase the number of data loader workers. *
Prefetch batches to GPU memory.

\subsection{Case 2: The ``Algorithm'' Bottleneck (The Latency
Cliff)}\label{case-2-the-algorithm-bottleneck-the-latency-cliff}

\textbf{Symptom}: Your real-time recommendation system fails to meet the
20ms latency SLA (Service Level Agreement). The GPU utilization is low,
and the batch size is 1. \textbf{Diagnosis}: The \textbf{Algorithm} is
too computationally deep for the sequential deadline. You are
latency-bound by serial operations. \textbf{The Fix}: Throwing more
hardware (Machine) won't help because latency is limited by the serial
execution of layers. You must change the Algorithm: *
\textbf{Quantization}: Switch to INT8 to reduce memory fetch time. *
\textbf{Pruning}: Remove redundant heads or channels. *
\textbf{Knowledge Distillation}: Train a smaller student model.

\subsection{Case 3: The ``Machine'' Bottleneck (The Compute
Wall)}\label{case-3-the-machine-bottleneck-the-compute-wall}

\textbf{Symptom}: GPU utilization is pinned at 99\%. Memory bandwidth is
unsaturated. Training is stable but takes 3 weeks. \textbf{Diagnosis}:
You have successfully fed the beast. The system is \textbf{Compute
Bound}. \textbf{The Fix}: You have hit the physical limits of the single
chip. * \textbf{Scale Up}: Move to a newer generation GPU (e.g., A100 to
H100). * \textbf{Scale Out}: Distribute training across multiple GPUs
(Data Parallelism). * \textbf{Lower Precision}: Switch from FP32 to BF16
(doubling theoretical TFLOPs).

\section{Troubleshooting Production
Systems}\label{troubleshooting-production-systems}

Identifying the root cause of performance bottlenecks requires
systematic elimination. Table~\ref{tbl-dam-troubleshooting} provides a
diagnostic matrix for common failure modes observed in production
deployments.

\begin{longtable}[]{@{}
  >{\raggedright\arraybackslash}p{(\linewidth - 6\tabcolsep) * \real{0.1600}}
  >{\raggedright\arraybackslash}p{(\linewidth - 6\tabcolsep) * \real{0.1429}}
  >{\raggedright\arraybackslash}p{(\linewidth - 6\tabcolsep) * \real{0.3600}}
  >{\raggedright\arraybackslash}p{(\linewidth - 6\tabcolsep) * \real{0.3257}}@{}}
\caption{\textbf{DAM Diagnostic Matrix.} Root cause identification and
remediation strategies for common ML systems
failures.}\label{tbl-dam-troubleshooting}\tabularnewline
\toprule\noalign{}
\begin{minipage}[b]{\linewidth}\raggedright
\textbf{Symptom}
\end{minipage} & \begin{minipage}[b]{\linewidth}\raggedright
\textbf{Likely DAM Culprit}
\end{minipage} & \begin{minipage}[b]{\linewidth}\raggedright
\textbf{Diagnostic Question}
\end{minipage} & \begin{minipage}[b]{\linewidth}\raggedright
\textbf{Recommended Action}
\end{minipage} \\
\midrule\noalign{}
\endfirsthead
\toprule\noalign{}
\begin{minipage}[b]{\linewidth}\raggedright
\textbf{Symptom}
\end{minipage} & \begin{minipage}[b]{\linewidth}\raggedright
\textbf{Likely DAM Culprit}
\end{minipage} & \begin{minipage}[b]{\linewidth}\raggedright
\textbf{Diagnostic Question}
\end{minipage} & \begin{minipage}[b]{\linewidth}\raggedright
\textbf{Recommended Action}
\end{minipage} \\
\midrule\noalign{}
\endhead
\bottomrule\noalign{}
\endlastfoot
\textbf{Low GPU Utilization} \textbf{High Latency (P99)} \textbf{High
Training Cost} \textbf{Silent Accuracy Drift} \textbf{Out-of-Memory
(OOM)} & \textbf{Data} \textbf{Algorithm} \textbf{Machine} \textbf{Data}
\textbf{Algorithm/Machine} & Is the data loader keeping up with the
accelerator? Is the model depth or width exceeding the latency budget?
Is the hardware utilization (\(\eta\)) below 30\%? Has the statistical
distribution (\(P_t\)) shifted from \(P_0\)? Does the model state fit in
available VRAM? & Implement prefetching and use binary formats. Apply
quantization (INT8) or structured pruning. Optimize CUDA kernels or use
spot instances. Trigger retraining and update active learning filters.
Use gradient checkpointing or reduce batch size. \\
\end{longtable}

\section{The Tooling Map}\label{the-tooling-map}

Once you have a hypothesis (e.g., ``I suspect I am Machine Bound''), you
need evidence to prove it. Abstract concepts must be measured with
concrete utilities. This map connects the theoretical components to the
specific Linux and Python profiling tools you should reach for.

\begin{longtable}[]{@{}
  >{\raggedright\arraybackslash}p{(\linewidth - 6\tabcolsep) * \real{0.1468}}
  >{\raggedright\arraybackslash}p{(\linewidth - 6\tabcolsep) * \real{0.2936}}
  >{\raggedright\arraybackslash}p{(\linewidth - 6\tabcolsep) * \real{0.2385}}
  >{\raggedright\arraybackslash}p{(\linewidth - 6\tabcolsep) * \real{0.3028}}@{}}
\caption{\textbf{DAM Tooling Map.} Profiling utilities for diagnosing
bottlenecks in each DAM component. Start with the primary tool for quick
triage; use secondary tools for deep-dive
analysis.}\label{tbl-dam-tooling}\tabularnewline
\toprule\noalign{}
\begin{minipage}[b]{\linewidth}\raggedright
\textbf{Component}
\end{minipage} & \begin{minipage}[b]{\linewidth}\raggedright
\textbf{Key Metric}
\end{minipage} & \begin{minipage}[b]{\linewidth}\raggedright
\textbf{Primary Tool}
\end{minipage} & \begin{minipage}[b]{\linewidth}\raggedright
\textbf{Secondary Tool}
\end{minipage} \\
\midrule\noalign{}
\endfirsthead
\toprule\noalign{}
\begin{minipage}[b]{\linewidth}\raggedright
\textbf{Component}
\end{minipage} & \begin{minipage}[b]{\linewidth}\raggedright
\textbf{Key Metric}
\end{minipage} & \begin{minipage}[b]{\linewidth}\raggedright
\textbf{Primary Tool}
\end{minipage} & \begin{minipage}[b]{\linewidth}\raggedright
\textbf{Secondary Tool}
\end{minipage} \\
\midrule\noalign{}
\endhead
\bottomrule\noalign{}
\endlastfoot
\textbf{Data} \textbf{Algorithm} \textbf{Machine} & Batch Load Time
FLOPs, Model Depth GPU Utilization, SM Occupancy & \texttt{tqdm}
(iterations/sec) PyTorch Profiler \texttt{nvidia-smi} & \texttt{iotop},
\texttt{dstat} (Disk I/O) DeepSpeed Flops Profiler Nsight Compute,
Nsight Systems \\
\end{longtable}

\section{The DAM Scorecard}\label{the-dam-scorecard}

To surpass qualitative guessing, use these efficiency ratios to grade
your system's performance against its theoretical limit. This ``Report
Card'' standardizes what ``good'' looks like.

\begin{longtable}[]{@{}
  >{\raggedright\arraybackslash}p{(\linewidth - 8\tabcolsep) * \real{0.1111}}
  >{\raggedright\arraybackslash}p{(\linewidth - 8\tabcolsep) * \real{0.1389}}
  >{\raggedright\arraybackslash}p{(\linewidth - 8\tabcolsep) * \real{0.3958}}
  >{\raggedleft\arraybackslash}p{(\linewidth - 8\tabcolsep) * \real{0.1667}}
  >{\raggedleft\arraybackslash}p{(\linewidth - 8\tabcolsep) * \real{0.1667}}@{}}
\caption{\textbf{The DAM Efficiency Rubric.} Use these three numbers to
characterize any ML system's maturity. \textbf{MFU (Model FLOPS
Utilization)} is the single most important metric for large-scale
training.}\label{tbl-dam-scorecard}\tabularnewline
\toprule\noalign{}
\begin{minipage}[b]{\linewidth}\raggedright
\textbf{Component}
\end{minipage} & \begin{minipage}[b]{\linewidth}\raggedright
\textbf{Metric}
\end{minipage} & \begin{minipage}[b]{\linewidth}\raggedright
\textbf{Definition}
\end{minipage} & \begin{minipage}[b]{\linewidth}\raggedleft
\textbf{Failing Grade (\textless)}
\end{minipage} & \begin{minipage}[b]{\linewidth}\raggedleft
\textbf{Passing Grade (\textgreater)}
\end{minipage} \\
\midrule\noalign{}
\endfirsthead
\toprule\noalign{}
\begin{minipage}[b]{\linewidth}\raggedright
\textbf{Component}
\end{minipage} & \begin{minipage}[b]{\linewidth}\raggedright
\textbf{Metric}
\end{minipage} & \begin{minipage}[b]{\linewidth}\raggedright
\textbf{Definition}
\end{minipage} & \begin{minipage}[b]{\linewidth}\raggedleft
\textbf{Failing Grade (\textless)}
\end{minipage} & \begin{minipage}[b]{\linewidth}\raggedleft
\textbf{Passing Grade (\textgreater)}
\end{minipage} \\
\midrule\noalign{}
\endhead
\bottomrule\noalign{}
\endlastfoot
\textbf{Data} \textbf{Algorithm} \textbf{Machine} & \textbf{I/O
Overhead} \textbf{Active Params} \textbf{MFU} &
\(\frac{\text{Data Wait Time}}{\text{Total Step Time}}\)
\(\frac{\text{Non-Zero Params}}{\text{Total Params}}\)
\(\frac{\text{Achieved FLOPs}}{\text{Peak FLOPs}}\) &
\begin{minipage}[t]{\linewidth}\raggedleft
\begin{quote}
10\% 100\% (Dense) \textless{} 30\%
\end{quote}
\end{minipage} & \textless{} 1\% \textless{} 50\% (Sparse)
\textgreater{} 50\% \\
\end{longtable}

\section{Scaling Laws vs.~The Information
Roofline}\label{scaling-laws-vs.-the-information-roofline}

Systems engineering requires distinguishing between \emph{growth
trajectories} and \emph{fundamental limits}.

\subsection{Scaling Laws (The Journey)}\label{scaling-laws-the-journey}

\textbf{Scaling Laws} are empirical power laws that predict \emph{how
fast} model performance improves as we increase resources. *
\textbf{Kaplan Scaling} (\citeproc{ref-kaplan2020scaling}{Kaplan et al.
2020}): Performance improves predictably with Parameters (\(N\)), Data
(\(D\)), and Compute (\(C\)). * \textbf{Chinchilla Scaling}
(\citeproc{ref-hoffmann2022training}{Hoffmann et al. 2022}): Defines the
\emph{optimal ratio} of these resources (e.g., \(D \approx 20N\) tokens
per parameter).

These laws are \textbf{economic guides}. They tell you: ``If I double my
compute budget, my error rate should drop by \(X\)\%.'' They assume the
information is there to be learned.

\subsection{The Information Roofline (The
Destination)}\label{the-information-roofline-the-destination}

The \textbf{Information Roofline} is the theoretical limit of what
\emph{can} be learned from the data, regardless of scale. * \textbf{The
Ceiling}: The \textbf{Bayes Error Rate} (the irreducible error inherent
in the data). * \textbf{The Slope}: \textbf{Information Density}
(Signal-to-Noise Ratio). * \textbf{The Bottleneck}: If your data has low
information density (e.g., noisy financial tickers), you hit the ``Data
Quality Wall'' long before you hit the Compute Wall.

\textbf{The Diagnostic Lesson}: Scaling Laws predict the \emph{slope} of
improvement. The Information Roofline predicts the \emph{ceiling}. If
your loss curve flattens \emph{before} the Scaling Law prediction, you
have hit the Information Roofline. Adding more GPUs (Machine) or
Parameters (Algorithm) at this point is futile; you must improve Data
Quality.

\section{Exercises}\label{exercises}

\begin{tcolorbox}[enhanced jigsaw, colbacktitle=quarto-callout-tip-color!10!white, breakable, leftrule=.75mm, colback=white, bottomrule=.15mm, title=\textcolor{quarto-callout-tip-color}{\faLightbulb}\hspace{0.5em}{Check Your Understanding}, toprule=.15mm, bottomtitle=1mm, coltitle=black, colframe=quarto-callout-tip-color-frame, left=2mm, opacityback=0, rightrule=.15mm, opacitybacktitle=0.6, toptitle=1mm, arc=.35mm, titlerule=0mm]

\textbf{Exercise 1: Component Identification} A production image
classification service runs on an A100 GPU. The \texttt{nvidia-smi}
output shows 25\% GPU utilization, while \texttt{iotop} reveals the disk
is saturated at 100\%. Which DAM component is the bottleneck? What are
two specific optimizations you would recommend?

\textbf{Exercise 2: Iron Law Analysis} Consider a transformer model with
7B parameters performing inference with batch size 1. The model requires
14 TFLOPs per forward pass. On an H100 GPU (1,979 TFLOPs peak FP16), the
measured latency is 50ms. Calculate the achieved utilization (\(\eta\)).
Is this system Data-bound, Algorithm-bound, or Machine-bound? Justify
your answer.

\textbf{Exercise 3: Scaling Law vs.~Information Roofline} Your team has
been training a sentiment analysis model. After scaling from 125M to 1B
parameters (8× increase), validation loss improved from 0.45 to 0.42
(6.7\% improvement). Chinchilla scaling would predict a
\textasciitilde15\% improvement for this compute increase. What does
this discrepancy suggest? Which DAM component should you investigate
first, and why?

\textbf{Exercise 4: Anti-Pattern Detection} A colleague proposes
upgrading from 4× A100 GPUs to 8× H100 GPUs because training is ``too
slow.'' Before approving the \$200K hardware purchase, what three
diagnostic questions would you ask? Map each question to the DAM
component it investigates.

\end{tcolorbox}

\begin{tcolorbox}[enhanced jigsaw, colbacktitle=quarto-callout-note-color!10!white, breakable, leftrule=.75mm, colback=white, bottomrule=.15mm, title=\textcolor{quarto-callout-note-color}{\faInfo}\hspace{0.5em}{Summary}, toprule=.15mm, bottomtitle=1mm, coltitle=black, colframe=quarto-callout-note-color-frame, left=2mm, opacityback=0, rightrule=.15mm, opacitybacktitle=0.6, toptitle=1mm, arc=.35mm, titlerule=0mm]

The DAM Taxonomy provides a systematic framework for diagnosing ML
systems bottlenecks:

\begin{itemize}
\tightlist
\item
  \textbf{Data (D)}: Information flow, constrained by bandwidth.
  Optimize via data selection and I/O pipelines.
\item
  \textbf{Algorithm (A)}: Mathematical logic, constrained by operations.
  Optimize via compression, pruning, and quantization.
\item
  \textbf{Machine (M)}: Physical execution, constrained by peak
  throughput. Optimize via hardware upgrades or improved utilization.
\end{itemize}

The Iron Law equation
(\(\text{Time} = \frac{D_{vol}}{BW} + \frac{Ops}{R_{peak} \cdot \eta} + L_{lat}\))
quantifies these constraints. Use Arithmetic Intensity to determine the
Data/Machine boundary, and the DAM Scorecard to evaluate system
maturity. Remember: diagnose before you optimize.

\end{tcolorbox}

\phantomsection\label{refs}
\begin{CSLReferences}{1}{0}
\bibitem[\citeproctext]{ref-hoffmann2022training}
Hoffmann, Jordan, Sebastian Borgeaud, Arthur Mensch, Elena Buchatskaya,
Trevor Cai, Eliza Rutherford, Diego de Las Casas, et al. 2022.
{``Training Compute-Optimal Large Language Models.''} \emph{arXiv
Preprint arXiv:2203.15556}, March.
\url{http://arxiv.org/abs/2203.15556v1}.

\bibitem[\citeproctext]{ref-kaplan2020scaling}
Kaplan, Jared, Sam McCandlish, Tom Henighan, Tom B. Brown, Benjamin
Chess, Rewon Child, Scott Gray, Alec Radford, Jeffrey Wu, and Dario
Amodei. 2020. {``Scaling Laws for Neural Language Models.''} \emph{arXiv
Preprint arXiv:2001.08361}, January.
\url{http://arxiv.org/abs/2001.08361v1}.

\end{CSLReferences}


\backmatter

\clearpage


\end{document}
